\title{N'EN PA LO TEN}

\author{Pièse icrita pe Le Digourdì}
\date{Téatro Splendor de Veulla, 9 avrì 2016}

\maketitle

\markboth{\MakeUppercase{\thetitle}}{\MakeUppercase{\thetitle}}

\fotocopertina{Foto/2016/gruppo.jpg}{Paolo Cima Sander (avouì son tseun Margot), Francesca Lucianaz, Marco Ducly, Simone Roveyaz, Jordy Bollon, Thierry Jorrioz, André Comé, Sophie Comé}{Marlène Jorrioz, Jo\"{e}l Albaney, Stéphanie Albaney}{2016}

\LinkPiese{N'en pa lo ten}{https://www.youtube.com/watch?v=dc_RIE5WGek}{.5}

\souvenir{Lo 2016 l'è it\'o l'an de la tranzich\'on euntre le fondateur di Digourdì é la relève. Lo premì Prézidàn di Digourdì, que pe \textit{privacy} ara querièn Pippo, l'a fit\'o son premì an comme ex-Prézidàn avouì eunna mia tro d'alcol\ldots mi pa aprì la pièse, douàn!

Sen partì vito lo mateun pe tsardjì la \textit{scénographie}. Tsardjà le-z-Ape é le camiontcheun, n'en bi dou creppe eun tchi Pierre Savioz é Remo Comé. Aprì, avouì totta la compagnì n'en fé maenda i restoràn Monte Emilius. Damadzo que le dijestif n'en bi-le eun tchi Michel Celesia é noutro \textit{ancien Président} l'è chortì de la cantin-a belle tsa. Can sen arrevoù i Splendor, la \textit{scénographie} l'ie dza belle plachaye; donque, pe Pippo l'ie l'aoua de l'apér\'o euntchì le Moldave, eunc\'o se l'ie catr'aoue é de lé a pouza fayè proué. Lé me rappello que n'en acséléroù é Pippo l'è prézentou-se pe la proua jénéralla complètemàn liquid\'o. Finque si lo palque, dimèn que résitave douàn lo pebleuque, l'ie eun fourma.

Fenì lo spettaclle, Pippo l'ie apersi-se que son \textit{taux} alcolémique l'ie cal\'o, mi pa prao pe èidji-no a dibarachì. A ditsardjì totta la \textit{scénographie} sen rest\'o, bièn euncats\'o, mé, Jordy é Simone.

Pouì deue, donque, que lo 2016 l'è it\'o l'an de la tranzich\'on euntre fondateur é relève, mi l'è it\'o eunc\'o l'an iaou n'en comprèi que faillè divin-ì bièn pi professionel.
}{André Comé}


%Primo anno vero in cui è iniziata la transizione tra vecchia e nuova parte
%Qualcosa su Richard che ha fatto la voce
%SOUVENIR:Primo anno vero in cui è iniziata la transizione tra vecchia e nuova parte
%Souvenir : Hard drinking per il primo presidente. Partiti presto per caricare la scenografia. Caricato tutto bevuto due colpi da Pierre e Remo. Poi pranzo a ME. Poi bevuto digestivo giù a pollein da Michel Celesia. Joel, pres uscente, era già caldo... Arrivato sul palco completamente fuori uso... E anche sul palco era bello caliente
%Dopo allestimento... Aperitivi dalla Moldave. Lì abbiamo accelerato. Dato il tasso alcolico anche post pièce, per sbarazzare siamo rimasti in tre: Jordy, Io e Simone. Molto incazzati... Forse è stato l'anno in cui abbiamo capito di diventate un pelo più professionali

\queriaouzitou{
\begin{itemize}
\item[$\bullet$] Margot, lo petchoù tseun protagoniste de la \textit{scène} finale, l'è itaye vardaye dérì le rid\'o pe lo pappa de Paolo Cima Sander. Pe la vardé tranquila, l'a faillì lèi baillì eunna matse de croquette\ldots for probablo que l'è dèi salla nite lé que l'a attacoù a èi tchica de problème de santé!

%margot "cagnolino" -> c'era papà dietro le quinte a darle dei croccantini.. da lì infatti ha iniziato a star male XD.. l'ha riempita come un'uovo!

%\item[$\bullet$]-> 2016 "scherzo di roveayz o joel... mi hanno legato le scarpe da cacciatore

\item[$\bullet$] La pièse \og N'en pa lo te\fg counte de fas\'on bièn ironique la fiacca di bur\'o pebleuque. Deun lo 2016, pe le Digourdì n'ayè bièn de-z-atteur que trailloon pe de-z-ouficho réjonal, mi pe seutta pièse gneun de leur l'a partésipoù. Gneun de leur l'ie refezose; l'ie fran it\'o eun case. Anse\ldots aprì la pièse sayoon tcheu triste de pa èi partésipoù a l'icriteua di teste, vi que l'arian i bièn de-z-idoù pe rendre eunc\'o pi comique la counta.

\item[$\bullet$] 	\og N'en pa lo ten\fg reprézente lo débù deun la compagnì de Richard Cunéaz, lo premì atteur itrandjì di Digourdì. Sa estraordinéa capasitoù, malgré eunc\'o dzoueun-o, d'eunterpretì de personadzo vioù ou tipique de la tradech\'on valdoténa, l'a pouchà lo directif a lo térì dedeun la compagnì, eun cllouzèn eun joueu pe le-z-orijine pompiolentse! Pe seutta pièse Richard pouye pa si lo palque, mi prite sa vouése derì le rid\'o.

\item[$\bullet$] Pe diye 	\og $\#$\fg (\textit{hastag} ou \textit{cancelletto}), lo sembole eumpléyà si Instagram pe décriye le fotografie, la compagnì l'a propozoù eun néolojisme: petchoudacllenda.

\item[$\bullet$] Eunna partiya di spettateur l'a pa tan appréchà ou l'è senti-se offenchaye pe la parodì si la fiacca di bur\'o pebleuque. Naturellamente, seutte creteuque  son itaye féte pe eunna partiya de sise que trailloon deun lo pebleuque. Mi, can mimo, recougnisèn que, a niv\'o de réalizach\'on, l'è cheur pa itaye eunna di pièse pi dzente.
\end{itemize}
}

%Curiosità:
%Cape Treisou : dal "capo" CVA Trisoldi... Magari non scriverlo

\Scenographie
\begin{itemize}
\item[$\bullet$] 1 streutteua de bouque pe reprézenté eun \textit{accueil} de eugn ouficho pebleuque;
\item[$\bullet$] 1 trebeillet\footnote{ \textit{Roue d'un moulin à eau, de baratte actionnée par l'eau - Nouveau dictionnaire de patois vald\^otain}. Deun seutta pièse eumpléyèn trebeillet comme néolojisme pe diye \textit{tornello}.} pe fé entré le-z-eumpléyà deun lo bur\'o é pe le fé teumbré lo cartelleun\footnote{ Calque su la paolla \textit{cartellino}.}. Pe le-z-\textit{handicapé} n'at eunc\'o eun cantchel;
\item[$\bullet$] 3 fondal gri, ate 4 mètre pe reprézenté le meur d'eun ouficho;
\item[$\bullet$] 1 pourtapaapleu avouì dedeun eun per de paapleu é eugn èima di fiolet;
\item[$\bullet$] 1 planta dérì laquella n'at catchaye 1 peura pe djouì i fiolet;
\item[$\bullet$] 5 table avouì 5 ordinateur é de baradziye de tsaque eumpléyà: plime, papì, documàn, botèille, frouite (pomme, banane pe Touéno, deun son casette), plante, tasse, mase de carte (deun lo téèn de Richard Foille);
\item[$\bullet$] 5 caèye pe l'ouficho;
\item[$\bullet$] 1 armadiette avouì dedeun 2 fezì de tsasse. Lo dérì de l'armadiette l'è iver pe pouèi fé entré Margot, lo pégno tseun de tsasse de Touéno Lo Gra. N'at eunc\'o d'atréssateua pe djouì i Ping Pong;
\item[$\bullet$] 1 pourtamantì avouì bièn de patrasse desì: de-z-arbeillemèn pe dou tsachaou, 1 tsapì, 1 per de biicllo pe colaté, 1 djacca di-z-esquì.  Dezò tcheu sise pateun n'a catchà 1 per de-z-esquì;
\item[$\bullet$] 1 casque é eugn arbeillemèn pe alì eun beseclletta (dezò la tabla de Richard Foille);
\item[$\bullet$] 1 atrésateua pe djouì i Golf (catchaye dezò la tabla de Ottino Spritz);
\item[$\bullet$] Deun l'\textit{accueil}: 1 cafetchiye, 1 grou mase de cllo , 1 coulan-a \textit{hawaïenne}, 1 déodoràn, 1 \textit{torche} \torcia, 1 djacca di Tsarvensoù Fiolet ;
\item[$\bullet$] 1 trotinette;
\item[$\bullet$] 1 saquette avouì 3 brioche é eun séléì;
\item[$\bullet$] 4 \textit{lampe frontale} (deun lo téèn de Touéno Lo Gra);
\item[$\bullet$] 1 pourtapacque, 1 cart\'on de Spritz é 1 paquette de papì;
\item[$\bullet$] 1 cabarè avouì 4 tase di cafì é eun vèyo de Limoncello;
\item[$\bullet$] 1 grou matérasse;
\item[$\bullet$] 1 beseclletta pe le mèinoù;
\end{itemize}

\setlength{\lengthchar}{4cm}

\Character[OTTINO SPRITZ]{SPRITZ}{Spritz}{Eumpléyà réjonal que lame l'aoua de l'apéritif, \name{Simone Roveyaz}}

\Character[THIERRY]{THIERRY}{Thierry}{Eumpléyà réjonal jamì entroù pe eugn ouficho,  \name{Thierry Jorrioz}}

\Character[SANDRINO]{SANDRINO}{Sandrino}{Valet di bur\'o réjonal, \name{Marco Ducly}}

\Character[GENE]{GENE}{Gene}{Vétchot di péì eun pench\'on, \name{Jordy Bollon}}

\Character[TOUÉNO LO GRA]{TOUÉNO}{Tueno}{Eumpléyà réjonal fagnàn, que meudze (sourtoù la tseur) é bèi to lo dzor (pocca d'éve);  \name{Paolo Cima Sander}}

%mettere zucchino
\Character[GEROMINE\\ COUCHOTTA]{GEROMINE}{Geromine}{Eumpléyaye réjonalla végane, salutiste é que lame lo \textit{Zumba}, \nameF{Stéphanie Albaney}}

%mettere foglia
\Character[RICHARD FOILLE]{RICHARD}{Richard}{Eumpléyà réjonal bièn gropoù a la nateua é i bièn di bitche, \name{Jo\"{e}l Albaney}}

\Character[FINE SAVENTA]{SAVENTA}{Saventa}{Eumpléyaye réjonalla que eumplèye tcheu le \textit{social} (Facebook, Twitter, Instagram) é sa totte le bague di-z-atre, \nameF{Marlène Jorrioz}}

\Character[COURSIÉ]{COURSIÉ}{Corriere}{\textit{Courrier} avouì pocca voya de traillì, \nameF{Francesca Lucianaz }}

\Character[ÉLECTRISIÈN]{ÉLECTRISIÈN}{Elettricista}{\'Electrisièn de la Réj\'on, \name{Thierry Jorrioz}}

\Character[SERVENTA]{SERVENTA}{Barista}{Serventa d'eunna cantin-a protso de Palase Réjonal, \nameF{Sophie Comé}}

\Character[GRAN DIRECTEUR\\ CARLO TR\`EISO\`U]{TR\`EISO\`U}{Treisou}{Lo Cap di bur\'o, bièn sportif é actif, \name{André Comé}}

\Character[]{TCHEUTTE}{Tcheutte}{.}

\Character[]{TOUÉNO É RICHARD}{Tuenoerichard}{.}

\setlength{\lengthchar}{4cm}

\DramPer

\act[Acte I]

\ridoiver

\scene[-- L'aoua di cartelleun]

\StageDir{Pe totta la pièse saré proyéttaye eunna vidéo si eun écran pozichon-où dérì la \textit{scène} preunsipalla. La vidéo moutre eunna grousa moutra que s'arite can la \textit{scène} partèi pe counté sen que capite a sall'aoua.}

\StartVideo{https://youtu.be/7tyzy-PbmtY}{Moutra}

\StageDir{Lo palque l'è eun bur\'o réjonal. A gotse n'en l'\textit{accueil} avouì lo trebeillet, iaou tcheu le-z-eumpléyà teumbron lo cartelleun. Fran a l'entraye, n'at eun pourtapaapleu. I mentèn di palque n'en sinque table é sinque quèye. Deussì le table n'at eun ordinateur é totte sorte de documàn, plime é foillette. N'en eunc\'o eunna planta, eugn armouare de feur é eun pourtamantì.}

\StageDir{Vouése foua campe \vouese : ``Sitte l'è eun di pi-z-eumpourtàn bur\'o  réjonal. Aprì de jénérach\'on de poleteucco vijonéo é la crize économique, lo peuple valdotèn l'a ottin-ì eugn ate niv\'o de maturach\'on: le campagnar, le posteill\'on, le bitchoulì, le mijì, le mas\'on, le cordan-ì é surtoù le-z-eumpléyà pebleuque l'an finalemàn comprèi que la productivitoù l'è la rèise di bien-\^etre de la sosiétoù é la seula rotta pe chotre de la crize. Dedeun le-z-ouficho pebleuque é dedeun l'ame de tsaque petchoù eumpléyà l'è nèisì eun nouo, euncrouayablo, misterieu é vichaou attatsemèn i travaille!".}

\StageDir{Teuppe \lemieBa, avouì pocca de lemie si l'écran que moutre l'aoua: chouì-z-aoue di mateun.}
\StageDir{Entre Ottino Spritz. L'è djeusto reteria-se. Saoute lo trebeillet. Vèi pa iaou beutte le pià é avouì lo téléfonne se fé de lemie.}

\begin{drama}
\Spritzspeaks\direct{B\^aillèn} Mondjeu,  se cllèrie ren. 
\StageDir{Boueuche l'artchaou contre quetsouza.}

\Spritzspeaks Ahi que mou! L’ie mioù se restavo euncoa avouì le-z-amì a nen bèye eunc\'o eun\ldots é na, n'i pa pouì\ldots é pequé? Pequé dèyo vin-ì seu traillì! Lo deleun l’è todzor eun dizastre. Bon, comenchèn la dzornoù avouì eun dzen cllopeun. \direct{Eun tramèn quetsouza desì sa tabla} Tramèn ià so\ldots  Mondjeu oubliavo! Lo \textit{badge}! Dèyo queti-lo a Sandrino, pouèi me teumbre llou a ouet'aoue.

\StageDir{Ottino pouze son \textit{badge} si la tabla de l'\textit{accueil}. Aprì s'itaoule si la tabla de llou.}

\Spritzspeaks Si cou banì a tcheutte!

\StageDir{La moutra proyéttaye si l'écran avanche. Ara l'è ouett'aoue di mateun.}

\StageDir{Vouése foua campe \vouese : ``La pountoualitoù deun le-z-ouficho pebleuque l'è fondamantala é a ouett'aoue totte le prézanse son réjistraye.}
\StageDir{Entre Thierry. L'è eun ganes\'on é dimèn bèi eun cafì.}

\Thierryspeaks L'è dza ouet'aoue! Me fa timbré lo cartelleun di-z-atre eumpléyà.

\StageDir{Teurie foua eunna rentse de cartelleun é le timbre eun aprì l'atro.}

\Thierryspeaks Si de Touéno lo Gra, si de Fine Saventa, si de Couchotta é si de Richard Foille.

\StageDir{Entre deun lo bur\'o é pouze le cartelleun si le table di-z-eumpléyà.}

\Thierryspeaks L'è dza ouette é sinque! Alèn fé eun cllopeun!

\StageDir{Thierry chor.}

\scene[-- L'è deleun]

\StageDir{Entre Sandrino.}

\Sandrinospeaks L'è deleun é la \textit{scène} recomenche. Alemèn le lemie.

\StageDir{Lemie \lemieSi .}

\Sandrinospeaks\direct{Eugn avèitsèn Ottino} L'atro l'è dza lé que drime.

\StageDir{Sandrino se plache a l'\textit{accueil}.}

\Sandrinospeaks Lo journal n'en atsetou-lo\ldots betèn si la cafetchie pouèi no no réchèn amoddo. \direct{Eugn ivrèn lo journal} N'a finque l'eunser di Fiolet é de la Rebatta! Lo avèitsèn aprì que n'i tro a fie\ldots mondjeu! Lo cartelleun de Ottino Spritz!

\StageDir{Sandrino pren lo cartelleun é galoppe ver lo trebeillet.}

\Sandrinospeaks Vito que piatro areuvve eun retar!

\StageDir{Sandrino timbre lo cartelleun é tourne a sa postach\'on.}

\StageDir{La moutra marque ouette é dji di mateun. Entre Gene.}

\Genespeaks \direct{A Sandrino} Bondzor!

\Sandrinospeaks Bondzor.

\Genespeaks Si vin-ì pe la prateuca de\ldots

\Sandrinospeaks  Ouè, ouè\ldots baillade-mé eun documàn.

\Genespeaks Va bièn.

\StageDir{Gene baille a Sandrino la carte d'identité. Sandrino marque lo non di vétchot si eun papì é aprì lèi baille eun \textit{pass}.}

\Sandrinospeaks Voualà lo \textit{pass}. Djeusto eun momàn\ldots

\StageDir{Gene pren lo \textit{pass} é atèn Sandrino, lequel va ver l’ouficho pe vère se n’a caqueun. Aprì, to todzèn, tourne a l'\textit{accueil}.}

\Sandrinospeaks\direct{A Gene} Mesieu, squezade mi n’a panco gneun disponiblo.

\Genespeaks L'è ouette é dji é n'a gneun? A ouet'aoue fa ataquì a traillì!

\Sandrinospeaks N'a panco gneun opératif. Me diplì.

\Genespeaks\direct{Eun s'inervèn} Mé n'i pa lo ten d'atendre, senque féo ara?

\Sandrinospeaks Tournade aprì.

\Genespeaks Ad\'on vardo lo \textit{pass}.

\Sandrinospeaks Mesieu dèyade me baillì lo \textit{pass}. 

\StageDir{Gene vionde l'itseua é partèi ià.}

\Sandrinospeaks\direct{Eun braillèn} Mesieu! Baillade-mé tourna lo \textit{pass}!

\Genespeaks Na, mé vardo lo \textit{pass}.

\Sandrinospeaks\direct{Comme douàn} Mesieu! Dèyade me baillì tourna lo \textit{pass}!

\Genespeaks Mé payo le-z-empoù é ad\'on mé vardo lo \textit{pass}.

\Sandrinospeaks\direct{Eun s'aprotsèn} Na, vardade pa lo \textit{pass}! Baillade-me-l\'o ara! 

\StageDir{Sandrino stchanque ià di man lo \textit{pass} a Gene é lèi ren lo documàn.}

\Genespeaks\direct{Eun chortèn} Maledecoù!

\Sandrinospeaks Ouè maledecoù! Mesieu!

\StageDir{Gene tourne eun dérì inervoù.}

\Genespeaks Ouè\ldots

\Sandrinospeaks Alade maque bèye le doze blan que bèyade tcheu le mateun é prégnade lo polet a la fenna!

\Genespeaks\direct{Eun chortèn bièn eunfastedjà} Ouficho pebleuque!

\scene[-- Medjì é bèye]

\StageDir{La moutra avanche é marque ouette é car di mateun.}

\StageDir{Vouése foua campe \vouese : ``Ouette é car. Finalemàn fé sa entroù lo premì eumpléyà réjonal: Touéno lo Gra. Eugn ommo to d'eun toque é llatoù i tradech\'on".}

\StageDir{Entre Touéno lo Gra. L'è conflo comme eun bal\'on. Pe le man l'at eun \textit{dossier} é eun saquette.}

\Tuenospeaks Salì Sandrino! Ad\'on? Tot amoddo?

\Sandrinospeaks To bièn, té?

\Tuenospeaks\direct{Eun se totsèn la panse} Mondjeu queun feun senà!

\Sandrinospeaks T'a medjà?

\Tuenospeaks\direct{Eun fièn eun petchoù rotto} Ouè. Té to bièn la feun senà?

\Sandrinospeaks Ouè\ldots

\Tuenospeaks Sen tourna seu a no a rontre le bal\ldots

\Sandrinospeaks\direct{Eun bloquèn Touéno} Ouè, ouè Touéno, brao!

\StageDir{Touéno tsertse de pasì déz\'o lo trebeillet mi areuvve pa, l’è tro gra.}

\Sandrinospeaks Te lo saoute? O fa t'ivrì?

\Tuenospeaks Na lo saouto eugn atro cou! 

\Sandrinospeaks Va bièn, areuvvo!

\StageDir{Sandrino teurie foua eun grou mase de cllo é comenche a tsertchì salla djeusta.}

\Tuenospeaks Acapou-la?

\Tuenospeaks\direct{Eun tsertsèn} Seutta l'è salla di mitcho\ldots ah la voualà!

\StageDir{Sandrino ivre la pégna pourta protso di trebeillet.}

\Tuenospeaks Bièn mersì!

\StageDir{Sandrino tourne a l'\textit{accueil}.}

\Tuenospeaks\direct{Bièn lagnà} Tourna seu\ldots l'è deleun. Tro de bague a fie.

\StageDir{Touéno pouze le bague de llou si sa tabla. Aprì se vionde ver Ottino, que l'è eunc\'o eun tren de drimì si la tabla.}

\Tuenospeaks Comèn l'è Ottino? Amoddo?

\StageDir{Ottino se rèche.}

\Spritzspeaks\direct{Eun se viondèn de l'atro coutì} Ouè, té t'a praou medjà?

\Tuenospeaks Ouè tot amoddo. Té lo feun senà? Bièn aloù?

\Spritzspeaks\direct{Bièn lagnà} Bondàn bièn!

\Tuenospeaks Ad\'on tranquilo, travailla maque tranquilo.

\StageDir{Touéno, to todzèn, tourne a seun poste é s'achouatte.}

\scene[-- Tendro]

\StageDir{La moutra marque ouette é demì di mateun.}

\StageDir{Vouése foua campe \vouese : ``Ouette é demì. Fé son euntraye Geromine Couchotta. Dipendenta réjonalle que lame lo spor, lo \textit{Zumba}, medjì pocca, san é naturel.}

\StageDir{Entre Geromine Couchotta, totta sourienta é eun fourma.}

\Gerominespeaks Bondzor Sandrino!

\Sandrinospeaks Oh bondzor Couchotta!

\Gerominespeaks Comme da cotima t'apreste to té?

\Sandrinospeaks Ouè apresto mé! Té dimèn beutta so	\ldots

\StageDir{Sandrino baille a Couchotta eunna coulan-a \textit{hawaïenne}.}

\Sandrinospeaks \ldots é itsaouda-té!

\StageDir{Sandrino fé partì eunna mezeucca brazilienna:}

\sound{https://www.youtube.com/watch?v=kk4uddaHdDE}{Samba do Brasil - Bellini}

\StageDir{Couchotta se beutte la coulan-a \textit{hawaïenne} é s'itsaoude. Sandrino levve l'atchaou di trebeillet é Geromine, a ten de mezeucca, lèi pase déz\'o sensa lo totchì, eun fièn lo Limbo. La mezeucca s'arite.}

\Gerominespeaks\direct{Contenta} Oueu l'è bièn aloù!\direct{A Sandrino} Mé dirio que  demàn pouèn bèichì ba eunc\'o de 4 cm!

\Sandrinospeaks D’acor Couchotta, bèichèn ba totte demàn!

\StageDir{Sandrino, douàn de tournì i seun poste, bèiche lo trebeillet é Couchotta s'aproste a sa tabla.}

\Gerominespeaks\direct{A Touéno} Bondzor Touéno! Tot amoddo?

\Tuenospeaks Salì Couchotta! Tot amoddo\ldots

\Gerominespeaks\direct{Ver Ottino} \ldots é seu Spritz?

\StageDir{Spritz boudze pa, conteneuvve a drimì. Geromine s'achouatte a sa plase.}

\Gerominespeaks\direct{A Touéno} Comèn l'è aloù lo feun senà? T'aré prao medjà?

\Tuenospeaks Mondjeu! Eugn amì de mé l'a pourto-me eun cartì de \textit{cerf}! Que bon! Lo desando n'en to medja-lo. Que bon! Que tendro\ldots

\Gerominespeaks Oh! Foula mé que n'i demando-te senque t'a medjà! Le repounse le si dza.

\Tuenospeaks \ldots é la demendze mateun n'en tchoué eun \textit{canard} é n'en medja-lo, que bon, l'ie grou pai! Aprì lo nite n'en medjà eun pégno lapeun que l'ie eunna baga! Mondjeu!

\Gerominespeaks	\direct{Bièn malechaye} Senque t'a fi?

\Tuenospeaks N'en medjà eun lapeun!

\Gerominespeaks Eun lapeun!\direct{Presta a plaoué} Mi na lo lapeun\ldots son tan sayo, tan delecatte, tan tendro\ldots

\Tuenospeaks \ldots tendro! Ouè, eunc\'o magàn que l'a maque an di, l'a to tritou-lo ba! Que bon; é pe clloure\ldots eunc\'o dou-z-où pe caì to ba.

\Gerominespeaks Vouì pamì acouté eunna paolla de sen que te di. Prao.

\Tuenospeaks Va bièn. Pitoù té senque t'a fé? 

\Gerominespeaks\direct{Bièn contenta de repoundre} Ad\'on n'i fé de bague trè-z-euntéressante. Lo desando iproù si allaye eun Veulla pe manifesté, pe difendre le droué di dzeleunne! 

\Tuenospeaks\direct{\'Eton-où'} Le droué di dzeleunne?

\Gerominespeaks Oué eunc\'o leur l'an de droué!

\Tuenospeaks\direct{Ironique} Tcheu van lo desando pe difendre le dzeleunne, normalle!

\Gerominespeaks Derian tcheut alé! La demendze, i contréo, avouì eun per de-z-amì sen divija-no pe totta la Val d'Outa é sen aloù ibofé le bat\'on di bitchoulì, sise que eumplèyon pe fouétì noutre saye é dzente vatse sensa drouette.

\Tuenospeaks \direct{Ironique} \textit{Wow}! Véo de bague eumpourtante t'a fé! Bièn aloù, pa fé de mou! Pitoù, senque t'a medjà? 

\Gerominespeaks N'i medjà eunna baga que te pouria lamé eunc\'o té. L'ie eun gonégeun avouì eunna reduch\'on de-z-andive é dou gran de sa roza de l'Himalaya! L'ie la feun di mondo!

\Tuenospeaks Te vèi que eunc\'o té te meudze la vianda\ldots é lo gonégeun! Aprì se son sise de Ferré avouì la trifolla de Feleunna, que can te la queurie nen areuvve eunna!

\Gerominespeaks Na Touéno! T'a pa comprèi ren. Lo gonégeun l'ie de \textit{soia}!

\Tuenospeaks De \textit{soia}?

\Gerominespeaks Ouè pequé i dzor de oueu avouì le-z-OGM\footnote{ \textit{Organisme Génétiquement Modifié}.} te pou medjì sensa fata de tchoué le bitche.

\Tuenospeaks\direct{Dispéoù} Soplé, prèdza-mé pa de seutte counte foule di-z-OGM, nen prèdzon to lo ten lo journal, la radi\'o\ldots na, na mé si pe lo veun é la tseur rodze. Prédzèn d'atro\ldots pitoù, l’ommo iao t'a quetou-lo?

\Gerominespeaks\direct{Triste} Touéno, tcheu le cou seutta dimanda. Te sa beun que n'i pa lo choze.

\Tuenospeaks Djeusto, l'è vrèi que té te lame pa lo salàn! 

\scene[-- La nateua di coucouille]

\StageDir{La moutra marque nou mouèn car di mateun.}

\StageDir{Vouése foua campe \vouese : ``Nou mouèn car. Sensa prisa entre noutro catrimo eumpléyà réjonal: Richard Foille. La sovegarda de la nateua l'è sa rèiz\'on de viya.}

\StageDir{Entre Richard Foille desù eunna \textit{trottinette} é s'arite douàn Sandrino.}

\Richardspeaks Bondzor Sandrino! Comèn l'è?

\Sandrinospeaks Amoddo, mi t'a atsetoù eun nouo moublo?

\Richardspeaks Ouè, n'i atsetoù eun nouo moublo é l'è comoddo, \direct{eugn ézajérèn} mi te diyo que l'è comoddo! L'è comoddo pequé pouì quetì la machina i mitcho\ldots

\Sandrinospeaks \ldots é te paye pa le postèdzo eun Veulla!

\Richardspeaks Sen l'è vrèi mi vouillao pa diye sen. L'è comoddo pequé eun quetèn la machina i mitcho èidzo la Réj\'on a reuspecté le \textit{Protocolli de Kyoto}! 

\Sandrinospeaks \textit{Protocolli de Kyoto}?

\Sandrinospeaks Ouè! Si fran contèn é aprì si moublo servèi da matte eunc\'o pe le pouse fin-e. Te sen pa \direct{eugn aneflèn l'er} queunta grama er que n'at? 

\Sandrinospeaks Ouè, mi sen l'è Touéno que l'è pasoù devàn\ldots l'a medjà pezàn lo feun senà é te sa\ldots

\Richardspeaks Mi na! Si eun tren de prédji di\ldots

\StageDir{Sandrino teurie foua eun déodoràn é comenche lo dziflé pe l'er.}

\Sandrinospeaks Betèn eunna mia de so que l'è mioù.

\Richardspeaks Mi na! Gava ià salla baga, queunta grama baga! So fé pa di bièn a la nateua; no no fa tsertchì de prézervé la nateua! Squersa pa!

\Sandrinospeaks Na na squerso pa.

\Richardspeaks Acoutta, èidza-mé a pourté si moublo deun l'ouficho.

\Sandrinospeaks Va bièn, ad\'on tsertsèn tourna la cllo.

\StageDir{Sandrino teurie foua lo grou mase de cllo é dimèn pren la \textit{trottinette} de Richard. Richard pase eun dziill\'on déz\'o lo trebeillet.}

\Richardspeaks Te me pase lo moublo, soplé?

\StageDir{Sandrino l'è eunc\'o eun tren de tsertchì la cllo djeusta pe ivrì lo trebeillet.}

\Sandrinospeaks\direct{Stouffieu} Mi va caquì seutte cllo! Fièn pouèi\ldots

\StageDir{Sandrino levve la \textit{trottinette} é la pase a Richard desì lo trebeillet.}

\Richardspeaks Ouè fièn pi vito pèi.

\StageDir{Sandrino tourne i seun poste.}

\StageDir{Richard, to contèn é pozitif, vionde i mentèn de l'ouficho avouì la \textit{trottinette}. Geromine l'è eun tren de bèye eunna tizan-a.}

\Richardspeaks\direct{I seun collègue} Bondzor, ad\'on v'ouite eun fourma?

\Tuenospeaks Ouè, tot amoddo.

\Richardspeaks Comenchèn la senà.

\Gerominespeaks Amoddo, \direct{eun somondèn sa tase} t'ou eunna pégna gotta?

\Richardspeaks Senque l'è sen?

\StageDir{Richard s'aproste pe aneflì, mi a demì mètre de distanse se teurie eun dérì.}

\Richardspeaks Na, na, soplé na!

\Tuenospeaks\direct{Eun avèitsèn la \textit{trottinette}} Richard mi de iaou t'areuvve? 

\Richardspeaks De mèiz\'on! Mi si moublo l'è formidablo, avouì so te fé de kilomètre é te t'apersèi gnenca! 

\Tuenospeaks Ouè mi tot eun pouchèn avouì le pià!

\Richardspeaks Djèique! Té avouì la panse que t'a\ldots  gnenca demì de té reste seu desì!

\Tuenospeaks\direct{Ironique} Té todzor eunna dzenta paolla pe mé! Can mimo, counta vèi, comèn l'è aloù lo feun senà?

\Richardspeaks Lo feun senà l'è bièn aloù! Pouzo si moublo é vo counto tot.

\StageDir{Richard pouze la \textit{trottinette} countre lo meur.}

\Richardspeaks Ad\'on demendze a doze é doze si vin-ì secretéo de la F.S.C.V.!

\Tcheuttespeaks De senque?

\Richardspeaks\direct{Ver Touéno} F.S.C.V., te sa pa senque l'è?

\Tuenospeaks Na.

\Richardspeaks Mi te sa ren té.\direct{Ver Couchotta} Té?

\Gerominespeaks Na, lo nom me di pa ren.

\Richardspeaks Fédérach\'on pe la Santé di Coucouille valdoténe!

\Tuenospeaks Oh mondjemé! Ara si queriaou, euspleuca vèi senque l'è seutta counta.

\Richardspeaks N'en eunna blita de travaille, sourtoù dérimente! Le coucouille fa l'èidjì, fa bailli-lei eunna man pe le fé saouté si, leur  vouillon tsire pe tèra, mi fa le pouchì si pe l'er\ldots mi aprì\ldots le dzi vouillon le tchoué!

\Gerominespeaks Senque? Vouillon tchoué le coucouille? Mi senque l'an fi-le, son tan saye.

\Tuenospeaks Mi l'è normal, avouì tcheu le dan que fan!

\Richardspeaks Mi queunse dan? Le coucouille son eumpourtante! 

\Tuenospeaks Mi iaou son eumpourtante? I mitcho de té!

\Gerominespeaks Pe la nateua son eumpourtante.

\Richardspeaks\direct{A Couchotta} Touéno comprèn ren. No le coucouille l'èidzèn, le pouchèn si pe l'er, lèi baillèn finque eunna man pe se reprodouiye\ldots

\Tuenospeaks\direct{``Incredulo"} Na, na, na, squeza-mé, finque se reprodouiye? Espleuca-mé seutta baga!

\Richardspeaks Ad\'on, pe se reprodouiye fa itre a dou\ldots eun la varde de devàn é lèi ivre eunna mia le-z-ale, l'atro de dérì nen pouche eunna, mi lèi baille eunna pégna man, djeusto pe sen que fat! Tsertsèn de fé lo poussiblo.

\Tuenospeaks Donque té te sarie secretéo pe pouchì eunna coucouille dedeun eugn atra?

\Richardspeaks Té te comprèn ren! Ita achouatoù si ta caèya é meudza pi de pomme!

\Gerominespeaks\direct{A Touéno} Touéno soplé l'è eumpourtàn, alèn eunc\'o no!

\Tuenospeaks Pe senque fie? Pe pouchì de coucouille eunna desì l'atra? Saré pa eunna baga utilla?

\Gerominespeaks Mi ouè!

\Richardspeaks\direct{A Couchotta} Té ouè que te comprèn, t'a eunna martse eun pi. Veun avouì no, n'en fata de dzi comme té deun la Fédérach\'on.

\Gerominespeaks Volontchì!

\StageDir{Richard satisfé s'achouatte i seun bur\'o .}

\Richardspeaks Can mimo ara quetade-mé pédre pequé n'i le papì de la F.S.C.V. que dèyo pourté eun devàn.

\scene[-- Petchoudacllenda\footnote{ Néolojisme eunventoù pe no Digourdì pe diye \textit{hastag}/\textit{cancelletto}/$\#$.}]

\StageDir{La moutra marque nou aoue di mateun.}

\StageDir{Vouése foua campe \vouese : ``Nou aoue. Finalemàn l'ouficho l'è i complette. Areuvve Fine Saventa, eumpléyà que sensa la tecnolojì di-z-àn 2000 sa pa iaou bouéchì la tita. Sensa lo téléfonne sa dzornoù sarie vouida é platta.}

\StageDir{Entre Saventa avouì eun saquette de brioche.}

\Saventaspeaks Bondzor Sandrino!

\Sandrinospeaks Bondzor.

\Saventaspeaks Eunc\'o oueu si passaye prendre doe brioche pe comenchì bièn la dzornoù. Te la vou i-z-armagnaye, chocolà ou vouida?

\Sandrinospeaks L'è todzor eun plèizì te vère, te pourte todzor de bon-e bague! Fièn chocolà.

\StageDir{Saventa baille la brioche i chocolà a Sandrino.}

\Saventaspeaks Acoutta, fièn la foto di mateun?

\Sandrinospeaks Mi ouè lèi manquerie!

\StageDir{Fine teurie foua lo téléfonne é fé eun \textit{selfie} avouì Sandrino. Si l'écran chor foua, pe eun per de secounde, la fotografiye de Saventa e Sandrino.}

\foto{https://www.instagram.com/p/BD_MwP8nEFL/}{Saventa $\&$ Sandrino}

\Saventaspeaks\direct{Eugn icrièn si lo téléfonne} Petchoudacllenda deleunmateun, petchoudacllenda sandrinotodzorveuste, petchoudacllenda sensabriochesecomenchepa. \direct{Eun avèitsèn lo trebeillet} Sandrino! Coudiye que t'a panco iver-me?

\Sandrinospeaks\direct{Eun prégnèn le cllo} T'arè pa prisa?

\Saventaspeaks N'i pa de ten a pèdre seu avouì té! Boudza-té!

\Sandrinospeaks Eun momàn que me fa acapé la cllo.

\StageDir{Sandrino comenche a tsertchì la cllo djeusta.}

\Saventaspeaks\direct{Ironique} Mogà fa nen gavé eunna mia di mase!

\Sandrinospeaks N'i pa lo ten de fé tcheu sise travaille. Ah èita-là séilla!

\StageDir{Sandrino ivre la pégna pourta protso di trebeillet é Saventa entre eugn ouficho. Aprì la fremme é torne a l'\textit{accueil}.}

\Saventaspeaks Bondzor amì! Bondzor a tcheutte.

\Tcheuttespeaks Salì Fine!

\Saventaspeaks\direct{Eun moutrèn Ottino} Lo noutro l'a tourna fé fita ieur lo nite?

\Tuenospeaks Djèique!

\Richardspeaks Mé n'i gnenca vi-lo! \direct{Ver Ottino} T'a prèi de couleur dernièramente!

\Spritzspeaks\direct{Eun se réchèn} Di pa de counte foule!

\StageDir{Ottino tourne drimì.}

\Saventaspeaks\direct{A Richard} Richard t'ioù armagnaye ou vouida?

\Richardspeaks Se t'a-z-armagnaye, mé vou matte pe le-z-armagnaye!

\StageDir{Saventa baille a Richard la brioche.}

\Saventaspeaks\direct{Ver Couchotta} Pe té que t'i ma pégna \textit{vegana}, n'i vi si la padze Facebook de Fenna Moderna que seutta senà l'è salla di seleì\ldots

\Gerominespeaks Ouè.

\Saventaspeaks \ldots é ad\'on so l'è djeusto pe té.

\StageDir{Saventa teurie foua eun seleì é lo baille a Couchotta.}

\Gerominespeaks \textit{Wow}! \direct{Eun gneflèn lo seleì} Lo flou l'è bon, mi l'è bio?

\Tuenospeaks Se l'a prèi-lo euntchì Bisson l'è finque trio!

\Saventaspeaks Ouè son kilomètre zéro!

\Richardspeaks Mi comèn fièn a diye kilomètre zéro? Prédzèn de Gressan-Tsarvensoù: n'a pa zéro kilomètre. N'a a di 1 kilomètre. Soplé fiade le pountuel!

\Tuenospeaks\direct{A Richard} Mi té t'a todzor quetsouza a diye! 

\Saventaspeaks\direct{A Richard} Todzor quetsouza que te va pa!\direct{A Touéno} Pe té Touéno l’è avanchaye renque salla vouida\ldots

\Tuenospeaks Bailla-me-là sé la vouida! Lèi cayo dedeun questouza mé!

\StageDir{Touéno pren la brioche, la ivre é lèi beutte eunna banane dedeun.}

\StageDir{Tcheutte lo avèiston mal.}

\Richardspeaks\direct{A Touéno} Se pou pa te vère. Te fé fran acro!

\Tuenospeaks T'a todzor quetsouza a diye!

\Richardspeaks T'areuvve cheur pa a 75 an!

\Tuenospeaks Tracàcha-te pa que avouì la grama lenva que t'a t'areuvve gnenca a 40!

\scene[-- Si pe Comboé]

\Saventaspeaks Can mimo! Mé n'ario eunna noella. \direct{Ver Touéno} Senque t'a fé desando l'éproù?

\Tuenospeaks Desando l'è lo dzor de la tsasse. Si aloù si pe Comboé a tsasse\ldots

\Saventaspeaks\direct{A Richard} \ldots é té? Eunc\'o té a tsasse pe Comboé?

\Richardspeaks A tsasse na, mi si aloù si pe Comboé.

\Saventaspeaks V'ouite pa euncrèija-vo?

\Richardspeaks Mi feguea-té!

\Tuenospeaks\direct{Eun moutrèn Richard} Seutta beurta baga? Mi penso fran pa.

\Saventaspeaks \'Ebeun, mé pe seutta bouite \direct{moutre lo téléfonne} n'i eunna \textit{vidéo} que dimoutre lo contréo\ldots vegnade maque veure! 

\StageDir{Tcheutte, mouèn que Ottino, se levvon é se plachon a l'entor de Saventa.}

\StageDir{Si la tèila l'è proyettaye eunna \textit{vidéo} iaou véyèn Touéno teryì eun creppe de fezì si eugn aoujì  plachà si eunna brantse de eunna planta, déz\'o laquella Richard l'ie eun tren de fé eunna boccon-où avouì sa fameuille.}

\StartVideo{https://www.youtube.com/watch?v=Rl-CojTfsVc&t=7s}{Touéno Lo Gra vs Richard Foille}

\StageDir{Tcheutte s'achouatton i leur poste, mouèn que Richard que ataque a deusqueté avouì Touéno.}

\Richardspeaks Té t'i eugn assasseun! T'a tchoué-me l'aoujì, mé n'ayoù caze prèi-lo eun man mon pégno aoujì!

\Tuenospeaks T'a fé-me scapé ià le do tchevreuille!

\Richardspeaks Ouè é mogà eunc\'o finque le tchevrèi de Marco Fragno!

\Tuenospeaks Mé payo lo carnet de tsasse!

\Richardspeaks Sen eun tren de prédji d'aoujì, t'a tchoué-me l'aoujì! T'a reuscoù de fé-me de mou, n'ayoù caze prèi-lo eun man! T'i eugn assasseun!

\Richardspeaks\direct{Ver Couchotta} Vrèi ou na? Couchotta! Bailla-mé eunna man!

\Gerominespeaks\direct{A Touéno} T'i eunna grama dzi!

\Richardspeaks Assasseun! Arrogàn!

\StageDir{Richard s'achouatte i seun bur\'o .}

\Tuenospeaks Arrogàn? Mi pren-te varda, grama lenva!

\Richardspeaks \`Eita ara se lèi penso me veun caze voya de plaoué\ldots

\Gerominespeaks\direct{A Richard} Te comprègno Richard\ldots

\Richardspeaks Pouo aoujì\ldots fièn-no coadzo, fa alì eun devàn.

\Tuenospeaks Mi pensa té sise dou! Comenchade a traillì, a fé quetsouza que l'è mioù.

\scene[-- Terièn i rèi]

\Richardspeaks  Acoutade, tsertsèn de pa lèi pensé\ldots pouo aoujì, pouo matasse. Ara tsertsèn de fé quetsouza de utilo pe l'ouficho.  Belote ou pinacola?

\StageDir{Richard teurie foua di cassette dou mase de carte.}

\Gerominespeaks Pe la pinacola fa aprotchì le table, se no fièn la belote nen baste eunna.

\StageDir{Silanse.}

\Saventaspeaks Se fa aprotchì le table\ldots a qui no dimandèn? 

\Tuenospeaks A Tchande! Atèn que lo queurio.

\StageDir{Touéno queurie Sandrino avouì lo téléfonne di bur\'o .}

\effet{https://soundcloud.com/user-234168361/telefono}{Téléfonne}[false]\label{tel}

\StageDir{Sandrino rep\'on.}

\Tuenospeaks Tchande, acoutta\ldots n'en eunna prateuca bièn eumpourtanta\ldots

\Sandrinospeaks Areuvvo i galoppe!

\StageDir{Beutton ba lo téléfonne. Sandrino avouì calme entre deun l'ouficho.}

\Sandrinospeaks Diade-mé.

\Gerominespeaks Sandrino te pouriye dimandé a sise di plan déz\'o se l'an voya de vin-ì sé pe no tramì le table\ldots

\Sandrinospeaks N'i senti-le douàn pe eugn atra baga é l'an deu-me que l'an pa lo ten.

\StageDir{Sandrino tourne i seun poste.}

\Richardspeaks \`Eita que eun réj\'on l'an fran voya de fé ren! Acoutade, vi que n'en pa gneun que no trame le table, fièn la belote.

\Tuenospeaks Fièn la belote. Mé me tramo pa, vegnade sé eun tchì mé.

\StageDir{Richard, Saventa é Couchotta aprotson le caèye a la tabla de Touéno. Richard a drèite de Touéno, Couchotta a sa gotse é Saventa a la gotse de Geromine. Dimèn Touéno fé de caro si la tabla.}

\Gerominespeaks Pe le coble?

\Tuenospeaks Terièn i rèi!

\Spritzspeaks\direct{Eun se réchèn} V'ouèide fata de eun catrimo?

\Tcheuttespeaks Na, na, na ita maque tranquilo!

\Spritzspeaks Ad\'on me viondo de l'atro coutì.

\StageDir{Ottino vionde lo queusseun é tourne drimì.}

\Tuenospeaks Terièn i rèi ad\'on!

\StageDir{Touéno teurie i rèi: Couchotta dèi djouì avouì llou.}

\Gerominespeaks Mi na torna avouì té, mi na. Te m'eunseurte to di lon. Aprì n'i pa voya de me tramì.

\Saventaspeaks Fièn que pe si cou pouèn djouì pouèi.

\Richardspeaks Tourna mé avouì Geromine? Perdèn todzor mé é lleu, lleu l'è fran eun dizastre.\direct{A Couchotta} Té t'arie fran fata de eunna périodda avouì Gallo\ldots ah se te driche.

\Tuenospeaks N'i micllà le carte\ldots Richard coppa\ldots

\StageDir{Teuppe \lemieBa .}

\StageDir{La moutra marque dji-z-aoue di mateun.}

\StageDir{Lemie \lemieSi .}

\StageDir{Entre Gene.}

\Genespeaks\direct{A Sandrino} N'a caqueun ara pe le-z-ouficho?

\Sandrinospeaks Vo dio to de chouite.

\StageDir{Sandrino queurie Touéno. Se sen lo son di \hyperref[tel]{téléfonne}\footnote{ Son a padze \pageref{tel}.}.}

\Tuenospeaks Tcheu le cou! Mi senque vouillon?

\StageDir{Touéno rep\'on i téléfonne.}

\Tuenospeaks Ouè\ldots

\Sandrinospeaks Acoutta n'a séilla eun pe eunna prateuca.

\Tuenospeaks Prateuca? Na, na, n'en pa lo ten.

\Sandrinospeaks Va bièn, mersì.

\StageDir{Beutton ba lo téléfonne.}

\Sandrinospeaks\direct{A Gene} Mesieu, l'an pa lo ten. Vo fa pasé aprì.

\Genespeaks\direct{Eun chortèn eun borbotèn} Inutillo!

\scene[-- De Spritz pe Ottino]

\StageDir{Entre eun \textit{courrier} avouì eun pourtapaque pe transporté eun cart\'on de Spritz é eun paquette de papì.}

\Corrierespeaks\direct{A Sandrino} Bondzor n'i eun cart\'on pe Ottino Spritz!

\Sandrinospeaks Va bièn, eun momàn\ldots

\StageDir{Mima scène di cllo, iaou Sandrino lèi beutte eunna blita de ten pe ivrì lo cantchel.}

\Corrierespeaks Ouè boudzade-v\'o, mé n'i pa to si ten.

\StageDir{Sandrino douàn de ivriye entre eun ouficho é se dirije ver le djouyaou de belote.}

\Sandrinospeaks\direct{Ver tcheutte} Gars\'on, betade ià le carte	\ldots

\Tcheuttespeaks Pequé?

\Sandrinospeaks L'è arrevoù lo coursié.

\StageDir{Tcheutte catson le carte é se plachon douàn l'écran de l'ordinateur de Touéno eun fièn seumbàn de travaillì. Sandrino chor de l'ouficho é tserste le cllo .}

\Corrierespeaks\direct{A Sandrino} Tcheu le cou seutta counta, mi pouade-v\'o pa marquì queunta l'et?

\Sandrinospeaks Tro seumplo pai!

\Corrierespeaks Djeusto, vo fa beun fé pasé lo ten.

\Sandrinospeaks Voualà! Acapou-la!

\StageDir{Sandrino ivre lo cantchel é tourne i seun poste. Lo coursié entre deun l'ouficho.}

\Corrierespeaks Bondzor. Senque v'ouite eun tren de fiye tcheu seuilla?

\Richardspeaks Fièn eunna retsertse\ldots

\Geromine \ldots de martchà.

\Corrierespeaks Ouè di martchà di demarse!

\Richardspeaks Vo tracachàde-vo pa de seutte bague.

\Corrierespeaks Acoutade, mé n'i fata de Ottino Spritz, Spritz Ottino.

\StageDir{Tcheu moutron avouì lo dèi la tabla de Ottino. Lo coursié s'aprotse a Ottino.}

\Corrierespeaks  Ah, drime?

\Tuenospeaks Na, na\ldots l'è eun tren de pensì.

\Richardspeaks Ottino! Can t'a eunna idì eumpourtanta, di-no-là!

\StageDir{Ottino se rèche.}

\Corrierespeaks\direct{Ver tcheutte} Mi l'è pa que v'ouèide eun poste seu eunc\'o pe mé.

\Richardspeaks Na, sen djeusto.

\Corrierespeaks\direct{A Ottino} V'ouite Spritz Ottino?

\Spritzspeaks Ouè.

\Corrierespeaks N'i eun cart\'on pe vo.

\Spritzspeaks L'atégnavo, betade maque sé si la tabla.

\StageDir{Dèi si momàn Ottino l'è réchà é avèitse si pe l'er to lo ten sensa fée ren.}

\StageDir{Lo coursié ditsardze lo cart\'on, mi lèi tsi ba pe tèra.}

\Corrierespeaks\direct{Ver tcheutte} L'è to ba! Caqueun veun m'èidjì?

\Tcheuttespeaks Na n'en pa lo ten.

\Corrierespeaks Ouè n'i comprèi, quetèn pédre. \direct{A Ottino} Dimèn betade eunna croué seuilla.

\StageDir{Ottino avouì la plima firme eun documàn si l'écran di téléfonne di coursié.} 

\Richardspeaks\direct{I coursié} Mi madama! Son-tì arrevoù le paquette de papì pe stampé le documàn?

\Corrierespeaks Ouè, vegnade vo le prendre.

\StageDir{Richard se levve é pren le paquette é le plache déz\'o son écran.}

\Richardspeaks N'ayoù fran fata pe beté-le déz\'o l'\textit{écran} de mé.

\StageDir{Lo coursié s'aprotse a la chortiya de l'ouficho é Sandrino areuvve avouì le cllo.}

\Corrierespeaks\direct{Ver tcheutte}  Cheur que v'ouèide pa eun poste eunc\'o pe mé?

\Saventaspeaks Sen i complet.

\Corrierespeaks Damadzo.

\Sandrinospeaks Si cou n'i to de chouite la cllo djeusta!

\StageDir{Sandrino ivre lo cantchel.}

\Corrierespeaks A forse! \textit{Au revoir}!

\StageDir{Lo coursié chor de scène, Sandrino cllou lo cantchel é tourne i seun poste.}

\scene[-- Eun megniye]

\StageDir{Teuppe \lemieBa .}

\StageDir{Se sen eun:}

\effet{https://soundcloud.com/user-234168361/black-out}{Gramo son d'électrisitoù}

\Tuenospeaks Tourna ià la lemie!

\Richardspeaks \`Eita que mogà va finque bièn pouèi.

\Tuenospeaks Comèn finque bièn?

\Richardspeaks Pequé pouèi consemèn tchica mouèn d'énerjì deun noutra Réj\'on!

\Gerominespeaks Ouè si d'acor.

\Richardspeaks Acoutade-mé, n'i eunna dzenta man, fa alé eun devàn. Cher collègue ``foua le lemie comme eun megniye"! 

\StageDir{Le catro belottiste teurion foua eunna \textit{lampe frontale} é se lo plachon si la tita. Dimèn areuvve eun électrisièn avouì eunna lemie eun man.}

\Elettricistaspeaks\direct{A Sandrino} Salì, si seuilla pe l'électrisitoù.

\Sandrinospeaks Areuvvo! Tsertso le cllo. Fiade-mé tchica de lemie, que dza l'acappo pa avouì la lemie\ldots feguea-té i teuppe!

\StageDir{Sandrino avouì l'èidzo de l'électrisièn tsertse le cllo é ivre lo cantchel.}

\Sandrinospeaks Voualà, entrade maque, queutto iver, can chortade cllouzade.

\Elettricistaspeaks Va bièn.

\StageDir{Sandrino tourne i seun poste.}

\Elettricistaspeaks\direct{Ironique ver le belottiste} Ouè, tcheu blette de tsa a fose de traillì. Avouì tcheu le-z-ordinateur que v'ouèide é gneun que le-z-eumplèye nen consomade de électrisitoù! 

\Saventaspeaks\direct{A l'électrisièn} Ouè prèdza mouèn é trailla.

\Gerominespeaks Pren gnenca pamì lo Wi-Fi.

\Saventaspeaks Pamì de connéch\'on!

\StageDir{L’électrisièn trafeutse eun momàn a l'anglle de l'ouficho. Aprì eun per de seconde le lemie se avion \lemieSi .}

\Elettricistaspeaks Voualà, to fé.

\Richardspeaks\direct{A Touéno} Oh finalemàn! Sensa llou sarian perdì!

\Tuenospeaks\direct{A l'électrisièn} Comme té nen n'a pamì!

\StageDir{L’électrisièn, douàn de chotre, s'aprotse a la tabla.}
\StageDir{Le belottiste gavon ià le \textit{lampe frontale} é recomenchon a djouì.}

\Tuenospeaks Ad\'on euncoa doe man.

\StageDir{Couchotta djouye eunna carta.}

\Elettricistaspeaks \direct{Eun avèitsèn le carte de Geromine} Mi na t'i eunna tsasotta! Djouya lo valet d'atoù!

\Gerominespeaks Ad\'on se t'i pi bon djouya té!

\Elettricistaspeaks Na n'i pa lo ten, deyo fé lo tor di-z-atre ouficho!

\StageDir{L'électrisièn chor.}

\Saventaspeaks \ldots é la dériye a mé! N'en gagnà! 

\Richardspeaks\direct{Eun se dispéèn ver Geromine} Mi t'a si rèi é seutta dame de belote, djouya-leu! Te me djouye seutte scartin-e!

\Gerominespeaks Mi se té te me di pa sen que t'a eun man mé pouì pa savèi sen que djouì!

\Richardspeaks Mi mé pouì pa te diye sen que n'i! Té pitoù, se t'a eugn ase dénonsa-l\'o!

\StageDir{Geromine é Richard se mandon a caqué é van a s'achouaté i leur poste. Dimèn Saventa s'aprotse a Touéno.}

\Saventaspeaks Touéno fièn eunna petchouda fotografie pe rapelé si momàn!

\Tuenospeaks Na, na, na.

\Saventaspeaks  Ouè soplé, pouai la beutto si Instagram é tcheutte pouon la veure.

\Tuenospeaks Ad\'on fièn seutta baga.

\StageDir{Saventa é Touéno se fan eun \textit{selfie} é, to de chouito, Saventa lo tsardze si Instagram.}

\Saventaspeaks Ad\'on petchoudacllenda lerèidelabelote, petchoudacllenda tchicaderepoùéaprìnotraillèn.

\StageDir{La fotografie l'è proyettaye si l'écran pe caque secounda:}

\foto{https://www.instagram.com/p/BD_LvPjHECV/}{Pouteun a Saventa}

\StageDir {Saventa s'achouatte i seun poste.}

\StageDir {Silanse.}

\scene[-- G.B. di '95]

\Saventaspeaks Richard! N'i acapoù si Facebook eun articllo pe té!

\Richardspeaks Pe mé?

\Saventaspeaks Lo titre l'è:``Euncrouayablo! Poue vatse maltrattaye!"

\Gerominespeaks Na! Senque l'è capitoù?

\Saventaspeaks Lo F.S.V.V., Fédérach\'on pe la Santé di Vatse Valdoténe, l'a denonchà eun grandjì de Tsarvensoù, G.B. di '95\ldots

\Tuenospeaks Qui?

\Saventaspeaks G.B. di '95.

\Tuenospeaks Cougniso pa.

\Saventaspeaks\direct{Conteneuvve a liye} \ldots
pe ``torteua contre vatse innosante". 
Lo Prézidàn de la Fédérach\'on, devàn lo dzeudzo, l'a accuz\'o lo grandjì  eun dièn: ``Son de personne cruelle! Groppon le quie di vatse i plaf\'on, sensa gneun motif é gavon a la bitche salla pocca d'euntimitoù que l'at!".

\Richardspeaks Djeusto! A fose eun que pense eunna mietta i bitche. Pensade a seutte poue vatse; todzor itaye topaye, mi ara lèi levvon la quia é tcheutte vèyon sen que l'an déz\'o. Mi vo semble lo case?

\Gerominespeaks T'a rèiz\'on, poue vatse.

\Tuenospeaks Mi comèn poue vatse? Lèi groppon la quia pe le-z-arì! 

\Richardspeaks Mi l'è comme se té te fusse sensa ganes\'on! Saré pa normal!

\Gerominespeaks\direct{A Touéno} Te sisa té eunna vatse, te sarie contenta?

\Richardspeaks Djeusto Couchotta! Té t'i eunna de no!

\Tuenospeaks Mi l'è normal gropé le quie pe le-z-arì! L'an todzor fé-lo.

\Richardspeaks Té t'i restoù eugn ommo d'eun cou!

\Tuenospeaks D'eun cou mi san é normal!

\Richardspeaks T'i maque restoù té pouèi! Avèitsa-té a l'entor!

\Tuenospeaks Mi avèitsa-teu é avèitsa le coucouille de té! Prédzen-nèn pamì!

\Richardspeaks Ouè quetèn pédre.

\StageDir{Silanse. Tcheutte fan ren.}

\StageDir{La moutra marque dji-z-aoue.}

\Richardspeaks Mondjemé! N'en dza fé dji-z-aoue.

\Gerominespeaks Oué dji-z-aoue.

\Richardspeaks Gneunca apesì, euncroyablo.

\Gerominespeaks Can eun l'at a fé lo ten pase.

\Richardspeaks\direct{A Ottino} Eunc\'o té t'a pa apesì\ldots

\Spritzspeaks Na ren.

\Richardspeaks Mi a queunt’aoua areuvve lo Gran Chef di Plan Dameun?

\Saventaspeaks A dji é demì.

\Richardspeaks Ad\'on vou eun devàn avouì le prateuque de la F.S.C.V.

\StageDir{Silanse.}

\scene[-- L'aoua di cafì]

\Tuenospeaks Mi se béisàn eun bon cafì?

\Tcheuttespeaks Ouè!

\Tuenospeaks L'è l'aoua di cafì. Ategnade eun momàn que queurio noutro Sandrino.

\StageDir{Touéno queurie Sandrino i \hyperref[tel]{téléfonne}\footnote{ Son a padze \pageref{tel}.}.}

\StageDir{Sandrino rep\'on.}

\Tuenospeaks Sandrino n'i eunna prateuca bièn eumpourtanta, te vériye inque?

\Sandrinospeaks Areuvvo to de chouito i galoppe.

\StageDir{Sandrino avouì lo pa de la fiacca s'aprotse a l'ouficho.}

\Sandrinospeaks\direct{A Touéno} Di-mé.

\Tuenospeaks Salì Sandrino, bailla eun cou de fi ba i Bar é di-lèi de pourté si le cafì.

\Sandrinospeaks D'acor! San dza to leur?

\Tuenospeaks Ouè comme da coutima san dza totte.

\Sandrinospeaks Va bièn. 

\StageDir{Sandrino tourne i seun poste é comande le cafì.}

\Gerominespeaks Mi Tsalendre queun dzor l'è sit an?

\Tuenospeaks Saventa avèitsa si ton téléfonne\ldots

\Saventaspeaks\direct{\'Etonaye} Eun demicro!

\Tuenospeaks Demicro? Ad\'on Sen-Itcheunne l'è eun dedzoù\ldots

\Richardspeaks \ldots é lo 27 pregnèn repoù\ldots

\Gerominespeaks \ldots 28 é 29 son eun desando é eunna demendze\ldots

\Saventaspeaks \ldots 30 é 31 comme le vallet de Romma sen eun \textit{mutua}\ldots

\Spritzspeaks \ldots lo premì de l’an sen i mitcho pe no reprendre de la piorna di dérì de l'an\ldots

\Tuenospeaks \ldots lo 2 é lo 3 pregnèn repoù\ldots

\Richardspeaks \ldots lo 4 é lo 5 son desando é demendze\ldots

\Gerominespeaks \ldots lo 6 l'è la fita di rèi é sen i mitcho\ldots

\Tcheutte \ldots é voualà eun pon de tréze dzor!

\StageDir{Entre eunna serventa avouì eun cabaret plen de tasse.}

\Baristaspeaks Bondzor Sandrino, te me ivre lo cantchel?

\Sandrinospeaks Ouè.

\StageDir{Sandrino se levve é gnouye la retsertse di cllo.}

\Sandrinospeaks\direct{A la serventa} Queunta l'è sel\'on té?

\Baristaspeaks\direct{Eun moutrèn eunna cllo} Seutta sel\'on mé.

\Sandrinospeaks Na pa seutta. Ah\ldots la voualà!

\StageDir{Sandrino ivre lo cantchel é la serventa entre.}

\Baristaspeaks Bondzor a tcheutte! Eunc\'o oueu plen de travaille?

\Gerominespeaks Sen tsardjà!

\StageDir{La serventa fé lo tor de tcheutte pe leur baillì lo cafì.}

\Baristaspeaks\direct{Ver Couchoutta} Pe té cafì queur avouì de lasì de \textit{soia}.

\Baristaspeaks\direct{Ver Touéno} Mesieu Touéno cafì \textit{corretto}.

\Baristaspeaks\direct{Ver Ottino} T'i finque réchà? 

\Spritzspeaks L'an récha-me!

\Baristaspeaks Pe té eun Limoncello.

\Spritzspeaks Saye petouda!

\Baristaspeaks\direct{Ver Saventa} \ldots é pe té Saventa eun \textit{cappuccino} avouì eun dzen sourì to pe té.

\Saventaspeaks Mersì, que dzen! Fièn eunna fotografie\ldots

\StageDir{Saventa fé an foto i \textit{cappuccino}. La fotografie l'è proyéttaye si l'écran pe caque secounda.}

\foto{https://www.instagram.com/p/BD_M4GRnEFf/}{Cafì di mateun}

\Saventaspeaks\direct{Eun icrièn si lo téléphonne} petchoudacllenda cafìlomateun, petchoudacllenda lobarnovoudibièn.

\Baristaspeaks\direct{Ver Richard} Lo cafì eunc\'o pe té\ldots é pe si cou n'en fenì.

\Richardspeaks Acouta\ldots lo vèyo l'è biodégradable?

\Baristaspeaks Ouè.

\Richardspeaks Lo cafì l'è de no-z-atre?

\Baristaspeaks Ouè de Gene.

\Richardspeaks Lo seucro de canna?

\Baristaspeaks Ouè comme tcheu le mateun!

\Richardspeaks Ad\'on lo bèyo!

\Baristaspeaks\direct{Ver tcheutte} Pe paì?

\Tuenospeaks Payo mé comme de coutima!

\StageDir{Touéno baille a la serventa de bon.}

\Gerominespeaks Tro jantilo Touéno, te no fé sise plèizì tcheu le dzor, t'i fran jantillo.

\Baristaspeaks Ad\'on vo queutto i voutro travaille, boun-a dzornoù!

\Tcheuttespeaks Boun-a dzornoù!

\StageDir{Sandrino areuvve pe ivrì lo cantchel. La serventa chor de l'ouficho.}

\Baristaspeaks\direct{Ver Sandrino} Sandrino té ren comme de couteumma?

\Sandrinospeaks Na mersì. N’i la cafetchiye\ldots semble pa, mi eugn \textit{euro} pe dzor son 30 i mèis é a l’an son 365 \textit{euro}\ldots é aprì n'i pa lo ten\ldots

\StageDir{Dimèn la tsambriye l'a pa cacou-lo é l'è chortia.  Tcheutte bèyon lo cafì.}

\scene[-- Lo Gran Directeur Carlo Trèisoù]

\StageDir{La moutra marque dji é demì di mateun. Sandrino galoppe ver l'ouficho.}

\StageDir{Dèi-z-ar la moutra plan planotte avancheré tanque a midzor.}

\Sandrinospeaks Gars\'on! L'è eun tren d'arevé lo cappe!

\StageDir{Tcheutte gavon ià le tasse di cafì é tsertson de betì eugn odre le table.}

\StageDir{Areuvve lo Gran Directeur Carlo Trèisoù. S'aprotse i trebeillet.}

\StageDir{Can lo vèyon le-z-eumpléyà s'ajiton é se levvon tcheutte.}

\Treisouspeaks Gneun que fé sen que dèi fiye seu dedeun, vo baillade-tì pa lagne? V'ouite arrevoù vito lo mateun é v'ouèide fé ren! Vegnade pa peu me deue que v'ouèide panco i lo ten!

\Tuenospeaks Na, na\ldots

\Gerominespeaks Squezade.

\Spritzspeaks Lo fièn pamì.

\Richardspeaks V'ouèide rèiz\'on.

\Treisouspeaks Que sise lo dérì cou que areuvvo é ma plase l'è pa presta!

\Richardspeaks Pard\'on. Fièn ara.

\Treisouspeaks Sistémade totte!

\StageDir{Partèi la tsans\'on:}

\sound{https://soundcloud.com/user-234168361/scherzi-a-parte-sottofondo}{Gbl 386 - Stomp cherzi - Giampiero Boneschi}\label{gbl}

\StageDir{Dimèn le-z-eumpléyà tramon le table é plachon eun grou materasse déz\'o lo trebeillet. Lo Directeur se tsandze avouì eun arbeillemèn sportif.}

\StageDir{La mezeucca s'arite.}

\Treisouspeaks Bon, si prest!

\StageDir{Pe se fé coadzo dimande de bouéchì di man a tcheutte, eunc\'o i pebleuque.}

\Richardspeaks Alé Chef!

\StageDir{Trèisoù saoute lo trebeillet avouì eunna pirouette eun l'er. Can tsi desì lo materasse, tcheu lèi boueuchon for di man, entuziaste.}

\Treisouspeaks Sensa voutro soutièn lèi la ferio pa!

\Saventaspeaks Bravo, bravo! Eunna fotografie tcheut eunsemblo!

\Treisouspeaks Na, maque avouì le doe feuille.

\StageDir{Saventa se fé eunna fotografie avouì Couchotta é lo Cape.}

\StageDir{La fotografie l'è proyéttaye si l'écran pe caque secounda.}

\foto{https://www.instagram.com/p/BD_MkyMnEEn/}{Pouteun pe Carlo Trèisoù}

\Saventaspeaks\direct{Eugn icrièn si lo téléphonne} petchoudacllenda SiL'èMonCappe, petchoudacllenda Trèisoùeugnézeumplopetcheutte, petchoudacllenda decappecommetélamammanènfépamì

\Treisouspeaks \ldots é rapelade-v\'o que se vouillade fiye tsemeun seuilla dedeun, déyade vin-ì de-z-atlet comme mé é saouté lo trebeillet mioù que mé. 

\Richardspeaks V'ouèide rèiz\'on! 

\Treisouspeaks\direct{Ver Touéno, eun lèi bouéchèn la man si lo ventro} Touéno, té te trebelerè eunna mia. Mi crèi-lèi é te lèi arreverè.

\Tuenospeaks Lèi prouo! Mersì Chef!

\Treisouspeaks Ara comenchèn a fiye quetsouza.

\StageDir{Tcheutte beutton eugn odre l'ouficho: gavon ià lo matérasse é plachon amoddo le table comme l'ion douàn. Lo cape se beutte la tsemize é le pantal\'on.}

\scene[-- Course pe le dipendèn]

\Treisouspeaks\direct{Ver Sandrino} Sandrino, beutta djeusto foua lo pannel  ``Uficho cllouzì pe course di dipendèn".

\Sandrinospeaks Ouè, l’è dza preste.

\StageDir{Sandrino plache lo pannel douàn lo trebeillet.}

\StageDir{Trèisoù, eun avèitsèn avouì supériorité le dipendèn, comenche a fé eunna léch\'on si comme prodouiye de pi.}

\Treisouspeaks Mé cher dipendèn, son pa de-z-àn comoddo pe noutra poua é dzenta Réj\'on!
Mi vo, é diyo vo-z-atre, pouade, eun traillèn seuilla achouatoù, tsandjì le bague! Vrèi ou na?

\Tcheuttespeaks\direct{Eun braillèn} Ouè!

\Treisouspeaks Pequé se mé vouillo tsandjì, pequé se vo v'ouèide voya de tsandjì\ldots to lo mondo pou tsandjì!

\Tcheuttespeaks\direct{Comme douàn} Ouè!

\Treisouspeaks Ara ataquèn lo course. N'i aprestoù eunna prézentach\'on pe vo-z-esplequì le noue metode pe macsimizé la produch\'on deun le-z-ouficho pebleuque. \direct{Ver Touéno} Touéno te me prite ton ordinateur?

\StageDir{Touéno s'ajite é comenche a clloure de vidéo/fotographie iverte si l'écran.}

\Treisouspeaks\direct{Eugn avèitsèn l'écran} \ldots é totte seutte fenne?

\Tuenospeaks Na, na, ren\ldots \direct{eun baillèn de creppe si lo clavier} ESC, ESC, ESC!

\Treisouspeaks Te per pa lo vicho!

\Tuenospeaks\direct{Todzor pi ajitoù} Na, ouè, voualà lèi sen\ldots voualà l'è totte amoddo!

\StageDir{Trèisoù tsardze la prézentach\'on é i mimo ten son joueu tsi desì lo revèille.}

\Treisouspeaks Ouè mi\ldots l'è dza dji é demì!

\Richardspeaks Ouè, donque?

\Treisouspeaks\direct{A tcheutte} Senque no fièn a dji é demì tsaque deleun mateun?

\Tcheuttespeaks Djeusto!

\Treisouspeaks L’è l’aoua di Ping Pong! Aprestade la tabla!
 
\StageDir{Le-z-eumpléyà tramon le table pe créé eunna tabla de Ping Pong.}

\StageDir{Partèi la tsans\'on \hyperref[gbl]{Gbl 386 - Stomp Scherzi - Giampiero Boneschi}\footnote{ Tsans\'on a padze \pageref{gbl}.}.}

\StageDir{Teuppe \lemieBa\ a drèite di palque.}

\StageDir{Entre lo vioù.}

\Genespeaks Si tourna seuilla\ldots

\Sandrinospeaks Me diplì Mesieu, se vegnavade sinque meneutte fé say\'on tcheutte disponiblo.

\Genespeaks Senque?

\Sandrinospeaks Se arevavade sinque meneutte fé say\'on tcheutte disponiblo.

\Genespeaks Senque?

\Sandrinospeaks\direct{Eun braillèn, eunna paolla pe cou} Se arevavade sinque meneutte fé say\'on tcheutte disponiblo!

\Genespeaks\direct{Malechà} N'i comprèi, l'ie pe diye ``euh, mi l'è pa poussiblo"!

\Sandrinospeaks Ad\'on diade ``euh, mi l'è pa poussiblo"! Aprì liade lo cartel\ldots

\Genespeaks\direct{Eun lièn eunna paolla pe cou} Uficho cllouzì pe course di\ldots! Ah na, ara vou veure\ldots

\StageDir{A lambo Sandrino bloque Gene.}

\Sandrinospeaks Na, na Mesieu! Itade seuilla, alade pa aoutre, v'ouèide pa lo \textit{pass}. 

\Genespeaks Eunna dzornaye perdiya é n'a gnenca eun cantchì iver pe alì cretequé sise pouo-z-aouvrì! 

\Sandrinospeaks Fiade eun baga: tornade demàn!

\Genespeaks Na, n'i pa lo ten; vo mando caqueun que vo rèche! Inutilo! Tchetchasoù! Mé payo le-z-empoù é vo ren! Inutilo!

\StageDir{Gene chor.}

\Sandrinospeaks Va bièn\ldots é co sitte l'è ià.

\scene[--  Ping Pong]

\StageDir{Lemie \lemieSi\ si to lo palque.}

\StageDir{La moutra marque onj'aoue. Véyèn Touéno que l'a djeusto gagnà contre Richard.}

\Treisouspeaks \ldots é ara Touéno contre Couchotta. Dériye partiya, djeusto eun tsandzo: qui gagne, gagne. Rapello le riille mesieu: si que gagne déside qui euntre vo-z-atre ite inque tanque choui-z-aoue pe fé le prateuque de tcheutte\ldots dimèn que le-z-atre son ià pe se divertì.

\StageDir{Touéno é Couchotta se plachon i dou coutì de la tabla; tcheu le-z-atre (eunc\'o Sandrino) avèitson.}

\Treisouspeaks Servicho pe Touéno.

\StageDir{Touéno baille eun creppe sèque é fé lo poueun. Saoute pe l'er to contèn, le-z-atre bouéchon di man é lèi fan la fita. Couchotta l'è dispéraye.}

\Treisouspeaks Gagne Touéno!

\Treisouspeaks\direct{Ver Touéno} N’ayoù pa la féi\ldots

\Tuenospeaks Mi ouè, n'i fran fé eunna dzenta gnacaye lé si la drèite, si la coueugne\ldots

\Treisouspeaks T'i eun can\'on! Jamì vi eunna gnacaye comme la tin-a! Mi ara di-me vèi. T'a dza désidoù qui dèi resté inque pe traillì totta l'iproù?

\Tuenospeaks Ad\'on\ldots l'ommo ou la fenna que dèi resté seu pe traillì tanque lo nite l'è\ldots 

\StageDir{Eunna mia de suspanse.}

\Tuenospeaks\direct{Eun braillèn} Couchotta!

\Gerominespeaks Mi Touéno na, tcheu le cou que te gagne té te fé resté inque mé. Devendro can djouyèn i plimeun te tchapperè eun paillano!

\Treisouspeaks\direct{Ver Couchotta} Vo feré la repeuille devendro, mi ara t'i té que te reste inque a  traillì.

\StageDir{Eun remersièn Touéno, le-z-eumpléyà tramon le table i leur poste. La moutra marque midz\`o.}

\scene[--  Feun dzornoù]

\Treisouspeaks Vi que l'è belle midz\`o, mé diriyo que pouèn alé no divertì!

\StageDir{Partèi la tsans\'on:}
\sound{https://soundcloud.com/user-234168361/canzone-finale}{Samba Degli Assenteisti - Bruno Zambrini}

\StageDir{Eun aprì l'atro le-z-eumpléyà chorton de l'ouficho, mi douàn de chotre queutton si la tabla de Geromine le prateuque que derè fé.\\ A ten de mezeucca, tsaqueun chor avouì quetsouza que dimoutre sen que feré pe totta l'iproù.\\ Saventa s'arbeuille pe alì colaté é teurie foua eun per de-z-esquì; Ottino chor avouì totta l'atrésateua pe djouì a golf; Sandrino se beutte eunna maille di fiolet di Tsarvensoù é chor avouì eugn èima é eunna peura; Richard s'arbeuille pe alì eun beseclletta é chor desì eun petchoù vél\'o di mèinoù; Touéno é Trèisoù s'arbeuillon pe la tsasse, prègnon le fezì é de l'armouére teurion foua lo tseun de la tsasse: eun cré barbontcheun.}

\StageDir{Fenèi la mezeucca.}

\StageDir{Lemie base.}

\StageDir{Couchotta, dispéraye, pouze la tita si la tabla.}

\StageDir{Vouése foua campe \vouese : ``Ouè, sit an le Digourdì l'an beun tchica ézajéroù avouì leur sarcasme, l'è vrèi. Mi l'amérian quetì deun tcheu vo eunna pégna refléch\'on. Sen-n\'o cheur de itre contèn de noutro travaille? No baillèn totte noutro eungadzemèn pe bièn réusì i rézultà final? No metèn tcheu le dzor la paye? I ten que no viquisèn, a vo la reponsa.}

\StageDir{Teuppe \lemieBa .}

\ridocliou

\DeriLeRido

\RoleNoms{Vidéo é son}{Laurent Barmaverain}
\RoleNoms{Mijì}{Renzo Bollon}
\RoleNoms{Camion}{Diego Bollon}
\RoleNoms{Vouése foua campe}{Lo dzouveun-o Seunteucco de la Repebleucca de Pompioù}
    
 \end{drama}
