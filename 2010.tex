\title{FORUM VALDOT\`EN}
\author{Pièse icrita pe Le Digourdì}
\date{Téatro Giacosa de Veulla, 7 mi 2010}

\maketitle

\fotocopertina{Foto/2010/gruppo.jpg}{Francesca Lucianaz, Marco Ducly, Ilaria Linty, Paolo Cima Sander, Simone Roveyaz, Jo\"{e}l Albaney, Laurent Chuc}{Jo\"{e}lle Bollon, Giada Grivon, Pierre Savioz, Jasmine Comé}{2010}

\LinkPiese{Forum Valdotèn}{https://www.youtube.com/watch?v=VQUzYb-3Mdw&list=PLBofM-NS_eLJUln45l7VH457fGak_Bk5O&index=8}{.5}

\souvenir{Pe mé Forum Valdotèn l'è itaye la pièse pi dzenta de totte. Lo 2010 l'è itoù l'an di premì cou: \textit{vidéo} pe eunna pièse, imitach\'on d'eun personadzo fameu (dzeudzo Santi Licheri), parodì d'eun programme télévizif. L'avanspettaclle l'ayè tsaoud\'o lo pebleucco, lo ritme de la résitach\'on martsè é dérì le rid\'o totte l'è aloù amoddo. Tsaqueun de sise petchoù boc\'on l'a fonchoun-où pe son contcho é betoù eunsemblo l'an cré\'o la pièse pi dzenta di Digourdì.
}{Francesca Lucianaz}

%Per me 2010 pièce più bella per il nostro gruppo -  l'anno delle prime volte: Primi video, prima  imitazione del giudice, ritmo alto, Prima pièce su programma televisivo,tutto è andato Perfetto (anche avan spectaclle) (è un insieme riuscito), prima volta danza sul palco
\queriaouzitou{
\begin{itemize}

\item[$\bullet$] Lo vrèi dzeudzo Santi Licheri, que Jo\"{e}l Albaney imite, l'è mor eun mèis douàn de la prézentach\'on de Forum Valdotèn (4 avrì 2010). Donque, sensa lo volèi, la pièse l'è itaye eunc\'o eun ommaje a la mémoué de Santi Licheri, fameu majistrà é personadzo télévisif.

\item[$\bullet$] Orijinellamente, le doe \textit{scène} prensipale sayòn deun l'odre eunvése: douàn la counta de Tchièn Frottapanse é aprì salla di dou-z-ipaou. I dérì momàn son itaye eunvertiye, vu que lo ritme de la premie l'ie pi ate.

\item[$\bullet$] La \textit{vidéo} de l'arrestach\'on di pouo Tchièn Frottapanse l'è itaye réalizaye avouì eun seul \textit{plan-séquence}\footnote{ \textit{Dans le plan-séquence, ces plans sont filmés sans interruption, ni montage. Un plan-séquence ne dispose donc pas de coupure et se caractérise par son action continue. Le spectateur peut ainsi suivre une action du début à la fin sans qu'il y ait d'arrêts}. Source: \href{https://www.cinecreatis.net/lexique/le-plan-sequence/}{CinéCréatis}.}. Catro meneutte de \textit{vidéo} que le-z-atteur son arrev\'o a eunterpretì to deun creppe. Mi l'è pa it\'o seumplo, di momàn que Rosina, Gilda é d'atre campagnar avouì l'Ape s'aprotsoon pe dimandé (mogà finque eun braillèn): \og Senque l'è capit\'o? \fg, \og Senque v'ouite eun tren de fé?\fg ou \og Pequé n'a le carabegnì?\fg.
\end{itemize}
}

\Scenographie
\begin{itemize}
\item[$\bullet$] 2 cllende de bouque pe le-z-eumputé di tribunal;
\item[$\bullet$] 1 grousa tabla relevaye é 1 quèya avouì le raoue pe lo dzeudzo \tribunale ;
\item[$\bullet$] 1 martelette de bouque pe lo dzeudzo \martelloGiudice ;
\item[$\bullet$] 1 plimetta USB \usb ;
\item[$\bullet$] 1 bourset avouì dedeun de DVD \dvd ;
\item[$\bullet$] De matériel pe baillì lo blan: 1 itchila \scala , 1 pénnel \pennello\ é 1 sidel;
\item[$\bullet$] 1 atéstà di \textit{Guinness dei primati} é 1 groussa botèille \bigBottle ;
\item[$\bullet$] 1 fondal maròn avouì i mentèn eunna tèila iaou veun proyettaye l'icrita ``FORUM VALDOT\`EN''. Si la mima tèila ser\'an proyettaye le vidéo de la pièse.
\end{itemize}

\setlength{\lengthchar}{3.5cm}

\Character[PIERRE]{PIERRE}{Pierre}{Prézentateur de la pièse, \name{Pierre Savioz}}

\Character[LAURENT]{LAURENT}{Laurent}{Prézentateur de la pièse, \name{Laurent Chuc}}

\Character[RITA DE L’ÉILLIZE]{RITA}{Rita}{L’istorique prézentatrise de la trasmechón \textit{Forum}, \nameF{Francesca Lucianaz}}

\Character[FENNA\\ DI MAQUIADZO]{MAQUIADZO}{Maquillage}{Femalla di \textit{staff} de la trasmech\'on que s'occupe de truqué le partésipàn, \nameF{Giada Grivon}}

\Character[PASCAL]{PASCAL}{Pascal}{Assistàn de Rita arbeillà comme eun valet de veulla, \name{Marco Ducly}}

\Character[EUMPRÉZÉYE]{EUMPRÉZÉYE}{Eumprezeo}{Eumprézéye euntsardjaye de baillì lo blan i meur di tribunal, \nameF{Jasmine Comé}}

\Character[SIMON]{SIMON}{Simon}{Botcha de l’eumprézéye, arbeillà comme eun eumbianqueun, \name{Simone Roveyaz}}

\Character[DZEUDZO\\ SENLIQUEUR]{SENLIQUEUR}{DzeudzoSenliquer}{Dzeudzo de la trasmech\'on \textit{Forum}, \name{Jo\"{e}l Albaney}}

\Character[TOBIE RASCARD]{TOBIE}{Tobie}{Ommo de Bertina, eumplouayé réjonal dzouvenno que lame fé la dzenta viya, \name{Laurent Chuc}}

\Character[BERTINA\\ RESETTA]{BERTINA}{Bertina}{Fenna dzouveunna propriétéa di bazar di péì é fenna de Tobie, \nameF{Jo\"{e}lle Bollon}}

\Character[TCHIÈN\\ FROTTAPANSE]{FROTTAPANSE}{Cien}{Eumputé sensa permì de gueuda, \name{Pierre Savioz}}

\Character[FRANCO TIC]{TIC}{Tic}{Maréchal calabrotte di carabegnì, \name{Paolo Cima Sander}}

\Character[CARLO TAC]{TAC}{Tac}{Carabegnì \carabiniere vénéchàn i premi-z-arme, \name{Simone Roveyaz}}

\Character[MADAMA TISSOT]{M.me TISSOT}{Tissot}{Dzeudzo di \textit{Guinness dei primati}, \nameF{Ilaria Linty}}

\Character[]{TCHEUTTE}{tcheutte}{\quad}

\DramPer

\act[\avanSpect\ Avanspettaclle \avanSpect]

\StageDir{\hspace*{2.5em}Lemie \lemieSi.}
\StageDir{\hspace*{2.5em}Si le \textit{proscenium} se prézente Pierre, arbeillà bièn élégàn.
}

\begin{drama}

\Pierrespeaks Squezade l'è capitoù eun problème. Eun de no di Digourdì, lo meun compagn\'on que derie itre seu avouì mé, se fé pamì acappé.

\StageDir{Di fon di téatre se sen Laurent braillì.
}

\Laurentspeaks Euh, eh! Atèn eun momàn vah!

\StageDir{Laurent galoppe \galoppe\ i mentèn di pebleucco é rejouèn Pierre eun saoutèn si lo palque. L'è to blette de tsa é l'a la djacca deun la man. Eunc\'o llou l'è élégàn.
}

\Pierrespeaks Mi iaou t'ie? Te semble lo case de fé atendre tcheutte seutte dzi? Iaou t'ie?

\Laurentspeaks \direct{Avouì lo flo i queur} Te pouè pa atendre seun meneutte douàn de comenchì?

\Pierrespeaks \direct{I pebleucco} Squezade! \direct{A Laurent} Se pou savèi iaou t'ie fenì?

\Laurentspeaks Squezade tcheut! Mé sayoù aloù aoutre i Splendor! 

\Pierrespeaks I Splendor?! Mi t'i ià de tita? Te sa pa que sen seuilla?

\Laurentspeaks Ouè mi n'i deutte: vi que sit an n'atte le-z-éléch\'on comunalle, a forse ivron si Splendor! Mi ren da fé! Pouèi n'i pa vi gneun, n'i vi to frémoù, si tracacha-me é si vin-ì sé a flamma. L'è aloù eunc\'o bièn que n'i acapoù eun poste vouido seu douàn, piatro\ldots !

\Pierrespeaks Ara que te me di so\ldots eugn éffé l'a fé drolo eunc\'o a mé; pequé seun meneutte fé n'i avèitchà foua de la fin-itra é n'i vi lo plasal di machine, seuilla eun fase, vouido.

\Laurentspeaks L'an pasoù si plasalle l'ie to plen! Eh mi sit an le dzi son organizou-se eh! Sinque pe machina é caqueun l'è eunc\'o finque viì ba a pià! Pequé te sa, deun le dérì-z-an le bague son tchica tsandjaye!

\Pierrespeaks Sen cheur, pequé fa deue que dèi lo premì janvieur n'en pamì\ldots

\StageDir{Pierre pren lo portafoille \portafolie\ é teurie foua eunna carte é la moutre i pebleucco.}

\Laurentspeaks \ldots la \textit{Carte Vallée}!

\Pierrespeaks Can mimo l'è beur queryi-la pouèi, pequé pe no Valdotèn l'ie de pi de eunna seumpla carte, l'ie lo sembole de l'autonomie, l'ie eunna amia, eunna de fameuille.

\Laurentspeaks Ouè! No Valdotèn belle can alaon a la mer\ldots pitoù oubliavon lo paréo, lo achiamàn\ldots mi la \textit{Carte Vallée}?! Ah salla na eh! Tcheutte avouì la \textit{Carte Vallée} deun lo portafoille!

\Pierrespeaks \ldots é mé si preste a betì la man \man\ si lo fouà \foua\ que \direct{i pebleucco} la ouèide eunc\'o tcheut pe lo portafoille! L'è-tì pa vrèi? Lévise la man qui l'a dza caya-la ià! 

\Laurentspeaks Eh, èita! Gneun! Cayade-la pa ià eh! Pequé n'i sentì de ouése que la réj\'on vou eumpléì seutta carte de magnie que tcheut le rézidàn entron gratouitamente pe le comeun pebleucco!

\Pierrespeaks Ad\'on na cayade-là pa ià!

\Laurentspeaks \ldots é n'i eunc\'o sentì que vouillon fé comme pe l'imondicha, que te baillon eun pri eun fonch\'on de la cantitoù que te fé.

\Pierrespeaks Seutta n'ayoù panco senti-la!

\Laurentspeaks Can mimo fa diye que sensa l'è totta eugn'atra viya, no sen itoù tro acoutemoù amoddo. Aprì can mamma Réj\'on baille eunna man, no lèi prégnèn le bri!

\Pierrespeaks Mi pe sen baste veure lo dérì de l'an! De tratteur, de-z-Ape, de porter, de machine. Tcheut eun quiya douàn le distributeur\distributeur . N'ayè de fameuille que eumplisaon seutte \textit{taniche} \tanica\ da pourté i mitcho\ldots é Graziano, eun noutro amì de Riccione, \direct{a Laurent} counta vèi senque l'a deu-no!

\Laurentspeaks L'a avèitcha-me pe le joueu é la deu-me \direct{avouì eun assàn romagneul}: ``\textit{Ma quando si diceva che il Valdostano l'ultimo giorno dell'anno fa sempre il pieno\ldots ah io immaginavo tutta un'altra cosa!}''.

\Pierrespeaks \ldots é lo 31 eun tchi no senque l'et?

\Laurentspeaks Da no lo 31 l'è eunc\'o finque Patr\'on! Mi a Tsarvensoù qui l'a fé lo Patr\'on? Gneun! Tcheutte ba i distributeur eun quiya avouì seutte \textit{taniche}!

\Pierrespeaks \ldots é gnenca lo premì de l'an sen it\'o tranquilo. N'ayè eun trimadzo pe sise mitcho! De passamàn eun avouì l'atro: fa beun beuté eun caro le \textit{taniche}. Pouèi n'en eumplì le crotte, le cantin-e, no chortao finque foua di fin-itre!

\Laurentspeaks Boneur que la Réj\'on l'è vii-no eun countre avouì seutta loué di \textit{Piano Casa}, que baille la possibilitoù d'alardjì lo mitcho di $20\%$, piatro iaou le beuttavon totte seutte \textit{taniche}!

\Pierrespeaks Can mimo Laurent fa deue eunna baga: la Réj\'on de tenzentèn fé de bique!

\Laurentspeaks Mi Pierre! Eunc\'o Paolo Comé, lo noutro pi for djouyaou di Fiolet fé de bique! Feguea-té la Réj\'on!

\Pierrespeaks Ouè mi seutta l'è fran grousa! Aprì a no de Tsarvensoù no totse de protso, pequé l'è capitaye djeusto dameun lo péì!

\Laurentspeaks Dameun Tsarvensoù? Mi senque l'è capitoù?

\Pierrespeaks Mi salla counta, bièn eumpourtanta, lé dameun lo péì!

\Laurentspeaks \ldots dameun lo péì\ldots Na! Di-me pa que aprì la \textit{tangenziale} a Comboé ara vouillon fé l'\textit{autogrill} i Peseun-e?!

\Pierrespeaks Mi na, mi senque t'a comprèi?! Mé si eun tren de prédjì di tréneun \tren\ de Cogne!

\Laurentspeaks Oh mondjemé, t'a fé-me viì eun creppe!

\Pierrespeaks Can mimo pe seutta counta di tren l'an caya-no ià de sou! 

\Laurentspeaks Mi Pierre, lo discoù di tréneun l'è tchica complecoù\ldots é damadzo que pa tcheutte lo san. Mi mé te diyo que le noutre politisièn son itoù pi veusto de sen que tcheu no seu dedeun pensèn! 

\Pierrespeaks Mi counta vèi ad\'on!

\Laurentspeaks Té pensa senque l'è capitoù eun Val d'Outa deun la dériye périodda\ldots

\Pierrespeaks Bah, deun la dériye périodda\ldots Ah! Lo \textit{Bon Chauffage}! A mé l'è bièn allo-me: si arrevoù a atseté $12\ 353$ \textit{pellet}, mi que t'ou de pi!

\Laurentspeaks Mi na! Mi que le \textit{Bon Chauffage}! Salle son totte de counte foule! Seutta l'è eunna baga bièn, bièn, bièn eumpourtanta é fa die que la tchica tracachà tcheutte!

\Pierrespeaks T'i eun tren de prédjì di tramblemèn de tèra? Mondje que pouiye que n'i prèi si dzor! N'i to vi boudjì la coutse, si to aloù ba a lambo pe le-z-itchilì tanque foua de foua. Se lèi penso\ldots

\Laurentspeaks Voualà! Mé l'è sé que te vouillavo! Té t'i convencù de seutta baga. N'an prédja-nen le journal \journal , le TG réjonalle, grou titre ``\textit{Epicentro a Bionaz}''\ldots mi son totte de counte foule!

\Pierrespeaks De counte foule? Mi t'i matte?

\Laurentspeaks Totte de counte foule! 

\Pierrespeaks Mi comèn? Mi se l'è pa itoù lo tremblemàn de téra, ad\'on senque l'è itoù?

\Laurentspeaks L'ie ba eun Veulla la \textit{trivella} que fiè la borna pe la \textit{Metropolitana}!

\Pierrespeaks Mi ad\'on ara mé n'i to comprèi comèn son alaye le bague! Voualà iaou catson le tréneun de Cogne!

\Laurentspeaks Eh ouè! Lo catson lé déz\'o! Can mimo n'en prao deu-nen eunc\'o sit an!

\Pierrespeaks Ouè!

\Laurentspeaks Aprì n'en fé eunc\'o tchica tar!

\StageDir{Laurent se teurie si la mandze pe avèitchì la moutra \orologio\ que l'è oublia-se de beutì\footnote{Erreur que l'è itoù bièn apréchà pe lo pebleucco que l'a bièn riette a seutta gaffe.}!}

\Laurentspeaks Mé dériyo de comenchì a prézentì la pièse.

\Pierrespeaks Betèn-no an mia eun plase\ldots

\StageDir{Pierre é Laurent se cllouzon la djacca é se beutton a poste la tsemize.}

\Pierrespeaks Vèyo que t'i pa fran ren crèisì de l'an pasoù! Me totse eunc\'o sit an lo mou i dzegnaou?

\Laurentspeaks Na! Sit an si organiz\'o-me\ldots

\StageDir{Laurent va dérì la tèila. Aprì eun per de sec\'onde torne avouì eun sgabel de bouque. Lo pouze protso a Pierre é lèi mounte desì. Ara l'è pi ate de Pierre é dèi donque pléì le dzegnaou pe itre a la mima atchaou de Pierre.}

\Pierrespeaks \textit{Mesdames et Messieurs}\ldots

\Laurentspeaks \ldots pièse icrita pe le Groupe Téatral le Digourdì de Tsarvensoù\ldots

\Pierrespeaks \ldots a vo Forum Valdotèn!

\StageDir{Teuppe \lemieBa .}


\act[Acte I]

\ridoiver

\scene[-- Aprestèn l'éillize]

\StageDir{La \textit{scénographie} l’è la copie di programme télévisif Forum: dou ban pe le-z-eumputé é eun ban pe lo dzeudzo. Eun \textit{scène} n’a Rita De l’\'Eillize que se boudze d’eun coutì a l’atro avouì eunna femalla di maquiadzo que lèi galoppe aprì pe la beutté eunna mia eun plase. Son panco eun dirette é donque le lemie son base. Pascal, l'assistàn de Rita, l'è eun tren d'apresté la plase di dzeudzo.}

\Ritaspeaks \direct{I caméramèn foua \textit{scène}} Apresta-mé la sinque\ldots Vouè salla va bièn! \direct{Ver la fenna di maquiadzo} Beutta-mé pa troppe de coleu \trucco\ i vezadzo que me fé viille da matte. Vouè pouèi va mioù! É Pascal iaou l’è ? L’è todzor eun treudda si ommo lé, te lo queurie é repón pa, te lo tsertse é l’è pa, can t’a fata l’è ià\ldots

\Pascalspeaks Rita si seuilla! Sayò djeusto eun tren de beutté eun plase le noutre bague pe la trasmichón. Semble pa avì mi se te te rèche pa a satt'aoue di mateun te fé pocca ou ren.

\Ritaspeaks É pensa a mé! Que dèyo me féye totta dzenta pe lo meun poubleuque.

\Pascalspeaks T’a de gran fastide té!

\Ritaspeaks Pensa que l’è dza la 2647 pountaye de Forum Valdotèn é si todzor ajitaye comme se fise la premie!

\Pascalspeaks Mé na. L’è eun an é chouì mèis que traillo seuilla é diyo trèi paolle pe dzor\ldots é chourtoù todzor le mime!

\Ritaspeaks Oh, acouta! De carabegnì comme té nen n'a plèin-a la Val d’Outa! Baillo eun caouse a eun bouèis\'on é nen chorton eun mouì! Euh que tracasse!

\Pascalspeaks Na, na: le nite le fio pa, de frette nen prègno pa\ldots na na Rita resto séilla!

\Ritaspeaks Mi vouè! \direct{Totta dousa} Resta seuilla avouì Rita, que sen belle afféchouou-no a té!

\StageDir{Entroun, eun fièn bièn de cazeun, dou-z-eumbianqueun, l’eumprézéye é lo botcha Simon. Pascal tsertse de le-z-aritì.}

\Pascalspeaks Senque fiade séilla? Pouade pa entrì!

\Eumprezeospeaks M’enteresse pa! Trama-té!

\Ritaspeaks Senque capite? 

\StageDir{Rita s'aprotse a l'eumprézéye é a Pascal. La fenna di maquillage lèi reste todzor protso. Dimèn Simon mounte l'itchila si la gotse di palque, ver lo fon.}

\Pascalspeaks Si pa te die.

\Ritaspeaks \direct{Inervaye \bienMalechaie} Euh madama no sen eun télévijón seuilla, n'en pa de ten a pédre, soplé!

\Eumprezeospeaks Restade maque tranquilla, ajitade-v\'o pa, l’è totte icrì neue si blan!

\StageDir{L'eumprézéye baille eun papì a Rita.}

\Ritaspeaks Vouì veure!

\Eumprezeospeaks  Vèyade si papì? L’è lo contrat avouì l’Endemol que m’autorize a baillì lo blan oueu devendro satte de mi!

\Ritaspeaks Mi pouade pa fé pi tar ou eugn atro dzor?

\Eumprezeospeaks Na pequé dèi demàn ivrèn lo cantchì si a Senta-Colomba é lé nen n’en pe pouza, rèizón eun pi pe fenì vitto oueu!

\Ritaspeaks Mi lo prée l'a totta seutta prisa! 

\Pascalspeaks Pouriye fé comenchì a la tsantì vi que son an blita.

\Eumprezeospeaks Na, Na, Na! No sen proféchonnel é surtoù n’i pa de ten a pédre, donque comenchèn, alé Simon!

\StageDir{L'eumprézéye comenche a dirijì lo travaille é Simon ataque a baillì lo blan. Le dou restéràn pe to lo ten di prosé pe eunna coueugne a gotse di palque, de fas\'on que se véisan pa tan.}

\Ritaspeaks Madama! Euh madama! Prèdzo avouì vo madama\ldots

\StageDir{L'eumprézéye se vionde eun baoudelièn.}

\Ritaspeaks Lé a l'anglle, vouì pa vo veure é pa vo sentì don! Sen eun télévij\'on seuilla!

\Pascalspeaks Rita, tracacha-té pa le avèitso mé!

\Ritaspeaks \direct{Ironique} Ah sit cou si tranquila!

 \StageDir{\Fv{Rita t’i presta? L’è djeusto fenì \textit{Mattino Cinque} é sampouza totse a no}}

\Ritaspeaks Vouè, vouè! \direct{A Pascal} Ià, ià! T’i eunc\'o seuilla, l’è panco ton momàn, pousa, totse a mé, ià, ià!

\StageDir{Pascal chor de la \textit{scène}. La fenna di maquiadzo porte a Rita le botte é lèi arendze le potte \labbra . Aprì chor eunc\'o lleu de \textit{scène}.}

\StageDir{\Fv{Attenchón, diretta de sé a ouette, satte, chouì, sinque, catro, trèi\ldots }}

\sound{https://www.youtube.com/watch?v=pzCaXpeXSgo}{Forum - Sigla}

\StageDir{Dimèn pase la fenna di maquiadzo avouì eun grou cartel avouì icrì:``BOUÉCHÌ DI MAN''. Fenì lo refrèn, la cllert\'o di lemie veun aoumentaye.}

\scene[-- Gagnì pe la viya]

\Ritaspeaks Bondzó a tcheut é bienvin-ì a eunna noua émichón de “Forum Valdotèn”. Si fran contenta perqué ieue n’en totchà eunna poueunte di 42$\%$ de \textit{share}! Mersì bièn a tcheut vo que no avèitsade chouèn. Bon adón comenchério to de chouitte avouì lo premì cas de seutta dzournoù. Veun Pascal!

\StageDir{Pascal entre.}

\Ritaspeaks Que dzen te veure Pascal!

\Pascalspeaks Mersì Rita! Douàn de comenchì, vouriyo djeusto saliì euncó mamma que l’a prèi eunna bronchite é l’è platta a la coutse.

\Ritaspeaks \direct{Jèinaye \imbarazzatoo} Vouè Pascal, la salièn tcheut euncó no é spéèn que tournise totte eun plase! Conteugna pe plèizì\ldots

\Pascalspeaks Ouè, pe lo premì case de oueu n'en mesieu Tobie Rascard que porte la fenna, madama Bertina Résetta.

\StageDir{Entroun tsaqueun di seun djet, se plachon dérì lo leur ban é salliyon lo poubleuque.}

\Ritaspeaks Bièn, adón vi que son tcheut i leur poste, fiyèn entré lo dzeudzo Senliqueur!

\StageDir{Rita chor de la \textit{scène}. Pascal va dérì lo ban di dzeudzo, soue lo campaneun \campanel\ é aprì doe seconde lo dzeudzo entre. La fenna di maquiadzo pase douàn lo pebleucco avouì eun cartel avouì icrì:``TCHEU DRETTE''.}

\Pascalspeaks \direct{I pebleucco} Tcheu drette! 

\DzeudzoSenliquerspeaks \direct{Dimèn que Pascal lo èidze a s'achouatté} Achouattade-v\'o maque, mersì Pascal. Adón n'en oueu lo cas de Tobie Rascard \direct{avèitse Tobie} diyo bièn?

\Tobiespeaks Vouè, vouè mesieu lo dzeudzo.

\DzeudzoSenliquerspeaks Vouè é Bertina Resetta djeusto?

\Bertinaspeaks Vouè mesieu lo dzeudzo.

\DzeudzoSenliquerspeaks Adón a vo la paola mesieu Rascard.

\Tobiespeaks Bondzó mesieu lo dzeudzo, oueu si seuilla pe eunna bagga bièn seumpla.

\DzeudzoSenliquerspeaks Ouè, se l'ie tan seumpla vegnavade pa seuilla!

\Tobiespeaks La baga l'è seumpla, mi avouì eunna fenna pai l'è pa seumplo ren se betì d'acor. Lo mèise pas\'o, feniya eunna pezanta demì dzornoù de traille, si passó bèye eugn apéritif avouì de-z-amì é n’i vi que a l’anglle de la cantin-a n’ayè eunna bour\'o de dzi. Queuriaou de seutta bagga si approtcha-me é n’i vi que djouyaon tcheutte a si nouo djouà que va tan de modda ara. L’é si djouà iaou te fa eundovin-ì dji numéró desì 20 pe arrevé a gagnì 4000 \textit{euro} i mèise pe 20 an sensa fata de féye ren, caze comme eun poleteucco de \textit{Place Deffeyes}.

\DzeudzoSenliquerspeaks Diyèn pa de counte foule va souplé; can mimo n’i comprèi de queun djouà prèdzade: lo Lot \lotto.

\Tobiespeaks Na mesieu lo dzeudzo\ldots

\DzeudzoSenliquerspeaks Vouè squezade-mé si troumpo-me vouillao diye lo \textit{Gratta e Vinci}.

\Tobiespeaks Na na sise son de djouà passó, lèi djouye pamì gneun. Sitta se queurie \textit{Pin for Five}.

\Bertinaspeaks Mi resta quèi Tobie que l’è dza dzen que mamma Réjón te paye lo \textit{bilinguisme}.  Mesieu lo dzeudzo, se queurie \textit{Win For Life}, gagnì pe la viya.

\DzeudzoSenliquerspeaks Vouè, vouè, ara n’i to cllèe! Contegnade maque avouì voutra conta mesieu Tobie.

\Tobiespeaks Bièn, tornèn i discoù: eun cou vi totte seutte dzi, n’i voulì proué eunc\'o mé a djouì  dji numérò. Malerezemàn, me nen vegnòn eun devàn maque satte, pouèi vi que la fenna l'a eunc\'o praou de chanse, tcheu le-z-àn gagne quèichouza a la lottìi di Sci Club é i tombol\'on de Tsalendre n’i téléfoun-ou-lèi \chiamare\ pe me fé èidjì. É sade comèn l’è allaye a fenì la conta?

\DzeudzoSenliquerspeaks Na diyade-mé\ldots

\Tobiespeaks Soun chortì tcheut é dji le numérò é pouèi n’en gagnà.

\DzeudzoSenliquerspeaks Istooo!

\Bertinaspeaks Vouè é dèi si dzor la noutra viya l’é vin-a eungn enfeue.

\DzeudzoSenliquerspeaks Senque vouillade diye?

\Bertinaspeaks Deun lo noutro mitcho l’è totta an desquechón; se pou pamì vivre: d'an premì n’ayòn prèdjà de eumplèyì le sou pe alardjì é valorizé lo petchoù bazaa i mentèn di péi que \ldots

\Tobiespeaks Mi pren-té varda! Feguea-té se n’i deu so, caya ba salla boteucca que te fé maque pédre de ten.

\Bertinaspeaks Mesieu lo dzeudzo, prèdze pouèi perqué vouriye spendre tcheu le sou pe fé la dzenta viya.

\Tobiespeaks Lamondjeu! Mi qui l'a fé-me-l\'o fé de te téléfoun-ì si dzor lé!

\Bertinaspeaks T'ise pa téléfoun-ou-me n’ayòn gneunca lo tracasse d’itre seuilla oueu.

\DzeudzoSenliquerspeaks \direct{Boueuche avouì lo martelette de bouque \martelloGiudice} Todzèn, todzèn, prèdze eun pe cou seuilla dedeun!

\Tobiespeaks Adón d’an premì n’i pensó de fée mèitchà di sou avouì la fenna, aprì l’è vi-me eun devàn que satte numér\'o si dji n’i cherdi-le mé! Donque me totsériye lo 70$\%$ di gagnadzo!

\Bertinaspeaks Mon cher, l’è mersì i trèi numérò que n’i bailla-te mé que n’en gagnà! Lo dzor, lo mèise é l’an di noutro mariadzo.

\Tobiespeaks Di pa de counte foule BERTINAAA! Lo 13, que l’è lo dzor que s’en maria-no n’ayò dza djouya-lo mé!

\Bertinaspeaks Sen maria-no lo 12 Touéno! \direct{Eun plaouen \piangere} Vèyade mesieu lo dzeudzo, comèn féyo a resté eunc\'o avouì eungn ommo pouèi!

\DzeudzoSenliquerspeaks V'ouèi pa tcheu le tor\ldots

\Bertinaspeaks L’è eumpousiblo se beutté d’accor avouì eugn ommo avouì le secotse créaye!

\Tobiespeaks Adón conteugna a te le fé pequé di banque.

\DzeudzoSenliquerspeaks \direct{Boueuche lo martelette \martelloGiudice} Silanse, tchica de sèriét\'o, mi perqué fiade pa \textit{fifty-fifty}, mèitchà pe eun?

\Tobiespeaks Mi commèn mèitchà pe eun? Fisse pe salla accrapiya n’ario gneunca djouyà ``la schedina'' é aprì perqué divijì eun partie igale di momàn que mé n'i djouyà satte numer\'o countre le trèi de lleu!

\Bertinaspeaks Ah vouè, l’è pouèi que se rèizoun-e? Ad\'on spleucca-mé perqué lo gagnadzo di bazar fa lo divijì can si maque mé a lèi travaillì dji-z-aoue pe dzor?!

\Tobiespeaks Mi planta-là lé!

\DzeudzoSenliquerspeaks \direct{Boueuche lo martelette \martelloGiudice} Squezade, l’è pa clléa eunna bagga, v'ouite eun comugnón di bièn?

\Tobiespeaks Malerezemàn vouè mesieu.

\Bertinaspeaks Ouè dzeudzo, me si mariaye bièn dzoueun-a é\ldots \direct{eun braillèn \bienMalechaie} bièn foula!

\Tobiespeaks Ah pe sen t’a pa fé de grou tsandzemèn!

\Bertinaspeaks Mi senque me fa sentì! Saloppe, permè-té pa!

\DzeudzoSenliquerspeaks \direct{Boueuche lo martelette \martelloGiudice} Ah deun si cas la loué prèdze cllèe. Se arrevade pa a vo beutté d’accor, fa divijì eun partiye igale: 50$\%$ é 50$\%$! Si articllo, l’articllo 111 di Code \textit{civil}, l’è pa valido deun lo cas de \direct{fé le corne \corna}, ``adulterio, incapacità di intendere é volere, non adempiere ai propri doveri di coniuge''\ldots

\Bertinaspeaks Ah vouè? Adón ara terièn to foua, mesieu lo dzeudzo! Me euntérèche pa pi de fé eunna beurta fegueua.

\Tobiespeaks Teuria maque to foua, n’i pa ren a catchì!

\Bertinaspeaks Ah ouè?

\Tobiespeaks Ouè!

\Bertinaspeaks Ouè?

\Tobiespeaks Ouè!

\Bertinaspeaks Ad\'on te te rappelle la tin-a dériye chortia pe traille a Romma?

\Tobiespeaks Vouè, lo \textit{Convegno} di fondachón \textit{anti-sismiche}, qui se la obliye, l’è itaye bièn stoufianta, é pe fenì n’i eunc\'o tchap\'o eugn’eunseurtaye di dirijàn perqué n’i perdì la machina di fotografiye \macchinaFoto\ de l’uficho!

\Bertinaspeaks Na mon chèe, t’a pa perdi-la, l'iye dézò lo sédil de la machina!

\Tobiespeaks \direct{\'Etoun-où é tracachà \ajitou} Dézò lo sédil de la machina ? É té te me lo di ara ?

\Bertinaspeaks Mesieu lo dzeudzo, vouriyo fé veure seutte fotografiye\ldots aprì véyèn se l’articllo 111 di code l’è eunc\'o valido.

\DzeudzoSenliquerspeaks Pascal sitta l’è lo teun traille.

\Tobiespeaks Na, na, na, na, na eh!  Baillo pa lo permì pe veure le foto! Son de bague eunterne di travaille, que l’an pa ren a veure avouì lo prosé.

\DzeudzoSenliquerspeaks Se voulièn rezoudre la situach\'on fa vère le foto, piatro\ldots

\StageDir{Dimèn Pascal pren la plimetta USB de Bertina é la porte eun réjie.}

\StageDir{Veugnon proyettaye eun per de foto\footnote{Malerezemàn le fotografie son pe l'archive priv\'o di dzeudzo Senliqueur! Donque la seula magniye pe le vère l'è eugn avèitchèn la vidéo de la pièse dèi la meneutta 25$:$33.} de Tobie avouì de femalle tchica peillotte \fennaTsada . Pa ren de pezàn mi djeusto de llou eumbrachà a de fenne pa tan toppaye.}

\DzeudzoSenliquerspeaks \direct{A Tobie} Complemàn, pa mal!

\Simonspeaks Que boun-e!

\Eumprezeospeaks Mi que te fé? Botchassa, trailla!

\DzeudzoSenliquerspeaks \direct{A Tobie} Mon cher, v'ouite mal plachà. Bon can mimo seutte foto le teugno pi mé perqué dèyo le beutté a verbal, eh Pascal?  \ok

\Pascalspeaks \direct{Eun catsèn to ià} Vouè, vouè to vrèi!

\DzeudzoSenliquerspeaks Me retério pe baillì eunna soluchón i voutro cas.

\sound{https://www.youtube.com/watch?v=pzCaXpeXSgo}{Forum - Sigla}[false]

\StageDir{Avouì lo jénérique de Forum, lo dzeudzo chor.}

\scene[-- La sentense de Senliqueur]

\Ritaspeaks \direct{De foua \textit{scène}} Ara la poublisitó!

\StageDir{La cllertoù di lemie bèiche.}

\Tobiespeaks \direct{Eun tsemièn ver Bertina to dispé\'o} Ah Bertina! Mi t'i foula? Te me porte seu eun télévij\'on é te fé veure seutte bague douàn tcheutte!

\Bertinaspeaks \direct{Malechaye \malecha} Ah te crèyo! T’a praou fé-me-nèn vère va! 

\Tobiespeaks Pe eunna petchouda fita \fitaa\ que l’è chortiya si dzor! É aprì se vèi pa ren d’atro si salle imadze. N’i maque fé caque foto avouì seutte femalle tchica plèizente, voualà lo totte!

\Pascalspeaks \direct{Eun tsertsèn de divijì le dou} Plan plan!

\Bertinaspeaks Pe fourteun-a que n’a pa eunc\'o lo restàn perqué piatro fayè féye Forum eun ``fascia protetta'' aprì 11 aoue di nite!

\Eumprezeospeaks N’i dza vi djeusto entraye de sen que l’è fé si seuilla. Se cougnisón dza di prédjì sise!

\Tobiespeaks Mi senque nen vouillade savèi vo? Contegnade a baillì lo blan va! Eumbianqueun de la \textit{mutua}!

\Simonspeaks Deh mamma réjón! Calma!

\Ritaspeaks \direct{A Tobie} Deh vo restade tchica pi tranquillo! \direct{Ver l'eumbianqueun} É aprì vo-z-atre ba lé baillade lo blan é eunmélade-vó de vo bague!

\Simonspeaks Rita l’a rèizón. Can eun sa pa, l’é mioù que restise quèi.

\Eumprezeospeaks Mi a té caqueun la dimando-te quèitsouza? Na, dimando?!

\Simonspeaks Na, na! Acouta, seu vou tanque a la reugga?

\Eumprezeospeaks Vouè, pa pi si eh! É pren-te varda de fé de dan!

\Simonspeaks Vouè, vouè, féyèn todzor \textit{pennellato} ou comencho lo \textit{spazzolato}?

\Eumprezeospeaks \textit{Pennellato} va bièn!

\Ritaspeaks \direct{Pitoù inervaye \stouffie\ ver le dou} Pe plèizì!

\Tobiespeaks Can mimo Rita, vouillavo djeusto présizé que mé n’i todzor travaillà pe la fameuille é eun Réjón s’i todzor fé-me eun toumeun pouèi \direct{Eun moutrèn avouì le man é lèi veun tchica lo plaouo \triste} Pe senque? Pe ren, pe eunna Bertina que vou me rouvin-ì!

\Eumprezeospeaks Mi planta-là lé va deun toumeun!

\Bertinaspeaks Vouè l’a fran rèizón!

\Tobiespeaks  \direct{Tsandze de ton,  tchica pi vendicatif} Ah mi se gagno mé tcheut le sou di \textit{Win For Life}, te me vèi pamì cheu: me n'en vou i Maldive!

\Simonspeaks Eunc\`o mé, eunc\'o mé!

\Eumprezeospeaks Té pensa maque a travaillì vah! Que te fa n'en féye eunc\'o de mètre caró pe atseuté eun beillette pe le Maldive!

\Bertinaspeaks \direct{A l’ommo} Vouè, te nen accape pi eunna aoutre per lé que lave, steurie é fé to lo reste i mitcho comme fiao mé. Critin-a que s’i!

\Tobiespeaks Ah tracassa-té pa! Que nen trouo beun eunna dzenta é dzoueun-a.

\Simonspeaks Comèn salle di foto?

\Eumprezeospeaks Se te la plante pa te féyo pi travaillì eunc\'o la demendze mateun!

\Simonspeaks Na, na, resto quèi!

\StageDir{\Fv{Dirette de sé a ouette, satte, chouì, sinque, catro, trèi, dou\ldots } Tcheut dimèn prègnon leur plase. Pascal soun-e lo campaneun \campanel .}

\Pascalspeaks Tcheu drette!

\StageDir{Lo dzeudzo entre é s'achouatte eun saoutèn desì la caèya. Pascal lo plache amoddo douàn lo ban.}

\DzeudzoSenliquerspeaks Adón, sentì la diquiarachón de M. Rascard é de Mma Resetta vo liyo la sentensa: vi le proue que l’an vi tcheutte, i sense de l’art. n. 435 ter comma 3 di Code \textit{civil}, lo mesieu seuilla prézèn l’a pa reuspétó le chouivèn articllo:
\begin{itemize}
\item[$\bullet$] art. n. 235 di Code \textit{civil} que prèdze di drouet de l’ommo ver la fenna;
\item[$\bullet$] art. n. 1352 di Code pénal \textit{ubriachezza molesta} \pionn;
\item[$\bullet$] eun mouì d’atre eunfrachón que n’i désidó de pa tin-ì countcho pe pa rendre la voutra situachón eunc\'o pi grave.
\end{itemize}
Eun considérachón de sen que n’i deu tanque ara n'i desidoù que lo gagnadzo dèi pi alé divijà d’eun seutta magniye:
\begin{itemize}
\item[$\bullet$] la tseufra de \textit{euro} 1000 a tcheut dou pe eumpléyì pe lo bazar;
\item[$\bullet$] \textit{euro} 2500 a Bertina pe sé ézijanse;
\item[$\bullet$] \textit{euro} 500 a Tobie pe sé voye.
\end{itemize}
eun pi pe vo mesieu:
\begin{itemize}
\item[$\bullet$]  la tseufra de \textit{euro} 5000 a baillì a l’Amministrachón réjonalla pe lo dan d’immadze que v'ouèi proquero-lèi.
\end{itemize}
Pouèi n'i desidoù, orevouà a tcheutte!

\tcheuttespeaks Orevouà mesieu lo dzeudzo.

\sound{https://www.youtube.com/watch?v=pzCaXpeXSgo}{Forum - Sigla}[false]

\StageDir{Avouì lo jénérique de Forum, lo dzeudzo chor.}

\scene[-- Réclamme tsarvensolentse]

 \StageDir{Pascal, Rita, Tobie é Bertina se pourton i mentèn di palque: Pascal é Rita i mentèn, Rita a gotse é Tobie a drèite.}

\Eumprezeospeaks Brao Merlo! A la plase de resté quèi i micho!  Bièn fé!

\Simonspeaks \direct{Eun dimontèn l'itchila \scala} Se te tsachon pi de la Réjón te queutto pi la mia plase.

\StageDir{Le dou eumbianqueun chorton.}

\Ritaspeaks \direct{A tobie} Mesieu vouillade diye quèichouza?

\Tobiespeaks Ah Rita\ldots \direct{levve le-z-ipale} senque vouillade que vo diiso? Pi mal, pouchè cheur pa alé!

\Bertinaspeaks Si cou te beutte ba la tita a fose!

\Ritaspeaks Pe plèizì madama, lo voutro ommo me semble dza prou démoralizó pe la sitouachón.

\Tobiespeaks Pensa que fegueua deleun eun Réjón avouì tcheut. Na, na, n’i fran lo moral dézò le botte! Féyo pi fran comme n’i deu; féyo la valiza \valigia\ é vou a tchertchì lo tsaa.

\Bertinaspeaks Ah, ouè, ouè, va mai a tchertchì lo tsaa! Tobie, mé magaa n’i ézajéró a dédiì tan de ten i bazaa, mi n’i jamì fé-te manqué la tsaleue de la fameuille, é té? Té l’è pouèi que t’a remersia-me? T’a tchertchà la tsaleue d’eungn atro djet!

\Ritaspeaks Me diplé bièn pe comèn l’è alló a fenì lo voutro cas è spéo que magaa vo tournise allé an mia pi d’accoo.

\Tobiespeaks N’i fran pa la fèi de tournì allé d’acoo avouì seutta tsapletta! Rapélla-té bièn Bertina que mamma de mé l’a todzoo deu-me que le pateun se lavoun i mitcho! Sbailla-té pa pi bon!

\Bertinaspeaks Ita mai tranquilo que n’i praou lavo-nen de patteun i mitcho deun tcheu sise-z-àn! Mi ara praou, can l’è trop l’è trop, y è an limitta a tot!

\Tobiespeaks Tsapletta!

\Ritaspeaks Bièn ad\'on mé vo salio, vo remersio bièn, salì!

\StageDir{Bertina é Tobie salion Rita, se mandon a caqué é chorton de \textit{scène}.}

\Ritaspeaks É voualà que lo premì cas de seutta émichón l'é allaye-se-nèn. \'E ara la reclamme!

\StageDir{Rita é Pascal chorton.  Vegnon proyettaye doe poublisitó:
\begin{enumerate}
\item  lo cafì \caffe\ d'Ampaillan, que se te lo bèi te fé eunna dzenta man (i fiolet), é la Betsì de Ferré, que te fé pa biqué;
\item la trifolla \trifolla\ de Feleunna que can te la queurie nen areuvve eunna.
\end{enumerate}
}

\StartVideo{https://youtu.be/cD-6tZxgx4s, https://youtu.be/9jccI7Xurl0}{Lo caf\`i d'Ampaillan é la bets\`i de Ferré, La trifolla de Feleunna}

\scene[-- Quatro é seuncanta!]

\StageDir{\Fv{Rita t'i presta? Tornèn comenchì''. Pascal é Tobie entron eun \textit{scène}}}
 
\Ritaspeaks Mondje, vouè, vouè!

\StageDir{\Fv{Presta pe la diretta, 6-5-4-3-2\ldots }}

\Ritaspeaks Bièn tornoù a tcheut. No contegnèn la trasmichón avouì eun nouo cas. Pascal prézenta-no le-z-eumputé.

\Pascalspeaks Vouè Rita. Adón n'en mesieu Tchièn Frottapanse contre dou carabegnì \carabiniere\ di veulladzo: mesieu Franco Tic é Carlo Tac.

\StageDir{Le-z-emputé entron é se plachon douàn lo leur ban.}

\Ritaspeaks Bièn comenchèn adón, fiyèn entré lo dzeudzo Senliqueur.

\StageDir{Pascal va dérì la caèya di dzeudzo é soun-e lo campaneun \campanel .}

\Pascalspeaks Tcheu drette!

\StageDir{Entre todzor la femalla di maquiadzo avouì lo cartel ``TCHEU DRETTE''. Entre lo dzeudzo.}

\DzeudzoSenliquerspeaks Bondzó a tcheutte, n'en seuilla pe lo secón cas de seutta dzornoù mesieu Tchièn Frottapanse que l'a portó lo Maréchal Franco Tic é lo Caporal Carlo Tac, la paolla a mesieu Tchièn.

\Cienspeaks Bondzó, mesieu lo dzeudzo, bondzó a tcheutte.

\DzeudzoSenliquerspeaks Voué n’en comprèi-lo, bondzó eunc\'o a vo, comenchade pe plèizì a conté la voutra conta.

\Cienspeaks Adón, l'iye lo 25 d’avrì; sade l'è eun dzor de fita é fran si dzor n’i fé la sin-a di tet, perqué sade si eun tren de fé lo mitcho é, l’è euncó bièn allo-me, perqué si arreuv\'o a prendre lo ``contributo'' pe beutté le labie i tet. Iaou arendzo l’è eun poste bondàn particuillì perqué l’è pe la \textit{zona A} pe lo mentèn istorique; can mimo n’i fé seutta sin-a é sade comèn l’è, criya Binno de l'eumprèiza, criya Cino di simàn, Louis di bouque é euncó la fenna Fonsine, mondjemé salla fenna fé de douse que son an boumba é\ldots

\DzeudzoSenliquerspeaks \direct{Boueuche lo martelet \martelloGiudice} Praou, praou, praou, copade la conta! N’i pa fata savèi tcheut le particuillì é souplé arrevade i beut.

\Cienspeaks Vouè, vouè, vouè, squezade-mé bièn mesieu lo dzeudzo, can mimo sayò eun tren de diye que n’en fé fita é a la diye totta n’en eunc\'o tchica ézajéró avouì lo bèye, perqué n’en comenchà avouì lo Muller de la Cave comme apéritif, aprì sen passó i Torrette, perqué avouì la tseue \carne\ t’i coudzì de bèye eun bon rodzo\ldots

\DzeudzoSenliquerspeaks Praou! Arrevade a la questchón pe plèizì!

\Cienspeaks Vouè, vouè, vouè, squezade-mé bièn mesieu lo dzeudzo. Adón eun cou fenì le dijestif é fé doo tsansón n’i prèi lo tsemeun di mitcho é malerezemàn n’i pamì pensó que la Réjión l’a beuttó a dispozichón de tcheut si dzen servicho de taxi que te porte i mitcho avouì poca sou, \textit{Allô Nuit} se criye; é n’i fé la counta foula de prendre ma machina é de me beutté a la gueudda. A eun sertèn poueun, to pe eun momàn la lemìye bleuvva, é voualà!

\DzeudzoSenliquerspeaks  Si stouffie d'acouté voutre conte foule que me spleccade ren, ara acoutèn lo carabegnì Franco Tic!

\StageDir{Franco Tic prèdzerè todzor avouì eun acsàn calabrotte.}

\Ticspeaks Mesieu lo dzeudzo, bondzò, la nite di 25 avrì 2010, mé, lo soussegnà Marechal Tic nèisì a Gioia Tauro lo 3/3/73 é lo caporal Tac seu, nèisì a\ldots nèisì a\ldots

\Tacspeaks Bassano del grappa,  lo 7/7/77.

\Ticspeaks  No no plachòn eun localitó Chef-Lieu de la quemeua de \textit{Tsarvensoddi} pe eun normalle controlle can a chouì-z-aoue é 23 di mateun no-z-appesisèn que l’è eun tren de arrevé an machina que avanchè eun magniye pa conforme i Code di tsemeun. 

\Cienspeaks Na, na, na mesieu lo dzeudzo mé guedavo tranquilo pe mon tsemeun!

\DzeudzoSenliquerspeaks \direct{Boueuche lo martelet \martelloGiudice}  Ouè, ouè, acouttade vo ouèide dza prèdjà, ouèide dza i lo voutro ten\ldots

\Ticspeaks Diavo: aprì que n’en bailla-lèi la tsasse eun bon momàn\ldots

\Cienspeaks Na, na, na mé si arretou-me to de chouite!

\DzeudzoSenliquerspeaks \direct{Boueuche lo martelet \martelloGiudice}  Pascal pensa lèi té!

\StageDir{Pascal s'aprotse a Tchièn é lo eunvite a se calmé.}

\Ticspeaks N’en arritó la machina \machina\ é, constatoù l’évidàn statte di mesieu que guedèe, n’en fé-lèi lo test alco\ldots alcoli\ldots eh lo test di pallontcheun é, selón le prescrichón de la loué, n’en reterià lo permì de gueudda.

\DzeudzoSenliquerspeaks Bièn comencho a comprendre sen que l'è accapitó, mi comprègno pa lo perqué vou ite seuilla mesieu Frottapanse!

\Cienspeaks Mesieu lo dzeudzo mé n’i bi, n’i guedó é n’i cheue pa fé amoddo, beutto pa eun doute so. Mesieu lo dzeudzo, prégnade-v\'o varda, djeusto que l'isan gavo-me totte! Mé can mimo, oueu si seuilla perqué la nite eun questchón n’i soufló dedeun salla pégna machina que l’an sise dou lé, é n’i soufló la valiya de 4,50! Si cheue de so, lèi beutto la man \man\ si lo fouà \foua.

\DzeudzoSenliquerspeaks Bièn, mi iaou vouillade arrevé?

\Cienspeaks Mon tracasse l’è que pe seutta bagga n’i pa de rési que me dimoutre la validitó de l'alcoltest perqué Tic é Tac seuilla, l’an pa voullì me baillì lo verbal.

\Ticspeaks Mi mesieu lo dzeudzo no n’en baillà-lèi-l\'o!

\Cienspeaks Vouè mi deussì si croué bocón de papì que vo criyade verbal l'é pa marcaye la valliya de véo n’i souffloù. Comprégnade mesieu lo dzeudzo?

\DzeudzoSenliquerspeaks Na pe ren. Mi diyade-mé: perqué vouillade eun resì, se n’i bièn comprèi, que dimoutrise que v'ouèi souffloù \soffiare\ a si valeue tan ate? Mé areuvvo pa a comprende! Areuvvo pa a comprende ren!

\Cienspeaks Perqué mesieu lo dzeudzo\ldots vouì entré pe lo  \textit{Guinness dei primati}!

\DzeudzoSenliquerspeaks Comèn lo \textit{Guinness}?!  L'an gavou-vo lo permì de gueudda, l'an reteria-vo la machina é vo\ldots vo vo tracachade di \textit{Guinness}?!

\Cienspeaks Mi vouè, mesieu lo dzeudzo. Mé si pa pe le bague matérielle, lo permì de guedda aprì eugn an te lo rendoun, le machine te le atseutte. Mi lo \textit{Guinness}, l'è lo \textit{Guinness}, entré pe lo \textit{Guinness} l'a pa de pri. É i dzor de oueu qui l'è que fé 4 é 50?

\DzeudzoSenliquerspeaks Eugn éffé v'ouèi pa tcheu le tor.

\Ticspeaks N'a cheur a se blagué, de n’abreuttì!

\Cienspeaks Vouè, mesieu lo carabegnì rapellade-v\'o bièn que l'è mioù itre pi\'on que marón, perqué la piorna passe\ldots

\Ticspeaks \direct{Malechà \malecha} Oh senque?! Restade tchica calmo!

\DzeudzoSenliquerspeaks \direct{Boueuche lo martelet \martelloGiudice} Todzèn avouì le paolle, piatro vo féyo chotre tcheut.

\Cienspeaks Na, na, squezade mesieu lo dzeudzo, si fé-me prendre tchica la man! Mesieu lo dzeudzo, vouillavo renque vo diye eunc\'o que mé oueu n’i portó seuilla avouì mé lo dzeudzo di \textit{Guinness}, madama Tissot. Soplé, pouèn la fé entré?

\DzeudzoSenliquerspeaks Va bièn, Pascal, fiade entré madama Tissot.

\Pascalspeaks Vouè, mesieu lo dzeudzo, to de chouite.

\StageDir{Pascal acompagne madama Tissot eun \textit{scène}. Madama Tissot prèdze eun patoué tchica amériquèn.}

\Tissotspeaks Bondzó mesieu lo dzeudzo, é a vo tcheut, vo porto le salì de l'assembloù NGRW \textit{National Record Guinnes of the World}. Si oueu seuilla eun tsardze de dzeudzo pe rendre offisiel lo \textit{record} de mesieu Tchièn Frottapanse.

\DzeudzoSenliquerspeaks \direct{Avouì acsèn amériquèn} Bièn madama, de senque v'ouèide fata, diade-mé.

\Tissotspeaks N’ario praou de eunna proua \textit{audio-video} ou icrita de salla valeue que tan vou ite eun tren de prédjì.

\Tacspeaks \textit{Video}? \textit{Ostregheta} se n’en lo \textit{video}.

\Cienspeaks T'a eun \textit{video}? Teuria foua adón!

\DzeudzoSenliquerspeaks \direct{Surprèi} A mi adón ouèide eunna vidé\'o de si dzor lé?

\Ticspeaks \direct{Fé de gramo joué \malechaa\ i Caporal} \textit{Video}? Vouè l’é la \textit{camera-car} que n’en todzó deussì noutre machine mi ara sarè ba i fon d'eun magazeun\ldots avouì tcheut le filmà que n’en eun cazerma lèi allérèn chouì mèis pe accapé fran si que tsertsade. \direct{Eun avèitsèn lo Caporal} Djeusto?

\Tacspeaks Na maréchal. Vo rappelade pa le dispozichón di comandàn? No fa todzor no porté aprì le réjistrachón di dérì mèis, eugn éffé le n’i fran seuilla. 

\StageDir{Tac ivre lo bourset é se beutte a tchertchì lo DVD \dvd.}

\Tacspeaks \ldots lo dzor di bat\'on d'or\ldots na, lo dzor de la fita de la birra \ldots na, ah voualà lo 25 d'avrì!

\DzeudzoSenliquerspeaks Bièn, bièn, Pascal prégnade la proua video é fiade-n\'o la vère.

\StageDir{Pascal pren la casetta é chor foua de \textit{scène}. Veun proyettaye eunna vidéo iaou le carabegnì aritton Frottapanse, lo maltratton é, aprì èi fé-lo soufflé, déclaron que la valia l'è de 4.50.}

\StartVideo{https://www.youtube.com/watch?v=CrRZhSjnl6k&t=110s}{Tchièn Frottapanse\ldots 4.50!}

\StageDir{Pascal tourne eun \textit{scène}.}

\DzeudzoSenliquerspeaks Ah, bièn si cou n’i praou de proue eun man, me reteurio pe désidé, Pascal la caèya pe plèizì.

\sound{https://www.youtube.com/watch?v=pzCaXpeXSgo}{Forum - Sigla}[false]

\StageDir{Pascal èidze lo dzeudzo a chotre.}

\scene[-- Atégnèn la sentense]

\StageDir{Rita se plache i mentèn de la \textit{scène}.}

\Ritaspeaks Voualà l’è arrevó lo momàn di téléfoun-aye! Dimandèn squeuza se no pouèn pa vo sentì tcheut mi sade tellamente nombreu, mi contegnade a proé, se sa jamì. Adón sentèn la premì téléfoun-aye de oueu\ldots ouè, qui n’en eun leugne?

\StageDir{\Fv{ \direct{Eun cognèn} Ouè, s’i Tchaspeun de Cogne. Bondzó a tuit, Rita t'i toujour pi dzenta, vouillavo diye a Tchièn que dèi remèasiì la fenna de mè que lou dzor de la singna de la séchòn de la rebatta eun tchi Pèatse la ramacha-me tchica veutchou piatro lo \textit{record} di cattrou é sinquenta lo pourtavo si a Dzéméillàn\ldots aou, te lo raccoumando! Salì é mersì Rita!}}

\Ritaspeaks Salièn é remersièn tcheutte Tchaspeun de Cogne pe la sin-a euntéressanta osservachón. Passèn a la seconda é deriye téléfoun-aye de seu matteun ouè, qui n’en eun leugne?

\StageDir{\Fv{\direct{Avouì acsàn de Fin-is} ``Ouè, mondje mondje n’i prèi la leugne Piero, que émosón Rita, 'on de-r-an que aprouo, mondje, mondje!}}

\Ritaspeaks Si fran contenta pe vo Madama!

\StageDir{``Ouè squezode, mé dze 'i Rina de Féic é, vuoillavo difendre Tchièn eun diyèn que le dou carabegnì a la pla'e de fouété le dzi pourion se compolté tchica miouc avouì le sitouayèn perqué son bièn malpoulic. Eun pouteun \bacino\ a Pascal!''.}

\Ritaspeaks Salì Rina, é brao Pascal, eunc\'o ba pe Fin-ise t’a de sucsé eh? Lo ten di téléfoun-aye l’è fenì pe oueu me diplé, ara no fa allé sentì senque l’a  no diye lo dzeudzo.

\Pascalspeaks Vouè Rita.

\scene[-- Eun Frottapanse a New York]

\StageDir{Pascal va dérì la caèya di dzeudzo é soun-e lo campaneun \campanel .}

\Pascalspeaks Tcheu drette!

\StageDir{Entre la femalla di maquiadzo avouì lo cartel ``TCHEU DRETTE''. Entre lo dzeudzo.}

\DzeudzoSenliquerspeaks Mersì Pascal. Adón diyo que lo cas l'è pi lèin-o de senque pensao. L'articllo n. 35 bis comma trèi di Code \textit{civil} prévèi que le proue \textit{audio} é \textit{video} van trèi cou sen que vatte eunna temouagnanse é son a tcheut le-z-éffé la vetó é la dimostrachón de senque l’è accapitó si dzor lé. Donque mesieu Frottapanse l'a tcheu le drouet pe demandé lo seun fameu \textit{Guinness dei Primati}.

\StageDir{Son tcheutte countèn eun \textit{scène}; tchica mouèn le carabegnì.}

\DzeudzoSenliquerspeaks Mi fenèi pa seuilla. Selón le-z-articllo 39-43-52 di Code Pénal, gneun-a aoutoritó pou se permettre de abuzé di pouè, le dou carabegnì Tic é Tac que, dedeun lo filmà djeusto vi, baillòn de caouse é lévòn le man countre si pouo mesieu, dèyon pe loué paì le dan pe la tseuffra de \textit{euro} 4000 a Frottapanse: 
\begin{itemize}
\item[$\bullet$] 3000 pe le dan moral que vou èi fé-lèi;
\item[$\bullet$] 1000 perqué n’i désido-lo mé, é mé si la loué.
\end{itemize}
Pouèi l'é itoù désidó. Lo cas l'è cllou.

\sound{https://www.youtube.com/watch?v=pzCaXpeXSgo}{Forum - Sigla}[false]

\StageDir{Lo dzeudzo chor. Comme douàn, tcheut se plachon i mentèn di palque.}

\Ritaspeaks Oh voualà! Que détchijón l’a prèi lo noutro dzeudzo. N'arriyo jamì deu-lo é gneunca lo poubleuque l'arriye pensó a tan.

\Cienspeaks Ah lèi reste bièn, a sise dou!

\Ticspeaks\direct{Ver Tac} Deh que boillón que n'en tchapó! Le 4000 \textit{euro} te le ren pi tcheu té eunc\'o pe mé!

\Ritaspeaks Mesieu Frottapanse si contenta pe vo perqué sise dou carabegnì se meuton maque so. De gran malédec\`o!

\Tissotspeaks Vouè si contenta eunc\'o mé pe vo, vi que v'ouèide tsertcha-me de New York; oumouèn n’i pa fé eun vouayadzo pe ren.

\Cienspeaks É ara pe lo \textit{Guinness}? Senque fa fé?

\Tissotspeaks Mesieu Frottapanse, sitte l’è eun atéstà que dimoutre la valeue di voutro \textit{record} é seutta\ldots

\StageDir{Pascal lèi pase eunna grousa botèille \bigBottle .}

\Tissotspeaks \ldots l’è lo pri que vo porto a non de l’assembló NRGW \textit{National Record Guinness of the World} pe vo euncoradjì a ameilloré eunc\'o de pi lo voutro \textit{record}!  Mogà pe arrevé a eun dzen sinque i pallontcheun! Vo attègno a New York deun lo meun ouficho pe légalizé lo tot.

\Ritaspeaks Ouè réjie fièn eunna foto!

\StageDir{Rita, Tchièn, madama Tissot se beutton eun pouza pe la foto.}

\Ritaspeaks Mesieu Frottapanse vouillade diye quèichouza a propoù de la sentanse?

\Cienspeaks Si bièn contèn perqué si pouì lo premì Frottapanse fameu deun lo mondo é seurtoù lo premì Frottapanse que vat a New York!

\Ritaspeaks Ouè mesieu é rappelade-v\'o de prédjì de noutra trasmechón can vo dimandoun pi di voutro \textit{record}.

\Cienspeaks San doute Rita. Pouì pa vo oubliì.

\Ritaspeaks  Bièn ad\'on mé vo remersio é vo salio. Complemàn! Salì!

\StageDir{Tcheut chorton sof Rita é Pascal.}

\Ritaspeaks Adón Pascal n'en fenì!

\Pascalspeaks Ouè, l’é itaye diya pe seuilla.

\Ritaspeaks Eh vouè, spérèn que bièn de dzi l’issan vi-no perqué la pountaye de oueu l’iye cheur pa da pédre. Adón salièn tcheut, a no revère demàn mateun a la mima aoura. Salì a tcheut!

\Pascalspeaks Poudzo!

\sound{https://www.youtube.com/watch?v=pzCaXpeXSgo}{Forum - Sigla}[false]

\ridocliou

\DeriLeRido

\RoleNoms{Collaborateur}{Flavio Albaney, Ester Bollon, Paola Lucianaz, Aldo Marrari, Fabrizio Pession, Livio Viano, Valentino (lo tecnisièn de Livio)}
\RoleNoms{Miijì}{Diego Bollon}
\RoleNoms{Souffleur}{Valeria Brunod}
\RoleNoms{Tramamoublo}{Jean-Pierre Albaney, Louis Bollon, Jerome Saccani}

\end{drama}
