\chapter*{Colophon} 

\shapepar{\hexagonshape}{%(vedi computer science per layout colophon)
Ce livre a été écrit avec \LaTeX (La(mport)\TeX), un logiciel gratuit de composition typographique créé par Leslie Lamport, qui utilise \TeX, une invention de Donald Ervin Knuth datant de 1982, comme moteur de composition.
La \textbf{police} utilisée est Charter, une police de caractères à empattements (serif) conçue par le renommé typographe Matthew Carter en 1987. Caractérisée par des formes claires et des contreformes ouvertes, elle est reconnue pour son excellente lisibilité, ce qui en fait un choix apprécié pour le corps de texte. Les \textbf{émoticônes} \ok\ ont été aimablement fournies par Icons8 (\href{https://icons8.com/icons/}{https://icons8.com/icons/}).
La couverture du livre, réalizé par PIPPO, représente bla bla.
}
% prendere spunto da colophon della dispensa Pantieri o computer science
\vfill 
\noindent\makebox[\textwidth][c]{%
  \busta\ \href{mailto:jbollon94@gmail.com}{jbollon94@gmail.com}
}
\noindent\makebox[\textwidth][c]{%
  \busta\ \href{mailto:digourdi.tsarvensou@gmail.com}{digourdi.tsarvensou@gmail.com}
}
