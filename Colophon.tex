\chapter*{Colophon} 

\shapepar{\squareshape}{%(vedi computer science per layout colophon)
Si livro l'è it\'o icrì avouì \LaTeX (La(mport)\TeX), eun lojisiel gratouì de compozich\'on \textit{typographique} réalizoù pe Leslie Lamport, que eumplèye \TeX, créach\'on de Donald Ervin Knuth deun lo 1982, comme moteur de compozich\'on.
FONT Nome, autore: "ArsClassica, uno stile
ispirato a Gli elementi dello stile tipografico di Robert Bringhurst"
Le-z-\textit{emoji} FACCINA son it\'o ditsardjà avouì libre lisanse pe lo site \href{https://icons8.com/icons/}{icons8.com}, eun projé a code iver GESTITO/MANTENUTO da ved. COMPUTER SCIENCE....
La COPERTINA di livro l'è itaye réalizaye pe PIPPO é COSA RAPPRESENTA?}
% prendere spunto da colophon della dispensa Pantieri o computer science
\vfill 
\noindent\makebox[\textwidth][c]{%
  \busta\ \href{mailto:jbollon94@gmail.com}{jbollon94@gmail.com}
}
\noindent\makebox[\textwidth][c]{%
  \busta\ \href{mailto:digourdi.tsarvensou@gmail.com}{digourdi.tsarvensou@gmail.com}
}
