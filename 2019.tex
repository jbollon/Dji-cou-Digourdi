\title{SÈIDEPOU(VO)ÈR}
\author{Pièse icrita pe Le Digourdì}
\date{Téatro Splendor de Veulla, 9 marse 2019}

\maketitle

\fotocopertina{Foto/2019/gruppo.jpg}{André Comé, Alessandro Garetta, Sophie Comé, Richard Cunéaz, Margot Jorrioz, Jordy Bollon, Marlène Jorrioz, Marco Ducly, Paolo Dall'Ara, Paolo Cima Sander, Christianne Albaney, Michel Comé, Jér\^{o}me Munier}{Stéphanie Albaney, Jo\"{e}l Albaney, Jo\"{e}lle Bollon, Julie Squinabol, Thierry Jorrioz, Aimé Squinabol, Eyvia Lucianaz, Fabien Lucianaz}{2019}

\LinkPiese{Sèidepou(vo)èr}{https://www.youtube.com/watch?v=FzFVa_uQL5A&t=1383s}{.5}

\souvenir{\textit{À mon avis}, SÈIDEPOU(VO)ÈR l'è la pi jénialla pièse di Digourdì a niv\'o d'orijinalité métatéatrale. N'en voulì, comme comédièn, no-z-aprotchì pe lo premì cou a la satire politique di Consèille Réjonal. L'eunterprétach\'on finale si lo palque l'è itaye lévetta é bièn efficase di momàn que lo teste l'ie \textit{autobiographique} é métatéatrale. Deun la partia \textit{autobiographique} di sujé, mé say\'o éffectivamente itaye élia, a l'éiadzo de 18 an,  Prézidanta de eunna compagnì d'atteur bièn pi gran que mé. Eun pi, di coutì métatéatral, mé say\'o pe dab\'on la premie fenna é pi dzouveunna Prézidanta di Digourdì que dèijè gagnì le-z-éléch\'on pe itre reconfermaye. Donque, itre la protagoniste de mé mima l'é it\'o doblemàn pi émouvàn! L'émoch\'on l'è eunc\'o redoblaye a la feun di spettaclle can to lo pebleuque l'a tsanto-me \textit{Joyeux anniversaire}, vi que lo dzor mimo fitoo 19 an! Ah\ldots a propoù de métatéatro: deun la pièse mon cher Paolo Dall'Ara me régale finque eun mase de fleur pe mon anniverséo! SÈIDEPOU(VO)ÈR: eun cateill\'on métatéatral que llate Digourdì é satire politique.}{Marlène Jorrioz}
\queriaouzitou{
\begin{itemize}
\item[$\bullet$] Lo titre\footnote{ Pe la grafie di BREL lo titre sarie ``SÈIDEPOU(VO)ER". Lo dérì acsàn l'è pa nésesséo, mi l'è gneunca eugn erreur.} de la pièse l'è eun djouà de paolle que souligne dou élémàn di mal gouvernemèn de la classe dirijante: la sèi de pouvoer (SÈI DE POUVOÈR) é la corruch\'on, l'itre pouer (SÈIDE POUÈR);

\item[$\bullet$] Lo final de la pièse l'è it\'o icrì é ricrì pi de dji cou\ldots é prooù pi de 100 cou! L'è itaye la \textit{scène} pi traillaye de totta l'istouére di Digourdì;

\item[$\bullet$] Lo dzor di spettaclle, la Compagnì l'è itaye prémiaye pe la seconda plase di Prix Magui Bétemps pe la pièse di 2018 ``Todzo pi Digourdì''! Eunna grousa soddisfach\'on pe le Digourdì, di momàn que sayòn jamì it\'o prémià pe si Concour.

\item[$\bullet$] Pe lo premì (é for probablo pe lo dérì cou) l'è pouyà si lo palque eun noutro cher collaborateur é \textit{auteur}. No spéèn todzor que tournise si lo palque, mi llou préfée partisipé foua di téatre, eun no baillen de jénialle-z-idoù pe réalizé de counte comique.

\item[$\bullet$] L'élémàn lo pi métatéatral de SÈIDEPOUVOÈR l'è, sensa doute, la campagne électorale que tcheu le-z-atteur l'an fé lo lon di senà devàn de la pièse. Le doe fach\'on l'an pa mol\'o de baillì de \textit{santini} ià pe la parotse, pe convencre le dzi a vin-ì a téatre pe voté lo nouo Prézidàn di Digourdì. A témouagnadzo de la pressante propagande féta pe la fach\'on de la lista numér\'o dou n'a seutta \textit{vidéo}:

\begin{figure}[H]
\centering
\video\hspace*{0.5mm} \textsc{\small Vota Jordy di Digourdì}\hspace*{0.5mm} \video\\\vspace*{2mm}
\qrcode[hyperlink, height=0.5in]{https://www.instagram.com/p/Bul3B3loADp/}
\addcontentsline{vds}{section}{Vota Jordy di Digourdì}
\end{figure}


\end{itemize}
}

\Scenographie
\begin{itemize}
\item[$\bullet$] Eun per de table \tabla\ a feur de caval avouì eun grou lenchoueu blan;
\item[$\bullet$] 12 caèye di Consèille réjonal \sedia ;
\item[$\bullet$] \textit{Statut} di Digourdì \livro ;
\item[$\bullet$] 1 campaneun \campanel ;
\item[$\bullet$]1 calandrì;
\item[$\bullet$] De-z-objé pe tsaque postach\'on di conséillì réjonal: microfonne, botèille d'éve \boteilledeve\ , de foillette, de plime\ldots ;
\item[$\bullet$] 1 per de-z-esquì\eski ;
\item[$\bullet$] De jiragn\'on sèque \frappe;
\item[$\bullet$] 1 tsaretta \tsaretta\ avouì de saouseuse \salan\ é 1 dzeleunna \gallina ;
\item[$\bullet$] 2 poltronne \poltrona ;
\item[$\bullet$] 1 tabla \tabla\ é eunna caèya d'ouficho \sedia ;
\item [$\bullet$] 1 pourta \porta ;
\item [$\bullet$] 1 coutse \lettodoppio ;
\item [$\bullet$] 1 per de menotte \manette .
\end{itemize}

\setlength{\lengthchar}{3.75cm}

\Character[AIMÉ]{AIMÉ}{Aime}{Prézentateur de la pièse, \name{Aimé Squinabol}}

\Character[JULIE]{JULIE}{Julie}{Prézentatrise de la pièse, \nameF{Julie Squinabol}}

\Character[PRÉZIDÀN]{PRÉZIDÀN}{Presidan}{La vrèya Marlène Jorrioz di Chef-Lieu, éffectivemàn Prézidàn di Digourdì deun lo 2019, \nameF{Marlène Jorrioz}}

\Character[SECRETÉO]{SECRETÉO}{Secreteo}{Ommo présì é secretéo de la Compagnì téatrala, \name{Fabien Lucianaz}}

\Character[JORDY]{JORDY}{Jordy}{Lo vrèi Jordy Bollon di Chef-Lieu, dézireu de reprendre la plase de Prézidàn, \name{Jordy Bollon}}

\Character[MARCO]{MARCO}{Marco}{Lo vrèi Marco Ducly de Feleunna, \name{Marco Ducly}}

\Character[SOPHIE]{SOPHIE}{Sophie}{La vrèya Sophie Comé di Chef-Lieu, \nameF{Sophie Comé}}

\Character[CIMA]{CIMA}{Cima}{Lo vrèi Cima Sander Paolo de Feleunna, \name{Paolo Cima Sander}}

\Character[RICHARD]{RICHARD}{Richard}{Lo vrèi Richard Cunéaz de Pompioù, \name{Richard Cunéaz}}

\Character[PAOLO DALL'ARA]{PAOLO DALL'ARA}{Dallas}{Lo vrèi Paolo Dall'Ara de la Prériye, \name{Paolo Dall'Ara}}

\Character[JO\"{E}LLE]{JO\"{E}LLE}{Joelle}{La vrèya Jo\"{e}lle Bollon di Chef-Lieu, \nameF{Jo\"{e}lle Bollon}}

\Character[EYVIA]{EYVIA}{Eyvia}{La vrèya Eyvia Lucianaz d'Ampaillan, \nameF{Eyvia Lucianaz}}

\Character[SERVENTA]{SERVENTA}{Usciere}{Eumpléyaye i servicho di Consèille réjonal \commessa ,  \nameF{Christianne Albaney}}

\Character[JEAN LUC]{JEAN LUC}{Jean}{Ommo de 25 an, fier de traillì i-z-eumplàn de Pila, \name{André Comé}}

\Character[SAVENTA]{SAVENTA}{Saventa}{Viille \viille\ de 80 an queriaouza comme eun pet é bièn atatchaye a seun jiragn\'on sèque, \nameF{Sophie Comé}}

\Character[POMPIOLÈN]{POMPIOLÈN}{Pompiolen}{Eugn ommo ansièn \viou\ de Pompioù, \name{Richard Cunéaz}}

\Character[GARDE DI COR]{GARDE DI COR}{gc}{Garde di cor de Bruno Aveuille, responsablo de l'iverteua de la Pourta, eunterprétoù pe \textsc{Alessandro Garetta} é \textsc{Jér\^{o}me Munier}}

\Character[NOTÉO\\ VIONDAPAPÌ]{VIONDAPAPÌ}{Noteo}{Ommo avouì la fiacca de traillì que fé to sen que di Bruno Aveuille, \name{Thierry Jorrioz}}

\Character[BRUNO AVEUILLE\footnote{Litérallemàn sarie Bruno Vipa, mi Bruno Aveuille \ape\ soun-e mioù.}]{BRUNO \ape}{Bruno}{Conducteur télévisif di renommoù programme \textit{Porta a Porta} de RAI 1, \name{Jo\"{e}l Albaney}}

\Character[POURTAPIZZE]{POURTAPIZZE}{Fattorino}{Gars\'on euntsardjà de pourté le \textit{pizze} \pizza i mitcho di dzi, \name{Michel Comé}}

\Character[SPETTATEUR I]{SPETTATEUR I}{SpectI}{Eun spettateur de \textit{Porta a Porta}, \name{Ronny Borbey}}

\Character[SPETTATEUR II]{SPETTATEUR II}{SpectII}{Eugn atro spettateur de \textit{Porta a Porta}, \name{Laurent Chuc}}

\Character[]{PRÉZIDÀN É JORDY}{Presidanejordy}{\hspace*{0cm}}
\Character[]{LE ROUZA}{Fucsia}{\hspace*{0cm}}
\Character[]{JO\"{E}LLE \'E PAOLO}{Joelledallas}{\hspace*{0cm}}
\Character[]{CIMA \'E MARCO}{CimaMarco}{\hspace*{0cm}}
\Character[]{TCHEUTTE}{Tcheutte}{\hspace*{0cm}}
\Character[]{AIMÉ É JULIE}{Tcheuttedou}{\hspace*{0cm}}

\DramPer

\act[Avanspettaclle]

\StageDir{\hspace*{2.5em}Lemie \lemieSi.}

\StageDir{\hspace*{2.5em}Aimé é Julie son i mentèn di \textit{proscenium}.}

\begin{drama}

\Aimespeaks \textit{Mesdames} é \textit{Messieurs} bonsouar! Sit an le Digourdì l’an dimand\'o a no de fiye eunna pégna euntroduch\'on.

\Juliespeaks Ouè, pequé comme sèide beun, dedeun la Compagnì n’a i eunna groussa disquech\'on, que l'a divijà le Digourdì eun doe fach\'on!

\Aimespeaks Ad\'on ara totse a no fiye eugn avanspettaclle pi nèutral.  Pequé oueu l’è lo dzor di-z-éléch\'on, iaou fa chèdre lo nouo Prézidàn di Digourdì é fa vardé lo silanse électoral!

\Juliespeaks Ouè, le-z-électeur dèyon pa itre eunfluenchà pe le doe fach\'on! 

\StageDir{Marlène, a cats\'on, queurie Aimé é lèi beutte eunna craotta rouze\craotta .} 

\Aimespeaks \direct{Eun tournèn i mentèn} \ldots é ren, véyade de chèdre amoddo\ldots é se sèide fran pa qui fa voté\ldots 

\StageDir{Aimé avouì lo dèi moutre la craotta. Dimèn Jordy l'a querià Julie pe lèi betì eun tita eun grou floque dzano\fiocco.}

\Juliespeaks Ouè tsertsade de itre responsablo é votade pe la fach\'on \direct{eun moutrèn lo floque} que vo plé de pi!

\Aimespeaks Mi devàn de voté fa comprendre comèn son allaye pe dab\'on le bague\ldots

\Juliespeaks Ad\'on a vo\ldots

\Tcheuttedouspeaks Sèidepou(vo)èr!

\act[Acte I]

\ridoiver

\scene[-- Lo Consèille di Digourdì]

\StageDir{\Fv{Eunna normalla réuni\'on di Digourdì}}

\StageDir{Eun scène n'at eunna tabla a fer de caval tott'apprestaye comme deun lo Consèille réjonal (microfonne, botèille d'éve \boteilledeve , plime, etc\ldots). La serventa l'è eun tren de controlé que totte sise a poste.}

\StageDir{Entron Madame le Prézidàn é lo secretéo.}

\Presidanspeaks \direct{Totta tracachaye \ajitou}
Ouè si pamì chiya d'èi la Compagnì
deun le man\ldots

\Secreteospeaks Senque te lo fé diye?

\Presidanspeaks Marco me baille todzor contre, salla premì fenna de Cima Sander veun pamì i proue dèi que n’en pamì bailla-lèi de partie eumpourtante é salla carogne de Sophie n'i la fèi que l'ise voya de me robé la poltronna.

\Secreteospeaks Mmm eugn éffé te pou belle èi rèiz\'on, te sa? Eunc\'o Jordy n'i la fèi que l’a an mia de sèi de pouvoer.

\Presidanspeaks \direct{Malisieuze \malisieu} Lèi penso pouì mé a fé ité quèi salla beurta rosa\ldots

\Secreteospeaks Bon bon!

\StageDir{Entre Marco. Sensa diye ren va s'achouaté. Aprì entron Jordy é Cima eun borbottèn quetsouza.}

\Jordyspeaks \direct{S'arite to d'eun creppe avouì Cima} Ad\'on, se te me baille fèi té deun lo 2019 te fé seutta partia iaou te fé eun grou monologue devàn 30 meulle personne \direct{Cima dimèn sondze seutta \textit{scène}} té to solette!

\Cimaspeaks Mi mé n'i todzor deu-lo que té t'iye an mentalitoù jénialla, té \textit{Giordi} t’a an martse eun pi!

\Jordyspeaks Ad\'on te sa sen que fé\ldots

\StageDir{Jordy s'achouatte é Cima s'achouatte protso de Marco.}

\Marcospeaks Senque conte noutro Bollon?

\Cimaspeaks La propozou-me an baga que no pouèn pa refezé, té te fa renque fé sen que te diyo mé.

\StageDir{A drèite entre Sophie é Jordy la queurie avouì eun mouvemèn de la man \vieni. Sophie s'achouatte protso de llou é dimèn Jordy lèi borbotte quetsouza pe lo bouigno \bisbigliare .}

\Sophiespeaks Mi te lo sa que avouì salla comandanta \direct{eun moutrèn Marlène} n'en pa droué de paolla \quei!

\Jordyspeaks \direct{Eun terièn foua eunna busta \busta\ } Té fé comme n'i deu-te\ldots é le bague tsandzon pouì!

\Sophiespeaks Iprouèn ad\'on!

\StageDir{Areuvvon le-z-atre atteur de la compagnì. Tcheutte s'achouatton.}

\Presidanspeaks Dimanderio a la serventa de no porté l’odre di dzor dimèn que lo consèillì secretéyo vérifiye le prézanse eun prosédàn pe appel nominal eugn ordre alfabétique.

\StageDir{La serventa baille a tcheutte lo foillette de l'odre di dzor é dimèn lo secretéo fé l’appel. Tsaqueun repoùn eun levèn la man \man .}

\Secreteospeaks Prosédèn avouì l’odre di dzor:
\begin{enumerate}
\item sujé de la pièse 2019;
\item bouite di lettre eun bouque ou eun feu pe lo nouo sièje;
\item mise a bilàn de la \og colach\'on\fg a Lugano \direct{avèitse Cima} de 2000 èiro; \StageDir{Tcheu repoundon avouì eugn \og Euh!\fg é avouì lo dèi lév\'o reprodzon Cima \dei ;}
\item cotach\'on eun Boursa de la Compagnì.
\end{enumerate}
La paolla i Prézidàn!

\Presidanspeaks \direct{Avouì eun ton politisièn} Mersì secretéo. Bonsouar a tcheutte, chère Digourdie é cher Digourdì. Oueu, lo 23 settembro 2018, no sen seuilla pe analizé lo sucsé de la pièse \og Todzo pi Digourdì\fg que la fé-no gagnì la seconda plase i Prix Magui Bétemps.
\StageDir{Tcheutte boueuchon di man \bouechiman .}
Mi comme vo sade l'è belle l’aoua de pensé i demàn, é lo demàn l’é lo Printemps Téatral 2019. Dimando ara a sise d'euntre vo que l'an eunna idoù de la propozé i Consèille.

\StageDir{Richard levve la man \man\ pe prendre la paolla. Jo\"{e}lle lèi fé vère que dèi la prendre eun gnaquèn eun bot\'on. Richard trebeulle an mia é aprì gnaque lo bot\'on djeusto.}

\Presidanspeaks L’a dimandoù de èi la paolla lo consèillì Cunèya é l’è deun sé drouè.

\StageDir{Richard se levve \footnote{ Dèi-z-ara tcheu le consèillì que prègnon la paolla se levvon pe prédjì.}, comenche a prédjì mi se sen ren pequé l'a pa allemoù lo microfonne. Tcheu lèi diyon de allemì lo microfonne. Richard se inerve é can areuvve finalemàn a allemi-lo se sen eun \og isto seutte \textit{porcherie}''. Tcheu le prézàn lèi repondon avouì eun\og shttt\fg.}

\Cimaspeaks \direct{Ver lo pebleuque} Squezade\ldots l'è de Jovençan!

\Richardspeaks Bondzor a tcheutte! Vouillo maque propozé eunna dzenta idoù: eunna pièse avouì dou vétchot devàn eun fornet \fornet, fan la counta de comèn son tsandjaye le séiz\'on, avouì eun bon dobl\'on de rodzo é\ldots

\Jordyspeaks \direct{Eun bloquèn Richard} Mi pe pleizì! Conséillì Cunéaz, achouatade-v\'o é contignade maque a lie la Vallée. Chère Digourdie é cher Digourdì, diyèn prao i tradechonnelle pièse iaou n'a todzor dou pouo campagnar que se plègnon de la burocrasì ou si lo fé que n'a pamì de demì sèiz\'on, diyèn prao i pièse iaou n'a todzor la fénna que fé de grouse corna a l'ommo. Chère Digourdie é cher Digourdi fièn de pièse pi fritse, pi lévette, pi éffervéchante, pi moussaye, pi plèizente. Chère Digourdie é cher Digourdì, si seu pe vo propozé eunna pièse su le Simpson\homer !

\Marcospeaks Ouè, brao Jordy! Te sa que riye Cima to dzano que rotte é bèi de biye si lo chofà!

\Dallasspeaks Chère Digourdie é cher Digourdì\ldots Madame lo Prézidàn\ldots l'an deu-me que oueu l'è voutro anniverséo é ad\'on n'i porto-vo eun dzen cad\'o, pe vo.

\StageDir{Paolo régale eun mase de fleur a Marlène. Aprì  reprèn son discoù.}

\Dallasspeaks Chère Digourdie é cher Digourdì, conséillì Bollon, mi de senque v'ouite eun tren de prédjì! De fé le Simpson? Avèitsade lo pebleuque: pappa é mamma de mé san pa senque son le Simpson; fa prédjì di bague de no-z-atre, de la Val d'Outa, de la poleteucca\ldots mogà pourian fé eunna pièse si noutra Prézidanta\ldots èitsade comme l'è dzenta!

\Presidanspeaks Bon-a idoù Paolo! Senque n'a-tì de pi téatral di noutro Consèille réjonal? Ah voualà\ldots pouèn fé la conta de comèn n'i gagnà le-z-éléch\'on pe vin-ì Prézidàn! \direct{Inervaye} L’a dimand\'o la paolla la conséilliye Comé é fièn prédjì eunc\'o lleu.

\Sophiespeaks Ad\'on gars\'on mé diyo na, na a eugn atra pièse que prèdze de no! Sarie lo sec\'on an si de no, aprì le dzi son stouffie \stufo\ é veugnon pamì no vère!

\Cimaspeaks Oué vèyo dza Ettore Champrétavy que l'è seuilla \direct{moutre Ettore i mentèn di pebleuque} que no fé la morale si lo fé que ara fièn renque de pièse \textit{autobiographique}.

\Joellespeaks Mi Paolo, que problème n'at? Disa sen que la voya, can n'en jamì acouto-lo?

\StageDir{Caqueun baille rèiz\'on a Jo\"{e}lle.}

\Eyviaspeaks Squezade mi la poleteucca l'è an baga que sentèn dza a la radi\'o, a la télévij\'on\ldots é mé eunc\'o finque i mitcho avouì pappa que me conte le counte de Place Deffeyes!

\StageDir{Partèi eun vacarno jénéral, tcheutte se prèdzon eun desì l'atro, tanque lo Prézidàn fé restì tcheu quèi}

\Presidanspeaks\direct{Eun braillèn é souèn lo
campaneun \campanel} Silaaanse! 

\StageDir{Tcheutte fan silanse, mouèn que Richard é lo secretéo, que s'apesèison pa que le-z-atre son tcheu quèi.} 

\Secreteospeaks \ldots se n'i deu-te que té te pourte lo veun \wine\ é mé la polenta\ldots

\StageDir{Tcheutte avèitson mal le dou.}

\Presidanspeaks Vo dou? Soplé, sen eun tren de prédjì de bague sérieuze. \direct{I Consèille} Vo rappello que mé si lo Prézidàn é tanque a proua contréa désido to mé!

\Marcospeaks Ita quèya tsaretta! T'i pa i micho iaou te pou comandé!

\Jordyspeaks Oué Madame lo Prézidàn, bèichade la crita, pequé sen deun eunna démocrasì é se no sen pa d'acor, vo tapèn ba de salla caèya! Vourio rappelé que tanque ara comme Prézidàn v'ouèide fé ren! Vourio rappelé a tcheutte qui l'è itoù lo Prézidàn,\direct{bièn fier} lo seul vrèi Prézidàn que l'a fé-vo gagnì la seconda plase \second\ di Prix Magui Bétemps!

\StageDir{Mèitchà Consèille boueuche for di man eun baillèn man forta a Jordy.}

\Presidanspeaks Consèillì Bollon, vo dimando de fé attench\'on a comme prédzade é de pourté eunna mia de reuspé pe seutta sala.

\Jordyspeaks Squezade!

\Presidanspeaks\direct{A Jordy} Vo pensade que sen que diyo mé l'è pa partadjà pe tcheutte? Mé si chiya que, se lo betèn i votach\'on, mé gagno!

\StageDir{Parmì le consèillì n'a qui boudze la tita pe diye de ouè é qui pe diye de na.}

\Jordyspeaks  Madama la Prézidanta va bièn. Mi se aprì la votach\'on v'ouèide pamì la majoranse vo v'ouèide fenì de fé lo Prézidàn!

\Dallasspeaks  Consèillì Bollon! Iaou pensade de itre? I martchà? Mi avèitsade que mouro que v'ouèide, me semblade lo Ministre Toninelli!

\Cimaspeaks \direct{A Paolo Dall'ara} Mi comencha a prédjì lo patoué comme se dèi, peuccaparmidjàn! Can mimo Jordy bailla pa fèi a sitte; pitoù, v'ouite eun tren de me diye que se salla pantoufla \pantoufla\ l'a pa la majoranse no no la gavèn di pià?

\StageDir{Partèi eun vacarno jénéral, tcheutte se prèdzon eun desì l'atro, tanque lo Prézidàn fé restì tcheu quèi.}

\Presidanspeaks\direct{Eun braillèn é souèn lo
campaneun \campanel}  Silaaanse!

\StageDir{Tcheutte fan silanse, mouèn que Richard é lo secretéo, que s'apesèison pa que le-z-atre son tcheu quèi.} 

\Richardspeaks \ldots  n'i deu-te que lo douse lo pourto mé\ldots

\Presidanspeaks Euncoa vo dou? Prao! Dimanderio a la serventa de no pourté lo \textit{Statut} é i noutro secretéo, se l'a pa d'atro a fiye, de no lie l'art. 4 - \'Elech\'on di tsardze.

\StageDir{Lo secretéo se beutte le lenette \lenette , teurie foua eun grou livro \livro\ avouì desì bièn de poussa, souffle desì pe la gavé, tsertse la padze djeusta é teussaille.}

\Secreteospeaks Ad\'on \og\textit{dans le cas où, à l’occasion d’une votation importante comme: le nombre de \og dobl\'on \fg de vin rouge à acheter pour chaque épreuve, le choix de la destination de la sortie annuelle ou le sujet d’une pièce, si le Président n’obtien pas 50\% + 1 des voix, il doit remettre sa charge}\fg.

\Sophiespeaks Aaah! 

\StageDir{Jordy, Cima é Eyvia, eunsemblo, diyon \og ah aaah!\fg. Aprì Cima baille eun creppe eun tita a Marco que l'ie distré é avouì eunna mia de retar eunc\'o Marco braille \og Ah ah aaah\fg.}

\Jordyspeaks Ad\'on\ldots \direct{eun countèn le catro conséillì que son avouì llou} Madame lo Prézidàn, se la matématique l'è pa eunna conta foula, v'ouèide le-z-aoue contaye!

\Presidanspeaks \direct{Tracachaye\tracachaie} Se lo riillemèn l'è pai fa lo reuspetté! Mi can mimo vourio vo rappelé que se oueu lo directif tsi, noutro spettacllo i Printemps Té\^atral l'è a forta reusca.

\Marcospeaks Molla de prédjì é fé-no voté!

\Presidanspeaks Mesieu lo secretéo vo sade sen que déyade fée. Pasèn i votach\'on.

\Secreteospeaks Ouè Madame lo Prézidàn! Ad\'on, lévisan la man sise que son a faveur de la pièse \og Le Simpson\fg\ldots

\StageDir{Levvon la man Cima, Jordy, Sophie é Eyvia. Marco l'a la tita pe le gnoule é ad\'on Cima lèi baille tourna eun creppe eun tita pe lèi diye de lévé la man. Marco levve la man é lo secretéo marque si lo verbal sinque vouése pe la pièse \og Le Simpson\fg.}

\Secreteospeaks Lévisa la man sise que son a faveur de la pièse \og Comèn n'i gagnà le-z-éléch\'on\fg\ldots

\StageDir{Levvon la man Marlène, Jo\"{e}lle, Paolo Dall'Ara, Richard é Fabien\ldots a cats\'on Paolo Dall'ara soum\'on eun per de beillette de 50 euro \cash\ a la serventa, que totta contenta levve eunc\'o lleu la man.}

\Secreteospeaks Eun, dou, trèi, catro, sinque \direct{eun véyèn la serventa} é chouì! Avouì chouì préféranse pe lo partì de majoranse la moch\'on l’è\ldots

\Sophiespeaks \direct{Pren to de chouite la paolla}
Na, na, na! Dèi can le servente l'an-tì drouet de
voté? \direct{A la serventa} Betade maque ià sise sou!

\Secreteospeaks  \direct{Jèinoù\imbarazzatoo} Ah ouè, ouè, djeusto\ldots ad\'on patta a sinque! Le Digourdì l’an pamì eun Prézidàn!

\Cimaspeaks Ah, ah, ah! I mitcho!

\Eyviaspeaks \'E ara comèn martse?

\Richardspeaks Ouè, fièn comme va de moda: no no micllèn tchica é no terièn foua eunna noua majoranse!

\Jordyspeaks Conséillì Richard Cunéaz, soplé, achouatade-v\'o é contegnade a lie la Vallée. No no micllèn pa; comme resite noutro \textit{Statut}, que can l'io Prézidàn sayò tot a mémouée\ldots \direct{sarcastique \diabolic} n'ayò pa fata de eun secretéo, eh Madame lo Prézidàn; comme resite noutro \textit{Statut}, can n'en pamì eun Prézidàn la paolla passe i peuple tsarvensolèn. Secretéo controllade maque\ldots la padze l'è la 3056.

\Secreteospeaks Na, na, mé lèi crèyo pa.

\StageDir{Pren la padze é vérifie.}

\Secreteospeaks Ouè l'a fran rèiz\'on! Lo nouo Prézidàn dèi itre votoù pe lo peuple tsarvensolèn! Ad\'on fa acapé eun dzor pe le-z-éléch\'on. \direct{Eun mioulèn eun dzor si lo calandrì \calendario} A désambre na pequé n'a la betchì; a janvieur na pequé mé si ià colaté; a févrì sen aoutre a Lugano; ah lo voualà! Lo premì dzor disponiblo pe \og l’\textit{election} dzor\fg l’è lo 9 marse 2019! Cher Digourdì, bon-a campagne élettoralla a tcheutte!

\StageDir{La fach\'on de Jordy se levve totta contenta eun fièn fita \fita. Marlène, i contréo, chor malechaye \malecha.}

\StageDir{Si l'\textit{écran} l'è proyettaye la vidéo de l'émich\'on ecstraordinére di TG de NO, iaou François Catro Lenve, djeusto foua di sièje di Digourdì, eunterviste la ex Prézidanta Marlène Jorrioz é Jordy. Marlène scappe ià de coursa; i contréo Jordy s'arite é rep\'on i questch\'on de François, sensa pédre l'occaj\'on, avouì le compagn\'on de llou, de fé campagne élettorale avouì de mésadzo populère.\\ Dimèn Marlène comande i seun fidèle Paolo Dall'Ara de acapé 800 vouése i Chef-Lieu; lèi euntéresse pa comèn é avouì queunse sou, vou 800 vouése.}

\StartVideo{https://www.facebook.com/thedigourdi/videos/578848659263718}{La chute di Prézidàn}

\scene[-- Campagne élettorale]

\StageDir{Lo palque seré divijà eun trèi poueun icllerià : a gotse n'a Jean Luc, a drèite eun Pompiolèn eun tren de coueillì de sec\^orie é i mentèn n'at eunna viille penchon-aye saventa que baille bèye i son jéragn\'on to sèque.}

\StageDir{Dèi-z-ara tsaque fach\'on l'aré deun l'arbeillemèn quetsouza de dzano\fiocco   (pe Jordy) ou rouze\craotta (pe Marlène).}

\StageDir{Lemie \lemieSi\ a gotse  é teuppe \lemieBa\ si lo reste di palque.}

\StageDir{Jean Luc l'è eun tren de tournì a mèiz\'on avouì eun grou per de-z-esquì \eski . Dimèn,  dérì de llou, Paolo é Jo\"{e}lle tsatcharon.}

\Dallasspeaks Acoutta Jo\"{e}lle, n'i prédzà avouì Madama la Prézidanta é l'a deu-me \og te fa partì avouì Jo\"{e}lle, alé i Chef-Lieu é pourté-me 800 vouése!\fg.

\Joellespeaks L'a deu 800? Ouè ad\'on fa comenchì ara! Mi senque dièn i dzi?

\Dallasspeaks No lèi contèn de conte foule, to sen que no passe pe la tita\ldots acoutta: sitta \direct{eugn avèitsèn Jean Luc} te lo cougnì?

\Joellespeaks Oue l’è de Tsarvensoù é travaille eun Pila.

\Dallasspeaks Ouè, parfé! Comencha té a prédjì, que l'è eun dzen gars\'on é té t'i eunna dzenta feuille!

\StageDir{Jo\"{e}lle s'aprotse a Jean Luc di coutì drèite, dimèn que Paolo s'aprotse de l'atro coutì.}

\Joellespeaks Bondzor Mesieu!

\StageDir{Jean Luc se vionde ver Jo\"{e}lle é sensa s'apersèi reusque de tchapé la tita de Paolo avouì le-z-esquì.}

\Jeanspeaks Oh Bondzor Jo\"{e}lle!

\Joellespeaks Tot amoddo?

\Dallasspeaks \direct{Eun s'aprotsèn a Jean Luc} Salì l'ommo!

\StageDir{Jean Luc se vionde ver Paolo é, comme douàn, sensa s'apersèi, reusque de tchapé la tita de Jo\"{e}lle avouì le-z-esquì.}

\Dallasspeaks Que dzen-z-esquì! Le pouzèn ba mogà, eh? Maque eunna meneutta.

\Jeanspeaks Mi ouè\ldots

\StageDir{Lagnà é eun souflèn \stouffie, Jean Luc pouze ba le-z-esquì} 

\Joellespeaks Que dzen que son!

\Jeanspeaks \direct{To fier} So son le nouo Salomon SC40!

\Dallasspeaks \direct{Eun fièn semblàn de savèi} Ah créyao SC35!

\Jeanspeaks \direct{Euncoa to fier} \ldots mi son de moublo! Van to solette! Niv\'o coupa di mondo! 

\Dallasspeaks Imajino\ldots traillà amoddo oueu?

\Jeanspeaks \direct{Todzor eun sboufèn} Trop! Si d'eun lagnà, to lo dzor beutta si le mèinoù é beutta ba le mèinoù; t'areuvve a la feun de la dzornoù que t'a renque voya de medjì eun bon platte de seuppa é d'acapé la coutse \letto . 

\Dallasspeaks Acoutade: no vo portèn ià maque eunna meneutta!

\Jeanspeaks Senque me vendade?

\Joellespeaks Na, na, vendèn ren. No sen djeusto seu pe vo propozé lo nouo projé pe le-z-éléch\'on di nouo Prézidàn di Digourdì.

\Jeanspeaks Le digourdì?

\Dallasspeaks Vo sade pa qui son le Digourdì?

\StageDir{Jean Luc gante la tita.}

\Joellespeaks Le Digourdì son\ldots

\StageDir{Paolo fé resté quèya to de chouite Jo\"{e}lle, pe pa que esplequisa a Jean Luc qui son le Digourdì.}

\Dallasspeaks Ah! Ahn\ldots sa pa qui son le digourdì\ldots

\StageDir{Paolo refléchèi \pense doe seconde é repette \og sa pa qui son le Digourdì\fg eugn avèitsèn Jo\"{e}lle pe lèi diye \og di pa ren\fg.}

\Dallasspeaks Le Digourdì l'è la Digourdì S.p.A., la sociétoù de Milan que l'è arrevaye eun Val d'Ousta é l'a atsetoù tcheu le-z-eumplàn!

\Joellespeaks \ldots é eunc\'o si de Pila!

\Jeanspeaks Ah, é dèi can?

\Dallasspeaks Dèi can? Dèi oueu lo mateun, vito, te sa que a Milan se rèchon lo mateun é atseuton a drèite é a gotse!

\Jeanspeaks Ah, sayò pa qué, pa lizì ren si lo journal.

\Joellespeaks Ouè é aprì l'an a queur lo bièn ètre di travailleur. Vo véo traillade?

\Jeanspeaks  Eh\ldots dèi can chor lo solèi tanque a can lo solèi va ià!

\Joellespeaks Tan ad\'on!

\Dallasspeaks \ldots é véyo vo gagnade?

\Jeanspeaks Féo chouì mèis a l'an, véo t'ioù que gagniso?

\Dallasspeaks Ad\'on ara v'ouèide la poussibilitoù de tsandjì voutra viya! Ara tsandze la Prézidanse di Digourdì é noutra Prézidanta vou beté la \textit{Rente di Citoyen}. T'a dza senti-nèn prédjì si le journal \journal ? Avouì la \textit{Rente di Citoyen} te gagne to l'an eunc\'o se te travaille pa!

\Jeanspeaks Ah dzen so!

\Dallasspeaks Acoutta! Se t'ioù tsandjì ta viya, lo 9 de marse te fa alé i téatro Splendor a voté!
 
\Joellespeaks \ldots é fa voté Marlène, Prézidàn Marlène.

\Dallasspeaks \ldotsé se te te lo rappelle pa, pensa\ldots
 
\Joelledallasspeaks \direct{Eun tsantèn} \og Te travaille pe trèi mèis, te gagne to l'an, vota Marlène Prézidàn!''

\StageDir{Paolo é Jo\"{e}lle baillon eun beillette élettoral de Marlène a Jean Luc.}

\StageDir{Eunna viille madama \viille\ chor di mentèn di palque (que l'ie sensa lemie) é se miclle deun lo discour de Paolo é Jo\"{e}lle.}

\Saventaspeaks Senque l'è seutta counta que sitte travaille trèi mèis é gagne to l’an é me avouì ma pégna pench\'on areuvvo gnenca a feun senà .

\Dallasspeaks Mondjemé madama, vegnade avouì mé que vo conto totta la counta! \direct{A Jean Luc} Gars\'on restade-mé bièn! Salì\ldots é te gagne pe to l'an!

\StageDir{Paolo é la madama se tramon i mentèn di palque dimèn que Jo\"{e}lle di eunc\'o doe paolle a Jean Luc.}

\StageDir{Lemie \lemieSi\ i mentèn di palque.}

\Joellespeaks Ad\'on lo 9 marse Lista num\'er\'o 1 - Pe lo bièn di péì!

\Jeanspeaks Ah tranquila!\direct{Eun moutrèn lo beillette élettoral} Mé la croueu la beutto seu desì! 

\Joellespeaks Ad\'on mersì é \textit{au revoir}!

\StageDir{Jo\"{e}lle rejouèn Paolo é la viille i mentèn di Palque.}

\StageDir{Teuppe \lemieBa\ a drèite di palque.}

\Dallasspeaks \direct{Ironique} Mondjemé Jo\"{e}lle avèitsa que dzen jiragn\'on que l'a seutta madama!

\Joellespeaks \direct{Avouì eunna mia de doute \doute} Ouè que dzen\ldots fran dzen\ldots dzen jiragn\'on!

\Dallasspeaks \ldots é te sa pa, te sa pa véyo pren de pench\'on seutta poua fenna avouì sise dzen jiragn\'on; madama, diade-lèi!

\Saventaspeaks Eunna mizée!

\Dallasspeaks \ldots é ad\'on madama vouillade tsandjì voutra viya? Sitte l'è lo momàn, v'ouèide dza sentì prédjì de \textit{Quota} 100? Son 100 meulle \textit{euro} to l'an! Lo premì de janvì si lo contcho!

\Saventaspeaks Na! Mondjemé!

\Joellespeaks Eh ouè madama\ldots

\Dallasspeaks Jamì vi eunna baga pouèi, vrèi madama? Mi lo 9 de marse fa alì voté\ldots

\Joellespeaks \ldots é fa voté Marlène, Prézidàn Marlène.

\Dallasspeaks \ldots é se vo rappelade pa\ldots

\Joelledallasspeaks \direct{Eun tsantèn} Pe èi de dzen jiragn\'on é aoumenté la pench\'on, votade noutra fach\'on!

\StageDir{Paolo é Jo\"{e}lle baillon eun beillette élettoral de Marlène a la viille madama.}

\Saventaspeaks Ouè, ouè va bièn, ouè cheur!

\Dallasspeaks  Oh madama! Mi véo v'ouite dzenta? Poui-dze vo baillì eun poteun?

\Saventaspeaks \direct{Emochon-aye\felice} Ouè eunc\'o dou!

\StageDir{Se baillon dou poteun dimèn que Jo\"{e}lle levve le joueu si pe l'er.}

\Saventaspeaks \direct{A Jo\"{e}lle} Ouè eunc\'o a té, t'i tan dzenta.

\Joellespeaks \direct{Eun salièn avouì eun poteun la madama} Ouè, ouè mersì é \textit{au revoir}; \direct{todzor bièn ironique} fran de dzen jiragn\'on!

\StageDir{Jo\"{e}lle é Paolo chorton de scène ver la drèite di palque to teuppe.}

\Saventaspeaks \direct{Ver lo pebleucco} Digourdì\ldots la fèi que sise Digourdì son pi digourdì de sen que diyon de itre\ldots

\StageDir{Teuppe \lemieBa\ i mentèn di palque é lemie \lemieSi\ a drèite.}

\StageDir{N'at eun vétchot que l'è eun tren de coueillì de sec\^orie. Dérì de llou areuvvon Cima é Marco avouì eunna tsaretta plèin-a de bague.}

\Cimaspeaks Ad\'on Marco ieur nite n'i deu-te: fa alé ià eun campagne élettorale é te fa me prègne de saouseuse \salan, de boudeun, de moutsetta, de beuro coloù\ldots é ara, ad\'on, te me euspleuque senque no nèn fièn de eunna dzeleunna \gallina ?

\Marcospeaks \direct{Eun terièn foua la dzeleunna de la tsaretta} Seutta dzenta dzeleunna  no serveré pi prao!

\Cimaspeaks  Mi a senque? Mi te me vèi entré i Limonet avouì eunna dzeleunna?

\Marcospeaks La fièn pi foua devàn de arrevé i Limonet; tracàcha-te pa!

\Cimaspeaks Marco, sen eun campagne élettorale!

\Marcospeaks N'i deu-te de ité tranquilo! Lèi penso mé.

\Cimaspeaks  Va bièn, attaquèn que sen dza eun retar.

\StageDir{Cima é Marco s'aprotson i vétchot.}

\Marcospeaks Salì mesieu!

\Pompiolenspeaks Salù!

\Marcospeaks Ouemma que dzente sec\^orie!

\Pompiolenspeaks Oh salì, salì!

\Cimaspeaks Comèn l'è, tot amoddo?

\Pompiolenspeaks Ouè, ouè, tanque sen pa a fé téra va todzor bièn!

\Cimaspeaks Ad\'on tot amoddo!

\Pompiolenspeaks Mi ouè coueillèn catro sec\^orie.

\Marcospeaks Èita Paolo que dzente sec\^orie que l'a lo mesieu\ldots

\Pompiolenspeaks Ah ouè sit an son mondiale! Ouè bon\ldots fa eunc\'o le lavé, se-z-adebé\ldots mi aprì seutte eun salada avouì dou-z-où \ou son mondiale!

\Marcospeaks Ah avouì dou-z-où? N'i sen que fé pe vo!

\StageDir{Marco teurie foua doze-z-où é le baille a Cima.}

\Pompiolenspeaks \direct{Pren eunna sec\^orie é la moutre a Paolo} Èitade seu que qui blan! Gneunca la fenna can l'ie dzoueun-a l'ayè eun qui pouèi!

\Cimaspeaks \ldots é ad\'on sade an baga? Avouì le-z-où de Feleunna \direct{lèi fé vère le-z-où}\ldots  voutre sec\^orie\ldots 

\Pompiolenspeaks Ah mi son pe mé? Ouè mi que jantilo que v'ouite mesieu!

\Marcospeaks \direct{Eun prégnèn la dzeleunna} Mi ouè mesieu fièn pai: prégnade eunc\'o la dzeleunna.

\Pompiolenspeaks Mi tro jantilo! 

\Cimaspeaks Mi ouè, fièn an baga: betèn to dedeun  voutra saquetta di sec\^orie.

\StageDir{Lo Pompiolèn ivre la saquetta é lèi beutte dedeun la dzeleunna é le-z-où.}

\Pompiolenspeaks Oh que dzen, pouèi eunc\'o la fenna l'è contenta, contenta!

\Cimaspeaks Ah voualà!\direct{Eun terièn foua eun beillette élettoral} Rappelade-v\'o maque, mesieu, lo 9 de marse, Lista num\'er\'o 2, Digourdì\ldots

\Pompiolenspeaks Eh? Senque l'è seutta porcaria?

\Cimaspeaks \direct{I pebleuque} Oh lamondjeu, sitte comprèn ren! \direct{I vétchot} Dèijade alì beté an croueu lo 9 de marse si la lista num\'er\'o 2.

\Pompiolenspeaks Ah ouè lista num\'er\'o dou\ldots pe le-z-où!

\Cimaspeaks \direct{I pebleuque} L'a pa comprèi ren! \direct{I vétchot} Dèijade beté an croué si la\ldots

\Pompiolenspeaks \direct{Eunfastedjà \malechaa } Ouè, ouè n'i comprèi! Lista num\'er\'o 2!

\StageDir{Paolo teurie foua eugn atro beillette élettoral.}

\Cimaspeaks Ouè é prégnade eunc\'o sitte pe la fenna, prégnade tcheu dou.

\Pompiolenspeaks\direct{Eun prégnèn lo sec\'on beillette élettoral} Ouè tcheu dou, pouèi dou pi dou\ldots catro!

\Cimaspeaks \direct{I pebleuque} Mondjemé l'a pa comprèi ren sitte. \direct{I vétchot} Dèijade beté an croueu si la\ldots

\Pompiolenspeaks \ldots amoddo, amoddo, itade mai tranquilo: voto pe vo, tracachàde-vo pa!

\Marcospeaks Ad\'on parfé! No ara alèn\ldots \textit{au revoir} mesieu!

\Cimaspeaks Tanque mesieu!

\StageDir{Marco é Cima se nen van ver la gotse di palque.}

\StageDir{Teuppe \lemieBa\ a drèite é lemie \lemieSi\ a gotse.}

\StageDir{Marco é Cima acappon Jean Luc.}


\Cimaspeaks\direct{A Jean Luc} Salì! Comèn l'è?

\Jeanspeaks Ah salì! Ouè mi que trafeutso oueu ià pe Tsarvensoù.

\Cimaspeaks Mondjemé que dzen esquì!

\Jeanspeaks Ouè, ouè son an bomba!

\Cimaspeaks \ldots é ad\'on comèn l'è, tot amoddo?

\Jeanspeaks Mi ouè amoddo! Te vèi pa comèn si to contèn? Son djeusto ià doe dzi que l'an deu-me que me payon pe to l'an; mé traillo maque chouì mèis a l'an, féo la sèiz\'on si eun Pila.

\Cimaspeaks \direct{Douteu \pense} Doe dzi? Mi comèn l'ion arbeillà? Comèn l'ion?

\Jeanspeaks Mi to terià a fita, eun gars\'on l'ayè la lenva londze, molè pa de prédjì! La feuille, pitoù, grachaouza\ldots

\Cimaspeaks \direct{A Marco} Aaah\ldots l'è lo peuccaparmidjàn! Lo Biélèis!

\Marcospeaks Mi espleuca vèi mioù: pequé t'i contèn?

\Jeanspeaks Mi pequé, d'abor, sise Digourdì ou Digourdì S.P.A. l’an atsetoù tcheu le-z-eumplàn é ara l'an a queur lo bièn étre de no travailleur: me payon to l'an!

\Marcospeaks Le Digourdì S.P.A.? Èita, me diplé te lo diye mi le Digourdì son pa eunna S.P.A.; le Digourdì sen no, fièn de Téatre.

\Cimaspeaks L'è eunna compagnì téatralla!

\Marcospeaks \ldots atsetèn pa de-z-eumplàn\ldots

\Cimaspeaks \ldots atsetèn pa seutte bague\ldots Marco\direct{eun moutrèn la tsaretta}, soplé, fé-lèi vère qui sen, sen que n'en no-z-atre.

\Marcospeaks Ouè to de chouite que lo vèyo an mia ba de moral si pouo ommo\ldots baillèn an mia de coadzo!

\StageDir{Marco teurie foua de la tsaretta eunna tsèina de saouseuse \salan\ é la baille a Cima.}

\Cimaspeaks \direct{Eun gneflèn la saouseuse} Seutta l'è la saouseuse de Feleunna! Sen que bon flo\ldots

\Jeanspeaks \direct{Eun gneflèn} \ldots mmm que bon!

\Cimaspeaks La saouseuse de Feleunna!

\StageDir{Cima beutte la tsèina a l'entor di cou de Jean.}
 
\Cimaspeaksèita Mi avèitsa comèn te reste bièn; é aprì lo flou que l'at!

\Jeanspeaks Mi l'è pe mé?

\Cimaspeaks Mi ouè, pe vo! V'ouite eugn ommo formidablo.

\StageDir{Marco teurie foua eugn atra tsèina é la beutte a l'entor di cou de Jean Luc.}

\Marcospeaks Eunc\'o seutta!

\Cimaspeaks Eunc\'o eugn atra pe la fenna: la dobla saouseuse de Feleunna!

\Jeanspeaks Mi mersì! Te fé té so?

\Cimaspeaks Ouè féo mé betsì i mitcho, mé é Roberta\ldots son an baga formidabla! Mi te fa rappelé an baga\ldots

\StageDir{Marco baille a Paolo eun beillette élettoral.}

\Cimaspeaks \ldots lo 9 de marse te dèi alì votì le Digourdì\ldots mi la Lista num\'er\'o 2, sise dzano, le pi onéto!

\StageDir{Cima baille a Jean Luc lo beillette élettoral.}

\Jeanspeaks \direct{Eun lizèn} \og Eunsemblo pe le Digourdì''.

\Marcospeaks \ldots pa sise que atseutton le-z-eumplàn!

\Cimaspeaks Mi na! Sen de dzi pi onnéto no, do saouseuse, do\ldots

\Jeanspeaks  Vo diyo an baga! Mé lamo le dzi comme vo, que prèdzon pouai téra téra, diyon le bague comme son! Pa de counte foule, que gneun comprèn ren!

\Cimaspeaks Mi na! 

\Jeanspeaks La croueu la beutto si vo dzano!

\CimaMarcospeaks Oh brao! Brao!

\Cimaspeaks Ad\'on bièn mersì\ldots mi can mimo\ldots la saouseuse de Feleunna l'a an martse eun pi!

\Jeanspeaks Ouè, èita t'isa eun boc\'on de pan eun pi\ldots

\Cimaspeaks Eh na, fièn pi eugn atro cou! Ara vo salio.

\Marcospeaks Ouè \direct{eun sarèn la man a Jean Luc} salì, no contegnèn noutro tor.

\StageDir{Marco pren la tsaretta é chor a gotse di palque. Cima dimèn salie eunc\'o llou Jean Luc é chor foua dérì de Marco.}

\StageDir{Teuppe \lemieBa\ a gotse é lemie \lemieSi\ a drèite.}

\StageDir{Entron Jo\"{e}lle é Paolo é reston dérì lo vétchot que l'è todzor eun tren de coueillì le sec\^orie.}

\Dallasspeaks Bon Jo\"{e}lle! Sen a bon poueun. Acoutta fièn eunc\'o sitte\direct{avèitse lo mesieu douàn llou} é aprì n'en belle que fenì\ldots acoutta\ldots comencha té que l'è eugn ommo!

\StageDir{Jo\"{e}lle s'aprotse i mesieu, a sa drèite, é Paolo s'aprotse a gotse.}

\Joellespeaks Bondzor mesieu!

\Pompiolenspeaks Oh bondzor bondzor! Que dzen vo vére: le fenne ià pe lo pro!

\Dallasspeaks Salì l'ommo!

\Pompiolenspeaks Ouè salì eunc\'o a vo\ldots

\Dallasspeaks Mi que dzen serdzette que v'ouite eun tren de coueillì!

\StageDir{Jo\"{e}lle se dispère pe sen que l'a deu Paolo.}

\Pompiolenspeaks\direct{Malechà \malecha} Senque?Le serdzette le-z-atsetade pi a sen de Bisson, don! Seutte son de SEC\^ORIE!

\StageDir{Jo\"{e}lle pren la paolla pe arendjì la gaffe de Paolo.}

\Joellespeaks  Mi que dzente sec\^orie! Aprì avouì le-z-où seutte son la feun di mondo! 

\Pompiolenspeaks Mi son mondiale! N’i belle que le-z-où; l'an djeusto bailla-me-leu dou mesieu seumpateucco. \`Eitsade que coqueun frique!  \direct{Eun lèi moutrèn le-z-où} L'an maque deu-me que dèyo voté eunna Compagnì téatralla, mi n'i pa tan comprèi, comprègno pa tan salle bague\ldots

\Joellespeaks\direct{Eugn avèitsèn Paolo} Ah n'i comprèi\ldots

\StageDir{Jo\"{e}lle avèitse Paolo dispéraye é eunsemblo tournon dérì lo mesieu pe prédjì dézò vouése. Dimèn lo vétchot conteugne a coueillì le sec\^orie.}

\Dallasspeaks Jo\"{e}lle senque fièn? Acoutta, mé lèi baillo $400$ euro, si pa sen que fiye.

\Joellespeaks Mi na fran pouèi! Ehi! \direct{Eun moutrèn lo vétchot} Acoutta\ldots

\Pompiolenspeaks Mi l'è todzor pi sèque!Gneunca le vatse cacon pamì comme eun cou! Maladetto! Ouè, saré pi prao pequé n'a pamì de sèiz\'on: lo dézer l'è a catro pa, l'an deu-me que l'è a Pon-Sen-Marteun é veun todzor pi si, todzor pi si! Ah iaou no allerèn fenì!

\StageDir{Paolo l'at eunna idoù. Avouì eun jeste, eunvite Jo\"{e}lle a tournì a l'attaque.}

\Dallasspeaks Deh l'ommo! Ad\'on l'è belle seque eh?

\Pompiolenspeaks An tappolette sit an, de bague moustre!

\Dallasspeaks Mi la sade eunna baga? Eunc\'o no sen di Digourdì, mi no alèn pa ià pe baillì de-z-où ou prédjì de téatre\ldots no\ldots no fièn de projé!

\Joellespeaks \ldots é de grouse bague!

\Pompiolenspeaks An comprèi\ldots

\Dallasspeaks\direct{Eun betèn eunna man si l'ipala di vétchot} Te vèi ba lé?

\StageDir{Paolo avouì lo dèi moutre eun poueun louèn.}

\Pompiolenspeaks \direct{Eun sarèn le joueu} Iaou? Vèyo pa tan bièn.

\Dallasspeaks Avèitsa sé, dèi le teun pià tanque ba lé i fon\ldots

\Pompiolenspeaks Ah dérì salla pesse!

\Dallasspeaks \ldots $4.452.000$ mètre car\'o de tsan avouì eugn eumplàn d'irrigach\'on a gotta pe le serdzette!

\Joellespeaks \ldots pe le sec\^orie! 

\Pompiolenspeaks Que dzenta baga!

\Dallasspeaks \ldots é ad\'on di-mé, véo te nen vou? Alé, spara si an tseuffra!

\Pompiolenspeaks Eun petchoù tsan pe mé é ma fenna.

\Dallasspeaks Mi alé beutta de pi! Lo doblo!

\Joellespeaks Ouè fièn trèi ou catro cou pi grou!

\Dallasspeaks \direct{Todzor avouì eunna man si l'ipala di vétchot} Te sa senque vouillon fé le-z-atre sé?

\Pompiolenspeaks Na, senque?

\Dallasspeaks Inque\ldots la TAV! Turin-Lyon passeriye seuilla!

\Pompiolenspeaks \direct{\'Eton-où \sorpreso} Eulla! Mi que malpolì!

\Dallasspeaks Mi no n'en deu de na! No seu vouillèn le serdz\ldots

\StageDir{Jo\"{e}lle avèitse Paolo avouì le flame dedeun le joué} 

\Dallasspeaks \ldots le sec\^orie!

\Pompiolenspeaks Ouè ouè! Penso beun!

\Dallasspeaks Ad\'on acoutta \direct{eun souplièn} lo 9 de marse vota pe no-z-atre.

\Joellespeaks Fa voté Marlène: Prézidàn Marlène.

\Pompiolenspeaks \ldots é beun itade maque tranquilo!

\Dallasspeaks Se te te lo rappelle pa\ldots pensa a\ldots

\Joellespeaks \ldots se pi de sec\^orie t'ou coueilìi\ldots

\Dallasspeaks \ldots vota Marlène \ldots

\Joellespeaks  \ldots di Digourdì!

\StageDir{Jo\"{e}lle lèi fé veure lo beillette élettoral de Marlène.}

\Pompiolenspeaks Ah ouè, ouè, n'i to comprèi. Betade maque deun la saquetta.

\StageDir{Jo\"{e}lle beutte lo beillette élettoral deun la saquetta; dimèn Paolo fé tsire pe tèra, dérì lo vétchot, 500 euro.}

\Joellespeaks  Fiade attench\'on mesieu!

\Dallasspeaks Deh l'ommo! V'ouèide perdì 500 euro, seu dérì\ldots

\Pompiolenspeaks Na, na, son pa de mé, la fenna me queutte pa alé ià di mitcho avouì sise sou.

\Dallasspeaks Na, na, son de vo mesieu \ok !

\StageDir{Paolo coueille le sou é avouì eun grou sourì le beutte deun le man di vétchot \cash .}

\Dallasspeaks Son de vo, son de vo, tegnade maque, pouèi vo rappelade pi amoddo!

\Pompiolenspeaks \direct{Eun pregnèn le sou} Ah ouè, ouè, n'i comprèi!

\Joellespeaks Bon ara alèn; \textit{au revoir}, salì mesieu!

\StageDir{Jo\"{e}lle pren Paolo pe lo bri é lo pourte ià, comme se l'ise pouiye que terise foua d'atre sou. Jo\"{e}lle é Paolo chorton a drèite.}

\Pompiolenspeaks \direct{Ver lo pebleuque} Eh eh eh! \direct{Eun countèn le sou} Mé n'i pi belle comprèi ren! Le rouze, le dzano, Marlène, Jordy, argh! Bague pe le dzouveun-o, salle bague lé; l'eumpourtàn l'è que oueu n'en pourt\'o i mitcho la pagnotta\ldots é la fèi avouì so\direct{eun moutrèn le sou} nen atseutto eunc\'o eun mouì! Ah l'è fran bièn aloù que sise rablasoque l'an fran comprèi ren: leur pensoon que mé fucho de Tsarvensoù, que votòo pe inque; na, na, na\ldots mé si eun Pompiolèn é si renque inque a la maodda!

\StageDir{Lo vétchot chor de scène a drèite.}

\StageDir{Teuppe \lemieBa\ a drèite é lemie \lemieSi i mentèn.}

\StageDir{Accapèn todzor la viille fenna eun tren de chouegnì le jiragn\'on sèque \frappe . Sentèn Cima prédjì solette dérì de lleu.}

\Cimaspeaks Maladetto d'eun botcha eumpestoù, avouì totte le bague que n'en a fiye, sen ià eun campagne élettorale\ldots senque l'a fi-me? La créou-me la raoua de la tsaretta! Maladetto, maladetto! Bon, sen eun retar\ldots féo mé to solette. \direct{Eun s'aprotsèn a la madama}Bondzor madama!

\Saventaspeaks Oh bondzor!

\Cimaspeaks Comèn l'è? Tot amoddo?

\Saventaspeaks Ouè, ouè.

\Cimaspeaks Senque v'ouite eun tren de fiye?

\Saventaspeaks Si eun tren de m'aprestì pe lo concour de la Quemeua Balconi Fioriti.

\Cimaspeaks Maladetto\ldots \direct{eun totsèn eunna foille sètse} seutte bague sètse, gnenca le tchivre de Fragno peuccon de bague pai!

\Saventaspeaks Prédzade pa pi mal di jiragn\'on de mé!

\Cimaspeaks Mi pequé, lèi tegnade co finque?

\Saventaspeaks Ouè l'è to l'iveur que le choueugno\ldots

\StageDir{Ver la drèite di palque sentèn la vouése de Marco que, eun braillèn\og Paolo si arrevoù\fg, entre eun scène avouì la tsaretta \tsaretta; mi eugn arrevèn to de coursa, Marco s'apersèi pa di jiragn\'on é le gnaque avouì la tsaretta.}

\Saventaspeaks \direct{Épouvantaye \paura} Euh! Mé dzen jiragn\'on! Mi senque v'ouèide fé? Vo sade matte! Na, na, na, se pou pa\ldots mé\ldots mé me sento pa bièn\ldots si fran pa bièn\ldots

\StageDir{La poua madama évaèi. Se beutte eunna man si lo queur é s'itaoule si la tsaretta de Marco sensa pamì diye ren.}

\Cimaspeaks Mi madama, na, avèitsade que l'a maque totcha-le\ldots madama?

\Marcospeaks \direct{Eun s'aprotsèn i jiragn\'on} Mi queun l'è lo problème? Senque n'at?

\Cimaspeaks Mi senque n'at? Avèitsa sen que t'a combin-où!

\Marcospeaks Euh pe seutte catro frappe! To si cazeun pe catro frappe?

\Cimaspeaks Mi avèitsa \direct{eun moutrèn la viille itaoulaye} sen que t'a combin-où! T'a tchoué-me la viille! Maladetto!

\Marcospeaks \'E ara?

\Cimaspeaks  \ldots é ara\ldots eh, n'en perdì an vouése ara!

\Marcospeaks Te di?

\Cimaspeaks Ouè! Avèitsa\ldots

\StageDir{Cima baille an cré piat\'o a la viille madama, todzor itaoulaye si la tsaretta, rèida é paralizaye.}

\Cimaspeaks L'è dza diya seutta!

\Marcospeaks Comèn l'è diya?

\Cimaspeaks Ouè\ldots atèn eun momàn\ldots

\StageDir{Cima pouze dou dèi si lo cou de la viille madama pe sentre lo queur \cuore .}

\Cimaspeaks Maladetto l'è eunc\'o finque frèida!

\Marcospeaks \direct{Bièn tracachà \tracachaie} Comèn frèida?\direct{Eun terièn foua de la secotse eun per de \textit{santini}} Mi ad\'on senque me nen fiyo ara de sise? 

\Cimaspeaks Bailla a mé\ldots

\StageDir{Cima pren eun beillette élettoral é tsertse de lo euncastré deun la man de la madama.}

\Cimaspeaks Ouè mi l'è dza diya comme eun bèrio!\direct{A Marco} Bailla-mé-nèn eunc\'o trèi que Camandona can areuvve son todzor a trèi\ldots

\StageDir{Cima pren la man gotse de la viille é la pourte desì l'atra é dimèn euncastre le \textit{santini} pe le dèi de la madama.}

\Cimaspeaks Mi maladetto que diya que l'è!

\Cimaspeaks \direct{Eun prégnèn la tita de la madama} Voualà! Vionda-té tchica\ldots aprì betèn la tsamba tchica pi \textit{sexy} \ldots 

\StageDir{Cima pren la tsamba drèite de la madama é l'encrouije avouì l'atra.}

\Cimaspeaks Ara alèn ià!

\Marcospeaks La tsaretta mé la queutterio inque.

\Cimaspeaks Mi queutta-là sé ouè! Avèitsa que cazeun t'a combin-où!

\Marcospeaks Sen ià!

\Cimaspeaks Ouè, ouè, ià! \direct{Eun avèitsèn pe lo dérì cou la viille} Ah madama! Vardade-v\'o di gamolle!

\StageDir{Teuppe \lemieBa\ si to lo palque.}

\scene[-- Pourta \porta\ a Pourta \porta]

\StageDir{Se avion le lemie \lemieSi. La tèila di rétroproyetteur l'è teriaye ba tanque pe tèra.}

\StageDir{La campagne élettorale l'è fenia. Sen lo 8 de marse deun le studi\'o télévisif de la RAI, caque meneutta douàn de la directe de Pourta a Pourta. Si la drèite l'è plachaye eunna pourta blantse. Eun scène n'a lo notéo Viondapapì que roule tot ajitoù a drèite é a gotse lo lon di palque. Avouì eunna vouése \textit{autoritère}, mi todzor ajitaye, baille le-z-odre pe comenchì la directe. Le tecnisièn plachon doe caèye comodde i mentèn di palque é, a drèite, eunna caèya é eunna tabla avouì eun \textit{ordinateur}.}

\Noteospeaks Tchica pi aoutre seutte caèye!

\StageDir{Dimèn doe garde di \textit{corp} \bodyguard\ entron é se plachon protso de la pourta blantse, eunna a drèite é l'atra a gotse.}

\Noteospeaks \direct{I garde di \textit{corp}} Ad\'on v'ouite preste?

\StageDir{Le doe garde di corp fan ouè avouì la tita, sensa diye eunna paolla.}

\Noteospeaks Bon\ldots alé que alèn eun leugne de sé a pocca\ldots 

\StageDir{\Fv{Eun leugne de sé a catro\ldots trèi\ldots dou\ldots eun\ldots}}

\StageDir{Partèi lo refrèn de:}

\sound{https://www.youtube.com/watch?v=EESHIpo4Lgk&list=OLAK5uy_k5XFAIrq2l8dbbAfgGd8JBzn81wuLVrcw&index=2}{Gone With the Wind | Max Steiner\\ Porta a Porta - Sigla}\label{sigla}

\StageDir{Eunna di garde di corp ivre la pourta. Lo notéo s'aprotse é avouì bièn d'entouziasme fé entré Bruno Aveuille.}

\Noteospeaks Bruno!

\StageDir{Lo notéo eumbrache Bruno é aprì va s'achouaté si sa poltronna. Dimèn Bruno se plache i mentèn di palque.}

\Brunospeaks Bonsouar a tcheutte, mersì, mersì bièn é bienvin-ì a seutta édich\'on ecstraordinéra\ldots \direct{ver la réjì} se pouèn tchoué la mezeucca, soplé!

\StageDir{Lo refrèn s'arite.}

\Brunospeaks \ldots édich\'on ecstraordinéra de Pourta a Pourta. Oueu éléch\'on eumpourtante a  Tsarvensoù: doe fach\'on, sinque é sinque pe coutì que pouràn tramì l'économì de leur péì. Saré eunna souaré bièn eumpourtanta, oueu 9 de marse 2019\ldots é\ldots ah ouè! Vourio rappelé eunna baga eumpourtanta: n'en la télévouése, pouade voté de mèiz\'on é pouade voté eunc\'o vo di pebleuque seuilla prézèn.

\StageDir{Dimèn lo notéo l'è lévo-se é l'è aprotsa-se a Bruno.}

\Brunospeaks \direct{I notéo} Dimanderio i notéo Viondapapì se pou esplequì le modalitoù pe voté di mitcho\ldots

\Noteospeaks L'è seumplo: l'è suffizàn téléfoun-ì i num\'er\'o 088\ldots ah na! 800 3453 é gnaqué lo bot\'on 1 pe Marlène ou 2 pe Jordy.
 
\Brunospeaks Voualà vo remersio. Pouèn fé partì la télévouése!

\StageDir{Lo notéo tourne s'achouaté.}

\effet{https://on.soundcloud.com/ovu3w}{Tchéqueun}[false]

\Brunospeaks \direct{Surprì} Oh! L'an soun-où\ldots Viondapapì alade vère. 

\StageDir{Lo notéo, djeusto achouatoù, se levve. Partèi lo \hyperref[sigla]{refrèn}\footnote{ \textit{Refrèn} a padze \pageref{sigla}.} é eunna di garde di corp ivre la pourta. Entre Marlène tot arbeillaye de rouza. Lo notéo lèi bèije la man. Marlène se plache i mentèn di palque iaou Bruno la salie é la fé achouaté jantilamente si la caèya a gotse.}

\Brunospeaks Ah, madama Dzourio bonsouar! Ad\'on, senque me contade? Comèn l'è alaye la campagne élettorale? 

\Presidanspeaks L'è itaye diya, mi n'en fé eun dzen travaille. Ma fach\'on l'a traillà que l'ie eun plèizì!
Dèyo fran le remersì tcheutte, sourtoù mon cher Paul!

\Brunospeaks Ouè, ouè, mé ad\'on vourio vo dimandì se v'ouèide caque-z-imadze de voutra campagne élettorale\ldots

\Presidanspeaks Ouè.

\StageDir{Marlène teurie foua eun VHS \vhs\ é lo baille a Bruno. Bruno l'avèitse avouì eunna mia de doute.}

\Brunospeaks \direct{I notéo} Si pa se n'en eunc\'o seutte bague\ldots Viondapapì\ldots prouade a vère de betì si lo VHS!

\StageDir{Lo notéo tsertse de fé partì lo réjistrateur, que l'è dézò la tabla.}

\Brunospeaks \direct{A Marlène, ironique} V'ouite bièn avàn avouì le tecnolojì!

\StageDir{Lo notéo se vionde ver la tèila di rétroproyetteur é gnaque eun bot\'on de la télécomanda \telecomando\ pe fé partì la vidéo. La télécommande martse pa.}

\Noteospeaks Torna le-z-alcaline \pile\ que van pa!

\Brunospeaks Vo fa le bléttì, Viondapapì! \direct{I pebleuque} Squezade, l'è la directe\ldots

\StageDir{Lo notéo lètse le pile, le plache amoddo deun la télécommande é gnaque tourna lo bot\'on: finalamente la vidéo partèi.}

\StageDir{Si la tèila son proyettaye sinque épizode de la campagne élettorale de Marlène, que tsertse de soumondre le \textit{santini} de sa fach\'on avouì de tecnique bizarre:
\begin{enumerate}
	\item eun tseun acappe dézò tèra lo beillette élettoral é lo pourte i son patr\'on;
	\item eun gars\'on acappe lo beillette élettoral dedeun eun paneun;
	\item eunna mamma acappe lo beillette élettoral dedeun lo pannoleun di mèinoù;
	\item eun djovinotte acappe lo beillette élettoral dedeun lo distributeur di gan de Paris;
	\item avouì lo \textit{metal detector } eun djouyao di fiolet acappe lo beillette élettoral déz\'o téra.
\end{enumerate}
}

\StartVideo{https://www.facebook.com/thedigourdi/videos/286763348886446}{Lista n$^\circ$ 1 - Pe lo bièn di péì}

\Brunospeaks Complemèn madama Dzourio pe la \textit{vidéo}\ldots

\effet{https://on.soundcloud.com/ovu3w}{Tchéqueun}

\Presidanspeaks La fèi que l'an soun-où!

\Brunospeaks Mmm na me semble pa\ldots

\effet{https://on.soundcloud.com/ovu3w}{Tchéqueun}[false]

\Presidanspeaks Ouè l'an soun-où!

\Brunospeaks Ah ouè!

\Brunospeaks Viondapapì, alade vère.

\StageDir{Lo notéo se levve avouì la fiacca. Partèi lo refrèn.}

\StageDir{Eunna di garde di \textit{corp} ivre la pourta é entre Jordy, arbeillà avouì eunna djacca é eunna craotta dzana. Lo notéo lèi sare la man pa tan convencù. Jordy tsemin-e ver lo mentèn di palque eun salièn lo pebleuque.}

\Brunospeaks \direct{Eun sarèn la man a Jordy} Bonsouar! Achouatade-v\'o sé\ldots

\StageDir{Bruno lèi moutre la caèya eun fase a Marlène. Jordy s'achouatte.}

\Brunospeaks Se mogà lo prochèn cou pouade pa vo eundrèimì si lo tchequeun de la pourta\ldots

\Jordyspeaks \direct{Bièn émochon-où é plen d'énerjì} Squezade Bruno, mi si élettrizoù d'itre seuilla! \direct{Ver lo pebleuque} Tchao mamma!

\Brunospeaks \direct{Eun calmèn Jordy} Euh pe plèizì! Comme n'i demandoù a Marlène Dzourio, vo dimando comèn l'è itaye seutta campagne elettorale.

\Jordyspeaks L'è tot aloù amoddo! La crotta de pappa l'è an mia vouida mi\ldots ah, pouì fé eun remersiemèn? Vourio remersiì la dzeleunna de tanta, que l'a fé-no de-z-où grou pèi \direct{eun moutrèn avouì le man la dimench\'on de l'ou} pe noutro-z-életteur. Salì eunc\'o a té Rosita! Dzen travaille!

\Brunospeaks \direct{C.d.} Va bièn, va bièn! Euh\ldots se v'ouèide prestoù eunc\'o vo eunna vidéo\ldots

\Jordyspeaks \direct{Eun terièn foua eunna plimetta} Ouè, n'i eunna plimetta \usb \ldots

\StageDir{Bruno pren la plimetta de Jordy.}

\Brunospeaks  Ah eunna USB\ldots

\Jordyspeaks Na eunna plimetta\ldots

\Brunospeaks \direct{Eun se dirijèn ver lo notéo} \ldots USB! 

\Jordyspeaks Plimetta!

\Brunospeaks\direct{A Viondapapì} N'en eunna USB?

\Noteospeaks N'en totte!

\StageDir{Viondapapì pren l'USB é l'atatse a l'ordinateur. Si la tèila son proyettoù sinque épizode de la campagne élettorale de Jordy, que tsertse de baillì le \textit{santini} avouì de viamèn euncrouayablo:
\begin{enumerate}
	\item eunna feuille s'aprotse a l'imondicha pe tapì ià eun cart\'on: dèi déz\'o le cart\'on chor foua Jordy pe lèi soumondre eun beillette élettoral;
	\item eunna madama l'è eun tren de lavé le pateun i boueuille: eun chortèn foua de l'éve Jordy  lèi soum\'on eun beillette élettoral;
	\item eun gars\'on s'aprotse i for di pan; can lo ivre troue Jordy, dza preste avouì lo beillette élettoral eun man;
	\item eun djovinotte l'è eun tren de galoppé si pe eun sentchì: eun chortèn foua di tombeun, Jordy lo bloque é lèi baille lo beillette élettoral;
	\item eunna feuille ivre son conjélateur é lèi troue dedeun Jordy avouì eun beillette élettoral eun man\ldots la feuille déside de lo clloure tourna deun lo conjélateur.
\end{enumerate}
}

\StartVideo{https://www.facebook.com/thedigourdi/videos/2286809794900921}{Lista n$^\circ$ 2 - Eunsemblo pe le Digourd\`i}

\StageDir{Plan planotte la tèila se teurie si é, dimèn, véyèn di fon di palque avanchì le doe fach\'on: a gotse salla de Marlène, tcheut arbeillà avouì eun détaille rouza, é a drèite salla de Jordy, arbeillà avouì quetsouza de dzano. Le doe fach\'on se plachon a l'entor de leur candidà.}

\Brunospeaks \direct{A marlène é Jordy dimèn que la tèila veun teriaye si} V'ouèide tcheu dou fé eunna dzenta \textit{vidéo}. \direct{A Jordy} Lo voutro l'è fran dzen\ldots \direct{A Marlène} mi lo voutro l'è bièn pi dzen ah ah ah!

\Presidanspeaks Mersì Bruno!

\Brunospeaks \direct{Eun se viondèn ver lo fon di palque} Voualà, entron eunc\'o le doe fach\'on,\direct{ver lo pebleuque} bouéchèn di man i doe fach\'on que son entraye ara eun trasmich\'on\ldots

\StageDir{Applodissemèn \bouechiman\ .}

\Brunospeaks Ouè prao pèi! Prao pèi! Vo remersio tcheutte, mi mé ara passerio i momàn pi eumpourtàn de la souaré: vourio dimandé a vo dou comèn vo pensade de rezoudre lo problème que n'a pamì de mèinoù, de dzouvin-o que vouillon entrì deun le compagnì de téatre, a la lemie de la crize \textit{démographique} que viquisèn, iao le fameuille fan pamì de mèinoù\ldots comèn pensade de rezoudre? Mé dirio, comme l'è djeusto que fise, que lo premì a prédjì \direct{avèitse Jordy avouì eun fase sourì} l'è\ldots Marlène!

\Presidanspeaks Mersì!

\StageDir{Bruno baille le-z-ipale a Jordy é s'aprotse a Marlène.}

\Presidanspeaks Ad\'on Bruno, mé pe repoundre a la dimanda fério eun pégno pa eun dérì\ldots

\Brunospeaks \ldots fien-nèn dou!

\StageDir{Dimèn que Marlène conte de counte foule, la fach\'on de Jordy se beutte le man pe le pèi ou gante la tita. I contréo, la fach\'on de Marlène, fé seumblèn de comprendre.}

\Presidanspeaks Mé me rappello que can no alloon a l’icoula no sayòn tcheu blan, mouèn que noutro amì Osama, que pe deue l'ie pi Valdotèn de tcheu no! Salle lé sayòn it\'o le-z-àn de la grousa nèi, lo pou fiave pamì de-z-où\ou , le dzeleunne tsantavon a catr'aoue de l'éproù é can alavon a djouì i fiolet n'ayè eunc\'o le coucouye\ldots me restade dérì? Sen que mé vouì diye l'è que no fa le-z-èidjì euntchì leur, piatro l'a pa de sanse: ou te lèi baille le pèiss\'on \pesce\ ou te lèi aprèn a pitchì. Mi pe fenì lo discoù, pe itre clléya, vouì fé an demanda a té Bruno é a vo tcheut que v'ouite seuilla: mi no Valdotèn iaou vouillèn alé?

\StageDir{La fach\'on de Marlène boueuche di man.}

\Brunospeaks \direct{A Marlène, dispéoù} Mondjemé! Cheur eun discoù \textit{vraiment} reutso de sostanse, \direct{ironique} complemèn! \direct{A Jordy} Dirio ara de passé la paolla eunc\'o a vo.

\Jordyspeaks Ouè ad\'on no sen bièn seumplo é concret\ldots

\Brunospeaks Poublisit\'o!

\StageDir{La trasmech\'on s'arite é Bruno tsemie a drèite é a gotse pe attendre la feun de la poublisit\'o. Dimèn, la fach\'on rouza se réunèi pe baillì de coadzo a Marlène é salla dzana tsertse de tranquilizé Jordy que comprèn pa pequé Bruno l'a voulì copé to de chouite son discoù.}

\StageDir{Aprì eun per de seconde Bruno dimande a tcheutte de tourné a leur plasse.}

\Brunospeaks \direct{I doe fach\'on} Euh squezade, squezade, se comenche! Trèi, dou, eun\ldots voualà no passèn la paolla a Bollon.

\Jordyspeaks Ouè dijao\ldots no sen bièn seumplo é concret. Ad\'on, n'a pamì de mèinoù? Fa fé pi de botcha. Fa tsertchì de fé fé i dzi pi de mèinoù\ldots comèn se fan le mèinoù? L'è que n'a salla magniye pe se protédjì: te lo eunfeulle \direct{eun moutrèn avouì la man} é te te divertèi. Ad\'on, avouì mon ésper de jinécolojì Paolo Cima Sander, n'en itedjà la soluch\'on: fa fé eunna pégna, mi fran cré borna, desì eun per de gan de Paris \condom, sensa lo diye a gneun, pouèi can lo eunfeulle baille campa i botcha!

\Brunospeaks \direct{Eunc\'o pi dispéroù de douàn} Mondjemé!

\Jordyspeaks Seumplo mi concret!

\Brunospeaks Ouè, ouè l'è eun dzen squerse mi pa pi eun di pi\ldots

\effet{https://on.soundcloud.com/ovu3w}{Tchéqueun}[false]

\Brunospeaks Qui l'è?

\Noteospeaks Mi n'en trèi fach\'on?

\Brunospeaks Attégnaon pa gneun\ldots

\StageDir{Partèi lo refrain. Eunna garde di corp ivre la pourta é entre eun pourtapizze.}

\Fattorinospeaks Bonsouar a tcheutte! N'i portu-vo le nou pizze \pizza\ \textit{maxi} que v'èi comandoù!

\Noteospeaks Na pa pe no.

\Fattorinospeaks Mi lo num\'er\'o 23 de Rue Festaz sade prao vo!

\Brunospeaks \direct{I pebleuque} Argh squezade, squezade l'a directe! \direct{I garde di corp} Fiade chotre, fiade-l\'o chotre sitte, soplé!

\StageDir{Le doe garde di corp eumbrancon lo  pourtapizze é lo tsachon foua.}

\Brunospeaks \direct{I pebleuque} Squezade, son de bague que capiton can te fé eunna trasmich\'on eun directe\ldots can mimo, mé n'ario voya de dimandé a vo se v'ouèide caque dimande a no fiye, pe noutro dou candidà. N'a caqueun que vou prédjì?

\StageDir{Eun spettateur levve la man.}

 \Brunospeaks \direct{I spettateur} Oué diade maque.

\SpectIspeaks Vouillao savèi iaou l'è lo ben di Splendor\ldots

\Brunospeaks I fon di corridoù a drèite! Euh se n'a de dimande eunna mia pi eumpourtante\ldots
 
\StageDir{Eugn atro spettateur levve la man.}

\SpectIIspeaks Bonsouar a tcheutte! N'ario an baga a dimandé i doe fach\'on: véyo l'è coutaye la campagne elettorale? Pa pe d’atro, mi maque pe savèi, pequé sise son de sou de tcheu no!

\Brunospeaks Ah! Pourie fé lo seunteucco sitte!\footnote{ Lo dzor de la pièse l'ie it\'o fran lo Seunteucco de Tsarvensoù, Borbey Ronny, a pouzé la dimanda.} \direct{I dou candidà} Ad\'on seutta l'è eunna dzenta dimanda, l'è pa comoddo repoundre. 

\StageDir{Bruno pe désidé qui dèi prédjì pe premì gnouye a fé la conta; lo notéo se levve, lèi pren lo bri é lo pounte ver Marlène.}

\Brunospeaks \direct{A Marlène} Totse a vo!

\Presidanspeaks Mersì! Mé si sitte poueun ferio repoundre a mon esper de économì Paul.

\Dallasspeaks \direct{Eunna mia ajitoù} Mersì Madama la Prézidanta, mersì Bruno, mersì i pebleuque pe la dzenta dimanda. Ad\'on, n'en pa totchà le secotse di Tsarvensolèn. N'ayòn pa de sou mi n'en i la chanse d'acapì eugn ommo, sayo, que la bailla-no pi de 800 meulle euro. N'en to spendi-le pe fé la rente di sitouayèn é aumenté le pench\'on di Tsarvensolèn; ara nen n'en pamì mi sen preste pe gagnì le-z-éléch\'on!

\Brunospeaks Euh se pouì vo féo eunna pégna dimanda: qui l'è lo financhateur?

\StageDir{Paolo s'aprotse ver Bruno.}

\Dallasspeaks\direct{Eun prédzèn todzèn i bouigno de Bruno} Si mé lo financhateur\ldots mi se vo baillo \direct{eun terièn foua de sou \cash} 400 euro vo diade ren a gneun?

\StageDir{Bruno pren le sou é se le beutte eun secotse, eun tsertsèn de pa se fé vère.}

\Brunospeaks Va bièn, donque\ldots Marlène vo fio eugn applodissemèn maque mé pequé vo lo metade.

\StageDir{Bruno s'aprotse ver Jordy pe lèi baillì la paolla. Can Jordy l'è preste pe ivrì botse, Bruno se vionde ver lo pebleuque.}

\Brunospeaks Contegnèn noutra trasmech\'on\ldots

\Jordyspeaks Bruno! \direct{Eun levèn la man, eunna mia surprì} Pouì repoundre eunc\'o mé ?

\Brunospeaks  Oh ouè, oubliavo\ldots ouè, ouè diade maque eunc\'o vo la voutra.

\Jordyspeaks  Ad\'on no sen bièn seumplo é concret: n'en pa spendì ren! N'en prèi noutra Ape é sen aloù i mentèn di veladzo pe prédjì avouì lo peuple, avouì le dzi! N'en an mia ézajéroù avouì la fita finalla\ldots euh Marco!

\StageDir{Marco teurie foua eun lon scontreun [SE POU DEU?] é lo baille a Jordy.}

\Jordyspeaks I Limonet n'en spendì 500 euro de Prosecchi, 450 euro de Spritz\ldots 1 euro de Martini Blanc? \direct{Douteu \doute} Qui l'a bi de Martini Blanc? Voualà\ldots mi pe lo reste n'en pa terià foua eun sou!

\Brunospeaks V'ouèi fé viondé l'économì di voutre péì dièn!

\Jordyspeaks Ouè!

\Brunospeaks Ah bièn, penso que sisa eunna dzenta baga so que v'ouèide fé. \direct{Eun tsemièn ver lo pebleuque} Mi mé ara me dilondzerio pa troppe, mogà caqueun di pebleuque\ldots mi caze caze fiyo mé eun discoù.

\StageDir{Lo notéo, eun véyèn Bruno todzor pi protso di palque, galoppe ver lli é l'arite.}

\Noteospeaks Bruno iaou te va? L'è dza saouto-nen ba eun sit an! Attench\'on! Pitoù\ldots fa clloure la télévouése.

\Brunospeaks Ah ouè! Ad\'on: dji\ldots

\Noteospeaks Nou\ldots

\Brunospeaks Ouette\ldots

\Noteospeaks Satte\ldots

\Brunospeaks Chouì\ldots

\Noteospeaks Sinque\ldots

\Brunospeaks Catro\ldots

\Noteospeaks Trèi\ldots

\Brunospeaks Dou\ldots

\Noteospeaks Eun\ldots

\StageDir{Lo notéo tourne a sa plase.}

\Brunospeaks GONG! Voualà, la télévouése l'è fenia é ara ferio voté vo di pebleuque, pequé l'è bièn eumpourtàn. Eun attendèn Viondapapì fiade le voutre contcho: avèitsade qui gagne avouì la télévouese é no ara vièn comèn va avouì lo pebleuque.

\StageDir{Lemie \lemieSi\ si to lo pebleuque.}

\Brunospeaks \direct{Ver lo pebleuque} Ad\'on comencherio avouì le vouése de Bollon Jordy: viyo levvon la man pe Bollon Jordy?

\StageDir{Totta la fach\'on dzana levve la man é eunvite lo pebleuque a voté. Bruno counte rapidamente é a basa vouése le dzi di pebleuque que voton pe Jordy.}

\Brunospeaks Mmm va bièn\ldots Viondapapì, marcade 4!

\Jordyspeaks Comèn catro vouése?

\Noteospeaks \direct{Avouì eun grou \textit{binocolo} \binocolo} T'a counto eunc\'o Gildo é Provino si i sondzon a l'anglle a drèite?

\Brunospeaks Na\ldots Provino vote pe Marlène, marcade 4!

\Brunospeaks \ldots é ara le vouése pe Marlène!

\StageDir{Totta la fach\'on rouza levve la man é eunvite lo pebleuque a voté. Bruno avouì de jeste platéal se beutte a counté an matse de vouése. Aprì tchica s'arite.}

\Brunospeaks  \direct{I notéo} Ad\'on lo Splendor l'a 471 plasse, gavade 4 é betade salle vouése pe Marlène Dzourio!

\Brunospeaks \direct{A Marlène} Ah, ah, le bague son eun tren de alé bièn eh?

\Brunospeaks Va bièn, ara fièn fé le contcho a Viondapapì é\ldots n'en eunc\'o lo caro pe an pégna dimanda.

\StageDir{Lo pourtapizze i mentèn di pebleuque pren la paolla.}

\Fattorinospeaks Bruno! Bruno! Seuilla! Ara veugno si lo palque pouèi me sentade mioù.

\StageDir{Lo pourtapizze pouye si lo palque.}

\Fattorinospeaks Mé si pi aloù vère: lo 25 l’è lo Gourmet, lo 24 l'è lo Fashon Café é lo 23 l’è lo Téatro Splendor de Veulla\ldots é ara me payade seutte pizze!

\Brunospeaks \direct{Ver le garde di corp}
Pe plèizì\ldots

\StageDir{Le garde di corp prègnon de pèise lo pourtapizze é lo pourton foua di palque, eun chortèn a goste.}

\Fattorinospeaks \direct{Malechà \malechaa} Me le dèyade paì seutte pizze!

\Brunospeaks \direct{Eunfastedjà \stouffie} Mi l'è pa poussiblo! Squezade, squezade a tcheutte! Me l'atégnao fran pa\ldots euh Viondapapì sen preste pe le rezultà?

\Noteospeaks  Ouè!

\StageDir{Lo notéo pourte eunna busta a Bruno é aprì tourne a sa plase. Bruno ivre la busta \busta\ pe avèitchì qui la gagnà le-z-éléch\'on.}

\Brunospeaks\ldots{Lentamente} Ad\'on\ldots lo nouo Prézidàn di Digourdì de Tsarvensoù, que resteré eun tsardze pe sinque an, saré Jorrioz Marlène ou Jordy Bollon? Saré\ldots Jor\ldots Jordy Bollon!

\StageDir{Pe eun per de seconde partèi lo refrèn de:}

\sound{https://www.youtube.com/watch?v=04854XqcfCY}{We Are The Champions - Queen}

\StageDir{La fach\'on dzana saoute si pe l'er totta entuziaste. I contréo, salla rouza se dispère. Stchoppon si pe l'er eunc\'o de \textit{coriandoli} \coriandoli . Le dzano s'eumbrachon, tsanton é saouton. Dimèn Bruno, eunna mia triste, se teurie si de moral eun se betèn a fé fita avouì le dzano.}

\Jordyspeaks Mersì, mersì! Paolo, Sophie, Eyvia alade i mentèn di pebleuque é remersiade tcheutte! Marco! Marco! Alèn bèye eun crep!\direct{I pebleuque} Mersì: avouì Jordy!

\ridocliou

\StageDir{Bruno é Viondapapì avanchon ver lo  \textit{proscenium} dimèn que la tèila se cllou.}

\Brunospeaks Notéo Viondapapì, fa diye que me l'atégnao.

\Noteospeaks Mé na!

\Brunospeaks Mé créyao deun si dzouvin-o. Metave ou na?

\Noteospeaks Ouè mioù llou que salla lé!

\Brunospeaks Mi ouè djaque! Marlène aprì l'è dzouvin-a é l'a lo ten de crèitre.

\Noteospeaks Mi aprì prèdze troppe, ite pa quèya!

\Brunospeaks Can mimo si contèn pequé a la feun l'an gagnà le dzouvin-o é le dzouvin-o que l'an voya de fé quetsouza l'è beun dzen ou na?

\Noteospeaks  Aprì si de Renzo n'en vi-lo crèitre!\direct{Eun moutrèn l'atchaou d'eun botcha.} L'ie pégno pouai.

\Brunospeaks No diyon de clloure!

\Noteospeaks  Te cllou té o cllouzo mé?

\Brunospeaks  Cllouzèn eunsemblo: ad\'on\direct{i pebleuque} vo remersio tcheutte, no véyèn tcheut lo prochèn desando a 21 aoue seuilla si RAI\ldots

\Noteospeaks \ldots 1!

\Brunospeaks  \ldots é pa si RAI\ldots
 
\Noteospeaks \ldots 2!
 
\Brunospeaks \ldots avouì eunna édich\'on de Pourta a Pourta noualla! Mersì é \textit{au revoir}!
 
\Noteospeaks Salì a tcheutte!
 
 \StageDir{Bruno é Viondapapì chorton de scène a gotse.}

\scene[-- Mi dez\`o le lenchoueu\ldots]

\ridoiver

\StageDir{Itaoulaye desì eunna coutse \lettodoppio\ avouì de lenchoueu rodzo, n'a Marlène avouì eunna londze sigalla eun botse é eun \textit{foulard} blan ator di cou.}

\Presidanspeaks \direct{Avouì eun ton satisfé \tipoconocchiali} L'è itaye eunna dzornoù londze é l'an gagnà sise di Simpson. Jordy l’è lo nouo Prézidàn. Tcheu le cou que n’a de-z-éléch\'on n’a qui gagne é qui per. Mi can mimo ma fach\'on l’a fé eun dzen traille. L'è-tì pa pai, pégna carottin-a \carota de mé?

\StageDir{\Fv{Mi ouè} (L'è eunna vouése de eugn ommo, mi se comprèn pa amoddo de qui).}

\Presidanspeaks Ah ouè, dèyo fran remersì mon Paul! Llou ouè que me lame: lèi fou eun petchoù sourì é fé totte sen que n'i voya. Eunc\'o Jo\"{e}lle, le dou eunsemblo l'an pourto-me an matse de vouése. L’è-tì pa pai pégno pedzeun \pulcino\ de mé?

\StageDir{\Fv{Ouè madama} }

\Presidanspeaks Eunc\'o pe le Digourdì, comme pe la poleteucca, can n'a de crize tsizon le majoranse é fa alé a noue-z-éléch\'on\ldots é finque si cou l’è itoù lo peuple que l’a desidoù! No pouèn pa lèi fé ren\ldots mi can mimo seuilla i mitcho si todzor mé que pourto le pantal\'on, l'è-tì pa pai ma pégna verdzassa \scoiattolo ?

\StageDir{\Fv{Ah ouè madama} Marlène se levve, tchoué pe tèra la londze sigalla é s'aprotse i pebleuque.}

\Presidanspeaks Vo fa savèi que la réjì l’è itaye de mé: avouì Jo\"{e}lle, Richard, Fabien é \direct{eun avèitsèn a gotse di palque} mon tan sayo Paul, 
n'en betoù si eun plan formidablo\ldots é i pouer di Digourdì lèi si todzor mé! L'è-tì vrèi ma dzenta itèila \stella ?

\StageDir{Jordy, eun galopèn, entre eun scène avouì dou vèyo de Champagne \spumante\ pe le man. Le pantal\'on son eunc\'o avortoillà a l'entor de eunna tsamba. L'è caze peillotte; l'a maque eun per de ganes\'on bordeaux é eunna craotta dzana \craottadzana\ llataye a la tita. Can areuvve douàn a Marlène se beutte a dzouiill\'on é, to sourièn \felice, lèi soum\'on eun vèyo.}

\Jordyspeaks Madame la Prézidante!

\Presidanspeaks \direct{Eun prégnèn lo vèyo} Mersì!

\StageDir{Avouì supériorité, Marlène caye ià lo vèyo que Jordy l'a eun man.}

\Presidanspeaks  Alade a la coutse!

\Jordyspeaks \direct{An mia épouvant\'o é élettrizoù} Ouè madama la Prézidanta!

\StageDir{Jordy se gave ià le pantal\'on eunc\'o avortoillà i pià é eun fricotèn se itaoule si la coutse. Dimèn Marlène, ver lo pebleuque, bèi a qui blan lo vèyo é lo tappe pe tèra.}

\Presidanspeaks La SÈI DE POUVOER!

\Jordyspeaks Ouè madama la Prézidanta\ldots vo rappelade senque v'ouèi prometti-me?

\Presidanspeaks Qui l'a deutte de prédjì?

\StageDir{Jordy bèiche la crita é avouì le bouigno ba se teurie todzor pi eun dérì, dimèn que Marlène s'aprotse.}

\Presidanspeaks Te fa ité quèi \quei! T'a comprèi? \direct{Eun pouchèn Jordy countre la tita de la coutse} Seuilla désido to mé!

\StageDir{Jordy l'è gnacoù contre la tita de la coutse é Marlène, todzor pi protso, lèi pren la man é la llatte a la coutse avouì eunna menotta \manette .}

\Jordyspeaks \direct{Eun fricotèn tot ecsitoù \eccitato} Oh mondjeu di paadì! Madama\ldots

\StageDir{Marlène pase dérì la coutse, se plache si l'atro coutì de la coutse, pren l'atra man de Jordy é la llatte comme l'a fé douàn.}

\Presidanspeaks V'ouite tcheu cobbla! Vo féo eun pégno sourì é fiade to sen que n'i voya! V'ouite tcheu foundì pe mé.

\Jordyspeaks Madama la Prézidanta vèyo que lamade le djouà POUER!

\StageDir{Marlène, eun silanse, fé tourna lo tor de la coutse dimèn que Jordy, llatoù, tsertse de se baoudjì.}

\Presidanspeaks \direct{Eun tsemièn ver lo pebleuque} Lo dzen di POUER l'è que désido mé can comenchon le danse!

\StageDir{Marlène baille le-z-ipale i pebleuque é se plache i pià de la coutse, douàn Jordy que l'è élettrizoù. Partèi la tsans\'on:}

\sound{https://www.youtube.com/watch?v=pzKGqcCUR9M}{Walkin' $\&$ Strippin' | Sonny Lester Orchestra}

\StageDir{Marlène comenche a se dizarbeillì: douàn lo foulard blan, aprì lo maillontcheun rouza\ldots Jordy avouì totte sé forse tsertse de eumbranquì Marlène, mi, comme eun tseun llatoù que vèi arrevé  lo piqué, l'è obledjà a attendre lo volèi de Madama la Prézidanta, que s'aprotse to todzèn. Marlène ara l'a maque eun cré coteill\'on que la toppe. Le rid\'o dimèn se cllouzon é la Prézidanta baille le-z-ipale i pebleuque: l'è contchentraye si Jordy. Can la tèila l'è caze cllouta, Marlène, finalamente, se gave ià lo coteill\'on.}

\Jordyspeaks Ahouuu! \inamourou\ \inamourou\ \inamourou

\ridocliou

\DeriLeRido

\RoleNoms{Vidéo}{Jo\"{e}l Viérin}

\RoleNoms{Tecnisièn vidéo é son}{Stéphanie Albaney, Margot Jorrioz}

\RoleNoms{Lemie}{Roger Comé}

\RoleNoms{Joueur}{Marcello Giometto}

\RoleNoms{Camion}{Louis Bollon}

\end{drama}