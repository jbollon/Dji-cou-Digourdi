\title{TODZO PI DIGOURDÌ}
\author{Pièse icrita pe Le Digourdì}
\date{Téatro Splendor de Veulla, 10 marse 2018}

\maketitle

\fotocopertina{Foto/2018/gruppo.jpg}{Sophie Comé, Ronny Borbey, Francesca Lucianaz, Paolo Cima Sander, Julie Squinabol, Jo\"{e}l Albaney, Michel Comé, Marco Ducly, Stephanie Albaney, Thierry Jorrioz, Simone Roveyaz, Aimé squinabol, Marlène Jorrioz, Paolo Dall'Ara, Richard Cunéaz}{Evi Garbolino, Jo\"{e}lle Bollon, Pierre Savioz, Jordy Bollon, Laurent Chuc, André Comé}{2018}

\LinkPiese{Avanspettaclle, Todzo pi Digourdì}{https://www.youtube.com/watch?v=IaFisR_YuYo&list=PLBofM-NS_eLJUln45l7VH457fGak_Bk5O&index=4, https://www.youtube.com/watch?v=MiAzDgBoOTY}{.45}

\souvenir{La pièse \og Todzo pi Digourdì \fg l'è itaye icrita eugn'occaj\'on di dji-z-àn de la Compagnì. Lo sujé di teste l'è la \textit{biographie ironique} di Digourdì.
Personellamente \og Todzo pi Digourdì\fg resterè todzor deun mon queur, pe bièn de rèizón.

L'è itaye la dériye pièse comme Prézidàn di Digourdì; l'è eunna pièse avouì laquella eugn  itrandjì pou comprendre amod\-do sen que vou deue itre Digourdì, di momàn que lo \textit{fil rouge} de la counta l'è l'esprì comique de totta la compagnì, avouì voya de riye pe fée riye; l'è itoù lo premì cou iaou la Compagnì l'a prooù a tsandjì \textit{style} téatral: \textit{scénographie} redouite i \textit{minimum}, \textit{scène} reprézentaye deun difièn conteste \textit{espace-temps}\footnote{ Eun premì tentatif l'ie itoù fé avouì la pièse \og Matte\ldots sen tcheut matte\fg{} deun lo 2013.}, \textit{mimétisme} é surtoù bièn de \textit{métathéâtre}.}{Jordy Bollon}
\queriaouzitou{
\begin{itemize}
\item[$\bullet$] La grafie di titre de la pièse l'è pa correcte:  \og Todzo\fg{} se devriye icriye \og Todzor\fg{}. Dièn pa lo non di/de la responsablo/a de seutta erreur pe pa provoqué de discuch\'on inutile. Can mimo, va la pèin-a de soulignì pequé n’en desidoù de vardé lo titre trompoù. Pe doe rèiz\'on: premì, pe no rappelé de fé pi attenchón avouì la grafiye di teste que no publièn; secón, di momàn que la pièse counte de fasón \textit{autoironique} sen que vou deu itre Digourdì, eun titre avouì eunna petchouda erreur l'ie bièn reprézentatif di sujé de la pièse mima.\\

\item[$\bullet$] Deun la scène X - La Nite di-z-Oscar, noutro Paolo Cima Sander repette, deun eunna meneutta é demì, nou cou \og can mimo\fg{} é chouì cou \og eumpourtàn\fg{}!

\refstepcounter{videos}
\begin{figure}[H]
\vspace*{-5pt}
\centering
\begin{subfigure}{.75\textwidth}
\centering
\video\hspace*{0.5mm} \textsc{\small Can mimo vs Eumpourtan}\hspace*{0.5mm} \video\\\vspace*{1mm}
    \qrcode[hyperlink, height=0.5in]{https://www.youtube.com/watch?v=tCJzWhbbfCQ}
\end{subfigure}%
\addcontentsline{vds}{section}{Canmimo vs Eumpourtàn}
\end{figure}

\item[$\bullet$] Deun l'avanspettaclle, le dou prézentateur son réel\-la\-men\-te lo Seunteucco é lo Vise Seunteucco de la Quemeua de Tsarvensoù. 
\end{itemize}
}

\Scenographie
\begin{itemize}
\item[$\bullet$] 1 poltronna moderna;
\item[$\bullet$] 1 chofà;
\item[$\bullet$] 1 peuccapoussa \aspirapolvere\ \textit{automatique};
\item[$\bullet$] 1 \textit{hoverboard};
\item[$\bullet$] 1 \textit{album} di foto;
\item[$\bullet$] 1 ban de eunna cantin-a iaou aprésté lo bèye pe le clliàn;
\item[$\bullet$] 2 pégne table, comme salle di cantin-e ;
\item[$\bullet$] 15 caèye que se pouon pléì, comodde pe itre tramaye eun pocca ten;
\item[$\bullet$] 1 volàn di poulmeun;
\item[$\bullet$] 1 \textit{pupitre};
\item[$\bullet$] 1 busta;
\item[$\bullet$]1 Oscar.
\end{itemize}

\setlength{\lengthchar}{3.5cm}

\Character[LAURENT]{LAURENT}{Laurent}{Premì prézentateur, lo Vise Seunteucco de Tsarvensoù, \name{Laurent Chuc}}

\Character[RONNY]{RONNY}{Ronny}{Sec\'on prézentateur, lo Seunteucco de Tsarvensoù, \name{Ronny Borbey}}

\Character[MAGÀN]{MAGÀN}{Magan}{Magàn Sophie \viille, sa eumpléì amoddo lo téléfonne é la tecnolojì, \nameF{Sophie Comé}}

\Character[PAGÀN]{PAGÀN}{Pagan}{Pagàn Paul \viou, euncó llou bièn tecnolojique, \name{Paolo Dall'Ara}}

\Character[NEVAOU]{NEVAOU}{Nevaou}{Nevaou de pagàn Paul é magàn Sophie, \name{Aimé Squinabol}}

\Character[NEVAOUZA]{NEVAOUZA}{Nevaousa}{Nevaouza de pagàn Paul é magàn Sophie, \nameF{Julie Squinabol}}

\Character[SERVENTA]{SERVENTA}{Serventa}{Serventa d'eunna cantin-a de Tsarvensoù,  \nameF{Stéphanie Albaney}}

\Character[FRANCESCA]{FRANCESCA}{Francesca}{La vrèya Francesca Lucianaz di Chef-Lieu, \nameF{Marlène Jorrioz}}

\Character[PIERRE]{PIERRE}{Pierre}{Lo vrèi Pierre Savioz di Tsatì, \name{Pierre Savioz}}

\Character[JO\"{E}LLE]{JO\"{E}LLE}{Joelle}{La vrèya Jo\"{e}lle Bollon di Chef-Lieu, \nameF{Jo\"{e}lle Bollon}}

\Character[JO\"{E}L]{JO\"{E}L}{Joel}{Lo vrèi Jo\"{e}l Albaney d'Ampaillan, \name{Jo\"{e}l Albaney}}

\Character[VÉTCHOT]{VÉTCHOT}{Vetchot}{Eun vétchot \viou\ di péì, tchica boutro, \name{Richard Cunéaz}}

\Character[CIMA]{CIMA}{Cima}{Lo vrèi Paolo Cima Sander de Feleunna, \name{Paolo Cima Sander}}

\Character[MARCO]{MARCO}{Marco}{Lo vrèi Marco Ducly de Feleunna, \name{Marco Ducly}}

\Character[SIMONE]{SIMONE}{Simone}{Lo vrèi Simone Roveyaz di Pon-Sià, \name{Simone Roveyaz}}

\Character[ÉLEVEUR]{ÉLEVEUR}{Spectateur}{\'Eleveur tsarvensolèn, \name{Thierry Jorrioz}}

\Character[MAMMA]{MAMMA}{Mamma}{Eunna mamma d'eun atteur di Digourdì, \nameF{Stéphanie Albaney}}

\Character[TSACHAOU I]{TSACHAOU I}{TsachaouI}{Eun tsachaou de Tsarvensoù, \name{Jordy Bollon}}

\Character[TSACHAOU II]{TSACHAOU II}{TsachaouII}{Eungn atro tsachaou de Tsarvensoù, \name{André Comé}}

\Character[FOTOGRAFE]{FOTOGRAFE}{Fotografe}{Vioù fotografe di Printemps Thé\^atral, an mia trambelìn, \name{Thierry Jorrioz}}

\Character[CHAUFFEUR]{CHAUFFEUR}{Chauffeur}{\textit{Chauffeur} di poulmeun pe la chortia annuella di Digourdì, \name{André Comé}}

\Character[JORDY]{JORDY}{Jordy}{Lo vrèi Jordy Bollon di Chef-Lieu, \name{Jordy Bollon}}

\Character[THIERRY]{THIERRY}{Thierry}{Lo vrèi Thierry Jorrioz di Chef-Lieu, \name{Thierry Jorrioz}}

\Character[STEPHANIE]{STEPHANIE}{Stephanie}{La vrèya Stephanie Albaney d'Ampaillan, \nameF{Stéphanie Albaney}}

\Character[CONDUCTEUR]{CONDUCTEUR}{Conducteur}{Conducteur télévizif de la nite di-z-Oscar, \name{Richard Cunéaz}}

\Character[ASSISTANTA]{ASSISTANTA}{Valletta}{Assistanta di conducteur télévizif, \nameF{Stéphanie Albaney}}

\Character[]{LE DOU NEVAOU}{Ledou}{\quad}

\Character[]{TCHEUTTE}{Tcheutte}{\quad}

\DramPer

\act[\avanSpect\ Avanspettaclle   \avanSpect]

\StageDir{\hspace*{2.5em}Lemie \lemieSi.}

\StageDir{\hspace*{2.5em}Laurent l'è si \textit{proscenium} que avèitse lo pebleuque.}

\begin{drama}

\Laurentspeaks Bonsouar é bienvenù a tcheutte a seutta souaré di Printemps Thé$\hat{a}$tral avouì le Digourdì de Tsarvensoù. Me semble ieur can no sen acapó lo premì cou comme compagnì é n'en désidoù de beté si le Digourdì. Dèi si dzor lé n'en fé tan de souaré é de chortie, todzor eun vardèn lo mimo esprì, la mima voya de riye é de fé riye. Donque, eugn occajón de noutro anniverséo, n’ayòn désidó de pourté inque si lo palque eun momàn bièn tipique, é sourtoù bièn téatral, de sen que l'ie la viya valdoténa: lo Consèille Réjonal. Mi aprì no no sen deu: mi le dzi veugnon sé pe riye ou pe plaoué? Pe riye! É ad\'on no no sen deu: na, na, na! L'è vrèi que lo téatro é la poleteucca son bièn gropoù; bièn de cou te comprèn pa qui fé la poleteucca é qui fé lo téatro. Mi i mimo ten, euncó no Digourdì pouèn pa tan cretequì la poleteucca; euncó no n'en noutro pégno euntsardzo eun poleteucca. La Joueunte comunalla de Tsarvensoù l'è compozaye pe cattro Digourdì si sinque! N'en la majoranse! \'E l'unique que l'è pa eun Digourdì, lo seul que l'a jamì resitó desì si palque (é de so dèi se baillì lagne)\ldots l'è noutro Seunteucco! Seunteucco, que comme tcheu le vrèi politisièn, l'è vin-ì eun salle de téatro, l'a saroù la man a tcheutte, l'è fé-se vére, l'a fé de grou sourì\ldots é ara iaou l'è aloù? L'è aloù ià! Mi l'a pa comprèi que sise que l'an resitoù devàn sayòn sise de Brutchón! Pa de Tsarvensoù! Mi se pou beté Seunteucco eun que prèdze gnenca noutro patoué? Mé ara dimando a tcheu vo\ldots mi se pou voté eun Seunteucco que\ldots

\StageDir{Ronny Borbey, Seunteucco de Tsarvensoù, entre eun \textit{scène}.}

\Ronnyspeaks Laurent! Avèitsa que\ldots se te vou conten-ì a fé lo Vise Seunteucco tanque lo 2020\ldots fé attench\'on a sen que te di!

\Laurentspeaks Mi té de iaou t’i entroù?

\Ronnyspeaks N'a pa d'eumpourtanse de iaou si entroù! Te semble lo case  de diye de bague di janre?

\Laurentspeaks T'a rèiz\'on\ldots

\Ronnyspeaks Ouè que n'i rèiz\'on!

\Laurentspeaks Mi ara que sanse l'a deusqueté douàn a totte seutte personne?

\Ronnyspeaks Na l'a pa de sanse. Surtoù pequé fa planté-la lé de deusqueté comme eun Consèille réjonal. Se comprèn jamì ren: eun chor de la majoranse, l'atro entre deun la majoranse, aprì l'atro euncoa chor de la minoranse, si de douàn entre eun minoranse\ldots se comprèn pamì ren! Mi prao, fa avèitchì i futur, djeusto?

\Laurentspeaks Ouè! Mé diyo: pequé profitèn pa de totte seutte personne pe comenché a icriye lo programme di prochèn-e-z-éléchón?

\Ronnyspeaks Djeusto! Mi te diyo eungn atra baga: fièn comme fan le Grilleun!

\Laurentspeaks Ah ouè le sinque-z-itèile!

\Ronnyspeaks Djeusto! Leu senque fan? Demandon i dzi se son d’acor, beutton eun votach\'on, fan le \textit{parlamentarie}. Comme lo PD que fé le \textit{primarie}.

\Laurentspeaks \ldots é ad\'on no senque fièn?

\Ronnyspeaks Fièn le tsarvensarie! Demandèn i dzi se son d’acor.

\Laurentspeaks Va bièn, comenchèn!

\Ronnyspeaks Té pensa a doe bague: i spor é a l'imondicha\ldots

\Laurentspeaks Ah l'imondicha! N'i dza deu-lo eun Consèille, a la majoranse é a la minoranse que l'imondicha l'è pa eunna compétanse de la Quemeua, mi de l'Unité!

\Ronnyspeaks \ldots é ad\'on pensa i tourisme. Va bièn?

\Laurentspeaks Ouè; é té pensa a la queulteua é i travó poubleucco.

\Ronnyspeaks Va bièn. Ad\'on atacca té!

\Laurentspeaks Atacco\ldots ad\'on\ldots spor\ldots n'i eunna idé dèi can sayò ate pouèi \direct{avouì la man moutre l'atchaou}, dièn dèi can sayò pégno: eunna grousa manifestach\'on, bièn eumpourtanta\ldots

\Ronnyspeaks \ldots pa la Becca!

\Laurentspeaks Na bièn pi grou! Vouillo pourté a Tsarvensoù le Jeux Olympiques!

\Ronnyspeaks A Tsarvensoù? Mi se pou pa! N'en lo caro pe totte le streutteue?

\Laurentspeaks Ouè! Fé-me esplequì! N'en tot: ba pe la plan-a, protso a la grandze, protso lo mitcho de Franco Lucianaz fièn eunna grousa streutteua pe saouté avouì le-z-esquì; aprì fièn lo restoràn iaou betèn traillì tcheu sise de Valpettaz!

\Ronnyspeaks Ah pouèi areuvvon le vouése de Valpettaz!

\Laurentspeaks Ouè,  30 ou 40 vouése! Aprì lo fondo\ldots lo fièn si a Tsan-Plan; betèn eunna grousa cabouetta pe le beillette, eunna dzenta cantin-a iaou fièn traillì sise di Combes é di Tsatì!

\Ronnyspeaks \ldots é lé d'atre vouése!

\Laurentspeaks Son 200 vouése! Aprì lo bob: lo fièn partì si i rascar de Combatechiye\ldots

\Ronnyspeaks Mi gneun reste si pe de lé! Queunte vouése prégnèn?

\Laurentspeaks Ouè mi la feun de la pista di bob la fièn devàn lo mitcho de Diego Bollon! Countèn Diego countèn le Bollon, countèn le Bollon countèn Sen-Sal\'o !

\Ronnyspeaks Le Bollon son tan é no voton tcheutte!

\Laurentspeaks Mi pe gagnì le-z-éléch\'on fa euncó avèitchì la partia basa. Lé fièn eunna grousa \textit{patinoire} é la fièn jéré a Franco Lombardo!

\Ronnyspeaks Lé son 300 vouése! Sen a poste! 

\Laurentspeaks Garantì,  300 vouése!

\Ronnyspeaks Me plé!

\Laurentspeaks Mé avouì lo spor si a poste. Ara té pensa a la queulteua.

\Ronnyspeaks Acouta-mé: no n'en cherdì le Digourdì pe fé lo \textit{court-métrage} si la Reconstituch\'on de la Quemeua; le DVD l'an i eun gran sucsé! Ad\'on senque fièn? Pourtèn a Tsarvensoù lo festival di \textit{cinéma} de Cannes é de Venize! Imajina-té le journal:\fg Festival di \textit{cinéma} de Tsarvensoù\og{}!

\Laurentspeaks Wow!

\Ronnyspeaks Imajina le \textit{star} que se fan le \textit{selfie} si la promenade!

\Laurentspeaks Ah ah ah! Mi iaou sarie noutra promenade?

\Ronnyspeaks Mi n'en euncó no la promenade: lo tsemeun de Veulla! Imajina-té le \textit{star} que van si é ba pe lo tsemeun de Veulla!

\Laurentspeaks Si é ba, ba é si!

\Ronnyspeaks Se fan le \textit{selfie} é aprì fièn finque la directe avouì Barbara d'Urso, avouì lo programme \og Iproù 5\fg{}!

\Laurentspeaks Pa mal!

\Ronnyspeaks Pouèi acapèn euncó le vouése di fenne que steurion!

\Laurentspeaks Ouè, djeusto, salle son de tsaplette.

\Ronnyspeaks Ara té pensa i tourisme\ldots

\Laurentspeaks Tourisme\ldots sel\'on mé fa fé eunna téléférique que partèi de Plase Chanoux, areuvve si i Chef-Lieu, aprì pase euncó protso sen di Borbey, conteneuvve tanque si i val\'on de Combouì é a la feun s'arite si i refuje Arbolle.

\Ronnyspeaks Pa mal!

\Laurentspeaks Ouè, pouèi le portèn tcheutte si eugn Arbolle, bèyon lo cafì pi bon i mondo (lo Cafì Ollietti), mi surtoù\ldots payon! Le fièn paì la tasse de \textit{soggiorno}!

\Ronnyspeaks Pequé sise no voton pa, pequé iton pa a Tsarvensoù.

\Laurentspeaks Ouè, no voton pa mi payon! Baillon de sou a la Quemeua, la Quemeua baille de sou i sitouayèn, le sitouayèn son contèn é le sitouayèn contèn no voton!

\Ronnyspeaks Sen a poste.

\Laurentspeaks Ouè! Mi pe gagnì fa todzor prendre eun considérach\'on eunna baga: le trav\'o poubleucco. Si so, queutto la paolla a té.

\Ronnyspeaks Donque fé-me pensì\ldots n'i pensoù. Ad\'on té te cougnì amoddo la Becca, mi te feré pi eugn atra manifestach\'on: Veulla - Tsarvensoù - Mont-Emilius, pouèi te va euncó pi si!

\Laurentspeaks Na si pa d'acor!

\Ronnyspeaks Atèn que t'euspleucco. Te sa senque fièn de la poueunte de la Becca?  La copèn!

\Laurentspeaks Naaa! Comèn copé la Becca?

\Ronnyspeaks Ouè!

\Laurentspeaks \direct{Dispéoù} Na! Copé la Becca! Na votade pa! Na!

\Ronnyspeaks Acoutta! Copèn la Becca pequé pouèi portèn lo solèi a Roulaz é prégnèn totte le vouése de Roulaz; no vote finque Anselmino que la jamì votou-no! Portèn aprì lo solèi a Feleunna! Counta véo de vouése a Feleunna!

\Laurentspeaks Mondjemé! Eunna sentèin-a!

\Ronnyspeaks \ldots é pourtèn finque lo solèi a Plan-Feleunna! Counta le vouése\ldots

\Laurentspeaks N'i pa praou de dèi! Va bièn, si caze d'acor; mi la Becca iaou la betèn?

\Ronnyspeaks N'i pensoù euncó a so: lo pro de la grandze de Plan-Feleunna. Atsetèn lo pro (le tsanéno l'an fata de sou pe se paì la \textit{badante}), plachèn la Becca é fièn Charvland, lo Gardaland de Tsarvensoù! L'è pa eunna dzenta idoù?

\Laurentspeaks L'è eunna idoù euncrouayabla! A Charvland lèi betèn a traillì sise de la Giradaz, sise d'Ampaillan, sise di Pont-Suaz. Aprì totte le-z-attivitoù di Pon travaillon de pi é se travaillon no voton!

\Ronnyspeaks Eh ouè!

\Laurentspeaks Mé diyo que sel\'on mé n'a la caèya pe 30 an!

\Ronnyspeaks Cheur\ldots mi n'en deu devàn que fa demandé i dzi se son d'acor: fa fé le Tsarvensarie.

\Laurentspeaks Mi n'a pa fata avouì si programme!

\Ronnyspeaks Fien-leu pequé no sen comme le Grilleun démocratique\ldots é propozèn eun programme fattiblo!

\Laurentspeaks No sen de Valdotèn, sen pa sé pe counté de counte foule! Sen que dièn lo fièn!

\Ronnyspeaks Se pou fiye! Ad\'on mé beutto eun votach\'on: countréo?

\StageDir{Ronny é Laurent avèitson lo poubleucco.}

\Laurentspeaks Me semble gneun\ldots té avèitsa si, que mé avèitso ba.

\Ronnyspeaks Gneun! Amoddo! Ara demando: favorable?

\StageDir{Ronny é Laurent avèitson tourna lo poubleucco.}

\Laurentspeaks Avèitsa qui la votou-no! La votou-no finque Siro Viérin! Mi demàn nèi!

\Ronnyspeaks N'en l'unanimitoù. Saren-no la man pe la fotografie avouì le journaliste\ldots

\StageDir{Ronny é Laurent se saron la man é sourion pe se fé fé eunna foto. Aprì chorton.}

\Laurentspeaks Ah na!

\StageDir{Ronny é Laurent tornon i mentèn di palque.}

\Ronnyspeaks Na sen oublia-no di Digourdì!

\Laurentspeaks Donque vo quetèn avouì lo spettaclle di Digourdì icrì eun occaj\'on di djijimo anniverséo de la compagnì! A vo\ldots

\Ronnyspeaks Todzor pi\ldots

\Tcheuttespeaks \ldots Digourdì!

\act[Acte I]

\ridoiver

\scene[-- L'\textit{album} di Digourdì]

\StageDir{Lemie \lemieSi\ a drèite di palque, iaou acapèn eun chofà é magàn protso d'eunna poltronna.\\ \Fv{Tsarvensoù, 4 janvieu 2050}
}

\Maganspeaks \direct{Eun se achouatèn si la poltronna} Oh\ldots si belle lagnaye! Ara m'achouatto eun momàn. Ah na! Me fa euncó poulité lé pe tèra. Ah! Mi n'i la soluch\'on! Lola!

\StageDir{Eun peuccapoussa \textit{automatique} chor foua di chofà é poulite sel\'on sen que comande Magàn.}

\Maganspeaks Oh saye! Voualà, poulita an mia pi a drèite\ldots pi a gotse\ldots Stop! Ara tourna i teun poste, a coutcha!

\StageDir{Lo peuccapoussa se bloque.}

\Maganspeaks Seutte baradziye, martson pa. Le beutto a poste mé.

\StageDir{Magàn se levve é catse lo peuccapoussa dézò lo chofà. Aprì s'achouatte tourna si la poltronna é teurie foua lo portable.}

\Maganspeaks\direct{Eun lièn} Véyèn sen que conte-tì lo moundo oueu\ldots

\StageDir{Entre pagàn avouì l’\textit{hoverboard}.}

\Paganspeaks Sophie! Si arevoù!

\Maganspeaks L'ie l’aoua bon!

\Paganspeaks \direct{Eun se boudzèn avouì l’\textit{hoverboard}} N'i fé eunna dzenta promin-ada oueu\ldots avouì la dzenta mezeucca de l’Orage n'i fé belle dou \textit{kilomètre}!

\Maganspeaks \ldots é t'i nienca tsizì! T'i fran eun gamba!

\StageDir{Pagàn fé dou tor si lo poste é aprì bèiche ba de l’\textit{hoverboard}.}

\Paganspeaks \`Eitsa que mé si eungn esper de seutte machine! Pa comme té, todzor dérì a si portablo é a Facebook é iPhone!

\Maganspeaks \`Eitsa que mé oueu n'i dza to fé, n'i dza to poulit\'o !

\StageDir{Pagàn s'achouatte si lo chofà.}

\Paganspeaks Ouè, poulit\'o avouì salle machine que t'a pe lo mitcho\ldots que nen si mé sen que te combeun-e pe lo mitcho! \direct{Eun terièn foua lo portablo} Ara fé-me lie lo journal.

\Paganspeaks \`Eitsa Sophie! Amélie Viérin noua Prézidante de la Réj\'on\ldots eh, Sophie, dza lo pappa Laurent l'ayè fé lo Prézidàn 30 an fé!

\Maganspeaks \ldots é lo pappagràn l'ayè fé-lo dza i seun ten!

\Paganspeaks Ouè, que dzen veure comèn le mitchì se tramandon, l'è dzen so!

\Maganspeaks T'a fran rèiz\'on!

\StageDir{Le dou nevaou entron eun galopèn avouì eun man eun grou album di fotografie. S'achouatton protso de pagàn.}

\Ledouspeaks Pagàn, pagàn! Senque l'è si grou livro?

\Nevaousaspeaks L'è icrì: \og Le Digourdì! \fg{}

\Paganspeaks Aimé, Julie! Mi iaou v'ouèide acapoù so?  

\StageDir{Pagàn pren lo livro eun man é lèi souffle desì pe lèi gavé ià la poussa\footnote{ Commé éffé spésial n'en betoù bièn de \textit{borotalco} deun l'\textit{album}, pe fé vère la poussa can pagàn lèi soufflè desì.}.}

\Paganspeaks\direct{Eton-où \ouaou} Mi sit\ldots \direct{avèitse Magàn} té t'a catcha-lo!

\Maganspeaks Na!

\Paganspeaks Si l'è lo livro avouì totte le foto di Digourdì! La pi renoumaye compagnì de téatro de la Val d’Ousta!

\Nevaouspeaks Mi Digourdì qui?!

\Paganspeaks Deh Aimé! Pourta de reuspé pe Le Digourdì! Le Digourdì l'an port\'o lo nom de Tsarvensoù ià pe to lo moundo!

\Nevaousaspeaks Ah ouè? Comèn l'arian fé? Ara éziston-tì eunc\'o ? Can son nèisì?

\Paganspeaks\direct{Bièn fier \tipoconocchiali} Ad\'on\ldots vegnade seu\ldots

\StageDir{Pagàn fé de caro si lo chofà i dou nevaou. Le nevaou s'achouatton.}

\Paganspeaks \ldots ara vo fio la conta\ldots l'ie l'an 2000 é\ldots ouette é\ldots

\StageDir{Teuppe \lemieBa\ si la drèite di palque. Lemie \lemieSi\ a gotse é i mentèn di palque.}

\scene[-- La nésanse di Digourdì]

\StageDir{Eun \textit{scène} n'at eun banc\'on é doe pégne table avouì eun per de caèye. N'a la serventa que poulite lo banc\'on é eun vétchot achouatoù a gotse que li lo journal pe son contcho.\\ Entron deun la cantin-a  Jo\"{e}l, Jo\"{e}lle, Pierre é Francesca.}

\Francescaspeaks\direct{A la serventa} Bondzor! T'a eunna plase pe cattro?

\Serventaspeaks Ouè, salla tabla \direct{eun moutrèn la tabla a drèite} va-tì amoddo?

\Francescaspeaks Ouè.

\StageDir{Le cattro-z-amì s'achouatton.}

\Serventaspeaks Sade dza senque bèye?

\Pierrespeaks Féo mé pe tcheutte, fièn cattro bire!

\Francescaspeaks  Na, pe mé te me fé eun ju de frouì i marteun sèque!

\Joellespeaks \ldots é pe mé i-z-ambrocalle!

\Joelspeaks\direct{Eun rièn} I marteun sèque, i-z-ambrocalle! Desandro bèyavade pa tan de ju de frouì\ldots ou mioù: ambrocalle é marteun sèque ouè, mi l'ian dedeun l'éve de viya!

\StageDir{Pierre ri eunsemblo a Jo\"{e}l.}

\Francescaspeaks Euh Jo\"{e}l Albaney! Fa-tì te rapélé que n'i portó té é llou \direct{eun moutrèn Pierre} i mitcho? \'E pe terì si sitta tanque si la Basteuille n'en finque rechà Viro!

\Vetchotspeaks Ouè! Mé alao si i fèye é sise se retèriaon!

\Pierrespeaks\direct{I vétchot} Conteneuvva a lie lo journal! Euncó té can t'ie dzoueun-o t'a fé de fite!

\Vetchotspeaks Penso beun! Mi l'an jamì pourto-me i mitcho le femalle! Todzor arrevó i mitcho avouì l’ape; é pe to diye, i ten de mé le femalle servichaoun pe d’atro!

\Joelspeaks Mi soplé! Li lo journal!

\StageDir{La serventa pourte lo bèye pe le-z-amì.}

\Serventaspeaks Ah ouè \ldots me fiade fran riye: acouté vo me semble de vère lo Charaban!

\StageDir{Silanse. Le cattro-z-amì penson.}

\Joellespeaks Ah lo Charaban\ldots sarie pa mal beté si eunna compagnì de téatro a Tsarvensoù!

\Joelspeaks Pierre! Pourian beté si eunna compagnì de téatro!
 
\Pierrespeaks Pequé pa? Pa mal comme idoù. 

\Joelspeaks\direct{Todzor a Pierre} N'arian djeusto fata de fenne!

\Pierrespeaks Eh ouè sen maque mé é té!

\StageDir{Le dou penson eun momàn dimèn que Jo\"{e}lle é Marlène comenchon a se amalechì pe pa itre considéraye.}

\Pierrespeaks\direct{A la serventa} Squeuza! Té te prèdze beun tchica patoué\ldots

\Serventaspeaks  Pa pi tan\ldots

\Joelspeaks Na mi te lo prèdze bièn! Amoddo!

\Pierrespeaks \ldots é sen a trèi! 

\Joelspeaks Mi n'a pa praou!

\Pierrespeaks\direct{I vetchot} A propoù, té que le femalle te le-z-eumpléyae pe d'atro\ldots

\Vetchotspeaks  Senque?

\Pierrespeaks\direct{I vetchot} Té que t'i plen de femalle, te sarie no deu se n'a de fenne que pouon itre euntéressaye a fé téatro?

\Vetchotspeaks  Argh! Que counte foule, n'i pa lo ten, n'i d’atro pe la tita. Aprì vo senque vouillade fiye? V'ouite cattro rabadàn nèisì ieur, senque nen sade vo de téatro? Soplé!
 
\Joelspeaks  Rappella-té que no dzouin-o fièn sen que n'en voya se n'en la pach\'on! \direct{A Pierre} Mé é té sen prest, n'arian fran voya de comenchì!

\Francescaspeaks Can mimo v'ouite fran pa seumpateucco! Lèi sen mé é Jo\"{e}lle! 

\Pierrespeaks  Mi ouè, té é Jo\"{e}lle! Djeusto, Jo\"{e}lle! Pe case te cougnì pa caque feuille que l'a voya de fé téatro?
 
\Joellespeaks \direct{Malechaye \malechaa} Mi fé-té feun! Mé é Francesca pouèn pa resité avouì vo?
  
\Joelspeaks Say\'on eun tren de squersé!

\Pierrespeaks L'ie pe riye.

\Francescaspeaks\direct{Ironique} Que riye!
\Joellespeaks\direct{Offenchaye} Fran seumpateucco! 

\Pierrespeaks Ad\'on fiade-mé fiye le contcho. \direct{Counte le-z-atteur  de la compagnì} Eun, dou, trèi, cattro é sinque avouì la serventa\ldots pe fé eunna compagnì no manque euncó caqueun.

\Joelspeaks Sen pocca pe ara\ldots

\Pierrespeaks  Ah! L'è vin-i-me eunna idoù! Le premì trèi que entron eun cantin-a le térièn dedeun!

\Joelspeaks Na Pierre n'i pouiye! Avèitsa que reusquèn!

\Pierrespeaks Mi ouè! Sen que capite, capite! A Tsarvensoù sen tcheut amoddo!

\Joelspeaks Na, na, na! \direct{Ver le feuille} Vo v'ouite chiye?

\Joellespeaks Si pa, prouèn.

\Pierrespeaks\direct{A tcheutte} Vo fiade de mé?

\Francescaspeaks Renque pe si cou.

\Joelspeaks Prouèn!

\StageDir{Entre Cima. Le cattro-z-amì se beutton le man pe le pèi.}

\Cimaspeaks \direct{Ver la serventa avouì pach\'on} Bonsouar! Tot amoddo? Te me fé eun dzén-epì tsa? Mi tsa! Maque comme t'i boun-a té a lo fiye \malisieu .

\Serventaspeaks Ouè va bièn.

\Francescaspeaks  Ouè mi sitte l'è de Feleunna é sa gnenca lo patoué!

\Cimaspeaks Feleunna? Feleunna! Mi té can te di Feleunna te di Paolo Cima Sander! Mi gars\'on, chourtade eunna mia de sise \textit{schemi}! Chortade de Tsarvensoù, bèichade ba de salla montagne, ivrade le joueu, végnade avouì no que n’en lo solèi to l’an!

\Joellespeaks\direct{Eun rièn} Na, na\ldots sitte l'è tro rabadàn, l'è eun de no!

\Cimaspeaks Rabadàn? Jo\"{e}llina, mé que n'i pasoù totta la viya a te prédjì, mi reusta quèya\ldots can mimo\ldots

\Pierrespeaks\direct{I seun-z-amì} Vo dimando squiza mi si cheur que le prochèn dou\ldots

\Joelspeaks Na Pierre! Mé n'i tourna  pouiye!

\StageDir{Entron Simone é Marco eun rièn é squersèn. Pierre l'è dispéoù.}

\Joelspeaks\direct{Eun rièn pe pa plaoué} Ah ébeun! L'è pa pi que t'i aloù eun meillorèn!

\Marcospeaks\direct{A Simone} \ldots é desando nite iaou t'i sparì?

\Simonespeaks Ouè Marco, n’ayoù mioù a fiye que itì avouì vo!

\Marcospeaks  Ouè va bièn mi t'a quetou-me a pià a l’Inside \malecha !

\Simonespeaks Veullanoua-Tsarvensoù a pià l'è pa que te lèi beutte tan\ldots

\Marcospeaks Ouè doe meneutte!

\Serventaspeaks Senque béyade?

\Simonespeaks Beutta maque doe bire.

\Pierrespeaks\direct{Ver lo banc\'on} Marco, Simone! Vo laméria fé téatro avouì no?

\Marcospeaks Acoutta mi de feuille nen v'ouèide?

\Joelspeaks Plen pèi!

\Simonespeaks Eunna baga eumpourtanta\ldots de chortie foua Val d'Ousta nen fièn?

\Pierrespeaks Ouè cheur! Tcheu le mèis!

\Marcospeaks Pe mé se pou fiye!

\Pierrespeaks Ad\'on gars\'on, vegnade tcheu seuilla!

\StageDir{Tcheutte se plachon a l'entor de la tabla avouì lo vèyo levoù.}

\Joelspeaks Fièn eun santé a la noua Compagnì de Tsarvensoù!

\Tcheuttespeaks Santé!

\StageDir{Teuppe \lemieBa\ a gotse é i mentèn di palque.}

\StageDir{Lemie \lemieSi\ a drèite di palque.}

\scene[-- Pe to lo moundo \moundo!]

\Paganspeaks Véo d'émoch\'on; 4son dza belle pasoù 40 an!

\Nevaouspeaks\direct{Eunna mia tro savèn} Seutta l'ie pa eunna compagnì de téatre! V'ouyavade eunna compagnì de la crotta!

\Paganspeaks Grama lenva que t'a té!  Pourta reuspé pe le Digourdì! Totsa-meu pa le Digourdì!

\Nevaousaspeaks Squeza-mé pagàn, poui-dze te dimandé eunna baga? Péqué lo nom Digourdì? 

\Paganspeaks\direct{Fier de seutta dimanda} Ah!  Seutta l'è eunna dzenta counta: ad\'on, eun dzor\ldots \direct{comenche a borboté, vague} dou de no son partì ià pe eun mariadzo; aprì can son tournoù i mitcho l'ie tar é l'ay\'on fata de pantchì d'éve\ldots se son betoù protso d'eun meur, mi l'è chortia la mamma de eun de sise dou\ldots la mamma l'a vouaillà : \og Deh! Me raccomando\ldots fiade pa tro le Digourdì! \fg{} . Voualà! De si momàn la Compagnì l'ayè son nom!

\Nevaouspeaks Ouè mi pagàn, se sitte l'è lo nom, vouillo pa imajin-ì le pièse que v'ouèide fé!

\Paganspeaks Aimé! Pourta de reuspé pe le Digourdì é rappella-té que le Digourdì l'an pourtoù lo nom de Tsarvensoù ià pe to lo moundo!

\StageDir{Pagàn se levve.}

\Paganspeaks Té pensa que can n'en fé la premiye pièse,  n’ayè plen de dzi, n’ayè de machine tanque i mentèn de Veulla, n'ayè finque de pompì volontéo de Cogne é Perloz, n’ayè 600 volontéo de totta la Val d'Ousta que l'an édja-no pe fé la premiye! Tourno diye: la premiye!

\Ledouspeaks \direct{Sourprì \ouaou} Pe la premiye 600 volontéo? 

\Maganspeaks\direct{Eun se levèn} Mi senque te counte? \direct{Eun prégnèn l’album é eun s'achouatèn} Acoutade mèinoù, de sen que counte pagàn coppade la mèitchà, vardade eun car é tsapotade euncó eunna mia! Vegnade seuilla  que vo counto mé\ldots

\StageDir{Teuppe \lemieBa\ a drèite di palque.}

\StageDir{Lemie \lemieSi\ a gotse é i mentèn di palque.}

\scene[-- Catro tsatte \gatto\ \gatto\ \gatto\ \gatto]

\StageDir{Si lo palque n'a 15 caèye plachaye si trèi feulle. Eun vétchot é eun éleveur bièn annouiyà son achouatoù si la secounda feulla.}

\StageDir{Eunna mamma entre to de coursa.}

\Mammaspeaks\direct{I vétchot, avouì lo flo queur} Squizade-mé! L'è-tì dza comenchà? 

\Spectateurspeaks Comme tcheu le demicro ataque a ouette é demì!

\Vetchotspeaks  N'a pa de prisa!

\Mammaspeaks Amoddo! Mé ara dèyo vardé eun per de plase pe de paèn.

\StageDir{La mamma comenche a se dizarbeillì pe vardé la plase i seun paèn. Se gave lo palt\'o, eunna maille, eunna botta é le plache si le caèye.}

\Mammaspeaks So pe tanta Filoméne, la siaou, lo botcha\ldots é l'onclle Gene! \direct{Eun gavèn ià eunna caèya é eun se gavèn eunna botta} Pe llou  que areuvve avouì la caèya avouì le raoue queutto eunna botta! 

\StageDir{La mamma finalemàn s'achouatte si la premì  feulla, totta émochon-aye. I contréo, lo vétchot é l'allévateur avèitson torse la mamma.\\ Silanse.}

\Mammaspeaks  Bon mancon seun meneutte! Spéèn que arrevisan tcheutte!

\StageDir{Silanse. Entron dou tsachaou. Eun de leur seuble eunna tsans\'on.}

\TsachaouIspeaks\direct{Eun quetèn  de seblé} N'i la fèi que n'en trompoù poste! \direct{I vétchot} Squezade, l'è-tì seuilla la réuni\'on pe la tsasse? 

\Vetchotspeaks Véyade-tì pa que l'è la réuni\'on di Comité di Bataille?

\TsachaouIIspeaks Mi v'ouèide robo-no la plase? Oueu totse a no, vo v'ouyavade devàn ieur, oueu totse a no! Lo devendro totse a no!

\Vetchotspeaks Na, na, na.

\Spectateurspeaks \ldots é oueu que dzor l'et?

\TsachaouIIspeaks Oueu l’è devendro.

\Vetchotspeaks Oh la fèi n'i pa verià lo calandrì!

\TsachaouIIspeaks T’ou veure que n'en trompoù tcheu dou é sen vin-ì de desandro!

\Mammaspeaks\direct{I tsachaou} Sssht! Te vèi pa que sen a téatro!

\TsachaouIIspeaks Téatro? A Tsarvensoù? Sarè 20 an que n'a pamì lo téatro a Tsarvensoù!

\Mammaspeaks Ouè mi seutta l'è totta eunna noua Compagnì! Le dzoueunno de ara l'an désidoù de beté torna si lo téatro a Tsarvensoù.

\StageDir{Le tsachaou, lo vétchot é l'allévateur son sensa paolle.}

\Mammaspeaks Lèi crèyo pa que sade ren de to sen!

\TsachaouIspeaks Na.

\Mammaspeaks L'an reumplì lo veladzo de manifeste, de volanteun, ià pe totta la parotse!

\TsachaouIspeaks Pa vi ren.

\Mammaspeaks L'an finque betoù-le si le-z-annonse di mor!

\TsachaouIspeaks\direct{\'Eton-où , ver l'atro tsachaou} Ah t'a comprèi ara qui l'ie si Digourdì! Mé pensò que l'ie eun rabadàn que l'ie mancoù.

\TsachaouIIspeaks Mi son ià de tita! Mé n'i vi icrì ouette é demì di nite é n'i pensoù: \og mi sarè-tì pa seutta l'aoua de fé le sépolteue! \fg .

\TsachaouIspeaks Ad\'on l'è nèisia eunna noua Compagnì\ldots ad\'on avèitsen-la!

\Vetchotspeaks Na, na, na, mé vou belle que i mitcho.

\TsachaouIspeaks\direct{Eugn aritèn lo vétchot} Mi na fièn eunna mia de prézanse, n'a gneun!

\StageDir{Le dou tsachaou s'achouatton si la trèijima feulla.\\ Bèichon eunna mia le lemie.}

\Mammaspeaks Oh comenchon!

\StageDir{La mamma, le tsachaou, lo vétchot é l'allévateur boueuchon di man \bouechiman .}

\StageDir{Silanse. Tcheut avèitson lo spettaclle.}

\Vetchotspeaks\direct{A l'allévateur} Mi senque l'è-tì salla beurta baga lé?

\StageDir{Lo vétchot moutre avouì lo dèi eunna personna douàn llou.}

\Spectateurspeaks Que rotobala!

\Vetchotspeaks Se comprèn pa se l'è eugn ommo ou eunna fenna!

\Spectateurspeaks Na.

\Vetchotspeaks Eunna baga pai\ldots clloure i mitcho é templì ià la cllo!

\Mammaspeaks\direct{Inervaye, ver le dou que son eun tren de prédjì} Deh! \`Eitsade que v'ouite eun tren de prédjì de la feuille de mé! Itade quèi! Grame lenve!

\StageDir{Silanse. Lo secón tsachaou comenche a tossèi for, caze s'apeutre. Son compagnón lèi baille de patèle si l'itseun-a, mi servèison a ren.}

\Spectateurspeaks\direct{Eun tapèn eun bombón i tsachaou} Tchappa eun bombón di prée é quèi!

\TsachaouIIspeaks Mersì!

\StageDir{Lo tsachaou meudze lo bombón é se calme. Silanse.}

\StageDir{Lo premì tsachaou tchoppe a riye bièn for. Tcheu le-z-atre riyon pa é avèitson mal lo tsachaou que molle pa de riye. Silanse.}

\Vetchotspeaks\direct{Eun sarèn le joueu} Mi! \direct{Ver lo spettateur} L'è-tì pa de Feleunna si lé?

\Spectateurspeaks\direct{Eun sarèn le joueu} Orco can ouè!

\Vetchotspeaks\direct{Digout\'o} San-tì pa que son tcheu de-z-itrandjì?

\Spectateurspeaks L'an fran recoilla-le tcheutte!

\Vetchotspeaks Se san dza l'italièn l'è dza eun méacllo! Prétègnon de vin-ì inque prédjì patoué!

\TsachaouIIspeaks Sssht! Mal polì que v'ouite pa d'atro!

\StageDir{Silanse. Tcheu stchoppon a riye mouèn que lo premì tsachaou, que se sen eunna mia foua plase. Silanse.}

\TsachaouIIspeaks \direct{A l'atro tsachaou} Salla lé i caro\ldots

\TsachaouIspeaks Sssht prèdza plan.

\TsachaouIIspeaks \ldots l'è-tì pa la feuille de Diego? La blonda aoutre lé!

\TsachaouIspeaks Ouè, ouè, l'è lleu\ldots que couése!\ok

\StageDir{Silanse. Lo vétchot s'eundrime. Se avion le lemie \lemieSi .}

\Mammaspeaks\direct{Eun se levèn é eun bouéchèn le man} L'è fenì! \direct{Ver l'allévateur} Oh l'è fenì!

\StageDir{L'allévateur l'è eunna mia dizorientoù, mi can mimo se levve é eun bouéchèn di man rèche lo vétchot.}

\Spectateurspeaks\direct{I tsachaou} L'è fenì! 

\StageDir{Lo vétchot se rèche. Le tsachaou se levvon é tcheut eunsemblo boueuchon di man pe fé compagnì a la mamma, totta euntuziasta de la pièse.}

\StageDir{Teuppe \lemieBa\ a gotse é i mentèn di palque.}

\StageDir{Lemie \lemieSi\ a drèite di palque.}

\scene[-- Fran de-z-atteur proféssionel!]

\Nevaousaspeaks Magàn, n'ayè fran eun pebleuque mal polì\ldots

\Maganspeaks Ouè pebleuque! Se te vou queri-lo pebleuque! N'ayè catro tsatte!

\Paganspeaks Mi senque te nen sa té? Sophie tourna aoutre a moungouì avouì salla baga lé, lo portable, Facebook, Amazon\ldots

\StageDir{Pagàn gave ià di man l'album a magàn. Magàn se levve é tourne s'achouatté si la poltronna. Pagàn s'achouatte si le chofà, todzor i mentèn di dou nevaou.}

\Maganspeaks \`Eitsa, pitoù d'acouté salle counte foule de té, ito pi belle achouataye inque tranquila.

\Paganspeaks\direct{I nevaou} Magàn l'è an mia viille é ad\'on se rappelle pa tan comèn son alaye le bague. Aprì sitte l'ie maque lo comenchemèn! Alèn eun devàn\ldots

\StageDir{Pagàn vionde le padze de l'album.}

\Nevaouspeaks\direct{Eun moutrèn eunna fotografie}  Seuilla? Senque fiavade?

\Paganspeaks Seuilla? Sé l'ie lo premì Printemps Thé\^atral, i téatro Giacosa de Veulla! Que sucsé! Que bouéchì di man! N'ayoon fé fran eunna dzenta pièse.

\Nevaousaspeaks Wow! Vo douàn d'entré si lo palque comèn vo aprèstavade? Senque fiavade douàn d'entrì?

\Paganspeaks\direct{Eunna mia surprì de seutta dimanda} No sayon de professioniste! Tsaqueun s'aprestave, prouave sa partiya, aprì n'ayè de \textit{coiffeur}, caqueun que fiè an mia de méditach\'on pe\ldots

\Maganspeaks \ldots senque t'i eun tren de countì? Todzor de counte foule. Mi soplè! Ara vo counto mé  sen que capitave dérì le rid\'o .

\StageDir{Teuppe \lemieBa\ a drèite di palque.}

\StageDir{Lemie \lemieSi\ a gotse é i mentèn di palque.}

\scene[-- Dérì le rid\'o]

\StageDir{No no trouèn dérì le rid\'o di Téatro Giacosa. Deun seutta scène tcheu le-z-atteur fan semblàn que lo rid\'o di fon di palque sisa lo rid\'o rodzo di téatro, de fas\'on que lo pebleuque sisa dérì le rid\'o eunsemblo i-z-atteur mimo.\\ Tcheu l'an lo tecste de la pièse eun man, caqueun l'è ajitoù, d'atre relassoù é caqueun d'atro finque tro relassoù!\\ Eun scène n'a Cima achouat\'o si eunna caèya.}

\Cimaspeaks \direct{Eun se levèn, bièn ajitoù, eun troulèn comme eun pedzeun} Na, na me rappello pa la battiya, pouì pa! Ad\'on l'ie seumpla: \og Mé te diyo qu'areuvvo de Feleunna\ldots caco, peucho, avèitso la leunna''\ldots na, na, na, si tro ajitoù!

\StageDir{Entre Pierre arbeillà comme eun prie, eun béyèn eun ju de frouite. Cima s'achouatte.}

\Pierrespeaks\direct{Bièn relassoù} Paolo! Senque t'a? Te me semble preste a partì eun guèra!

\Cimaspeaks Na, na, na, si tro ajitoù!

\Pierrespeaks Ita tranquilo.

\Cimaspeaks\direct{Eun tremblèn} Na, na, na, si tro ajitoù!

\Pierrespeaks Te la sa la partia?

\Cimaspeaks Na! N'i eunna battiya seumpla: \og Mé dze si de Feleunna\ldots \fg voualà n'i oublia-la.

\StageDir{Entron Marco é Jo\"{e}l. Eunna mia pi\'on \pion, se teugnon si eun avouì l'atro.}

\Joelspeaks\direct{Eun tsemièn torse avouì Marco} Marco! Tourna fiye salla de douàn que te fiyè bièn.

\Marcospeaks\direct{Eun se medzèn le paolle} Queunta? Ah ouè, \direct{eun tsantèn} \og eun desì meulle lèi l'a fi!\fg.

\Joelspeaks Mi brao! Mi t'i meilloroù! Ara te boueucho lo ten\ldots

\StageDir{Jo\"{e}l se baille de patèle si la tsamba pe bouéchì lo ten. Paolo é Pierre avèiston la scène dispéroù.}

\Marcospeaks\direct{Eun tsantèn (caze) a ten avouì Jo\"{e}l} \og Eun desì meulle lèi l'a fi!\fg .

\Joelspeaks Mi brao!

\Marcospeaks Mersì, mi ara fièn eunna baga? Prouèn la pièse.

\Joelspeaks Ouè!

\Cimaspeaks Mi senque prouèn?! Viade pa comèn v'ouite combinoù?

\Joelspeaks\direct{Eun se medzèn le paolle} Senque n'at? N'a quetsouza que va pa?

\Cimaspeaks Te vèi pa comèn ti combinoù?

\Joelspeaks\direct{Comme douàn} Pequé té t'i combinoù mioù de mé?

\Cimaspeaks Quetèn pédre\ldots \direct{eun braillèn} n'en lo spettaclle seutta nite!

\Joelspeaks Bon sen beun a poste; sen aloù bèye eunna goloù é sen arrevoù lo pi vito poussiblo é sen arevoù ara\ldots é ara prouèn\ldots

\Cimaspeaks Mi v'ouite deun eun stat! Iaou v'ouite aloù? Di Moldave?

\Marcospeaks Na, sen aloù i Café du Téatro, que l'è pi protso!

\Cimaspeaks Quetèn pédre. Na, na, na!

\Joelspeaks Te comprègno pa. Tcheu le cou t'i ajitoù. Lé de foua \direct{moutre lo rid\'o di fon} lèi seràn catro tsatte.

\Cimaspeaks Catro tsatte?! Mi té t'i ià de tita! T'aré bi eunna \textit{cisterna} de rodzo!

\Marcospeaks\direct{A Jo\"{e}l} Va vère véo de dzi n'at.

\Joelspeaks\direct{Eun s'aprotsèn i rid\'o di fon} N'aré catro dzi\ldots

\StageDir{Jo\"{e}l se beutte a dziill\'on é avèitse déz\'o lo rid\'o .}

\Joelspeaks\direct{Eun rièn} Mondjeu! Mondjeu! L'è plen de dzi!

\Cimaspeaks\direct{Malechà} Mi t'i té plen!

\Joelspeaks N'i jamì vi tan de dzi pouèi! Gnenca a la fèira de Sent-Or.

\Cimaspeaks Oh mondjeu, na, na, na!

\Marcospeaks\direct{A Cima} Ita tranquilo! Acoutta, fièn an baga\ldots vou mé vère véo de dzi n'at\ldots llou \direct{moutre Jo\"{e}l} l'è normal que l'a vi-nèn an matse: vèi doblo!

\StageDir{Marco s'aprotse i rid\'o di fon é, can beutte la tita i mentèn di rid\'o, Jo\"{e}l lo pouche. Marco fenèi foua di rid\'o , mi to de chouite tourne dedeun, iao acappe Jo\"{e}l eun tren de riye.}

\Marcospeaks\direct{Malechà} Mi t'i mar\'on! Mi queunte fegueue te me fi fiye?!

\Pierrespeaks Sssht!

\Joelspeaks Mi se squerse!

\StageDir{Entre Francesca.}

\Francescaspeaks Ad\'on, mi senque l'è to si vacarno? Vo siade panco tsandjà? Senque v'ouite eun tren de fiye? Dji meneutte é comenchèn! I galoppe, vito vo tsandjì!

\Marcospeaks Ouè Francesca, eunna maille é eun per de pantal\'on é sen preste.

\Joelspeaks Eunna mailletta é si preste!

\StageDir{Marco, Pierre é Jo\"{e}l chorton.}

\Francescaspeaks \ldots é tcheu le-z-atre iaou son?

\Cimaspeaks Mi que nen si. Oh na, na, lèi la fièn pa! Na, na, na!

\Francescaspeaks Oh Cima! Calma té\ldots ouè\ldots sit l'è \textit{andait}! 

\StageDir{Entre Simone eun saoutèn é avouì eun vazet de \textit{zuccherini} alcolique.}

\Simonespeaks\direct{To contèn} Ouélla Franci! Agouta euncó té eun \textit{zuccherino}!

\StageDir{Simone soum\'on eun \textit{zuccherino} a Francesca.}

\Cimaspeaks\direct{Todzor pi ajitoù} Ouè senque l'è oueu? L'è fita?

\Francescaspeaks Euh Simone Roveyaz! Soplé, dji meneutte é comenchèn!

\Simonespeaks\direct{Eun moutrèn Cima} \ldots é sitte? T'a pouiye eh?

\StageDir{Simone, comme se Cima fuche eun petchoù tseun, lèi tappe i vol de \textit{zuccherini}. Cima tsertse de le medjì i vol.}

\StageDir{Entre Jo\"{e}lle eun galopèn, an mia inervaye.}

\Joellespeaks Sssht! Oh mi prédzade plan! Se sen totte lé dérì! \direct{A Francesca}  A propoù, Jo\"{e}l iaou l'è? Dièn proué eun per de bague.

\Francescaspeaks N'i spedi-lo se tsandjì eunsemblo i-z-atre é l'a deu-me \og eunna mailletta é si preste\fg \ldots panco vi-lo!

\Joellespeaks Oh mondjeu!

\Cimaspeaks\direct{Eun tremblèn todzor si la caèya} Na, na, na, la spountèn pa si cou!

\Francescaspeaks Sen stra eun retar!

\Joellespeaks Lo si\ldots é aprì fa co fé la foto comme souvenir de la première i Giacosa!

\Francescaspeaks Oue la ferèn se n'en lo ten!

 \Joellespeaks Pouèn euncó pa la fiye! Pe comèn pouriye chotre seutta foto: sitta \direct{eun moutrèn Cima} que tremble comme eun pedzeun, l’atro lé \direct{moutre Simone} que areuvve pa a resté rito, l'atro que molle pa de bèye!
 
 \Francescaspeaks Si pa sen que te diye. Euncó mé me fa lie lo boc\'on de mé\ldots
 
\StageDir{Entre Jo\"{e}l.}

\Joelspeaks\direct{Eun fièn eunna pirouette pe se moutré} Me voualà preste pe lo \textit{referendum}!

\Joellespeaks Ouè mi Jo\"{e}l, mi comèn t'i arbeillà?

\Joelspeaks Pequé? Pe la pièse di \textit{referendum}.

\Joellespeaks T'i arbeilla-te pe la pièse de l'an passoù! Seutta l'è la noua pièse, l’opetaillo moderno.

\Joelspeaks Isto! La noua pièse! L'opetaillo moderno.

\Joellespeaks Rècha-té va! Soplé, galoppa te betì eun pijama ou quetsouza que semblise a eun pijama!

\Joelspeaks Mi so semble eun pijama!

\StageDir{Jo\"{e}l comenche a chotre di palque.}

\Francescaspeaks\direct{Ver Jo\"{e}l} Na, Jo\"{e}l iaou t'i eun tren d'alé? Ara restèn tcheutte seu que de sé a dji meneutte no fa fé la fotografie. 

\StageDir{Entre Pierre.}

\Francescaspeaks Boneur que l'è arrevoù euncó Pierre. \direct{Ver Pierre} A propoù Pierre Savioz: t'a pensoù té i souffleur?

\Pierrespeaks Ouè le joueur. Aprì areuvvon Marco é Simon.

\Francescaspeaks Na le souffleur!
 
\Pierrespeaks Ah le souffleur. Mi Francesca! Sel\'on té no n'en fata di souffleur? Mi na, n'i pa crià gneun, no n'en pa fata!
 
\Cimaspeaks Senque? N'en pa le souffleur, voualà l'è feniya. T'ayè eunna baga a fiye, eunna! \direct{Tsache ià Pierre} Mi va ià soplé! \direct{Ver Jo\"{e}lle} Euh Giada, na euh Esterina, na Jasmine\ldots

\Joellespeaks Jo\"{e}lle.

\Cimaspeaks Soplé va querì Pepe é Paoletta, dimanda-lèi se pouon fé le souffleur vi que leur cougnisson la pièse.

\Joellespeaks Ouè, ouè mi té achoutta-té é calma-té\ldots \direct{Ver tcheutte} mé vou tsertchì Pepe é Paoletta é tcheu le-z-atre pe la foto\ldots \direct{Ver Simone} é té Simone planta-la-lé de tsemin-ì é de rempleure le dzi de seucro que aprì san pa iaou van!

\StageDir{Jo\"{e}lle chor é entre Richard, lo fotografe avouì eunna grousa machina di foto de la Kodak.}

\Fotografespeaks Voué gars\'on, v'ouite preste pe la fotografie?

\Joelspeaks Oh l'è arrevoù Richard pe la fotografie! Veun maque.

\Joellespeaks \direct{Dèi foua scène} Arrevèn euncó no!

\StageDir{Avouì Jo\"{e}lle entron d'atre atteur de la pièse: Francesca Lucianaz, Giada Grivon, Jasmine Comé, Laurent Chuc, Ester Bollon, Ilaria Linty. Tcheutte se beutton eun pozech\'on: se plachon eun demisercllo eun baillèn le-z-ipale i pebleuque. Richard l'et i fon di palque.}

\Fotografespeaks Le pi ate protso i pi base.

\Joelspeaks Senque vou diye?

\Fotografespeaks I meun trèi diade fromadzo! Eun, dou\ldots trèi!

\Tcheuttespeaks \direct{Eugn eurlèn} Fromadzo!

\Fotografespeaks Na l'a pa martchà! Fa refiye.

\Cimaspeaks Mi eun pi feun pouavade pa lo acapé?

\Fotografespeaks I meun trèi: eun, dou\ldots trèi!

\Tcheuttespeaks \direct{Eugn eurlèn} Fromadzo!

\Fotografespeaks A poste!

\StageDir{Jo\"{e}l accompagne foua Richard. Dimèn tcheu le-z-atre se beutton eun demisercllo ver lo pebleuque é s'apreston pe lo motte final.}

\Joelspeaks Ad\'on Richard mersì de totte é a la prochène.

\StageDir{Richard chor é Jo\"{e}l rejouèn le-z-atre.}

\Joelspeaks Alé Gars\'on, ara si reprèi-me eunna mietta! Mondjeu! V'ouite preste? I meun trèi, noutre motte\ldots eun, dou\ldots

\StageDir{Dimèn entre eun scène Richard é se tappe i mentèn di demisercllo.}

\Fotografespeaks\direct{Eun braillèn} Fromadzo!

\Tcheuttespeaks \direct{Bièn malechà} Mi va ià! Feulla ià!

\StageDir{Richard veun tchachà foua. Lo groupe tourne se betì eun demisercllo, tcheutte avouì eunna man pouzaye desì salla de Jo\"{e}l.}

\Joelspeaks Eun, dou, trèi\ldots

\Tcheuttespeaks\direct{Eun braillèn} Todzor pi Digourdì!

\StageDir{Tcheutte chorton a gotse, mouèn que Jo\"{e}l que chor ver lo fon, dérì lo rido, pe baillì lo bienvenù i pebleuque é pe prézenté la pièse l'\og Opetaillo Moderno\fg{}.}

\StageDir{Teuppe \lemieBa\ a gotse é i mentèn di palque.}

\Joelspeaks Bonsoir a tcheutte. Oueu n'en la Compagnì di Digourdì de Tsarvensoù que pe lo premì cou deun l'istouére pouye si lo palque di Giacosa. Vo queutto donque a la pièse icrita pe leur \og L'opetaillo moderno\fg{}!

\StageDir{ Lemie \lemieSi\ a drèite.}

\scene[-- Plen de sou!]

\Maganspeaks Voualà: sen l'è sen que l'è capitoù dérì le rid\'o di premì Printemps!

\Nevaouspeaks\direct{Ver pagàn} Ouè, v'ouyavade cheur pa de esper de téatro!

\Maganspeaks Na! Digourdì pa pe ren!

\Paganspeaks\direct{Ver magàn}  Mi té te dèi todzor betì lo bèque i mentèn! Acoutta, va aoutre a lavé le-z-éze.

\Maganspeaks Le-z-éze? Mi te sa pa que le-z-éze se lavon pamì a man?

\Paganspeaks Comèn na? Comèn se fé?

\Maganspeaks Te féo vère. \direct{Eun terièn foua son téléfonne} N'i djeusto ditsardjà eunna applicach\'on.

\StageDir{Magàn gnaque eun bot\'on si son téléfonne.}

\Maganspeaks\direct{A vouése ata} Fuffi!

\StageDir{\Fv{Ouè Sophie}}

\Maganspeaks\direct{Comme douàn} Lava le-z-éze!

\StageDir{\Fv{Va bièn}}

\Maganspeaks\direct{A pagàn} Té te sen de tapadzo di platte que se lavon?

\Paganspeaks Mi sentì senque? Sento té que t'i matta!

\Nevaousaspeaks Magàn! Ouè, la lavaplatte l'è eun martse.

\Maganspeaks Vou aoutre vère. Si pa tan chiya de seutta modernitoù.

\StageDir{Magàn chor.}

\Paganspeaks Mondjemé! Sen mal betoù. \direct{I nevaou} Acoutade, magàn l'è eunna mia ià de tita. Can mimo, Aimé é Julie\ldots sitte l'ie maque lo comenchemèn. Aprì n'en fé eunna matse de spettaclle, sen fé-no eunna matse de sou é avouì sise sou n'en fé de fite, de maende, de sin-e é pi finque de dzente chortiye!

\Nevaousaspeaks De chortiye? Wow! Iaou v'ouite aloù?

\Paganspeaks Sen aloù ba a Pollein\ldots na\ldots Polonia! Aprì Londra, San Francisco,  Mosca, Dubai é pi finque Rio! Mi ara vo counto de can sen aloù eun Piém\'on!

\Nevaouspeaks Na, na, counta-no vèi de can t'i aloù a Dubai ou a Rio.

\StageDir{Entre magàn.}
 
\Maganspeaks Ouè Paul, counta vèi de can t'i aloù a Rio, té que t'a tan pouiye de l’avi\'on.

\Paganspeaks\direct{Malechà} Ara itade quèi tcheutte! Ara vo counto de can sen aloù i Ca' Brusà ba pe Alba! 

\StageDir{Teuppe \lemieBa\ a drèite.}

\StageDir{Lemie \lemieSi\ a gotse é i mentèn di palque.}

\scene[-- Chortiya bièn digourdia]

\StageDir{Acapèn nou caèye plachaye pe reprézenté eun poulmeun. Douàn lo sédil di chauffeur, l'iè  eun volàn. Totte le-z-ach\'on di-z-atteur que dèyon euntérajì avouì lo poulmeun (iverteua di pourte ou di finestreun) seràn mimaye.}

\StageDir{Lo chauffeur l'è achouatoù deun lo poulmeun to solette, eunna mia eunfastedjà.}

\Chauffeurspeaks Sise de Tsarvensoù son todzor eun retar: pe la chortiya di For\ldots eun car d'aoua, le pompì volontéo demì aoua\ldots é ara sise di téatro l'an dza trèi car d'aoua bondàn! Atégnèn euncoa. Dimèn controlèn que sise totte a poste: le lemie lèi son, lo clacson martse é le panavèyo martson\ldots se fé lo temporal sen a poste. Ara atégnèn.

\StageDir{Entron Jordy é Simone. S'aprotson i finestreun, prouon a prédjì i chauffeur mi se sen ren.}

\Chauffeurspeaks Sento pa!

\StageDir{Lo chauffeur teurie ba lo finestreun.}

\Jordyspeaks Mé é Simone sen arrevoù! Atégnèn seuilla de foua le-z-atre.

\Chauffeurspeaks Na, na vegnade si! Pouyade maque si que aprì alade a bèye eun cafì é vo vèyo pamì; si beun comèn martson le bague.

\StageDir{Simone ivre lo:}

\effet{https://soundcloud.com/user-234168361/portellone-posteriore}{Portell\'on}

\StageDir{Simone é Jordy entron deun lo poulmeun. Jordy cllou lo portell\'on\footnote{ Dèi-z-ara tcheu le cou que lo portell\'on se ivre ou se cllou, se sen lo son di portell\'on.}. Entron eun \textit{scène} Thierry é Jo\"{e}l. Jo\"{e}l l'at eun petchoù saque pe la mordiya.}

\Chauffeurspeaks\direct{Eugn avèitsèn Thierry é Jo\"{e}l} Sise? Sarèn pa de vo sise?

\Jordyspeaks Ouè, son de no!

\StageDir{Thierry ivre lo portell\'on.}

\Joelspeaks Bondzor, pouèn entré?

\Chauffeurspeaks Vo v'ouèide pa comprèi. Vo v'achouatade devàn.

\Joelspeaks Pequé?

\Chauffeurspeaks Pequé se itade mal, se vo tourne si to sen que v'ouèide bi ieur, oumouèn n'a lo finestreun! 

\Joelspeaks Que stoufiàn!

\StageDir{Thierry cllou lo portell\'on.}

\Thierryspeaks Que boutro!

\StageDir{Jo\"{e}l ivre la pourta di poulmeun. Thierry entre pe premì pe se plachì protso di finestreun. Eun passèn desì Thierry, Jo\"{e}l entre  é se plache i mentèn. Thierry cllou la pourta.}

\Chauffeurspeaks\direct{Eun moutrèn lo saque de Jo\"{e}l} So senque l'è?

\Joelspeaks So l'è la mordiya.

\Chauffeurspeaks Ad\'on la mordiya va dérì!

\StageDir{Lo chauffeur tchappe lo saque é lo tappe dérì.}

\Joelspeaks Mi pequé?

\Chauffeurspeaks Pequé se no ariton fan de counte!

\Thierryspeaks\direct{Dezò vouése a Jo\"{e}l} Que malpolì si chauffeur!

\Joelspeaks Ouè t'a rèiz\'on.

\StageDir{Entron eun scène trèi feuille: Marlène, Jo\"{e}lle é Stephanie. Se plachon douàn lo poulmeun pe se fé eun selfie.}

\Joelspeaks\direct{I chauffeur} Seutte son avouì no.

\Chauffeurspeaks Bièn, eunna mia de feuille fa beun!

\Joelspeaks Ouè nen n'en maque trèi. Fan todzor de foto! Ara pouèn diye sen que n'en voya, mi can entron fa fé attench\'on.

\StageDir{Le feuille prouon a ivrì lo portell\'on, mi areuvvon pa. Simone se levve é lo ivre. Le trèi feuille entron é salion tcheutte.}

\Stephaniespeaks Squezade lo retar, mi n'ayé lo passadzo a livel d’Ampallian ba\ldots

\Simonespeaks Ouè, fa clloure lo portell\'on!

\StageDir{Simone se levve é comme douàn cllou lo portell\'on.}

\Chauffeurspeaks\direct{Ironique ver le feuille} Ouè é euncó finque lo \textit{semaforo} di Bournì! Can mimo, pouèn partì? N'at euncó eun poste vouido\ldots

\Simonespeaks Ouè pequé fa passé prendre Marco a Feleunna.

\Chauffeurspeaks Ah, va bièn.

\Joelspeaks\direct{I chauffeur} Feleunna l'è djeusto douàn d'arrevé a Pollein.

\Chauffeurspeaks Si beun iaou l'è Feleunna!

\StageDir{Lo chauffeur aleumme lo poulmeun, mi lo moteur vou pa nen sèi de partì.}

\effet{https://soundcloud.com/user-234168361/motorino-davviamento}{Tentatif aviemàn moteur}

\Joelspeaks\direct{Eun rièn} Comenchèn bièn!

\StageDir{Lo chauffeur reproue.}

\effet{https://soundcloud.com/user-234168361/accensione}{Aviemàn moteur}

\StageDir{Finalemàn lo moteur s'aleumme.}

\StageDir{Lo chauffeur beutte la rétro.}

\Chauffeurspeaks Vo que v'ouite ba i fon, diade-mé se n'i de caro dérì.

\Jordyspeaks Ouè te diyo mé. Veun maque! Veun, veun, veun\ldots

\StageDir{Lo poulmeun totse contre lo meur é tcheutte boueuchon la tita countre lo sédil.}

\Jordyspeaks Praou pouèi!

\Chauffeurspeaks Ouè n'i praou sentì! Tacaleun!

\StageDir{Lo chauffeur beutte la premiye é partèi.}

\Joelspeaks Sen belle partì, alèn ba pe lo Piém\'on, medjì de tseur\ldots 

\StageDir{Silanse.}

\Joelspeaks N'at an mia de silanse desì si poulmeun\ldots

\Thierryspeaks Beteun de mezeucca!

\StageDir{Thierry alondze la man pe avì la radi\'o, mi lo chauffeur lèi baille eunna dzifla.}

\Chauffeur Na, beutto mé!

\StageDir{Lo chauffeur tsertse eun per de canal mi n'a maque de publisit\'o.}

\Chauffeurspeaks Tchouéyèn, maque de publisitò.

\Joelspeaks Ad\'on fièn eunna tsans\'on no! Jo\"{e}lle attaca-là té!

\Joellespeaks Va bièn. \direct{Eun tsantèn} Le Digourdì\ldots

\Tcheuttespeaks\direct{Eun tsantèn} \ldots le Digourdì, le Digourdì de Tsarvensoù, le Digourdì de Tsarvensoù, le Digourdì de Tsarvensoù.

\Joelspeaks Seutta l'è fran dzenta, la fièn euncó can vegnèn si.

\Chauffeurspeaks Va bièn. Can mimo no aprotsèn\ldots èitade: lé n'a lo premì sate de Feleunna. Ouè l'an fé-lo fran ate\ldots

\StageDir{Tcheutte (douàn la premiye feulla, aprì la secounda é a la feun la trèijima) sauton si can lo poulmeun pase desì lo \textit{dosso}.}

\Joelspeaks Ouè son fran ate, sourtoù ara que l'an levou-lo. Son beun dzen é comoddo; mi n'a todzor de dzi que se lamenton\ldots

\Chauffeurspeaks Djèique! Baillon maque de problème\ldots

\Joelspeaks\direct{Eun contegnèn son discoù} \ldots pren eunna Jeep se te ite eun Val d'Outa.

\Chauffeurspeaks\direct{Eun contegnèn son discoù} \ldots ou te va a 15 a l'aoua ou te difé la machina\ldots

\Joelspeaks Djèique.

\Chauffeurspeaks Nen fan eun tsaque 100 mètre!

\Joelspeaks Vrèi! Djeusto pouèi. Ah! Arrevèn protso a Feleunna.

\StageDir{Dimèn, tcheu le-z-atre s'arbeuillon pe se toppé di frette que comenche a lèi itre.}

\Chauffeurspeaks Iaou ite Marco?

\Joelspeaks Ba lé, 50 mètre a drèite.

\Thierryspeaks\direct{A Jo\"{e}l} Toppa-té que fé frette.

\StageDir{Jo\"{e}l, eunsemblo a tcheu le-z-atre, teurie foua di saque de palt\'o, de gan, de bounet pe se toppé di frette. Dimèn lo polmeun s'arite.}

\Chauffeurspeaks Mondjeu que frette! Tsandze fran l'er! Mé n'i pa prèi ren dérì.

\StageDir{Caqueun baille i chauffeur eun bounet.}

\Joelspeaks\direct{Eun fenissèn de s'arbeillì} Mé si organizou-me bièn, pequé avouì lo frette que fé seu i mèis d'avrì fa pa squersé!

\Chauffeurspeaks Ouè pensade de alé lo querì? Seu fé frette!

\Joelspeaks Ouè, ouè vou mé! Vo itade seuilla!

\StageDir{Jo\"{e}l bèiche ba di poulmeun. Se sen eugn:}

\effet{https://on.soundcloud.com/2ZWj9FGPMw3ZSJAD6}
{Oua frèide}

\Joelspeaks\direct{Eun braillèn} Marco! Marco!

\StageDir{Entre Marco eun mailletta.}

\Marcospeaks\direct{A Jo\"{e}l} T'a fenì de fé lo mar\'on?

\Joelspeaks Mi comèn te fé sensa eun palt\'o? Beutta eun palt\'o que fé frette!

\Marcospeaks Ah, ah! Tcheu le-z-àn la mima battiya\ldots sen a avrì mar\'on!

\Joelspeaks Ouè, pouya si que fé frette!

\StageDir{Marco ivre lo portell\'on é pouye si lo poulmeun. Jo\"{e}l, tot i galoppe, cllou lo portell\'on, ivre la pourta é se tappe desì a Thierry pe entré.}

\Thierryspeaks\direct{A Jo\"{e}l} Ouè mi que couése dzalaye!

\Joelspeaks T'a sentì?

\StageDir{Thierry cllou la pourta. Se sen pamì lo ravadzo de l'oua dzalaye que soufflè foua di poulmeun. Le chauffeur partèi. Tcheutte se dizarbeuillon.}

\Joelspeaks Comenche l'Adret. Fenì Feleunna comenche a fé tsa. Te pase lo pon é te reste dza mioù.

\Chauffeurspeaks\direct{Eun avèitsèn douàn} Grousa seutta rionda! L'è noua?

\Joelspeaks Na l'è tchica que l'an fé-la.

\StageDir{Lo chauffeur entre deun la rionda avouì bièn de vélosité. Vionde lo volàn é tcheutte se plèyon ba a leur drèite.}

\Thierryspeaks Plan, va plan!

\Joelspeaks Ara praou chor foua!

\Chauffeurspeaks Fièn euncó eun tor!

\Tcheuttespeaks Na, na, na!

\Thierryspeaks Plan, va plan!

\StageDir{Lo chauffeur vionde de l'atro coutì lo volàn é tcheutte tournon a itre achouatoù drette.}

\Joelspeaks\direct{I chauffeur} Mi comèn se fé a prendre de rionde pouai! Can te entre te rallente!

\Chauffeurspeaks Mi n'i pa fé-la mé seutta rionda!

\Joelspeaks Mi soplé di pa de counte foule!

\Thierryspeaks Penson de itre i couscrì!

\Chauffeurspeaks Ouè avouì vo desì semble beun de itre i couscrì. Can mimo ara tanque a Muntal Deura no ariterè pa gneun.

\Joelspeaks Spéèn.

\StageDir{Silanse.}

\StageDir{To d'eun creppe, le Digourdì achouatoù dérì lo chauffeur sauton si pe l'er comme se quetsouza sise pasoù dézò le ràoue di poulmeun. Lo chauffeur arite to de chouite lo poulmeun.}

\Joelspeaks Mi senque l'è capitoù?

\Chauffeurspeaks Si pa. Avèitsèn dérì se n'a quetsouza.

\StageDir{Tcheutte se viondon pe avèitchì. Gneun vèi ren.}

\Joelspeaks Mi avèitsèn euncó eun cou.

\StageDir{Tcheutte eunsemblo se viondon pe avèitchì. Gneun vèi quetsouza.}

\Chauffeurspeaks Ad\'on bèicho ba pe vère.

\StageDir{Lo chauffeur ivre la pourta é bèiche ba di poulmeun. Can cllou la pourta Jo\"{e}l ataque a prédjì. Dimèn lo chauffeur mioule sen que n'a dérì lo poulmeun.}

\Joelspeaks Jordy eun \textit{chauffeur} pouèi te pouè accapi-lo maque té!

\Jordyspeaks N'en voulì reusparmì pe pa prendre Pelanda? 

\Joelspeaks Ouè mi se pou pa prendre de \textit{chauffeur} comme sitte!

\Thierryspeaks\direct{A Jo\"{e}l} Ah can t'ie té Prézidàn!

\StageDir{Lo chauffeur ivre la pourta.}

\Chauffeurspeaks Gars\'on\ldots n'i sétchà eun tèiss\'on.

\Joelspeaks Eun tèiss\'on? N'at eun deun to lo Piém\'on é t'a tchapou-lo té?

\Chauffeurspeaks Eh ouè.

\Joelspeaks Ara vèi té sen que fa fiye.

\Chauffeurspeaks Ara l'è mioù querì le Garde Forestier pe pa èi de counte.

\StageDir{Can lo chauffeur cllou la pourta Jo\"{e}l ataque a prédjì.}

\Joelspeaks Mi l'è pa poussiblo. A té, Jordy, te baillèn pamì de-z-entsardzo.

\Thierryspeaks\direct{A Jo\"{e}l} Pregnèn no lo moublo?

\Joelspeaks Te di?

\Thierryspeaks  Mi ouè tante sitte l'è pa bon a gueudé.

\Jordyspeaks Senque?

\Joelspeaks\direct{Ver tcheutte} Mé é Thierry n'en i eunna idou! Se prégnèn no lo poulmeun é ataquèn ba?

\StageDir{Tcheutte, mouèn que Jo\"{e}lle, son d'acor é entuziaste. Lo chauffeur l'è euncó eun tren de prédjì i téléfonne. Jo\"{e}l pouye si lo poste di chauffeur.}

\Joellespeaks\direct{Dispéraye} Jo\"{e}l soplé, fièn pa le Digourdì!

\StageDir{Jo\"{e}l beutte eun martse. I mimo ten areuvve lo chauffeur.}

\Chauffeurspeaks Mi senque v'ouite eun tren de\ldots

\StageDir{Partèi lo refrain de:}

\sound{https://www.youtube.com/watch?v=tgw1yEcWpTU}{Ruda tańczy jak szalona - Czadoman}

\StageDir{Lo poulmeun partèi ià a flama. Lo chauffeur lèi galoppe dérì.}

\StageDir{Aprì eun per de seconde a totta vélositoù, la mezeucca se tchoué, totta l'ach\'on rallente.}

\Joellespeaks\direct{Lentamente, a Jo\"{e}l} Attench\'on, la viille!

\StageDir{A drèite entre eunna viille madama. L'et eun tren de traversì la rotta. Tot a rallentateur, Jo\"{e}l frène, tcheutte boueuchon la tita countre le véyo ou le sédil. Dimèn, areuvve euncó lo chauffeur tot a galoppe, mi boueuche la tita countre lo dérì di poulmeun.}

\StageDir{Teuppe \lemieBa\ a gotse é i mentèn di palque.}

\StageDir{Lemie \lemieSi\ a drèite.}

\scene[-- Hollywood]

\Nevaouspeaks Ad\'on te vèi que sise Digourdì son pa itò de super star. V'ouite nèisì pe eunna cantin-a, lo voutro nom l'è chortì pe eunna souaré de fita, pe la premiye pièse n'ayè catro tsatte, i premì Printemps fiavade  maque de confuj\'on é di chortie\ldots prédzen-nen pa!

\Paganspeaks \'Eitsa que grama lenva que t'a! Fiade attench\'on: pourtade reuspé pe le Digourdì! Rappelade-v\'o  que le Digourdì l'an\ldots

\Ledouspeaks \ldots pourtoù lo nom de Tsarvensoù pe to lo moundo!

\Paganspeaks Ouè fran pai!\direct{Eun viondèn la padze de l'album} Avèitsade vèi sen que l'è capitoù aprì. Aprì eun per de-z-àn n'en fé eunna matse de sou é la Val d’Outa l'ie viin-a tro petchouda pe no é sen partì aoutre pe l’Amérique!

\Nevaousaspeaks Comèn v'ouite aloù aoutre pe l’Amérique?

\Paganspeaks  Totte l'è comenchà mersì a eun \textit{court-métrage} que n'en fé avouì Alessandro Stevanon. To lo moundo l'a vu si \textit{court-métrage} é l'an euncó finque queria-no a Hollywood; mi ara lamerio sentì la vouése de magàn\ldots di vèi se l'è pa vrèya seutta counta?

\Maganspeaks Ouè, pe eun cou me totse bailli-te rèiz\'on. Mi can mimo n'i pa comprèi sen que l'an vi eun vo sise de Hollywood pe finque vo baillì eugn Oscar!

\Nevaousaspeaks Eugn Oscar? Qui de vo la gagnà l’Oscar?

\Paganspeaks Se me rappello amoddo\ldots

\StageDir{Teuppe \lemieBa\ a drèite.}

\StageDir{Lemie \lemieSi\ a gotse é i mentèn di palque.}

\scene[-- La nite di-z-Oscar \oscar]

\StageDir{Eun scène, arbeillà bièn élégàn, n'a lo prézentateur de la nite di-z-Oscar. L'a douàn llou eun pupitre.}

\Conducteurspeaks Bonsouar, mersì a tcheutte, voutro \textit{acceuil} toujour si chalereu m'émochoun-e todzor. Mi l'è pa pe mé que v'ouite inque, tellamente nombreu pe reumplire totte le caèye di si magnifique téatro. V'ouite inque, \textit{mesdames} é \textit{messieurs}, pequé l'è arrevoù lo momàn que tcheu no, que to lo mondo l'è eun tren d'attendre: l'asségnach\'on de l'Oscar pe lo meillaou atteur protagoniste. L'è itaye bièn bataillaye. Vo rappello ara le sinque finaliste, euncó se vo sade bièn que maque eun de leur gagneré, maque eun nom l'è icrì deun la busta, maque eun de leur entrerè deun l'istouére. Le sinque finaliste son:
\begin{enumerate}
\item Leonardo DiCaprio;
\item Julia Roberts;
\item George Clooney;
\item Angelina Jolie;
\item Paolo Cima Sander.
\end{enumerate}

Complemèn a tcheu leur. De caque momàn devrie arrevé la busta é finalemèn noutra queriaouzitoù serè satisfète.

\StageDir{Entre eunna feuille élégante avouì eunna busta rodze é l'Oscar. Lo conducteur can la vèi se émochoun-e é la bèije eun per de cou. Aprì pren la busta.}

\Conducteurspeaks\direct{A la feuille} L'è salla djeusta? Pa comme l'atro an que l'ie eugn atra busta?

\Vallettaspeaks Spéèn!

\Conducteurspeaks L'Oscar pe lo meillaou atteur protagoniste va a\ldots

\StageDir{Lo conducteur ivre la busta é teurie foua eun papì.}

\Conducteurspeaks \ldots Paolo Cima Sander!

\StageDir{Paolo Cima Sander l'è achouatoù i mentèn di pebleuque. Can sen son nom se levve eun pià tot émochon-où. Eunna lemie s'aleumme fran iaou l'è achouatoù.}

\Cimaspeaks\direct{Bièn émochon-où} Oh mondjeu, mi na, lèi crèyo pa, mi fran mé. Attégnade que areuvvo ba. 

\StageDir{Partèi la tsans\'on:}

\sound{https://www.youtube.com/watch?v=Ycg5oOSdpPQ}{Katchi - Ofenbach vs. Nick Waterhouse}\label{katchi}

\StageDir{Paolo, tellamente ajitoù, miclle eunna mia d'anglé avouì lo patoué.}

\Cimaspeaks Areuvvo! \textit{Oh my God}! Mersì a \textit{everybody}!

\Conducteurspeaks Ouè t'attégnèn!

\StageDir{Paolo s'aprotse i palque.}

\Cimaspeaks Me l'attégnavo pa!

\StageDir{A couza de l'émoch\'on, Paolo lèi beutte bièn de ten pe arrevé . Lo conducteur, ad\'on, tsertse de fé passé lo ten.}

\Conducteurspeaks Ad\'on lo attégnèn. Qui se l'attégnave? Complemèn a Paolo, mi euncó a tcheu le-z-atre que se refàn pi eugn atro an\ldots

\Cimaspeaks\direct{Eun braillèn} N'i sbaillà chortiya! Areuvvo!

\Conducteurspeaks Ouè t'attégnèn. Eugn éffé seutta l'è sa souaré; l'attégnèn volontchì, n'en pa de prisa.

\Cimaspeaks\direct{Comme douàn} Areuvvo!

\Conducteurspeaks\direct{A la feuille, pe fé pasé lo ten} Aprì pensa que dzen: pe eunna personna, que tanque ieur itave deun petchoù veladzo de montagne, arrevé inque a Hollywood, arrevé i \textit{temple}, i Mont Olympe, a l'éillize di \textit{cinéma} mondial.

\Vallettaspeaks Euncrouayablo!

\Conducteurspeaks Ah, voualà Paolo Cima Sander!

\StageDir{Paolo areuvve dézò lo palque é comenche a saliì le dzi. La mezeucca s'arite.}

\Cimaspeaks\direct{Avouì la ranfan-a} Me voualà! \textit{Of course, ok, ok}. \textit{I don't} me l'attégnavo pa. \direct{Eugn eumbrachèn eunna madama} Oh madama moratsade-me vèi!

\StageDir{Paolo conteneuvve a saliì le dzi.}

\Cimaspeaks Mersì, mersì, \textit{one, two, three}\ldots \textit{don't worry be happy}!

\Conducteurspeaks Prèdza maque patoué Paolo!

\Cimaspeaks\direct{Eun calabrot} \textit{Ni vidimu ndu giardinu}! Mersì! \direct{Ver lo conducteur} Iaou son le-z-itchilì?

\StageDir{Lo conducteur moutre iaou son.}

\Cimaspeaks\direct{Eun pouyèn si lo palque} Mondjeu di paadì!

\StageDir{Paolo areuvve si lo palque é salie lo conducteur é la feuille que lèi baille l'Oscar.}

\Cimaspeaks Oh mondjeu! Mi comèn diade seu a Hollywood \og \textit{tutto bagnato di caldo}\fg{}?

\Conducteurspeaks Blet de tsa!

\Cimaspeaks Ad\'on si fran blet de tsa! \direct{Eugn avèitsèn l'Oscar} Mi eunc\'o finque Oscar\ldots pensao pa a eunna baga pouai aprì itre itoù Secretéo di Consor de l'\'Eve de Feleunna.

\StageDir{La feuille chor.}

\Cimaspeaks Ara senque fa diye?

\Conducteurspeaks Sen que t'a voya, lo palque l'è de té.

\Cimaspeaks\direct{Todzor émochon-où} Na, na, trop eumpourtàn si Oscar. Fa remersì la fameuille: Mami Roby é Papi Tony\ldots 

\StageDir{La feuille entre é dézò vouése baille eun mésadzo i conducteur, dimèn que Paolo conteneuvve avouì le remersiemèn.}

\Cimaspeaks \ldots trop eumpourtàn Oscar; se sit an l'è l'Oscar eugn atro an seré lo Nobel!

\Conducteurspeaks Ouè cheur!

 \Cimaspeaks\direct{Todzor bièn émù é ajitoù} Ad\'on fa remersì euncó le vezeun! La vezin-a Miranda, Armando é Lucia que son bièn eumpourtàn; mi sourtoù comèn pa remersì le-z-amì de mé de Feleunna é Plan-Feleunna, le-z-amì valdotèn crèisì avouì mé: Rocco, Turi, Mimmu \textit{u pulici}, Tommaso Buscetta, Salvatore Savastano, Tommaso \textit{u pistulero} é Johnny \textit{a munnezza}. Mi can mimo\ldots Oscar! Que dzen! Can mimo si arrevoù seuilla avouì la fameuille, le-z-amì\ldots mi comèn pa remersì le-z-amì Digourdì! L'an édja-me, son restou-me protso, l'an pourto-me seuilla a Hollywood, djeusto?

\Conducteurspeaks Ouè Hollywood!

\Cimaspeaks Le Digourdì que son crèisì avouì mé!

\Conducteurspeaks L'è to vrèi Paolo é te pou le remersì, te pou lo braillì a tcheutte! Pequé me diyon ara de la réjì que n'en eunna sourprèiza pe té: Paolo, \textit{mesdames} é \textit{messieurs}, n'i fran l'onneur de querì inque si lo palque tcheu le Digourdì, totta la compagnì deun laquelle Paolo l'è crèisì é que la fé-lo vin-ì lo bon atteur que ara l'è douàn no! Réjì! Mezeucca!

\sound{https://www.youtube.com/watch?v=Ycg5oOSdpPQ}{Katchi - Ofenbach vs. Nick Waterhouse}[false]

\StageDir{Lo conducteur queurie, eun aprì l'atro, tcheu le Digourdì si lo palque. Tsaque Digourdì, arbeillà élégàn pe la nite di-z-oscar, attraverse lo parterre eun salièn lo pebleuque é pouye si lo palque pe saliì Paolo.}

\Conducteurspeaks N'i lo plèizì de querì si lo palque:
\begin{itemize}
\item[$\bullet$] Pierre Savioz;
\item[$\bullet$] Ester Bollon;
\item[$\bullet$] Giada Grivon;
\item[$\bullet$] Laurent Chuc;
\item[$\bullet$] Jasmine Comé;
\item[$\bullet$] Francesca Lucianaz;
\item[$\bullet$] Marco Ducly;
\item[$\bullet$] Ilaria Linty;
\item[$\bullet$] Marlène Jorrioz;
\item[$\bullet$] André Comé;
\item[$\bullet$] Jo\"{e}lle Bollon;
\item[$\bullet$] Jo\"{e}l Albaney;
\item[$\bullet$] Jordy Bollon;
\item[$\bullet$] Simone Roveyaz;
\item[$\bullet$] Thierry Jorrioz.
\end{itemize}

\StageDir{La mezeucca s'arite.}

\Conducteurspeaks Voualà son tcheue inque. \textit{Mesdames} é \textit{messieurs}, a vo le Digourdì!

\StageDir{Teuppe \lemieBa\ a gotse é i mentèn di palque.}

\StageDir{Lemie \lemieSi\ a drèite.}

\scene[-- An matse de DVD]

\Nevaouspeaks \ldots é queun l'ie lo titre de si \textit{court-métrage} que l'a fé gagnì l’Oscar a Cima?

\Paganspeaks L'ie\ldots \og RECONSTITUTION – La vèille di gran dz\^o''.

\Nevaousaspeaks \ldots é senque counte?

\Paganspeaks L'ie la counta\ldots \direct{Ver Sophie} de senque counte?

\Maganspeaks L'ie la counta de la \textit{reconstitution} de noutra Quemeua: sinque Tsarvensolèn, deun lo 1946, l'an preì eun man Tsarvensoù.

\Nevaouspeaks Se pou veure?

\Paganspeaks Ouè! N'en euncó lo DVD!\direct{A Sophie} Iaou t'a catcha-lo?

\Maganspeaks Lé avouì le bague de té!

\Paganspeaks Ouè, t'aré caya-lo ià!

\Nevaousaspeaks Eun DVD? Ad\'on to lo moundo l'arè vi-lo!

\Paganspeaks To lo moundo! L'atsetavon si Amazon, l'avèitsavon si l'iPhone, l'atsetavon euntchì no; mi lo cou que n'en vendi-nen de pi l'ie l'an que n'en fé la pièse pe le dji-z-àn di Digourdì! L'ie lo 10 marse 2018 i Téatro Splendor!

\Nevaouspeaks Mi iaou i Splendor?

\Paganspeaks Iaou? \direct{Eugn avèitsèn amoddo lo pebleuque} Lé! Can te chor a gotse! Me rappello que l'an bailla-no eunna matse de sou! N'en to vendì é vo sade senque l'è capitoù? L'è arrevoù eugn ommo que l'a bailla-no pe eun DVD 100 euro!

\Nevaousaspeaks Mi qui l'ie si matte?

\Maganspeaks L'ie lo Seunteucco deun cou: Ronny Borbey!

\Paganspeaks Que matte Ronny! L'è belle aloù eun pench\'on euncó llou ara\ldots

\Maganspeaks Ouè!

\Nevaouspeaks Ah ouè. V'ouyavade fran feun vo Digourdì, v'ouite itoù todzor de Digourdì!

\Paganspeaks Ouè. Aimè, te sa pequé?\direct{Eun se levèn} Pequé itre Digourdì vou deu itre: abile, actif é euntelledzèn! \'E no lo sen todzor itoù, dèi can sen nèisì! No sen todzor\ldots 

\StageDir{Se avion le lemie \lemieSi\ si to lo palque. A gotse n'a tcheu le Digourdì, que, eunsemblo avouì pagàn, magàn é le dou nevaou, eurlon ver lo pebleuque:\
\begin{center}
\ldots pi Digourdì!
\end{center}
}

\ridocliou

\DeriLeRido

\RoleNoms{Mezeucca, éffé sonore}{Michel Comé}

\RoleNoms{Lemie}{Alessandro Chiono}

\RoleNoms{Mijì}{Renzo Bollon}

\end{drama}
