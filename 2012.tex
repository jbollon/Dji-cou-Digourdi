\title{EUN DROLO DE DISTRIBUTEUR}
\author{Pièse icrita pe Joël Albaney}
\date{Téatro Giacosa de Veulla, 11 mi 2012}

\maketitle

\fotocopertina{Foto/2012/gruppo_red.jpg}{Pierre Savioz, Ilaria Linty, Francesca Lucianaz, Stéphanie Albaney, Stefano Vermillon, Jo\"{e}l Albaney, Simone Roveyaz, Giada Grivon, Jasmine Comé, André Comé, Jordy Bollon}{Jean-Marc Savioz, \'Etienne Linty, Ester Bollon, Aimé Squinabol, Laurent Chuc, Philippe Vermillon, Marco Ducly, Julie Squinabol, Michel Ventrice, Lilys Galloni, Paolo Cima Sander, Sophie Comé, Jo\"{e}lle Bollon}{2012}

\LinkPiese{Eun drolo de distributeur}{https://www.youtube.com/watch?v=8yMGnxab5aY&list=PLBofM-NS_eLJUln45l7VH457fGak_Bk5O&index=9}{.75}

\souvenir{Me rappello pocca bague di pièse, n'i pa eunna grousa mémouée.
Mi can mimo me rappello amoddo comèn l'è nèisiya la compagnì di Digourdì!

L'ie lo 2007 é sayoo eun tren de guedé pe bèichì ba eun Veulla. S'ayoo presque foua d'Ampaillan, can sento a la radi\'o eugn'annonse publisitère d'eunna compagnì téatrale é to de chouite diyo a mé mimo: \og mi pequé seu a Tsarvensoù n'en pa eunna compagnì de téatro?\fg. Areuvvo i Pon-Sià é acappo le trèi de l'Ave Maria: Giada, Ester é Jasmine. Teurio ba lo finestreun é, tot entuziaste, lèi diyo: \og Oh mi pequé fièn pa eunna compagnì téatrale? An matse de Quemeue nen an eunna. Pouèn pa pa èi la noutra!\fg. Dèi si dzor gnouo a pompé de dzi pe terì si eunna compagnì. A la feun lo \textit{zoccolo duro} que me baillè fose l'ie la classe di '84: Jo\"elle, Jo\"el, Ester, Jasmine; aprì n'en eunc\'o terià dedeun d'atre amì: Francesca ('83), Giada é Pierre ('85). 

Caque ten aprì no-z-accapèn eun tchi Gidio (restoràn Monte Emilius) pe no-z-organizé é tsertchì d'atre dzouveunno. Sayoon eun tren de no-z-achouaté deun la pégna tsambretta a gotse di banc\'on, can Jo\"el se prézente avouì eun que n'ayoon jamì vi: botte blantse (que gneun betè), \textit{jeans} to strachà é polo de la Fred Perry avouì lo colette terià si! Me prézento: \og Salì, mé si Laurent\fg; \og \textit{Ciao, io sono Paolo}\fg. Ester m'avèitse comme pe diye \og mi sitte sa gneunca prédjì patoué\fg. Eun pi l'ie pi vioù de no!\\ \'E beun… sitte l'ie lo super Paolo Cima Sander: l'è itaye la vrèya rivélach\'on di Digourdì (pa fata de counté véo de cou lo pebleucco l'a riette grase a llou!). 

Dèi salla premiye réuni\'on n'en comenchà a no troué pe eunna sala de la viille bibliotèque (eun cou eunc\'o finque pe eunna sala dézò Mèiz\'on Quemeua), pe liye de teste téatral é fée de premiye proue. Eun teste que n'en prouoù mi jamì prézentoù l'è \og Piéreun lo stchapeun\fg\ !

Aprì fayè chédre eun Prézidàn. Can n'en fé la réuni\'on n'i to de chouite pompoù Jo\"el  \og Ouè té t'i adatte pe fé lo Prézidàn, t'i fran eun gamba, t'i lo top!\fg\ldots é pouèi n'en accapoù qui sariye aloù douàn lo dzeudzo d'eun lo case saoutisan foua de problème légal.

N'en pa moloù de no-z-accapé é de proué de teste téatral tanque lo dzor iaou sen senti-no preste pe prézenté eun premì spettaclle a totta la populach\'on de Tsarvensoù, lo 17 mi 2008 a la bibliotèque communale de Tsarvensoù. L'entuziasme que n'ayoon l'ie contajeu é lo pebleuque l'a to de chouito comprèi-lo. L'è itoù, sensa doute, l'eungrédiàn secré de noutro sucsé.

Pouèi son nèisì le Digourdì… ah na! Pouèi l'è comèn l'è nèisiya la compagnì. Mi pequé \og Digourdì\fg é pa \og Le compagn\'on de la fiacca\fg ou \og Le botchassa de Tsarvensoù\fg?

Eunna nite mé é Simone no no reteriavon aprì eun mariadzo. L'ie beun tchica tar mi n'ayoon pa voya de no reterì. Adon désidèn d'ataqué si pe Cogne. Passèn donque i mitcho de Simone pe prende le cllo de la machina. Mé attégnao foua di mitcho é for probablo que Simone l'arè fé de cazeun eugn entrèn i mitcho, pequé, aprì eun per de seconde, sa mamma chor foua si la louye é no di: \og Iao alade?\fg. Simone lèi rep\'on: \og Alèn si eun Cogne\fg. Dispéraye la mamma no braille: \og Si eun Cogne?! Fiade pa tro le digourdì!\fg.
Attaquèn si pe Cogne é dimando a Simone: \og Mi senque vou deue digourdì?\fg; \og Boh, si pa, mamma me lo di todzor\fg. Dèi salla nite n'i mémorizoù lo numér\'o de téléfonne de Simone comme \og Digourdì\fg!

Aprì tchica de ten, lo lon d'eunna proua, sen deu-no: \og Mi queun nom no no baillèn?\fg. Aprì trèi meneutte propouzo: \og Le Digourdì!\fg, \og Mi senque vou diye?\fg; avèitsèn sen que di lo dichonnéo: \og Abile, actif, euntelledzèn\fg\ldots \og Sen no!\fg.

Voualà comèn son nèisì Le Digourdì!}{Laurent Chuc}

\queriaouzitou{
\begin{itemize}
\item[$\bullet$] Pe Jordy Bollon, futur Prézidàn di Digourdì, é André Comé, futur Vise-prézidàn, \og Eun drolo de distributeur \fg reprézente leur débù deun la compagnì. \\

\item[$\bullet$] Pe lo premì cou, pouyon si lo palque de mèinoù é de megnotte.\\

\item[$\bullet$] Pe la majorité di Digourdì, seutta l'è la pièse mouèn dzenta ou que l'a baillà mouèn de soddisfach\'on.\\

\item[$\bullet$] Michel Ventrice l'è l'atteur lo pi ansièn que l'isa jamì resit\'o si eun palque avouì le Digourdì.
\end{itemize}
}

\Scenographie
\begin{itemize}
\item[$\bullet$] 1 distributeur \textit{automatique} d'éve, ate pi de dou mètre é lardzo trèi pa. L'at eunna brotsetta i mentèn, iaou le dzi pouon terì l'éve. Desì la brotsetta lèi son dji bot\'on pe séléchouì le caletaye d'éve. Eun pi, desì la machina, n'at eun grou pleuro iaou le cantognì micllon de poutreungue pe apresté le caletaye d'éve que le sitouayèn séléchouon;
\item[$\bullet$] 1 grou lenchoueu pe toppé la machina;
\item[$\bullet$] Pe le cantognì desì la machina, totte sorte de baradziye pe apresté le caletaye d'éve (1 modeun, de pousse, de benziya, de-z-erbe\ldots ) é de confettì\footnote{ \textsc{Larouss.fr} - \textit{Mince rondelle de papier coloré qu'on lance par poignées lors de certaines festivités (carnaval). À l'origine, confetti est un pluriel (confetto = boulette de plâtre frais roulée, confectionnée à la main, remplacée plus tard par un petit rond en papier dans les festivités du carnaval). Ce mot niçois et italien s'est imposé en français comme un singulier.}};
\item[$\bullet$] Pe le sitouayèn, de botèille, de tsaèn, de boteill\'on (eun vèyo ou eun plasteucca) pe terì l'éve;
\item[$\bullet$] Pe tcheu le sitouayèn, de \textit{tessere} pe paì l'éve di distributeur;
\item[$\bullet$] 1 ribàn de la Val d'Outa, de grouse fosette é 1 opinel;
\item[$\bullet$] 1 grou cartel de cart\'on avouì l'icrita: \og SISENSASOU\fg;
\item[$\bullet$] 1 banquetta \panchina ;
\item[$\bullet$] 1 pal de la lemie;
\item[$\bullet$] 1 cartel avouì icrì de comunicach\'on pe la populach\'on.
\end{itemize}

\setlength{\lengthchar}{3.25cm}

\Character[ATTILA]{ATTILA}{Attila}{Cantognì cost\'o todzor arbeillà eun man\-dze queur\-te é avouì tan de-z-àn d'espérianse eun Quemeua, \name{Simone Roveyaz}}

\Character[DIEGO]{DIEGO}{Diego}{Cantognì dzouvin-o avouì pocca d'espérianse, \name{Paolo Cima Sander}}

\Character[SEUNTEUCCO]{SEUNTEUCCO}{Seunteucco}{Seunteucco fenna de Tsarvensoù; fé dzenta fegueua l'è son objectif, déterminaye é sensa tracasse pe le trebelach\'on di-z-atre, \nameF{Francesca Lucianaz}}

\Character[DON\\ CARMELITO]{DON CARMELITO}{Don}{Don Carmelito, lo préye di péì, \name{Jordy Bollon}}

\Character[SERVÀN\\ DE MESSA]{SERVÀN DE MESSA}{Chir}{Servàn de messa de Don Carmelito, \name{Philippe Vermillon}}

\Character[POPULACH\'ON]{LE DZI}{Dzi}{La tsantiì di péì: \textsc{Joël Albaney, \\Ester Bollon, Paolo Cima Sander, \\Michel Ventrice};\\ Lo fotografe \textsc{André Comé};\\ Lo cap di-z-alpeun: Marco \textsc{Ducly};\\ Dou mèinoù de la \textit{Gaie Famille}:\\ \textsc{Aimé Squinabol} é \textsc{Julie Squinabol};\\ Le personadzo de la pièse é trèi mèinoù di péì: \textsc{Jean-Marc Savioz}, \textsc{\'Etienne Linty} é \textsc{Lilys Galloni}.}

\Character[VALLET\\ DE VEULLA]{VALLET}{Vallet}{Vallet de veulla fenna, jantilla é dilijanta, \nameF{Joëlle Bollon}\\}

\Character[CLAUDINE]{CLAUDINE}{Claudine}{Fenna veullatchiye que pourte ià pe lo péì son pé\-gno tseun, catchà dedeun la boursetta, \nameF{Stéphanie Albaney}}

\Character[GIACO]{GIACO}{Giaco}{Gars\'on dzouvin-o é tchica malédecoù,\\ \name{Pierre Savioz}}

\Character[GIADA]{GIADA}{Giada}{Femalla que pase pe case protso i distributeur, \nameF{Giada Grivon}}

\Character[PIERLUIGI]{PIERLUIGI}{Pierluigi}{Comandàn di-z-alpeun,\\ \name{Marco Ducly}}

\Character[SEUR\\ CLARETTA]{SEUR}{Seur}{Seur que l'at a queur lo bièn spirituél (é économique) de l'èillize, \nameF{\\Sophie Comé}}

\Character[V\'ETCHOT]{V\'ETCHOT}{Vetchot}{Ommo ansièn di péì, \name{\\Michel Ventrice}}

\Character[PROSPERO]{PROSPERO}{Prospero}{Vioù bièn veuste que lame s'euntremelé deun le-z-affère di-z-atre, \name{\\Joël Albaney}}

\Character[VOLONTÉO\\ DI SECOUR]{VOLONTÉO}{Volontero}{Volontéo di premì secour, fier d'itre volontéo é di drouet que lèi baille, \name{\\Pierre Savioz}}

\Character[LUCIANO]{LUCIANO}{Luciano}{Ommo si la carentèin-a que fé le \textit{turni} a la Cogne, bièn prisoù,\name{\\André Comé}}

\Character[MATASSA]{MATASSA}{Matassa}{Femalla sensa sou que dimande l’armoun-a, \nameF{Stéphanie Albaney}}

\Character[JACK]{JACK}{Jack}{\textit{Gay} cobloù avouì John,\\ \name{Laurent Chuc}}

\Character[JOHN]{JOHN}{John}{\textit{Gay} cobloù avouì Jack,\\ \name{Stefano Vermillon}}

\Character[NAIMA]{NAIMA}{Naima}{Femalla tunizine que tsemie avouì eun grou tsaèn si la tita é avouì a per le seun sinque mèinoù, \nameF{Ilaria Linty}}

\Character[FRANÇOIS]{FRANÇOIS}{Francois}{Lo pi\'on di péì, \name{\\Marco Ducly}}

\Character[TROUPE RAI]{TROUPE RAI}{Journaliste}{Journaliste fenna, eunterprétaye pe\\ \textsc{Giada Grivon}, é opérateur \textit{vidéo} de RAI 3, \name{André Comé}}

\Character[]{MACHIN-A}{Machina}{\hspace{1cm}}


\DramPer

\act[Acte I]

\scene[-- Me raccomando!]

\StageDir{Lemie \lemieSi\ si lo proscenium.}
\StageDir{ Di dou coutì di palque entron Attila (a drèite) é Diego (a gotse). Attila l'at avouì sé eun abrosaque é l'è arbeillà avouì eunna crè flanella. Diego, i contréo, l'è bardoù tanque i cou é se frotte le man pe lo frette.}

\begin{drama}
\Diegospeaks\direct{Eun baillèn la man \strettadimano\ a Attila} Ad\'on comèn l'è?

\Attilaspeaks \direct{Eun sarèn for la man a Diego} Amoddo!

\Diegospeaks\direct{Se totse la man} Ahia! Maladetto! Ad\'on sen eun retar?

\Seunteuccospeaks T'a vi lo Seunteucco?

\Attilaspeaks Na\ldots

\StageDir{A gotse entre lo Seunteucco. Tsemie bièn fièira é se plache i mentèn di dou.}

\Seunteuccospeaks Bondzor! Ad\'on n’i volì prédjì avouì vo dou, devàn que l’inaougurach\'on de oueu l’aviproù, pe vo rappelé voutra fonch\'on a la machina que n'en plachà é que demàn allèn ivrì.

\Diegospeaks Tourna? Mi se n'en fé pe trèi mèis lo course i BIM tcheu le feun senà é n'ayè dza si floque d'eun Assésseur que no féyè eunna tita pèi a forse de no diye sen que no déyèn féye a seutta fameuza machina de l'éve.

\Seunteuccospeaks Vouè lo si, mi vi que vo cougniso, gate pa repeté eun cou de pi le bague. Ad\'on, can le dzi\ldots

\Attilaspeaks\direct{Per la pachense} Eunc\'o? Mi vouè, no fa mandé a la brotsetta la caletoù d’éve que le dzi l’an cherdì!

\Seunteuccospeaks Bon, bon, mioù pèi. L’eumportàn l’è que demàn a choui-z-aoue di mateun v'ouisa tcheu dou a voutra plase perqué n’arè la quiya di dzi a prende l’éve é mé n’i pa voya de féye beurta fegueua!

\Diegospeaks Qui vou-teu que se lévise pèi vitto, de demendze, i fret a prende d’éve?

\Attilaspeaks Pe mé, can arrevèn a ouet aoue n’at praou é d'avanse.

\Seunteuccospeaks Se n’i deu choui-z-aoue, l’é choui-z-aoue, va-ti bièn? A do-z-aoue de l’iproù vo baillon pi lo tsandzo d'atre dou\ldots é pe lo nite de dji a chouì s'i eun tren de pensì de criì torna vo.

\Diegospeaks Tourna no dou?

\Seunteuccospeaks Ouè!

\Attilaspeaks Na, na, na! La fenna sensa de mé drimme pa lo nite.

\Diegospeaks \direct{Malisieu} Ouè, l’an beun deu-me-lò\ldots \malisieu

\Diegospeaks \direct{Suspecteu} Senque l'an deu-te?

\Seunteuccospeaks\direct{Eun copèn la counta di dou} Seutta l’é eunna situach\'on d’émerjensa!

\Diegospeaks Mi comèn trèi tourni? Véyo coute to so a la quemeun-a?

\Seunteuccospeaks So son pa de-z-affé que vo regardon!

\Diegospeaks Na, sarèn pa de problème de no dou mi de totta la popolach\'on de Tsarvensoù\ldots

\Attilaspeaks \ldots é squezade-mé, avouì totte seutte spèize, dedeun lo pat de stabilitoù comèn fiade a resté?

\Seunteuccospeaks Mi pensa pitoù a la tin-a de stabilitoù can te torne lo nite i mitcho! Can mimo, se n'i deu choui-z-aoue l'è choui-z-aoue, pa eunna meneutta pi tar! Bon ara mé vou, pequé sarèn dza tcheut lé m’atendre. Salì é a choui-z-aoue! 

\StageDir{Lo seunteucco chor.}

\Attilaspeaks Mi di-me té se fa féye no si travaille? Avouì to sen que n'en a féye eun quemeun-a!

\Diegospeaks Mi l'è pa ren sen. L'è que no fa resté lé a la machina tanque no baillon pa lo tsandzo!

\Attilaspeaks \ldots é aprì féye le trèi tourni perqué le dzi veugnon a prende l'éve a totte-z-aoure! Ah, mi mé me féyo pi appresté pe la fenna eun paneun avouì eunna grousa \textit{cotoletta} eumpanaye.

\Diegospeaks La \textit{cotoletta}? I fret? \frette

\Attilaspeaks Mi que fret!

\Diegospeaks Te reste pi totte desù lo \direct{ironique} petchoù estomac  que t'a! Mé n'i deu a la choze de me presté eun vazet de Danette, pouèi si a poste pe to lo dzor\ldots é se n'i eunc\'o tchica pi fan dimando eun toque de ta \textit{cotoletta}!

\Attilaspeaks Mi \direct{eun moutrèn lo seun grou bri} gnenca pe sondzo! T'a maque da te organizé devàn, comme mé!

\Diegospeaks Que jantillo! \'E mé que te rècho todzor can areuvve lo seunteucco ou la jeunte desù lo poste de travaille! 

\Attilaspeaks Mi di pa de conte foule, mé si todzor bièn réchà! Ah\ldots pe bèye senque te porte di mitcho?

\Diegospeaks Da bèye? Mi se n'en lo distributeur de l'éve seuilla!

\Attilaspeaks Te bèi pi té sen. Mé n'i todzor avouì mé eunna gotta de veun \direct{teurie foua lo boteill\'on di saque} perqué lo paneun fé sèi.

\Diegospeaks \direct{Avouì eun ton malisieu, eun avèitsèn lo boteill\'on} Eun boteill\'on de dou litro! Cheur d'avèi-nèn praou? 

\Attilaspeaks Penso beun; aprì se nen n'i pa praou baillo eun cou de fi a la fenna que veun sé avouì lo bis.

\Diegospeaks Ah! Bon l'è dza tar, senque fièn? Alèn?

\Attilaspeaks Ouè, ouè si qué que la Pro-Loco l’at aprestoù eunna mordia pe tcheut. 

\Diegospeaks Ad\'on vito alèn!

\StageDir{Chorton tcheu dou. Se tchouèyon le lemie \lemieBa .}

\scene[-- L'inaougurach\'on]

\ridoiver

\StageDir{Se avion le lemie \lemieSi .\\ Totta la pièse vionde a l'entor d'eunna machina de distribuch\'on  de l'éve, plachaye a gotse di palque. La particularitoù de si distributeur l'è que baille tan de calitoù d’éve. La machina l’at maque eun robinet é dameun lo distributeur l'è plachà eun grou pleuro. Pe l'inaougurach\'on la machina l'è topaye pe eun grou lenchoueu. \\ Eun scène, a gotse, douàn lo distributeur, n'at Attila, eun vallet de veulla, lo Seunteucco é lo préye avouì son servàn de messa; a drèite, n'at totta la populach\'on : lo cap di groupe di-z-alpeun avouì leur gonfal\'on, la tsantiì, eun fotografe, dou mèinoù de la Gaie Famille é bièn d'atre dzi.\\ Si lo fon di palque n'at eunna tèila iaou l'è proyettaye l'icrita \textit{Acqua Santa Colomba}. Pe clloure, comme décor, a gotse l'è plachà eun cartel avouì icrì de mésadzo pe le sitouayèn é a drèite n'at eunna banquetta é eun pal de la lemie.}

\Seunteuccospeaks \textit{Mesdames} é \textit{messieurs}, sitouayèn de Tsarvensoù, bondzor a tcheut é bienviì a l'inaougurach\'on de seutta moderna machina, frouì de la meillaouza tecnolojì!

\StageDir{Le dzi a drèite boueuchon di man.}

\Seunteuccospeaks Diavo, machina que baille totte le calitoù d’éve poussible é immajinable!

\StageDir{Le dzi boueuchon di man.}

\Seunteuccospeaks Mersì, mersì. Diavo, que baille totte calitoù d’éve poussible é immajinable é que ma jeunte é mé n'en battéyà: La Bouteucca de l’éve!

\StageDir{Le dzi tornon bouechì di man \bouechiman .} 

\Seunteuccospeaks Mersì, mersì. Seutta l'é la premiye Bouteucca de l'éve que ivrèn eun Val d'Outa. Vio de cou atsetèn de botèille d'éve que sen pa de iaou veugnon? \'E ad\'on l’Amministrach\'on diqué l’a fé? L'a pensoù de beté 18 caletaye d’éve\ldots

\StageDir{Lo vallet de Veulla bloque lo Seunteucco é lèi di de pa sparé tro ate avouì le num\'er\'o.}

\Seunteuccospeaks \ldots 15 caletaye d’éve\ldots

\StageDir{Comme douàn: lo vallet de veulla consèille i Seunteucco de calé.}

\Seunteuccospeaks \ldots Dji caletaye d’éve
 diffiente!
 
 \StageDir{Le dzi tornon bouechì di man todzor pi for.} 
 
\Seunteuccospeaks  Dji caletaye d’éve
 diffiente pe bailli-vo la possibilitoù de chédre queunta vo euntéresse. Mersì a eun pégno èidzo réjonal n'en pouì beutté a si distributeur eun \textit{software} de dérì jénérach\'on. Pe chédre l'éve que vo euntéresse baste eunsérì \direct{eun terièn foua eun paquet de \textit{tessere}} seutta \textit{tessera}\ldots
 
  \StageDir{Dimèn lo Seunteucco baille le \textit{tessere} a la populach\'on.} 
 
\Seunteuccospeaks \ldots  chèdre la tipolojì d’éve volua é voualà que chor lo prodouì que v'ouèi comandoù. Eh ouè mé cher sitouayèn sit cou pouade eumpléyé l'éve de voutra quemeun-a pe féye de tot: se pou seumplemàn la bèye naturella ou \textit{mousseux}, eumpléi-la pe varì sertène patolojì é, se voulade, finque la vardé desù la crédense comme éve bénédiya!

\StageDir{Le dzi tornon bouechì di man bièn for.} 

\Seunteuccospeaks \ldots é fenèi pa lé! L’amministrach\'on comunala l’a pensoù a eun discour sosial: eun se trouèn seuilla a féye la quiya, pe prende l’éve, le dzi l’an la possibilitoù de se cognitre é de sosializé! Pe tcheut sise motif vouillo remersié totte le-z-aoutoritoù: ma jeunte é to lo Consèille comunal, le consèillì réjonal, lo chef di groupe di-z-alpeun, lo serdzèn di pompì, Monseur de noutra parotse\ldots

\Donspeaks \direct{Eun nen pouèn pamì} Amen! \stufo

\StageDir{Lo Seunteucco s'arite é comprèn d'itre tro tchatchar\'on.}

\Seunteuccospeaks \ldots é tcheu vo-z-atre seuilla prézèn!

\StageDir{Le dzi boueuchon di man bièn for.}

\Seunteuccospeaks Dimandériyo ara a Monseur, Don Carmelito, se pou no baillì eunna pégna bénédich\'on a seutta machina, a la Bouteucca \direct{eun gavèn ià lo lenchoueu que toppe la machina} de l'éve!

\StageDir{Le dzi boueuchon tourna di man. Dimèn lo prée avanche, ver lo mentèn di palque, é lo fotografe se plache douàn le dzi pe fé de fotografie.}

\Donspeaks Mersì Madame lo Seunteucco. Vouillò djeusto remersié eunc\'o mé l'amministrach\'on pe sen que l'a fé pe no parrouassièn. L'éve\ldots

\Dzispeaks Amen!

\Donspeaks \ldots l'è eunna rezoursa que l'a queuttou-no noutro Bon Djeu, mi que chouèn l'é pa appréchaye pe le dzi. L'éve,

\Dzispeaks Amen!

\Donspeaks que l'a fé é feré todzor de miacllo l'é lo bièn pi eumportàn pe vivre\ldots pe la viya de l’ommo. \'E aprì que dimandé de pi a noutro Seunteucco que, mersì a sa sensibilitoù, l’a beuttoù eunc\'o la calitoù de l’éve \ldots

\Dzispeaks \ldots Amen!

\Donspeaks bénédiya?

\Dzispeaks \textit{Ora pro nobis}!

\StageDir{Dimèn lo Seunteucco se fé fé de foto \flash\ pe la Vallée.}

\Donspeaks Ad\'on a si poueun baillerio eunna pégna bénédich\'on a la Bouteucca de l'éve.

\StageDir{Don Carmelito, avouì l'èidzo di servàn de messa, pren l'aspersoir pe la bénédich\'on.
Lo chef di groupe di-z-alpeun baille l'\og attenti\fg é tcheutte repoundon i seun odre. A si poueun, lo prée di caque paolle eun lateun é bénédèi la machina.\\ Aprì la bénédich\'on, la tsantiì ataque eun \textit{Salve Regina} pocca eunton-où.} 

\Seunteuccospeaks\direct{Bièn digoutaye pe la tsantiì} Mersì, ouè, mersì. \direct{I vallet de veulla} Dimanderio ara le fosette\forbici .

\Valletspeaks\direct{Eun s'ajitèn} Na n'i oublià le fosette!

\StageDir{Lo Seunteucco avèitse mal lo vallet, lequel se vionde é spédèi Attila, lo cantognì, a tsertchì de fosette. Dimouèn lo Seunteucco tsertse de contegnì lo discoù.}

\Seunteuccospeaks Ad\'on remerserio euncoa  Monseur de noutra parrotse, lo cappe di-z-alpeun, le pompì, la Pro-Loco que l'a aprestou-no eun petchoù \textit{casse-cro\^{u}te} é aprì remersio eunc\'o la tsantiì\ldots Na! La tsantiì na! Vo remersio tcheu vo seuilla prézèn!

\StageDir{Dimèn dou mèinoù de la Gaie Famille, douàn la Bouteucca de l'éve, sopendon eun lon ribàn rodzo é neur. I mimo ten arreuve Attila avouì de grou sicateue; le baille i vallet, lequel le moutre i Seunteucco. Sensa tan d'atre soluch\'on, lo Seunteucco pren lo sicateue é finalemèn l'è preste pe copé lo ribàn pe l'inaougurach\'on di distributeur.}

\Seunteuccospeaks Decllaro iverta la Bouteucca de l'éve!

\StageDir{Le dzi boueuchon di man, mi lo Seunteucco arreuve pa a copì lo ribàn. Pe l'èidjì arreuve Attila que avouì eun grou opinel lo coppe a mèitchà.\\
Se tchouèyon le lemie \lemieBa .}


\scene[-- George é l'éve naturella]

\StageDir{L’è choui-z-aoue di mateun. Se sen lo:}

\effet{https://soundcloud.com/user-234168361/verso-del-gallo}{Chan di pou}

\StageDir{Desì la Bouteucca de l'éve, iao l'è plachà eun grou pleuro, n'a le dou cantognì que son dza prest pe fé leur servicho. Naturellamente son panco bièn réchà \sbadiglio .\\ A drèite, iao n'at eunna banquetta pe s'achouatì, areuvon le premì dzi: eunna mamma avouì eunna petouda é eunna madama (Claudine) bièn arbeillaye avouì eun crè tseun deun sa boursa. La mamma, rapidamente, avanche ver la machina é teurie eun litre d'éve. Dimèn, Attila, avouì eun grou modeun pe la polenta, modde le potreungue que Diego tappe dedeun lo pleuro. Can la mamma l'a fenì de terì l'éve, chor foua avouì sa feuille é la madama avouì lo tseun s'aprotse i distributeur.}

\Claudinespeaks Oh, a forse! \direct{I tseun} Vitto George! Féyèn totte a galoppe, pèi aprì te pourto a féye la cacca!

\Machinaspeaks \direct{Avouì eunna cadense électrique} Eun-se-rì la \textit{tes-se-ra}.

\StageDir{Claudine teurie foua la \textit{tessera} é l'enserèi deun la machina. Dimèn areuvve eun mètre di-z-esquì (Giaco) avouì an quése de botèille. Semble bièn pris\'o .}

\Machinaspeaks \direct{Comme douàn\footnote{ Dèi-z-ara lo icrièn: C.d.}} Ché-dre la ca-le-toù de l’é-ve.

\Claudinespeaks\direct{I tseun} Donque senque no prégnèn? Senque te di té? Prégnèn naturella!

\StageDir{Gnaque lo bot\'on de l’éve cherdiya. Le dou cantognì l’an pa comprèi queunta éve l’é itaye cherdiya.}

\Machinaspeaks \direct{C.d.} Pré-djì pi for é di-ye i mi-cro-fon-ne l’é-ve cher-di-ya.

\Claudinespeaks Ouff, que complecoù! \direct{Eun s’aprotsèn i microfonne é eun prédzèn pi for} Na-tu-rel-la!

\StageDir{Le dou cantognì, can sentèison la caletoù cherdiya, se boudzon pa é reston tranquilo a leur plase. Dimèn areuvve eugn'atra fenna (Giada) que s'achouatte tranquila si la banquetta.}

\Machinaspeaks Pla-chì le bo-tè-ille.

\Claudinespeaks Ouff, que complecoù!

\StageDir{Claudine plache le botèille.}

\Machinaspeaks Ché-dre la can-ti-toù.

\Claudinespeaks Prègno dou litre. 

\Giacospeaks\direct{Eun tsemièn nervezamente} Ad\'on no no boudzèn?! Nen pouì pamì! Dèyo fé les\'on mé é l'è do-z-aoue de ten que si protso \direct{eun moutrèn Giada} a seutta stoufianta!

\Claudinespeaks \direct{Épouvantaye} Eun momàn, tchica de pachense!

\Giacospeaks Pachense? Pachense pe senque? Pe prende d'éve naturella? Mi va ba eun Djouiye la prendre! Saquetta, mi rècha-té lo mateun! 

\Claudinespeaks\direct{Offenchaye} Mé me rècho totte le mateun pequé n'i George da porté féye le sin-e bague\cacca .

\Giacospeaks Qui t'a?

\Claudinespeaks\direct{Eun moutrèn la boursetta} George!

\Giacospeaks Senque l'è eun lapeun \coniglio ?

\Claudinespeaks L'è lo meun tseun!

\Giacospeaks Ah té t'i eunna de salle veullatchiye que vardon le tseun pe le mitcho é can l’an fata de fé leur bague le porton ià desì le terrèn di-z-atre.

\Claudinespeaks Que dzen carattéro! Voualà n’i fenì, si cou pouì alé-me-nèn!

\StageDir{Claudine se nen va eun borbottèn. Giaco se plache douàn la machina.}

\scene[-- Dji litre, pa de pi!]

\Giacospeaks\direct{Avouì eun ton amical} Tchao Machina!

\Machinaspeaks \direct{Todzor avouì eunna cadensa électrique} Tchao Gia-co!

\Giacospeaks Ad\'on, véyèn tchica senque n’at: mousseu na, éve naturella, éve djeusto que boudze é éve que bouleuque\ldots Oh voualà la min-a: microfiltraye i Tsarboùn Actif Ionizaye.

\Giadaspeaks Ah, l’é ti bon-a salla?

\Giacospeaks\direct{Bièn spac\'on} Seutta? Seutta l'é la rèina di-z-éve! Te fé féye de rotte que te le sen tanque a Seneun! 

\Machinaspeaks  Eun-se-rì la \textit{tes-se-ra}.

\Giacospeaks Eun momàn de pachense!

\StageDir{Pierre eunserèi la \textit{tessera} é plache eun boteill\'on. Aprì selechoue la caletoù é le dou cantognì gnoun-on a tappé ba deun lo pleuro totte sorte de pousse blantse.}

\StageDir{Aprì doe seconde se sen eun son deun seblé: lo vallet de veulla  entre eun scène é se plache douàn Giaco.}

\Valletspeaks \direct{A Giaco} Mesieu, se pou pa! 

\Giacospeaks Te di a mé?

\Valletspeaks Ouè, lo rèillemèn prévèi la possibilitoù de prende dji litre é vo v'ouèi chouì boteill\'on, que vou diye doze litre, que vou diye dou litre eun pi!

\Giacospeaks Ah v'ouite bon-a a fé le contcho! Ad\'on mé beutto pa plen le boteill\'on!

\Valletspeaks Se pou pa, la machina sa pa que dèi s'arité a dji litre!

\Giacospeaks\direct{Ironique} Le \textit{software} moderno! Bon, vou diye que eumplèiso pouì maque sinque boteill\'on.

\Valletspeaks Ouè!

\StageDir{Giaco conteneuvve a eumpleure le boteill\'on. Areuvve eunc\'o eun alpeun que s'achouatte si la banquetta pe atendre.\\ Dimèn le dou cantognì traillon pe vouidjì l'éve cherdiya.}

\Valletspeaks No no boudzèn?

\Giacospeaks Eun momàn de pachense!

\Giadaspeaks Mesieu Rotto, no no baoudzèn!

\StageDir{Aprì èi eumplì lo seunquimo boteill\'on, Giaco pren la cassa é s'apreste a partì.}

\Giacospeaks Orvouar! Ah se v'ouèide fata de eunna les\'on, vo queutto lo beillette \direct{eun terièn foua di pourtafoille dou beillette é eun le baillèn i vallet é a Giada}, é queriade-mé! Tanque! 

\StageDir{Giaco chor é finalemèn Giada teurie son litre d'éve.}

\Giadaspeaks \direct{I vallet, eun chortèn avouì sa botèille} Mersì é orvouar!

\Valletspeaks Orvouar.

\scene[-- Pe lo bièn de l'\`Eillize]

\StageDir{Totse a Pierluigi, lo chef di groupe di-z-alpeun. Se plache douàn la machina avouì lo barlet de 25 litre. Lo vallet de veulla, que vionde todzor per lé a l'entor di distributeur, can vèi lo grou conténiteur seuble to de chouite.}

\Pierluigispeaks Mi se pou pa féye eugn'esséch\'on? L’é pa pe mé, mi pe lo sièje di groupe!

\StageDir{Dimèn si la banquetta s'achouatte eunna Seur.}

\Valletspeaks Se l’é na, l’é na! Seuilla son tcheut igale.

\Pierluigispeaks N'i comprèi que sen tcheut igale, mi comèn fio oueu la viproù avouì le meun-z-alpeun sensa l'éve de sen Grat?

\Valletspeaks Ou vegnade avouì de botèille eun riilla ou piatro bèyade l'éve i rubinet de mèiz\'on. Va bièn?

\Pierluigispeaks Ouff! Fio pi que téléfon-ì a Piero\ldots mogà me pourte le botèille djeuste.

\Valletspeaks Ouè queriade Piero.

\StageDir{Pierluigi pren lo téléfonne é criye son amì alpeun. Lo vallet l'eunvite jantillamente a se tramì avouì son conténiteur pe quetì la plase i-z-atre. Dimèn entre eun vétchot (Prospero) avouì chouì botèille é s'achouatte protso de la Seur. L'è vioù, mi avouì sa baquetta tsemie leste.}

\StageDir{Partèi eunna réclamme que sponsorize l'éve de \textit{Sainte Colombe}\footnote{ Malerezamente sen pa arrevoù a troué seutta \textit{vidéo}. A mémoué, pouèn deu que n'ayè eunna boteille d'éve que colatè ba pe eun prou. Ester Bollon, métresse di-z-esquì, la chouivèi tanque i fon di prou. A la feun, pren la boteille é braille: \og L'\'eve de Senta-Colomba, se te la bèi l'è eunna bomba!  \fg.}.} 

\StageDir{Eunna personna attraverse lo palque.}

\Seurspeaks Mesieu lo vallet!

\Valletspeaks Ouè Madama!

\Seurspeaks Poui-dzò passì?

\Valletspeaks Ouè vegnade maque.

\Seurspeaks Mersì!

\StageDir{La Seur se levve é s'aprotse i distributeur.}

\Seurspeaks\direct{A Pierluigi} Bondzor Mesieu! Mi v'ouèide vi que dzenta la poublisitoù?! Finque paì Messner pe fiye la reclamme; la quemeua l'aré spendì cheur pa pocca!

\Pierluigispeaks Ah cheur! Se prégnaon la Chiabotto cheur lèi coutave mouén!

\Seurspeaks Sen si beun pa.

\StageDir{Comme douàn, la machina dimande la \textit{tessera}.}

\Seurspeaks\direct{Eun refléchisèn} Ad\'on mé la \textit{tessera} iaou l'è que n'ayoù pouzo-la? Ah ouè, inque pouèi\ldots

\StageDir{La Seur se teurie si la soutane pe terì foua di tsaus\'on la \textit{tessera} é dimèn Pierluigi é lo vallet avèitson queriaou si drolo de comportemèn. Aprì èi prèi la \textit{tessera}, la Seur l'eunsérèi deun lo distributeur.}

\Machinaspeaks Chédre la cale-cale-cale\ldots

\StageDir{Pierluigi baille eun creppe i distributeur. Dimèn entre eugn atro vétchot que s'achouatte protso a Prospero.}

\Machinaspeaks Caletoù de l'éve.

\Seurspeaks  \'Eve de \textit{Sainte Colombe}!

\Pierluigispeaks Mi squezade-mé: salla l'è la pi tchiya; nen va la pèin-a?

\Seurspeaks Ouè, ouè créyade-mé, Mesieu. Pensade que mé si arrevaye pocca fé de Lourdes é eunc\'o aoutre per li prèdzon de noutra éve. L'éve de \textit{Sainte Colombe} comme éve de guérizòn!

\Pierluigispeaks Mi pensa té.

\Seurspeaks Ouè é inque a dou dzor torno aoutre avouì le maladdo di péì. Ad\'on pensavo de pourté eun per de litre é de le vendre. Que restise euntre no\ldots \direct{eun bèichèn la vouése} inque la payo satte santime é aoutre la vendo a seun euro i litre! Pe lo bièn de l’èillize eh!

\Pierluigispeaks Ouè cheur\ldots lo bièn de l'èillize! Mi eun éffé l'è pa mal comme magniye pe féye le sou. Caze caze dèyo prédji-nen avouì lo directif: se arrevisan eunc\'o no a vendre la noutra de sen Grat\ldots magà eunsemblo i \textit{vin brulé} a la véillà de Tsarvensoù\ldots

\Seurspeaks Crèyade-mé que l'è eun bon affiye. Voualà! Fenì eunc\'o mé. Vo v'ouèi dza prèi la voutra?

\Pierluigispeaks Na. N'i pa pouì pequé lo conténiteur de mé \direct{eun desuèn lo vallet} va pa bièn.

\Seurspeaks Ah\ldots eummajin-o que vo prégnade l'éve de sen Grat, vrèi Mesieu?\direct{Se fé lo signe de la croueu} Se vèi de louèn!

\Pierluigispeaks Ah cheur\ldots é aprì se t'i alpeun te la paye eunc\'o mouèn. L'é cllè que te fa alì eun Quemeua fiye eun papì asprése pe seutta dzenta é comodda ajévolach\'on. Mi squezade\ldots se vouillade pouade vegnì eunc\'o vo amì di-z-alpeun! Pouai pe tsaque sinque litre que teriade reusparmiade eunna santima!

\Seurspeaks\direct{\'Emochon-aye}  Eunna santima! Mersì Mesieu que v'ouèide di-me seutta bagga fran dzenta! Ara fa scapì, mi lèi penso pi! Mersì é\ldots eunna pégna bénédich\'on a tcheutte.

\StageDir{Baille la bénédich\'on é aprì chor.}

\scene[-- Prospero]

\StageDir{Ià la Seur, Pierluigi vèi arrevé Piero.}

\Pierluigispeaks\direct{Eun chortèn} Oh l'è arrevoù Piero!

\StageDir{Di momàn que n'a pa gneun douàn lo distributeur, Prospero se levve de la banquetta, pren le botèille é avouì la baquetta trampèille ver lo distributeur. I mimo ten, Pierluigi entre é acappe Prospero douàn de llou. Tsertse, donque, de lo pouchì ià.}

\Prosperospeaks\direct{A Pierluigi} Euh na, na! Lèi si mé ara! Fiade pa lo feun!

\Pierluigispeaks Mi senque diade? Mé l'é dza eun momàn que si seuilla é ara totse a mé.

\Prosperospeaks Mi lèi penso fran pa. Mesieu lo vallet sit fé lo feun!

\StageDir{Lo vallet seuble é se beutte i mentèn di dou. Dimèn areuvve eunc\'o l'atro vétchot pe queriaouzé.}

\Valletspeaks \direct{A Pierluigi} Te pou pa fiye comme t'a voya té! Te fa torna fiye la quiya. \direct{I vétchot} Vo Mesieu, soplé, réstade achouatt\'o lé que lèi penso mé!

\StageDir{Lo vallet acompagne lo vétchot a la banquetta é aprì tourne protso i distributeur pe controlé la discuch\'on.}

\Prosperospeaks\direct{A Pierluigi} La quiya l'é la quiya.\direct{I vallet} L'an la plima eun tita é penson de itre va saì qui! Alisan ba a l'Adunata!

\Pierluigispeaks Mi Mesieu lo vallet, si seuilla dèi sinque mouèn car é pouì pa tournì i fon djeusto perqué n'ayoù pa le conténiteur djeusto.

\Valletspeaks Vouillade eun verbal?

\Prosperospeaks Ouè, vouillade eun verbal?

\Valletspeaks \direct{A Prospero} Souplì, lèi si mé pe seutte bague!

\Prosperospeaks Ouè, ouè, mi fiade-lèi maque eun dzen verbal. Se lo meutte!

\Pierluigispeaks\direct{Démoralizoù} Ad\'on vouì pa lo verbal; \direct{ironique} tourno aoutre eun Veulla é voualà! Tante n'i to lo dzor! 

\Valletspeaks Ouè, sayo!

\StageDir{Pierluigi pren son conténiteur é s'achouatte si la banquetta.}

\Prosperospeaks\direct{I vallet} Te vèi que lèi fièn pouiye a sit eh?!

\Valletspeaks Ouè, ara totse a vo, mi boudzade-v\'o! 

\Machinaspeaks Eun-se-rì la \textit{tes-se-ra}.

\Prosperospeaks\direct{I vallet} Cougnì vito!

\Valletspeaks Ouè boudzade-v\'o que n'a tchica de dzi, soplé.

\StageDir{Prospero teurie foua di pourtafoille la \textit{tessera} é l'eunfeulle deun lo distributeur.}

\Machinaspeaks Eun-se-rì la \textit{tes-se-ra}.

\Prosperospeaks Mi n'i djeusto eunseri-la! Se vèi que la li pa!

\StageDir{Prospero pren la \textit{tessera} é la poulite: la lètse é lèi souffle desì. Aprì tourne la catchì dedeun.}

\Machinaspeaks Ché-dre la ca-le-toù de l'éve.

\Prosperospeaks\direct{Eun plachèn la botèille} Ad\'on la caletoù de l'éve\ldots me rapello pamì, n'ayoù icri-lo seu mateun si lo beillette mi n'i perdi-lo\ldots senque l'ie\ldots ah ouè!  V-Power de Mori\'on! Dèi can bèyo so n'i pamì de mou a l'itsin-a é avouì la fenna totte semble pi amoddo. Si pa se me compregnade Madame lo vallet\ok !

\StageDir{Le cantognì tappon ba pe lo pleuro de drole pouse.}

\Valletspeaks\direct{Eun l'eunvitèn a pensé i sin-e botèille} Ouè, ouè soplé !

\StageDir{Entre eun volontéo avouì an \textit{tanica} de 15 litre é s'achouatte si la banquetta. Dimèn Prospero teurie sa éve avouì bièn de calma.}

\Prosperospeaks Mi\ldots eh ouè\ldots l'a fi beur totta la senà é baille tourna beur! Ren a fiye! Seutta senà l'a pa voya de chotre si solèi!

\Prosperospeaks\direct{Eun avèitsèn foua di palque} Tchao, tchao! Ouè sen seu a proué lo moublo nouo! Ouè, ouè, la fenna lèi va beun, l'è i mitcho tranquila. Mi son tcheu dou de té? Mondje se l'è crèisì lo premì! Mi l'è igala a mamma, toutouteun a la mamma! Que dzen botcha; te queutto que la botèille l'è plèin-a. Tchao, tchao!

\StageDir{Prospero plache eugn'atra botèille é conteneuvve a fé la conta.} 

\Prosperospeaks\direct{I vallet} Si pa se le cougnisade\ldots

\Valletspeaks Eh na!

\Prosperospeaks Llou l'a marià\ldots atèn\ldots me rapello pamì se\ldots na \direct{i vallet} vo cougnisade pa!

\StageDir{Prospero se beutte a seblì \seble\ eun pégno refrain.}

\Prosperospeaks\direct{I dzi achouattoù} Deh vouillade savèi eunna conta que fé riye?

\StageDir{Le dzi achouattoù si la bantse repoundon: \og Na mersì!\fg.}

\Prosperospeaks Ad\'on vo la counto! N’atte eun Italièn, eun Fransé é eun Angllé\ldots 

\Vetchotspeaks \ldots mi l'ie pa eun Maroqueun?

\Prosperospeaks Mi te la sa pa! Comèn te fi a saèi que l'ie eun Maroqueun?

\StageDir{Douàn de contenevì, Prospero s'apersèi que la botèille l'è plèin-a! Galoppe (eun trampeillèn) ver lo distributeur pe plachi-nen eunna vouida.}

\Prosperospeaks\direct{Tourna ver le dzi achouattaye} Ad\'on tourno diye: n'ayè eun Maroqueun, eun Fransé é eun Angllé\ldots é si cou n'at lo Maroqueun. Son ià pe lo bouque é eun l'a frette pequé l'è iveur. Ad\'on sitte di a l'atro:\og Beutta de bouque \legna\ i fornet pequé fi frette\fg . L'atro lèi di:\og Nen n'ario! Lo ferio bièn volontchì!\fg .

\StageDir{Prospero se beutte a riye to solette. Tcheu le-z-atre s'avèitson sensa diye ren.}

\Prosperospeaks Seutta l'è eunna di pi dzente que si!

\scene[-- 25 litre pe le volontéo]

\Volonterospeaks \direct{A Prospero} Ouè, v'ouite a poste Mesieu?

\Prosperospeaks Ouè si arrevoù i dérì litre! 

\StageDir{Lo volontéo se levve avouì sa \textit{tanica} é s'aproste a Prospero.}

\Prosperospeaks Iaou alade-v\'o avouì si conténiteur?

\Volonterospeaks Mé si volontéo! No volontéo n'en la convench\'on! Tanque a 25 litre pouèn terì.

\Prosperospeaks Na, na, seu pouade maque teryì tanque a 15 litre!

\Volonterospeaks Na mi mé n'i beun de papì \direct{eun tsertsèn deun le secotse}\ldots voualà!\direct{Eun terièn foua eun foillette} L'é to icrì neur desì blan: volontéo di secour, 25 litre!

\StageDir{Prospero pren lo foillette é se beutte a lie eunsemblo i vallet de veulla. Dimèn s'euntremelle eunc\'o lo vétchot.}

\Prosperospeaks\direct{I vétchot} Mi sit senque vou?! L'è tcheu le momàn seu!

\Valletspeaks\direct{Eun prégnèn pe eun bri lo vétchot} Soplé! Alade vo achouattì que seu lèi penso mé!

\StageDir{Lo vallet accompagne lo vétchot a la banquetta.}

\Volonterospeaks\direct{I vallet} Todzèn i mesieu \direct{eun moutrèn lo vétchot} que l'a tchica de problème de santé\ldots

\Valletspeaks Vouillade eun verbal eunc\'o vo?

\Volonterospeaks Na, na.

\Prosperospeaks\direct{I vallet} Vegnade vèi sé!

\StageDir{Eunsemblo lion lo foillette. Lo vallet comprèn to de chouite que l'è totte eun riilla, mi Prospero lèi beutte bièn pi de ten a lie é semble pa tan convencù.}

\Prosperospeaks Ouè semblerie beun tot a poste.

\Volonterospeaks Me baillade lo papì?

\StageDir{Prospero ren lo papì i volontéo.}

\Prosperospeaks Ad\'on teriade maque.

\StageDir{Prospero gave ià la dérie botèille di distributeur é reprèn sa \textit{tessera}.} 

\Prosperospeaks \ldots pregnèn la \textit{tessera}, vi que salle que n'i queuttoù dedeun le-z-atre machine l'an foutu-me seun \textit{euro}! Seutte machine!

\StageDir{Dimèn lo vallet lèi pren lo tsaèn di botèille pe bailli-lei eunna man é pe fé pasé lo volontéo.}

\Prosperospeaks Mi iaou alade avouì le min-e botèille!

\StageDir{Prospero avouì eun sate pren lo tsaèn, mi lo vallet lo molle pa. Comenchon avouì bièn d'énerjì se terì ver leur lo tsaèn, tcheu dou eun braillèn: lo vallet vou maque baillì eunna man, Prospero vou pa itre èidjà. Aprì eun per de seconde, lo tsaèn tsi pe tèra. Lo volontéo galoppe per recoillì le botèille.}

\Prosperospeaks\direct{I vallet} Lazar\'on!

\Valletspeaks Ah sitte l'è lo remersiemèn que me baillade?! Mé \direct{eun tramèn lo tsaèn de douàn lo distributer} vo baillo eunna man\ldots

\Prosperospeaks \ldots é tourna!

\StageDir{Lo volontéo baille a Prospero eunna botèille que l'ie tsiziya.}

\Volonterospeaks\direct{A Prospero} Mesieu lo \textit{pacemaker} l'è a poste?

\Prosperospeaks\direct{Inervoù}  Ouè! Vouillè \direct{ver lo vallet} me pourtì  ià le botèille!

\Valletspeaks Na, n'i maque bailla-vo eunna man!

\Prosperospeaks\direct{Todzor pi malechà é offenchà} Mi n'i pa fata de vo pe sen! N'i fata d'atro!

\StageDir{Eunc\'o eunna mia malechà, Prospero pren son tsaèn é se plache protso di distributeur. Dimèn lo volontéo se plache douàn le distributeur é eunserèi la \textit{tessera}.}

\Volonterospeaks Donque\ldots \direct{eun lièn le caletaye} sen Grat, Mori\'on, Tsarb\'on\ldots ah voualà: l'éve \textit{effervescente} naturelle de Ponteille. D'abor le medeseun dion que l'è l'éve pi boun-a pe le maladdo!

\Prosperospeaks \'Eve?

\Volonterospeaks Ponteille! Salla que reboudze ba pe lo tor\'on!

\StageDir{Lo volontéo plache lo conténiteur déz\'o la brotsetta. Entre Luciano eun saoutèn la quiya.}

\Pierluigispeaks\direct{A Luciano} Oh mi iaou alade? N'a la quiya!
 
\Lucianospeaks Acoutta, mé si eun retar!

\Pierluigispeaks \ldots é senque m'eunteresse que t'i eun retar?!

\StageDir{Lo vallet eunterveun pe bloqué la discuch\'on que dimèn s'itsaoude euntre Luciano, Pierluigi é lo vétchot. I mimo ten, pe queriaouzé, s'aprotse eunc\'o Prospero.\\ Aprì eun per de seconde la situach\'on se calme é Luciano vèi lo grou conténiteur di volontéo.}

\Lucianospeaks Ouè mi sitta véo de litre l'at a terì?


\Prosperospeaks\direct{Eun prégnèn da par Luciano} Na l'é totte a poste.  N'i dza totte avèitchà mé. 

\StageDir{Lo vétchot se levve eugn atro cou de la banquetta pe acouté sen que Prospero di, mi i mimo ten lo vallet lo reprodze.}

\Prosperospeaks Ad\'on n'i dza totte avèitchà mé é sitte pou terì tanque a 25 litre pe dzor pequé l'è di secour.

\Lucianospeaks Vouè mi mé a do-z-aoue vou travaillì. Si dza eun retar! 

\Prosperospeaks Attégnade pi eun momàn eun pi, comme tcheut! No n'en tcheut atégnì!

\Lucianospeaks Ouè mi oueu l'é demicro!

\Prosperospeaks \'E beun é ad\'on?

\Lucianospeaks V'ouite todzor a tsapléttì é vo sade pa le bague pi eumportante. Lo demicro l'éve avouì lo gas l'é bièn pi gazaye!

\Prosperospeaks Mi pe dab\'on? \confuso\ Comèn fiade a savèi so?

\Lucianospeaks La fenna de mé, l'atro dzor, l'é vin-a seuilla i distributeur devàn que l’iverteua é dèi avèi sentì, si pa da qui, seutta dzenta novalla. Mi diade-l\'o pa a tcheut\ldots perqué piatro fièn trop de quiye lo demicro é a mé totsériye magà partì demì aoura devàn de méiz\'on.

\Prosperospeaks Na, na, tranquilo diyo ren a gneun!

\Lucianospeaks \direct{Ver lo volontéo} Vouè mi sitte l'è eunc\'o seuilla. \direct{A Prospero} Acoutade: pouì queuttì a vo lo tsaèn di botèille?

\Prosperospeaks Mi n'i pa pouì to si ten. Mé n'i eunc\'o lo courtì, le jirani\'on\ldots n'i pa lo ten!

\Lucianospeaks Jirani\'on d'iveur?

\Prosperospeaks Mi a no d'Ampaillan areuvon douàn!

\Lucianospeaks\direct{Eun baillèn eun man a Prospero lo tsaèn} Fiyo passì pi tar la fenna lo reteryì! \direct{Eun lèi baillèn la \textit{tessera}} Voualà la \textit{tessera} é supergazaye de Fontan-a Tsada.

\Prosperospeaks Ouè vèyo sen que pouì fiye. Si pa cheur\ldots 

\Lucianospeaks Mi ouè l'è an bagga de dji meneutte de ten. Mersì!

\Prosperospeaks Salì.

\StageDir{Luciano galoppe ià. Dimèn lo volontéo l'a fenì d'eumpleure lo conténiteur de 25 litre.}

\Prosperospeaks\direct{I volontéo} Sit cou l'è plen!

\Volonterospeaks Eh ouè, pouèi le maladdo son contèn! Orvouar!

\StageDir{Lo volontéo chor eun trèinèn la \textit{tanica} plèin-a d'éve.}

\scene[-- La flemma di vétchot \viou]

\Valletspeaks\direct{I vétchot}  Voualà Mesieu totse a vo ara! Plan planotte, alade maque.

\StageDir{Avouì an flemma que la mèitchà baste, lo vétchot se levve é, to todzèn, s'aprotse i distributeur. }

\Vetchotspeaks\direct{I vallet} Mi senque fa fiye?

\Valletspeaks Se vouillade l'éve, teriade l'éve\ldots mi vo manque la botèille. Se vouillade pouade la dimandé i Mesieu \direct{eun moutrèn Prospero} que l'è bièn jantillo\ldots

\StageDir{Aprì èi eunserì la \textit{tessera}, lo vétchot s'aprotse a Prospero.}

\Vetchotspeaks \direct{Eun prégnèn eunna botèille di tsaèn de Luciano} Mesieu, que v'ouite tan jantillo, pouì prendre eunna botèille?

\Prosperospeaks\direct{Aprì èi avèitchà mal lo vétchot} Mi ouè teriade maque voutra éve.

\StageDir{Lo vétchot tsertse de plachì la botèille, mi sa man tremble trop. Prospero eunterveun eunfastedjà.}

\Prosperospeaks Mi èita! Deh so va rotte! L'è nouo! Se v'ouite pa bon mandade la fenna! Beutto mé!

\StageDir{Prospero pren la botèille é la plache dez\'o la brotsetta.}

\Valletspeaks Todzèn!

\StageDir{Aprì èi terià l'éve, lo vétchot reteurie la \textit{tessera} é, accompagnà pe lo vallet, chor foua.}

\StageDir{Pierluigi s'aprotse i distributeur, teurie finalemèn l'éve pe le seun-z-alpeun é chor.}

\StageDir{Prospero pren lo tsaèn de Luciano é teurie l'éve. Dimèn entre eunna fenna poua, avouì eun grou cartel de cart\'on avouì l'icrita \og SI SENSA SOU\fg é s'achouatte pe téra pe dimandé l'armoun-a \elemosina .}

\Matassaspeaks Avèitsa sé que dizastre! Gnenca eun poste libro pe pouzì le dzin-aou, i mentèn de sise escremèn de tseun é tsatte! Le dzi l’an fran pa de reuspet pe no pouro matasse sensa sou.

\Prosperospeaks Que no euntéresse a no!

\StageDir{Dimèn que la botèille se reumplèi, Prospero s'aprotse a la fenna pe liye lo cartel.}

\Prosperospeaks Senque l'è so?\direct{Eun lièn mal} Si-sensa-sou, sisensa-sou, sisensasou, si-sensasou!

\scene[-- John, Jack é l'amour]


\StageDir{Entron, man deun la man, Jack é John, eunna cobla \textit{homosexuelle}.}

\Jackspeaks\direct{Tot(ta) grasieu(za), ver Prospero} Oh Prosper!

\Johnspeaks\direct{C.d.} Oh Prospé!

\Prosperospeaks Oh mondjeu!

\StageDir{Le dou s'aprotson a Prospero pe bailli-lei eunna man: tsertson de lèi passé le botèille pe terì l'éve.}

\Prosperospeaks Beuttade ba seutte botèille! N'i pa fata de vo-z-atre! Alade ià!


\StageDir{Avouì la baquetta é lo tsaèn pe le man, Prospero tsertse de le vardé louèn de llou é tchica pe cou se trame di distributeur.}

\Johnspeaks Ouè mi que caratéo!

\Jackspeaks\direct{A John} Avèitsa Pippi! L'é todzor dzen tournì seuilla, na? \inamourou

\Johnspeaks Ouè, lo poste iaou sen cougni-no, meneun! 

\Jackspeaks Mersì a si poste ma viya l'a prèi fourma. N'i torna vi le couleur a la plase de vère totte blan é ner!

\Johnspeaks Te te rapelle s'i dzor li? Mé me lo oubliériyo jamì! T'a fé-me passì devàn é mé n'ayoù dza perdì lo queur pe té. Mé l'iyo eun tren de chédre la caletoù pouèi é\ldots

\StageDir{John se plache douàn lo distributeur.}

\Jackspeaks \ldots é mé n'i moutrou-te comèn se eumpléyave l'\textit{écran}, pouèi\ldots

\StageDir{Jack se plache dérì John, eun l'eumbrachèn.}

\Johnspeaks Que dzen momàn!  \'E totte mersì a si distributeur.

\StageDir{John, euforique, eumbrache Prospero, que, digout\'o, se reveurie.}

\Matassaspeaks Baillade caque tsouza pe eunna poua femalla sensa sou!

\Jackspeaks Mi avèitsa seutta! Mi senque te vou? \direct{Eun tsertsèn pe le secotse} Mé n'i pa de pise, té John?

\Johnspeaks Mmm\ldots

\Jackspeaks\direct{\malisieu} Fé-me sentì se ta de pise\ldots

\StageDir{Jack, to todzèn, s'aprotse a John pe lo prendre avouì sin-e man, mi John recule jèinó tanque i poueun de pitì le pià de Prospero, lequel, tourna eunfastedjà, avouì la baquetta baille eun creppe si lo qui de John.}

\Johnspeaks Ahia!

\Jackspeaks\direct{A Prospero } T'a fé mou a John! \malecha

\Prosperospeaks\direct{A John} Mi fé-té feun!

\Johnspeaks T'a fé-me de mou!

\Jackspeaks Queutta pédre John, veun sé. \direct{A la fenna sensa sou} T'a lo léteur di carte?

\Matassaspeaks Na.

\Jackspeaks \'E ad\'on ren. N'en pa de pise.

\Matassaspeaks Ah, ad\'on demàn m'atresso pi eunc\'o mé. Prègno pi lo lecteur di carte!

\Jackspeaks  Ouè, ouè, va bièn, pourta sen que t'a voya. \direct{A John} John! Cherdèn la caletoù de l'éve.

\Johnspeaks Douàn cherdèn la lenva.

\Jackspeaks  Djeusto! Ad\'on: italièn\ldots

\Johnspeaks Italièn\ldots naaa!

\StageDir{Dimèn entre eunna femalla d'orijine tunizine avouì seun mèinoù.}

\Jackspeaks Fransé?

\Johnspeaks Naaa!

\Jackspeaks Patoué?

\Johnspeaks Patoué!

\Jackspeaks Oh! Patoué particuillé \eccitato !

\Johnspeaks Particuillé!

\StageDir{Jack é John se totson lo dèi eun fièn le maron-ette. Dèi-z-ara la machina prèdze avouì eunna vouése féminile.}

\Machinaspeaks Eun-se-rì la \textit{tes-se-ra}.

\Jackspeaks John t'a té la \textit{tessera}?

\Johnspeaks Ouè.

\StageDir{John se viounde pe moutrì lo qui a Jack é lèi fé comprendre que la \textit{tessera} l'è dedeun la secotse di pantal\'on. \'Emochon-où, Jack la pren é la eunserèi deun la machina.}

\Machinaspeaks Tan-que i fon!

\Jackspeaks\direct{Eun pouchèn la \textit{tessera} tanque i fon} Tanque i fon!

\Machinaspeaks \textit{Tes-se-ra} pa va-li-da.

\Jackspeaks Pa valida! John! L’è pa salla djeusta! Iaou l'è salla djeusta?

\Johnspeaks Ah ouè, adoùn l’è dedeun lo bourset.

\StageDir{John s'aprotse a Jack eun lèi moutrèn lo bourset. Le dou moungouyon eun momàn pe tsertchì la \textit{tessera} deun lo bourset. Can l'acappon fan eun viondolet é a la feun l'eunserèison deun lo distributeur.}

\Machinaspeaks Pla-chì le bo-tè-ille.

\Jackspeaks Le ou\ldots

\Johnspeaks \ldots la?

\Machinaspeaks\direct{Avouì eunna vouése grave é pouissante} Eunna botèille!

\StageDir{John pren eunna botèille deun la boursa.}

\Johnspeaks La botèille! Pégno Jack, prégnèn torna l'éve que n'en prèi si dzor li?

\Jackspeaks Ouè! Raffor plus plus plus! Eun prodouì formidablo!

\Johnspeaks Euh Prospé! Seutta nite te veun danchì avouì no ?

\Prosperospeaks N'i pa lo ten pe sen! Fiade vo voutre bague!

\Jackspeaks Tante no t'envitaon pa Prosper!

\StageDir{Jack é John prègnon leur botèille plèin-a d'éve.}

\Jackspeaks Ara no prégnèn la botèille é la pourtèn i mentèn di verte de Comboé comme si cou lé é la ditoppèn\ldots é can la ditopèn ferè: pouf!

\Prosperospeaks Mi sampouza fio pi mé pouf!

\Johnspeaks Jack, la ditoppèn pi plan plan que lo pouo \textit{forcello} pouise s'accoblì tranquillamente! 

\StageDir{Le dou salion Prospero. John comenche a chotre ver la gotse.}

\Jackspeaks John! Mi atèn-mé!

\StageDir{John tourne eun dérì é Jack lèi pren la man. Tcheu dou sourièn, chorton eunsemblo.}

\scene[-- 1000 litre!]

\StageDir{La fenna tunizine s'aprotse i distributeur avouì seun sinque mèinoù é eun grou tsaèn desì la tita deun loquel n'at eunna matse de botèille.}

\Naimaspeaks\direct{Ver la fenna poua} Bondzor! \direct{A Prospero} Bondzor! Mèinoù, boudzade-v\'o, dérì de mé!

\Prosperospeaks Euh na, na! V'ouèide tro de litre! Me semble tchica trop de conténiteur!

\Naimaspeaks Ad\'on véyade de pa me aggredì de seutta magniye; mé n'i drouet a chouì litre pe mèinoù é ouette pe mé que si mamma sensa ommo.

\Prosperospeaks Fiade pa tan le feun! Fiade maque vère le documàn.

\Naimaspeaks Le documàn?! Qui v'ouite-v\'o pe me dimandì le documàn? 

\Prosperospeaks Madame lo vallet!

\StageDir{Eun seblèn entre lo vallet.}

\Valletspeaks Senque n'at eunc\'o?

\Naimaspeaks Me dimande le documàn!

\Valletspeaks Ah fiade veure le documàn!

\Naimaspeaks Le documàn? Dimandade a mé le documàn?

\Valletspeaks Ouè, ouè a vo. V'ouèide le documàn?

\Naimaspeaks N'i pa de documàn. Dimandade a salla lé \direct{eun moutrèn la fenna poua} le documàn!

\Valletspeaks Na lo dimando a vo.

\Naimaspeaks Mé n'i pa le documàn, n'i maque la carte \textit{oro}. \direct{Eun baillèn la carte i vallet} L'a bailla-me-là la \textit{Communauté de Montagne} é n'i 100 litre d'éve pe mèis gratouì. Mé é le meun mèinoù!

\Prosperospeaks \direct{Eun avèitsèn la carte} So l'è la carte \textit{oro}?

\Naimaspeaks Ouè \textit{oro}, carte d'or. \direct{Eun se reprégnèn la carte} Maque no n'en drouet a so.

\StageDir{Dimèn le mèinoù son achouato-se protso de la fenna poua.}

\Naimaspeaks \direct{Ver le mèinoù} Mi ad\'on mèinoù! Si de lé! Mi l'è pa poussiblo! Vegnade sé, alèn quiì l'éve ara.

\StageDir{Naima baille eunna botèille a tita i seun mèinoù é tcheut eunsemblo comenchon a terì l'éve. Dimèn entre François, lo pi\'on di péì.}

\Francoisspeaks Squezade, qui l'è lo dérì?

\Naimaspeaks Si mé la dériye.

\Francoisspeaks\direct{Eun moutrèn la banquetta} Ad\'on me beutto sé.

\StageDir{Lo vallet, eun vièn François eunna mia eun dificult\'o, lèi baille eunna man pe reustì drette.}

\Prosperospeaks François! L'a fèi t'a trompoù poste !

\Francoisspeaks Na, na.

\Prosperospeaks \ldots pequé sé baillon maque l'éve!

\Francoisspeaks Sitte l'è lo poste djeusto. Pequé demàn n'i le-z-analize. Ad\'on le vioù diyon que se te bèi eunna botèille d'éve de sent'Anna te tchoué tcheu le battère, tcheutte! Proua eunc\'o té!

\Prosperospeaks Mi mé n'i pa fata. \direct{Ver Naima} Euh madama aritade-v\'o, soplé. Pouade fiye passì lo mesieu que tante l'a djeusto eunna botèille? An pégna botèille!

\Naimaspeaks Pensa té se dèyo baillì eunna man a llou! Mé n'i totte le botèille di mèinoù seu!

\Prosperospeaks Fièn maque pasé douàn llou, l'a maque eunna pégna botèille!

\Naimaspeaks Mi penso fran pa, maque de drouet vouillade. Pa poussiblo!

\Prosperospeaks Baillade-mé eunna man, soplé!

\Naimaspeaks \ldots é ad\'on baillèn eunna man!

\StageDir{Naima é Prospero prègnon eun pe coutì lo pouo François, mi, aprì eun mètre, Prospero lo queutte tot a Naima, laquelle, eun trebelèn, areuvve a pourté François i distributeur. \\ Eun vièn an mia doblo, François eunserèi la \textit{tessera} é plache la botèille. Dimèn, Naima trame le seun botcha, louèn de François que l'è todzor preste a tsire.}

\Prosperospeaks \direct{A François} Ad\'on t'areuvve a séléchouì l'éve?

\Francoisspeaks Ouè, ouè\ldots sent'Anna.

\StageDir{François gnaque lo bot\'on de l'éve sent'Anna. \\ Se bèichon le lemie \lemieBa\ é partèi lo refrèn de :}

\sound{https://www.youtube.com/watch?v=Eef69_st5oI}{UEFA Champions League anthem}

\StageDir{Lemie \lemieSi .}

\Valletspeaks Vo imajinade gnenca sen que l’è capitou-vo! La sosiétoù que fé seutte machine l’a vendi-vo ara lo litre numér\'o 1000!

\StageDir{De dameun la Bouteucca de l'éve, le dou cantognì sparon pe l'er de confettì. \\ Areuvve eunc\'o lo Seunteucco.}

\Seunteuccospeaks Mondjemé, n’i fallì galoppì to de chouite eun sé! Mi que dzenta noalla! N’en no, a Tsarvensoù, lo venqueur di litre numér\'o 1000! \direct{I vallet} Va criì to de chouite lo TG3.

\Seunteuccospeaks\direct{A François, eun lèi sarèn la man} Complemàn, complemàn de la par de totta la jeunte, complemàn!

\StageDir{François veun tellamente sopat\'o pe lo Seunteucco que reusque de tsire pe tèra.}

\Naimaspeaks Mi pensa té sitte! Se passavo mé avouì meun litre, lo prégnò mé lo litre numér\'o 1000! \arrabbiato

\StageDir{Entron eugn opérateur di reprèize RAI é eunna journaliste. Dimèn, le dou cantognì bèichon ba di distributeur.}

\Valletspeaks\direct{I Seunteucco} Mondje que chanse! L'ian seuilla dérì  a fiye eun servicho desù le jiragn\'on de Mariolino Bonadé.

\Journalistespeaks \direct{I Seunteucco} Qui l'a gagnà?

\Seunteuccospeaks\direct{Eun moutrèn François} Llou, llou!

\StageDir{La journaliste se plache protso de François é l'opérateur di reprèize se pozechoun-e pe le reprendre. François l'è vard\'o drette grase a Attila; protso de leur n'at eunc\'o Prospero que salie la camérà.}

\Journalistespeaks Bondzor a tcheutte! Sen seu pe sentì sen que l'at a no diye lo Mesieu que la terià lo litre num\'er\'o 1000 a la Bouteucca de l'éve. Ara vèyèn sen que l'a gagnà. \direct{Eun aprotsèn lo microfonne ver François} Ad\'on Mesieu\ldots

\Francoisspeaks\direct{Eun se medzèn le paolle} Francois Treuncaillette.

\Journalistespeaks François comèn vo sentade aprì avèi i eunna satisfach\'on di janre?

\Francoisspeaks Eh?

\Journalistespeaks Diavo, comèn vo sentade aprì avèi i eunna satisfach\'on di janre?

\Francoisspeaks Ah, a mé l’è pa tchandjà ren.

\StageDir{Lo Seunteucco eunterveun eun prégnèn lo microfonne.}

\Seunteuccospeaks Ad\'on fa diye que no sen contèn d'èi sé a Tsarvensoù lo gagnàn di litre num\'er\'o 1000 de la Bouteucca de l'éve. La sosiétoù que l’a vendi-no la machina bailleré a Mesieu François eun bon de 200 litre pe caletoù d'éve da dzoure devàn lo 31/12/2012.

\StageDir{Prospero s'aprotse i microfonne.}

\Prosperospeaks Se va bièn nen avancheré trèi car!\riye

\Journalistespeaks Ouè ad\'on de Tsarvensoù l'è totte, mersì a tcheutte é orvouar a l'an que veun!

\ridocliou

\DeriLeRido
\RoleNoms{Réjì}{Paola Corti}
\RoleNoms{Collaborateur}{Valentina Ferré}
\RoleNoms{Mijì}{Renzo Bollon}
\RoleNoms{Souffleur}{Jasmine Comé}
\RoleNoms{Tramamoublo}{Jean-Pierre Albaney}

\end{drama}



