\souvenir{\textit{À mon avis}, SÈIDEPOU(VO)ÈR l'è la pi jénialla pièse di Digourdì a niv\'o d'orijinalité métatéatrale. N'en voulì, comme comédièn, no-z-aprotchì pe lo premì cou a la satire politique di Consèille Réjonal. L'eunterprétach\'on finale si lo palque l'è itaye lévetta é bièn efficase di momàn que lo tecste l'ie \textit{autobiographique} é métatéatrale. Deun la partia \textit{autobiographique} di sujé, mé say\'o éffectivamente itaye élia, a l'éiadzo de 18 an,  Prézidanta de eunna compagnì d'atteur bièn pi gran que mé. Eun pi, di coutì métatéatral, mé say\'o pe dab\'on la premie fenna é pi dzouveunna Prézidanta di Digourdì que dèijè gagnì le-z-éléch\'on pe itre reconfermaye. Donque, itre la protagoniste de mé mima l'é it\'o doblemàn pi émouvàn! L'émoch\'on l'è eunc\'o redoblaye a la feun di spettaclle can to lo pebleuque l'a tsanto-me \textit{Joyeux anniversaire}, vi que lo dzor mimo fitoo 19 an! Ah\ldots a propoù de métatéatro: deun la pièse mon cher Paolo Dall'Ara me régale finque eun mase de fleur pe mon anniverséo! SÈIDEPOU(VO)ÈR: eun cateill\'on métatéatral que llate Digourdì é satire politique.}{Marlène Jorrioz}