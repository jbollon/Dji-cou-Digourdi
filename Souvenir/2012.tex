\souvenir{Me rappello pocca bague di pièse, n'i pa eunna grousa mémouée (a l'universitoù itedzoo 100 padze deun dou dzor é lo dzor aprì n'ayoo dza totte oublià!).
Mi can mimo me rappello amoddo comèn l'è nèisiya la compagnì di Digourdì!

L'ie lo 2007 é sayoo eun tren de guedé pe bèichì ba eun Veulla. S'ayoo presque foua d'Ampaillan, can sento a la radi\'o eugn'annonse publisitère d'eunna compagnì téatrale é to de chouite diyo a mé mimo: \og mi pequé seu a Tsarvensoù n'en pa eunna compagnì de téatro?\fg. Areuvvo i Pon-Sià é acappo le trèi de l'Ave Maria: Giada, Ester é Jasmine. Teurio ba lo finestreun é, tot entuziaste, lèi diyo: \og Oh mi pequé fièn pa eunna compagnì téatrale? An matse de Quemeue nen an eunna. Pouèn pa pa èi la noutra!\fg. Dèi si dzor gnouo a pompé de dzi pe terì si eunna compagnì. A la feun lo \textit{zoccolo duro} que me baillè fose l'ie la classe di '84-'85: Jo\"elle, Jo\"el, Pierre, Ester, Jasmine, Ester, Simone é Francesca ('83). 

Caque ten aprì no-z-accapèn eun tchi Gidio (restoràn Monte Emilius) pe no-z-organizé é tsertchì d'atre dzouveunno. Sayoon eun tren de no-z-achouaté deun la pégna tsambretta a gotse di banc\'on, can Jo\"el se prézente avouì eun que n'ayoon jamì vi: botte blantse (que gneun betè), \textit{jeans} to strachà é polo de la Fred Perry avouì lo colette terià si! Me prézento: \og Salì, mé si Laurent\fg; \og \textit{Ciao, io sono Paolo}\fg. Ester m'avèitse comme pe diye \og mi sitte sa gneunca prédjì patoué\fg. Eun pi l'ie pi vioù de no!\\ \'E beun… sitte l'ie lo super Paolo Cima Sander (pa fata de counté véo de cou lo pebleucco l'a riette grase a llou!).

Dèi salla premiye réuni\'on n'en comenchà a no troué pe eunna sala de la viille bibliotèque (eun cou eunc\'o finque pe eunna sala dézò Mèiz\'on Quemeua), pe liye de teste téatral é fée de premiye proue. Eun teste que n'en prouoù mi jamì prézentoù l'è \og Piéreun lo stchapeun\fg\ldots ah! Deun salla pièse n'ayè eun directeur de si pa senque… é te sa qui lo fiyè? Paolo Cima Sander! Eugn éffé deun mon portable n'i eunc\'o ara salvoù son contacte comme \og Paolo directeur\fg!

Aprì fayè chédre eun Prézidàn. Can n'en fé la réuni\'on n'i to de chouite pompoù Jo\"el  \og Ouè té t'i adatte pe fé lo Prézidàn, t'i fran eun gamba, t'i lo top!\fg\ldots é pouèi n'en accapoù qui sariye aloù douàn lo dzeudzo d'eun lo case saoutisan foua de problème légal.

N'en pa moloù de no-z-accapé é de proué de teste téatral tanque lo dzor iaou sen senti-no preste pe prézenté eun premì spettaclle a totta la populach\'on de Tsarvensoù, lo 17 mi 2008 a la bibliotèque communale de Tsarvensoù.

Pouèi son nèisì le Digourdì… ah na! Pouèi l'è comèn l'è nèisiya la compagnì. Mi pequé \og Digourdì\fg é pa \og Le compagn\'on de la fiacca\fg ou \og Le botchassa de Tsarvensoù\fg?

Eunna nite mé é Simone no no reteriavon aprì eun mariadzo. L'ie beun tchica tar mi n'ayoon pa voya de no reterì. Adon désidèn d'ataqué si pe Cogne. Passèn donque i mitcho de Simone pe prende le cllo de la machina. Mé attégnao foua di mitcho é for probablo que Simone l'arè fé de cazeun eugn entrèn i mitcho, pequé, aprì eun per de seconde, sa mamma chor foua si la louye é no di: \og Iao alade?\fg. Simone lèi rep\'on: \og Alèn si eun Cogne\fg. Dispéraye la mamma no braille: \og Si eun Cogne?! Fiade pa tro le digourdì!\fg.
Attaquèn si pe Cogne é dimando a Simone: \og Mi senque vou deue digourdì?\fg; \og Boh, si pa, mamma me lo di todzor\fg. Dèi salla nite n'i mémorizoù lo numér\'o de téléfonne de Simone comme \og Simone Digourdì\fg.

Aprì tchica de ten, lo lon d'eunna proua, sen deu-no: \og Mi queun nom no no baillèn?\fg. Aprì trèi meneutte propouzo: \og Le Digourdì!\fg, \og Mi senque vou diye?\fg; avèitsèn sen que di lo dichonnéo: \og Abile, actif, euntelledzèn\fg\ldots \og Sen no!\fg.

Voualà comèn son nèisì Le Digourdì!}{Laurent Chuc}
