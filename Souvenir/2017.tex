\section*{Lo souvenir di Seunteucco}
\og N'en voulì counté l'istouére moderne de noutra Quemeua avouì eun langadzo directe, \textit{visuel}, eun proupouzèn i Digourdì eun défì: dzoure leur caletaye é leur espérianse téatrale pe réalizé eun \textit{court-métrage}. Imajin-ì la vèille di premiye-z-éléch\'on tsarvensolentse aprì la Secounda Guerà mondiale, l'a bailla-no la possibilitoù de rappelé tcheu sise que l'an i a queur Tsarvensoù, mimo aprì le-z-àn teuppe di \textit{fascisme}; mi surtoù de commémoré le sinque Père de Tsarvensoù (comme lamo le querì mé!), que l'an portoù la Quemeua i premiye-z-éléch\'on de Tsarvensoù reconstituite.
\\Personellamente, l'è it\'o émouvàn veure comme se pou baillì viya a eun boc\'on d'istouére avouì l'\textit{art cinématographique}. Mimo lo fé que le-z-atteur sayoon de dzouveun-o Tsarvensolèn l'è eun souvenir prèsieu que n'i\ldots pequé, eunsemblo, n'en cougnì, é eun caque magniye viquì, eunna padze eumpourtanta de noutra Communoté.
%Anche il fatto che gli attori fossero tutti giovani e del territorio ha permesso che anche loro venissero a conoscenza di una pagina che ignoravano.
%Personalmente, è stato emozionante veder ricostruire all'interno di un set cinematograrica la storia del nostro comune
%Perché abbiamo voluto raccontare la storia moderna del nostro comune utilizzando un linguaggio immediato e proponendo ai diogurdì una sfida, ossia di cimentarsi nella realizzazione di una pièce teatrale.
%Ripercorre la storia della notte prima delle elezioni, ci ha permesso di ricrodare tutti coloro chehanno sempre avuoto a cuore il nostro comune nonostante gli anni bui della sopressione e soprattutto i cinque,c he amo def padri fondagori, che in prima persona hanno portato il comune alle libere elezioni del 46 continuando per la maggior parte l'impegno nella legislatura successiva.
\fg{}
\newline
\newline
\hspace*{\fill} \textit{Ronny Borbey}

%
