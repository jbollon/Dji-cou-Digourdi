\souvenir{Sayoon mé, pappa é Peppino Camandona (\textit{bonanima}), lé a medjì la trippa a la Caritas de la Pagoda; eun momàn tchica particulié, i mentèn di-z-ansièn, di pouo, di maladdo.
\\ L'ion eun tren de fé la counta can Peppino ataque a no countì de si cou que l'an faillì pourtì si pe Cogne an quése di mor avouì eunna Fiat. Can son arrevoù si, son entroù pe lo mitcho é l'an accapoù, a gotse, lo mor deun la coutse é a drèite, deun l'atra tsambra, l'abeuque que teriè ba de grappa!
\\ Ver mèitchà matin-où, cllouzon lo mor deun la quése é passon l'atra mèitchà de la matin-où eun compagnì di paèn, di-z-amì di mor é\ldots de l'abeuque! 
\\ Dimèn l'a attacoù a nèire détchis; naturellamente gneun l'è ren\-di-se contcho de ren é pe l'aoua de la sepolteua n'ayè belle demì mètre de nèi. La Fiat l'ie pa 4x4, mi sitte l'ie pa lo dérì di problème. Lo vrèi défì l'è it\'o pourté foua si le-z-ipale la quése sensa se euntsabotté pe la nèi\ldots l'an beun i tchica de difficult\'o, mi a la feun totte l'è aloù amoddo.
\\ Seutta donque l'ie l'ispirach\'on que l'a fé nèitre \og Tanta betsii\fg. Lo dzor aprì de la trippa de la Caritas, n'i terià ba le personadzo, eun per de-z-ach\'on é avouì Jo\"el Albaney n'en désidoù de betì la betsii i poste de l'abeuque. Aprì avouì to lo reste di groupe n'en djount\'o de-z-id\'o comique, comme le Tirolèis é la danse tsaque cou que la paolla \textit{kartoffen} l'ie prononchaye.
\\ \'E naturellamente lo final\ldots to lo Splendor l'ie pléyà eun dou di riye. Eun \textit{crescendo} de ritme, \og é ara comèn fièn, maladetto n'en pa la quése, comèn fièn, iaou la betèn, l'è pa poussiblo sise dzouveun-o é l'\textit{internet}, é ara comèn fièn\ldots \fg{}; lé mé é Pierre no no avèitsèn, aprì avèitsèn la mii\ldots doe secounde aprì sen foua di palque dézò lo bouéchì di man di pebleuque.}{Paolo Cima Sander}
%Con Joel poi mi sono visto e ho detto prendiamo spunto da questo fatto. Io tiro giù una scaletta di personaggi e azioni. Con joel decidiamo di mettere la betsii al posto dell'alambicco. Poi con la vecchia squadra (franci, pierre...) abbiamo scritto il resto (i tirolesi betsii). Mi ricordo bene del balletto kartoffen.
%Poi geniale la chiusura... climax pazzesco! Guardato la mii e siamo usciti trionfanti!
%Era un tripudio di idee


%idea nata alla mensa della caritas della pagoda (Peppino camandona), c'era la trippa, (eravamo io papà e peppino)... contesto particolar (anziani, senza tetto).

%Stavamo mangiando... tra una cosa e l'altra... peppino racconta di una macchina particolare (fiat) per un funerale vanno a Cogne (con cassa da morto nella macchina). Entrano in casa, a sx hanno il morto e a dx hanno l'alambicco in funzione nell'altra stanza.
%Chiudono il morto verso metà mattinana.
%poi si mettono a fare la counta con gli altri e anche a bere... per la sepoltura erano caldissimi!
%Ma nel mentre ha nevicato di botto ed erano un po' alticci. Dovevano reggere la cassa sulla neve con un po' di difficoltà... e poi sono andati al funerale.