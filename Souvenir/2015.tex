\souvenir{L'idoù de résitì eunsemblo a la compagnì Le Beurt é Boun de Pollein l'ie chortiya foua de la tita de Jo\"{e}l Albaney é Michel Squinabol. Pe totte é doe le compagnì l'è itaye la premiye espérianse de collaborach\'on.

Me rappello eunc\'o que l'è itoù lo premì cou iao tcheu le-z-atteur l'an partesipoù a l'icriteua di teste. Sen divija-no eun trèi groupe (micllà avouì Tsarvensolèn é Pollentch\'on) pe icriye le trèi-z-acte de Disco Flama. Tsaque groupe l'ayè fé eunna approfondite retserte pe caractérizé amoddo l'époque deun laquelle lo local (Disco Flama) l'ie reprézentoù: costume, mezeuque, \textit{scénographie} é eunc\'o finque le dzi que l'an viquì réellamente lo local.

Collaboré avouì Pollein l'è itaye eunna dzenta espérianse. Mi can mimo, eunc\'o avouì eugn atra compagnì, sen  arrev\'o a fé de pastisse (é sitte me lo rappello amoddo pequé la fegueua n'i fé-la mé). A la feun di spettaclle, n'en fé eunna danse de groupe, le Tsarvensolèn avouì eunna maille rodze é le Pollentch\'on avouì eunna bleue. Mé si lo seul que l'è entroù eun \textit{scène} avouì eunna maille diffienta, comme lo pedzeun neur i mentèn di pedzeun dzano. 

Ah! La coulpa l'ie pa de mé, mi de si macao d'eun Jordy Bollon que, dérì le rid\'o, i mentèn de la confuj\'on, l'ayè tchapou-me la mailletta. 
}{Marco Ducly}