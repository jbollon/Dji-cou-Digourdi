\souvenir{La pièse \og Todzo pi Digourdì \fg l'è itaye icrita eugn'occaj\'on di dji-z-àn de la Compagnì. Lo sujé di teste l'è la \textit{biographie ironique} di Digourdì.
Personellamente \og Todzo pi Digourdì\fg resterè todzor deun mon queur, pe bièn de rèizón.

L'è itaye la dériye pièse comme Prézidàn di Digourdì; l'è eunna pièse avouì laquella eugn  itrandjì pou comprendre amod\-do sen que vou deu itre Digourdì, di momàn que lo \textit{fil rouge} de la counta l'è l'esprì comique de totta la compagnì, avouì voya de riye pe fée riye; l'è itoù lo premì cou iaou la Compagnì l'a prooù a tsandjì \textit{style} téatral: \textit{scénographie} redouite i \textit{minimum}, \textit{scène} reprézentaye deun difièn conteste \textit{espace-temps}\footnote{ Eun premì tentatif l'ie itoù fé avouì la pièse \og Matte\ldots sen tcheut matte\fg{} deun lo 2013.}, \textit{mimétisme} é surtoù bièn de \textit{métathéâtre}.}{Jordy Bollon}