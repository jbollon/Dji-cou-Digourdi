\chapter*{Annexe}
\markboth{\MakeUppercase{Annexe}}{\MakeUppercase{Annexe}}
%\pagestyle{plain}
\section*{Version numérique}\label{vers_num}
Ce dépôt
\begin{center}
\centering
\github\ \hspace*{0.5mm} \href{\detokenize{https://github.com/jbollon/Dji-cou-Digourdi}}{\color{RawSienna}{\textsc{GitHub}}} \hspace*{0.5mm} \github\\
%\vspace{-5pt}
 \vspace*{2mm}
\qrcode[hyperlink, height=0.5in]{\detokenize{https://github.com/jbollon/Dji-cou-Digourdi}}
\end{center}
\noindent  regroupe l'ensemble du code utilisé pour créer Dji Cou Digourdì, ainsi que sa version PDF. À partir de cette version numérique, il est possible d'accéder directement aux fichiers originaux employés par les Digourdì pour reproduire les effets sonores et les morceaux musicaux intégrés aux pièces. Un simple clic sur les titres correspondants suffit pour les consulter.

De plus, cette version numérique permettra d’assurer la mise à jour continue des liens et des QR codes associés aux contenus multimédias, même en cas de modification ou de déplacement de ceux-ci.
\newpage
\section*{Un peu de chiffres!}
Cette section représente fidèlement la déformation professionnelle de l'auteur, lequel n'a pas résisté à la tentation d'extraire quelques statistiques descriptives de Dji Cou Digourdì. Pour en faciliter la lecture, quelques précisions sont nécessaires :
\begin{itemize}
\item[•] Est considéré comme mot tout nom commun ou adjectif (les noms propres sont exclus);
\item[•] Les avant-spectacles et les vidéos ne sont pas pris en compte  dans le calcul des répliques\footnote{ L'ensemble des paroles prononcées par un personnage sans qu'il soit interrompu par un autre (IT - \textit{Battuta teatrale}).} ni dans celui des paroles ou des mots;
\item[•] Le nombre d’acteurs mentionnés ne comprend que ceux ayant prononcé au moins une réplique.
\end{itemize}