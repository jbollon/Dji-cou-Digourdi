\chapter*{Annexe}
\markboth{\MakeUppercase{Annexe}}{\MakeUppercase{Annexe}}
%\pagestyle{plain}
\section*{Version numérique}\label{vers_num}
Ce dépôt
\begin{center}
\centering
\github\ \hspace*{0.5mm} \href{\detokenize{https://github.com/jbollon/Dji-cou-Digourdi}}{\color{RawSienna}{\textsc{GitHub}}} \hspace*{0.5mm} \github\\
%\vspace{-5pt}
 \vspace*{2mm}
\qrcode[hyperlink, height=0.5in]{\detokenize{https://github.com/jbollon/Dji-cou-Digourdi}}
\end{center}
\noindent  contient tout le code utilisé pour créer Dji cou Digourdì, ainsi que sa version PDF. Depuis la version numérique, il est possible d'accéder aux fichiers originaux employés par les Digourdì pour reproduire les effets sonores et les morceaux musicaux intégrés aux pièces. Il suffira de cliquer sur les titres correspondants.

De plus, la version numérique permettra de maintenir à jour les liens et les QR codes de tous les contenus multimédias, même s'ils venaient à être modifiés pour une raison quelconque.
\newpage
\section*{Un peut de nombres!}
Cette section représente fidèlement la déformation professionnelle de l'auteur, lequel n'a pas résisté à la tentation d'extraire des statistiques descriptives de \textit{Dji Cou Digourdì}. Pour les interpréter, quelques précisions sont nécessaires :
\begin{itemize}
\item[•] On considère un mot un nom (non propre) ou un adjectif;
\item[•] Les avant-spectacles et les vidéos ne sont pas inclus dans le calcul des répliques\footnote{ L'ensemble des paroles prononcées par un personnage sans qu'il soit interrompu par un autre (IT - \textit{Battuta teatrale}).} et des paroles/mots. De plus, le nombre de pièces ne comptabilise que les acteurs ayant prononcé au moins une réplique.
\end{itemize}