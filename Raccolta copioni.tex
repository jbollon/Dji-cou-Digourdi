\documentclass[11pt, twoside, openright]{book}

% DA CITARE FONTE: https://icons8.com/icons/set/light-o

% aggiungere in piccolo "a cura di jb"

%%% Font, caratteri speciali e lingua %%% 
\usepackage[T1]{fontenc}
\usepackage[utf8]{inputenc}
\usepackage[french]{babel}

%%% Per ringraziamenti a forma di cuore
\usepackage{shapepar}
\usepackage{lipsum} % For placeholder text

%%% simbolo mail
\usepackage{fontawesome}

%\raggedbottom %annulla riempimento della pagina in verticale con spazi aggiuntivi tra capoversi o ambienti

%%% Stile della pagina %%%
\pagestyle{headings} %Ved. p. 35
\usepackage{emptypage}
%%%%%%%%%%%%%%%%

\usepackage{graphicx}
\usepackage[paperheight=9in, paperwidth=6in]{geometry}
%\usepackage[a4paper]{geometry}
\usepackage{afterpage}
\usepackage[svgnames, dvipsnames]{xcolor}

\usepackage{pagecolor}
%\usepackage{gfsdidot}
\usepackage{charter}
\usepackage{qrcode}
\usepackage{etoolbox}
\renewenvironment{quote}{%
   \list{}{%
     \leftmargin0.1cm   % this is the adjusting screw
     \rightmargin\leftmargin
   }
   \item\relax
}
{\endlist}


\DeclareRobustCommand{\augiefamily}{%
  \fontfamily{augie}\fontseries{m}\fontshape{n}\selectfont}
\DeclareTextFontCommand{\textaugie}{\augiefamily}

\usepackage[object=vectorian]{pgfornament}

\usepackage{changepage}

\usepackage{titling}
\patchcmd{\titlepage}{\setcounter{page}\@ne}{}{\message {Successfully patched titlepage.}}{\message {Failed to patch titlepage.}}


\usepackage[skins]{tcolorbox}

%%%%%%%%%%%%%%%%%%%%%%%%%%
% Dedication's package
\usepackage{emerald}
\pdfmapfile{=emerald.map}
%%%%%%%%%%%%%%%%%%%%%%%%%%

\newtcbox{\whiteshadowbox}[1][]{%
  enhanced, 
  center upper,
  fontupper=\large\bfseries,
  drop fuzzy shadow southeast,
  boxrule=0pt,
  sharp corners,
  colframe=blue,
  colback=white!10,
  before={\begin{center}},
    after={\end{center}},
  #1%
}
%%%%%%%%%%%%%%%%%%%%%%%%%
\usepackage{tocloft}


%%% PACKAGES
\usepackage{dramatist}

\usepackage{titling}% Allows for multiple \titles within one document

\let\oldmaketitle\maketitle
\renewcommand{\maketitle}{%
\newpage
\phantomsection
  \addcontentsline{toc}{chapter}{\thetitle}
\oldmaketitle

\markboth{\MakeUppercase{\thetitle}}{\MakeUppercase{\thetitle}}
}

\renewcommand{\casttitlename}{\includegraphics[scale=.14]{emoji/personadzo.png}\hspace{.5mm} Personadzo \includegraphics[scale=.15]{emoji/personadzo.png}}
\renewcommand{\actname}{}
\renewcommand{\scenename}{\textit{Scène}}
\renewcommand{\printscenenum}{%
\scenenumfont \thescene}

\renewcommand{\StageDir}[1]{%
    \begin{stagedir}
    #1
    \end{stagedir}
    %\vspace*{.5cm}
}

\setlength{\afteractskip}{10pt}
\setlength{\beforesceneskip}{10pt}
\setlength{\aftersceneskip}{10pt}
\setlength{\aftercasttitleskip}{20pt}


%%%%%%%%%%%%%%%%%%%%%%%%
% FOTO COPERTINA
%%%%%%%%%%%%%%%%%%%%%%%%
\newcommand{\fotocopertina}[4]{
\cleardoublepage 
%\newpage
\vspace*{\fill}
\begin{center}
\textbf{\LARGE Le Digourdì di #4}
\end{center}

\begin{figure}[h]
\centering
\whiteshadowbox{
\includegraphics[height=4cm,keepaspectratio]{#1}
}
\end{figure}
\noindent
\textbf{Drette}: \textit{#2.}
\newline
\newline
\textbf{Achouatoù}: \textit{#3.}
}
%%%%%%%%%%%%%%%%%%%%%%%%

%%%%%%%%%%%%%%%%%%%%%%%%
% LINK PIESE
%%%%%%%%%%%%%%%%%%%%%%%%
\newcommand{\LinkPiese}[3]{
\vspace*{\fill}
\begin{figure}[h]
\centering
\foreach \x [count=\i] in {#1} {
 \foreach \y [count=\j] in {#2} {
  \ifnum\i=\j
      \begin{subfigure}{#3\textwidth}
  \centering
    \video\hspace*{0.5mm} \textsc{\x}\hspace*{0.5mm} \video\\
 \vspace*{2mm}
\qrcode[hyperlink, height=0.75in]{\y}
  \end{subfigure}%
   \fi
 }
}
\end{figure}
\vspace*{\fill}
}
%%%%%%%%%%%%%%%%%%%%%%%%

%%%%%%%%%%%%%%%%%%%%%%%%
% Souvenir
%%%%%%%%%%%%%%%%%%%%%%%%
\newcommand{\souvenir}[2]{
\section*{Lo souvenir de l'atteur}
\og #1\fg{}
\newline
\newline
\hspace*{\fill} \textit{#2}
}
%%%%%%%%%%%%%%%%%%%%%%%%

%%%%%%%%%%%%%%%%%%%%%%%%
% Queriaouzitoù
%%%%%%%%%%%%%%%%%%%%%%%%
\newcommand{\queriaouzitou}[1]{
\subsection*{Queriaouzitoù}
#1
}
%%%%%%%%%%%%%%%%%%%%%%%%

%%%%%%%%%%%%%%%%%%%%%%%%
% Scénographie
%%%%%%%%%%%%%%%%%%%%%%%%
\newcommand{\Scenographie}{
\newpage
\begin{center}
\Large
\textsc{\scenographie\ \textit{Scénographie} \scenographie}
\end{center}
}
%%%%%%%%%%%%%%%%%%%%%%%%

%%%%%%%%%%%%%%%%%%%%%%%%
% CAST
%%%%%%%%%%%%%%%%%%%%%%%%
\newcommand{\name}[1]{eunterprétoù pe \textsc{#1}.}
\newcommand{\nameF}[1]{eunterprétaye pe \textsc{#1}.}
%%%%%%%%%%%%%%%%%%%%%%%%

%%%%%%%%%%%%%%%%%%%%%%%%
% Dérì le Rido
%%%%%%%%%%%%%%%%%%%%%%%%
\newcommand{\DeriLeRido}{
%\newpage
\cleardoublepage 
\thispagestyle{empty}
\begin{center}
\textbf{\LARGE Derì le rid\'o}
\end{center}
\vspace{.2cm}
}

\usepackage{pgffor}
\newcommand{\RoleNoms}[2]{
\begin{center}
 \textbf{#1}\\\vspace{.05cm}
  \foreach \x in {#2} {
    \x\\
  	}
  	\vspace{.01cm}
\end{center}
}
%%%%%%%%%%%%%%%%%%%%%%%%  

%%%%%%%%%%%%%%%%%%%%%%%%
% VIDEO
%%%%%%%%%%%%%%%%%%%%%%%%
\usepackage{float}
\newcommand{\listvideos}{Liste di \textit{vidéo}}
\newlistof{videos}{vds}{\listvideos}

\newcommand{\StartVideo}[3]{
\refstepcounter{videos}

\begin{figure}[H]
%\vspace*{-5pt}
\StageDir{\hspace{2.5em}Teuppe \lemieBa .}
\centering
\foreach \x [count=\i] in {#1} {
 \foreach \y [count=\j] in {#2} {
  \ifnum\i=\j
      \begin{subfigure}{.75\textwidth}
  \centering
    \video\hspace*{0.5mm} \textsc{\small\y}\hspace*{0.5mm} \video\\\vspace*{2mm}
    \qrcode[hyperlink, height=0.5in]{\x}
  \end{subfigure}%
  \ifnum\i=1
   \vspace{2.5mm}
   \fi
  \addcontentsline{vds}{section}{\y}
   \fi
 }
}
\ifstrempty{#3}%
    {\StageDir{Lemie \lemieSi .}
    {}
%\vspace*{-5pt}
}%
\end{figure}%
#3
\ifstrempty{#3}
    {}
    {\vspace*{-2em} \StageDir{Lemie \lemieSi .}
    }
}
%%%%%%%%%%%%%%%%%%%%%%%%
\usepackage{subcaption}
%%%%%%%%%%%%%%%%%%%%%%%%
% MEZEUCCA
%%%%%%%%%%%%%%%%%%%%%%%%
\usepackage{xparse}
\usepackage{xstring}

\newcommand{\listsounds}{Liste di mezeucque}
\newlistof{sounds}{snd}{\listsounds}

\NewDocumentCommand{\sound}{m m O{true}}{%
  \refstepcounter{sounds}

  \begin{figure}[h!]
    \centering
    \soundcloud \hspace*{0.5mm} \href{#1}{\color{RawSienna}{\textsc{#2}}} \hspace*{0.5mm} \soundcloud%
    \vspace{-5pt}
    % \vspace*{2mm}
    % \qrcode[hyperlink, height=0.5in]{#1}
  \end{figure}

  \IfStrEq{#3}{true}{%
    \addcontentsline{snd}{section}{#2}
  }{}
}
%%%%%%%%%%%%%%%%%%%%%%%%

%%%%%%%%%%%%%%%%%%%%%%%%
% Effet
%%%%%%%%%%%%%%%%%%%%%%%%
\newcommand{\listeffets}{Liste di-z-éffé sonore}
\newlistof{effets}{eff}{\listeffets}

\NewDocumentCommand{\effet}{m m O{true}}{%
  \refstepcounter{effets}

  \begin{figure}[h]
    \centering
    \soundcloud \hspace*{0.5mm} \href{#1}{\color{RawSienna}{\textsc{#2}}}\hspace*{0.5mm} \soundcloud%
  \end{figure}%
  \vspace{-0pt}
  \IfStrEq{#3}{true}{%
    \addcontentsline{eff}{section}{#2}%
  }{}
}
%%%%%%%%%%%%%%%%%%%%%%%%

%%%%%%%%%%%%%%%%%%%%%%%%
% Foto
%%%%%%%%%%%%%%%%%%%%%%%%
\newcommand{\listfotos}{Liste di foto}
\newlistof{fotos}{fot}{\listfotos}

\newcommand{\foto}[2]{
\refstepcounter{fotos}

\begin{figure}[h]
\centering
\instagram \hspace*{0.5mm} \textsc{#2}\hspace*{0.5mm} \instagram\\
 \vspace*{1mm}
\qrcode[hyperlink, height=0.5in]{#1}
\end{figure}
\addcontentsline{fot}{section}{#2}
}
%%%%%%%%%%%%%%%%%%%%%%%%

\newcommand{\ridoiver}{
%\vspace*{-.55cm}
\begin{center}
\raisebox{-.2mm}{\includegraphics[scale=.1]{emoji/ridoIver.png}} {\large\textsc{Se ivre lo rid\'o}} \raisebox{-.2mm}{\includegraphics[scale=.1]{emoji/ridoIver.png}}
\end{center}
}

\newcommand{\ridocliou}{\vspace*{15pt}
\begin{center}
\raisebox{.1mm}{\includegraphics[scale=.07]{emoji/tsilateila.png}} {\large \textsc{tsi la tèila}}  \raisebox{.1mm}{\includegraphics[scale=.07]{emoji/tsilateila.png}}
\end{center}
\vspace*{.25cm}
}

\addto\captionsfrench{\renewcommand{\contentsname}%
    {Pièse}%
}
\newcommand{\moublotitlename}{Moublo}
\newcommand{\moublo}{\newpage \centering\textsc{\moublotitlename}}

%%%%%%%%%%%%%%%%
%%% File sty personali  %%%

%%% Emoticons %%%
\usepackage{emoticons}

%%%%%%%%%%%%%%%%

%%% Collegamenti ipertestuali %%%
\usepackage{hyperref}
\hypersetup{
    colorlinks=true,
    linktoc=all,
    linkcolor=Brown,
    urlcolor = Sepia
}





\begin{document}

\frontmatter
%----------------------------------------------------------------------------------------
%	TITLE PAGE
%----------------------------------------------------------------------------------------
\afterpage{%
\newgeometry{margin=0cm}

\pagecolor{LightSkyBlue}\afterpage{\restorepagecolor}

\begin{titlepage}
\vspace*{\fill}
\vspace*{-1cm}
	\begin{center}
	\textcolor{White}{ % Red font color
		{\fontsize{40}{40}\selectfont \augiefamily Dji Cou Digourd\`i}
	}
	\vspace*{\fill}
	\end{center}
	
	%\vspace{0.025\textheight} 	
	%\rule{0.3\textwidth}{0.4pt} 	
	%\vspace{0.1\textheight} 
	
	%{Pièce icrita pe\\\vspace{0.25cm} \Large \textsc{Le Digourdì de Tsarvensou}} % Author name
	
	%\vfill % Whitespace between the author name and publisher
	
	%------------------------------------------------
	%	Publisher
	%------------------------------------------------
%\begin{center}
%\includegraphics[scale=0.075]{07_official_bianco_stampa.png}\\[0.5\baselineskip] 
%\end{center}
	
	%{\large\textsc{the publisher}} % Publisher
	
	%\vspace{0.1\textheight} % Whitespace under the publisher text
	
	%------------------------------------------------
	%	Bottom rules
	%------------------------------------------------
	
	%\rule{\textwidth}{0.4pt} % Thin horizontal rule
	
	%\vspace{2pt}\vspace{-\baselineskip} % Whitespace between rules
	
	%\rule{\textwidth}{1pt} % Thick horizontal rule
	
\end{titlepage}
\nopagecolor

\clearpage
\restoregeometry
} 

\clearpage
\thispagestyle{empty}
% Ved. p.34 per ordine su dispensa latex (anche per copyright : latex, emoticon)

\chapter*{Colophon} 

\shapepar{\squareshape}{%(vedi computer science per layout colophon)
Si livro l'è it\'o icrì avouì \LaTeX (La(mport)\TeX), eun lojisiel gratouì de compozich\'on \textit{typographique} réalizoù pe Leslie Lamport, que eumplèye \TeX, créach\'on de Donald Ervin Knuth deun lo 1982, comme moteur de compozich\'on.
FONT Nome, autore: "ArsClassica, uno stile
ispirato a Gli elementi dello stile tipografico di Robert Bringhurst"
Le-z-\textit{emoji} FACCINA son it\'o ditsardjà avouì libre lisanse pe lo site \href{https://icons8.com/icons/}{icons8.com}, eun projé a code iver GESTITO/MANTENUTO da ved. COMPUTER SCIENCE....
La COPERTINA di livro l'è itaye réalizaye pe PIPPO é COSA RAPPRESENTA?}
% prendere spunto da colophon della dispensa Pantieri o computer science
\vfill 
\noindent\makebox[\textwidth][c]{%
  \busta\ \href{mailto:jbollon94@gmail.com}{jbollon94@gmail.com}
}
\noindent\makebox[\textwidth][c]{%
  \busta\ \href{mailto:digourdi.tsarvensou@gmail.com}{digourdi.tsarvensou@gmail.com}
}

\newpage 

\clearpage
\begin{center}
\thispagestyle{empty}
\vspace*{\fill}
\Large
\ECFJD
Pe noutra Pach\'on\\
Téatro, Patoué\\
Lenva, \'Emoch\'on\\
rèise de digourdia Amitié
\vspace*{\fill}
\end{center}
\newpage

% pe lo Comédie é lo Patoué
% Istouére é Lenva
% simàn de digourdia Amitié

% cercare di togliere il "de la" 

%%% Indice generale*
\tableofcontents
\clearpage
\listofvideos
\listofsounds
\listofeffets
\listoffotos
\chapter*{Préfachon} 
Dimandé a Daniel

\chapter*{}
\heartpar{Un grand merci profond et sincère à Daniel Fusinaz, pour avoir cru en cette œuvre, pour la patience et la passion qu'il a déployées dans la relecture de l'ensemble des textes, et pour l'honnêteté intellectuelle avec laquelle il m'a initié à la graphie BREL.
Je remercie également Diego Lucianaz, pour ses conseils, ses suggestions et sa disponibilité sans faille.
J’embrasse également Marlène Jorrioz, qui a peaufiné l’introduction de Dji Cou Digourdì, avec un français soigné, reflétant à la fois la passion de l’auteur et l’amour qu’elle porte pour Le Digourdì.
%Éventuels sponsors à mentionner (?) (Commune de Charvensod, Lucianaz...)
}


%Introduzione non numerata
\chapter*{Introduction} 

%IDEA: APPENDICE CON STATISTISCHE: parole totali, nome più frequente, verbo più frequente ecc.... parola più lunga...; totale e per pièce

% DA QUALCHE PARTE SCRIVERE CHE per i suoni e le musiche c'è un link su soundcloud per trovare ciò che abbiamo creato o salvato. Le canzoni o le sigle non ci sono se non sono state create da noi. Si possono trovare su yt.

% DA QUALCHE PARTE SCRIVERE:  non sono state inserite le indicazione di scena in merito agli ingressi/uscite da dx/sx, se non strettamente necessario... MOtivo: per non appesantire la lettura del racconto con note tecniche di regia.

\paragraph*{Pourquoi \textit{Dji Cou Digourdì}}
\textit{Dji Cou Digourdì} est né de la nécessité de rassembler toutes les pièces de théâtre interprétées par la compagnie Le Digourdì de Tsarvensoù à l'occasion du Printemps Thé\^atral. Presque toujours, les dernières versions des textes sont modifiées à la main lors des dernières répétitions, période durant laquelle, pour une étrange raison, naissent les répliques et les idées les plus irrévérencieuses et efficaces. Par conséquent, au fil des ans, les textes théâtraux, archivés quelque part dans les ordinateurs des acteurs, sont restés figés sur une ancienne version du scénario, peu fidèle à la représentation jouée en direct sur scène.

De plus, les pièces des Digourdì ont souvent intégré au scénario des contenus multimédias, tels que des audios, des bruitages, des vidéos et des images. Ainsi, outre le désir de témoigner fidèlement des pièces interprétées sur la scène du Printemps Thé\^atral, \textit{Dji Cou Digourdì} est également un archivage des contenus multimédias produits par la compagnie théâtrale \textit{tsarvensolentse}.

Bref, \textit{Dji Cou Digourdì} se présente aux lecteurs comme un témoignage et un catalogue de l'art théâtral des Digourdì.

\paragraph*{Le contenu}
\textit{Dji Cou Digourdì} rassemble les dix premières pièces interprétées par la compagnie Le Digourdì de Tsarvensoù à l'occasion du Printemps Thé\^atral de 2009 à 2019\footnote{ Au total, cela fait 11 ans ; cependant, en 2017, Les Digourdì n'ont pas participé au Printemps Thé\^atral. La raison de cette absence est expliquée dans le chapitre dédié à l'année 2017.}. Chaque pièce correspond à un chapitre et s'articule de la manière suivante :
\begin{itemize}
\item[$\bullet$] Couverture de la pièce avec titre, auteurs, lieu et date de la représentation ;
\item[$\bullet$] Photo avec la liste des acteurs et un QR Code pour visionner la vidéo intégrale de la pièce ;
\item[$\bullet$] Entretien avec un acteur comprenant un souvenir personnel et quelques curiosités ou anecdotes liées à la pièce ;
\item[$\bullet$] Description de la scénographie et des principaux accessoires de scène ;
\item[$\bullet$] Liste et description des personnages (par ordre d'apparition) avec l'indication de l'acteur qui l'interprète ;
\item[$\bullet$] Éventuel avant-spectacle ;
\item[$\bullet$] Texte théâtral ;
\item[$\bullet$] Liste des collaborateurs qui ont travaillé en coulisses.
\end{itemize}

\paragraph*{L'habillage}
C'est avec plaisir que je commente également l'habillage qui a été choisi pour agrémenter le contenu de \textit{Dji Cou Digourdì}. J'ai personnellement veillé à ne pas me borner à la rédaction d'un simple volume constitué d'une série de pièces de théâtre classées chronologiquement. Étant donné que Les Digourdì, depuis leur création, ont toujours cherché à proposer un théâtre frais et jeune, j'ai décidé que le recueil devait présenter divers éléments d'originalité, afin de rendre la lecture agréable, colorée et multimédia.

Pour rendre la lecture agréable, j'ai choisi de rédiger \textit{Dji Cou Digourdì} avec \LaTeX, un logiciel gratuit de composition typographique, conçu pour produire des documents d'une qualité typographique élevée. La grande flexibilité de \LaTeX m'a permis de transformer la mise en page graphique d'un script de théâtre classique, propice à un acteur qui doit lire aisément une réplique, en une mise en page fonctionnelle pour un lecteur intéressé par l'intrigue du scénario. En effet, la beauté d'une composition typographique, outre le texte, réside dans le fait que la disposition du matériel à lire ou à consulter n'attire pas l'attention sur elle-même, mais conserve cette sobriété qui permet au lecteur de recevoir le message sans distractions inutiles et sans buter sur la lecture à cause d'espacements irréguliers ou de changements constants de style de caractères\footnote{ Fazio, Luciano. \textit{Introduzione all'arte della composizione tipografica con LaTeX}. Università degli Studi di Messina, n.d., \url{https://mat521.unime.it/~fazio/tesi/GuidaGuIT.pdf}}. Par conséquent, je suis certain que le lecteur appréciera la composition typographique agréable de \textit{Dji Cou Digourdì}.

Pour rendre le texte plus coloré, je me suis inspiré du livre \og Computer Science Distilled: Learn the Art of Solving Computational Problems\fg\footnote{ Waisman, Wladston Ferreira Filho. \textit{Computer Science Distilled: Learn the Art of Solving Computational Problems}. Code Energy LLC, 2017.}, dans lequel les auteurs utilisent des émoticônes pour égayer la lecture de thèmes liés à l'informatique. J'ai exploité cette idée non seulement pour embellir le texte et lui donner une touche d'originalité, mais aussi pour rendre plus didactique la compréhension de certaines expressions ou mots du patois. Le lecteur trouvera donc souvent une émoticône représentant un objet, une émotion ou un animal juste après le mot correspondant : \textit{martelette}\martello, \textit{innamoroù}\inamourou, \textit{gadeun}\gadeun.

Enfin, compte tenu de la fonction d'archive que possède \textit{Dji Cou Digourdì}, j'ai souhaité rendre facilement accessible au lecteur tout le matériel multimédia produit par Les Digourdì au sein de leurs pièces. Grâce à des QR Codes, il est en effet possible d'accéder aux plateformes YouTube, Facebook, Instagram et SoundCloud pour apprécier les vidéos, les photos, les bruitages et les musiques\footnote{ Les bruitages et les musiques sont accessibles via la version en ligne du présent livre. Version archivée au sein du dépôt GitHub (voir \hyperref[vers_num]{Annexe - Version numérique}).} réalisés par Les Digourdì.

\paragraph*{Méthode}
Maintenant que nous avons clarifié la raison d'être de \textit{Dji Cou Digourdì}, son contenu et son habillage, c'est le temps de décrire brièvement sa réalisation.

La première phase a été dédiée à récupérer les copies papier et/ou numériques des textes théâtraux, en privilégiant les versions les plus à jour. Par la suite, j'ai recherché, archivé puis visionné, seconde par seconde, toutes les vidéos des représentations théâtrales des pièces de 2009 à 2019. Ce fut sans aucun doute la phase qui a demandé le plus de temps. En regardant et en écoutant chaque réplique, j'ai mis à jour tous les scripts, y compris avec les improvisations qui, en tant que telles, n'auraient jamais pu être transcrites dans le script original. De plus, j'ai structuré toutes les pièces en scènes, en leur donnant un titre éloquent.

Une fois terminée la transcription de toutes les pièces, la phase de révision linguistique a commencé. En collaboration avec le BREL\footnote{ Bureau Régional Ethnologie et Linguistique.}, l'intégralité du texte de \textit{Dji Cou Digourdì} a été corrigée sur le plan orthographique et lexical.

Parallèlement à ces deux phases principales, j'ai collecté tout le matériel multimédia produit par Le Digourdì\footnote{ Malheureusement, une seule vidéo n'a pas été archivée car elle a été égarée on ne sait où. La vidéo représentait parodiquement la publicité pour l'eau de Santa Colomba « qui, si tu la bois, est une bombe ! » (voir pièce « Eun drolo de distributeur »).} et je l'ai rendu disponible, si ce n'était pas déjà le cas, sur YouTube \href{https://www.youtube.com/@the_digourdi}{\yt}, pour les vidéos, et sur SoundCloud \href{https://soundcloud.com/user-234168361/sets}{\pppp} pour les bruitages et les chansons.

De plus, pour chaque pièce, j'ai mené un entretien avec un ou plusieurs acteurs afin de recueillir ces souvenirs, ces coulisses et ces émotions qui rendent l'interprétation d'une pièce unique et irremplaçable. Les entretiens ont été menés de la manière la plus spontanée possible, individuellement ou en groupe, autour d'un apéritif qui a aidé l'acteur à revivre toute son expérience personnelle avant, pendant ou après la pièce.

Je conclurai par quelques notes méthodologiques concernant la scénographie, les costumes, les avant-spectacles et la citation des collaborateurs.
J'ai rédigé la description de la scénographie dans le seul but d'énumérer sommairement les éléments les plus marquants du contexte dans lequel le scénario se développe, avec une énumération approximative de certains accessoires de scène utilisés par les personnages. Cette liste ne se veut absolument pas une reproduction fidèle de la fiche technique détaillant tous les accessoires et les scénographies.

En ce qui concerne les costumes, j'ai choisi de ne pas les décrire dans une section dédiée. En cas de tenues particulières, celles-ci sont indiquées en note ou dans la description des personnages.
Enfin, j'ai transcrit les avant-spectacles lorsque le script ou la vidéo était disponible\footnote{ En 2014, Jo\"elle Bollon et Laurent Chuc ont chauffé le public avec un avant-spectacle dont, malheureusement, ni le texte ni aucune preuve vidéo n'ont été retrouvés.}. J'ai également cherché à citer, à la fin de chaque pièce, les collaborateurs qui ont travaillé en coulisses. Cependant, la partie finale des remerciements a parfois été coupée de la vidéo. Par conséquent, à travers des photos et des entretiens, j'ai tenté d'identifier les personnes qui ont collaboré tout au long de la production théâtrale, mais je ne peux certainement pas garantir de les avoir toutes citées. Si quelqu'un a été omis, je jure que ce n'est rien de personnel !

\paragraph*{Notes stylistiques}
À mesure que le travail progressait selon la méthodologie décrite dans le paragraphe précédent, j'ai dû faire des choix stylistiques.

En premier lieu, le patois étant une langue minoritaire, plusieurs mots n'existent pas ou se sont perdus avec le temps. Pour combler le vide lexical du patois, il est d'usage courant de s'appuyer sur la langue française. Cependant, dans certains cas, j'ai suivi le conseil de Diego Lucianaz\footnote{ Mes plus sincères remerciements lui sont adressés pour ses nombreuses consultations concernant la recherche de mots ou d'expressions en patois.}, fervent défenseur de Le Digourdì pour le rôle que la compagnie joue dans la diffusion du patois auprès des jeunes. Il m'a suggéré de proposer des néologismes plutôt que d'utiliser le français dans le cas où un mot ne pourrait être traduit. À titre d'exemple, le terme « aspirapolvere » peut être traduit du français par « aspirateur » ; cependant, une solution originale et très théâtrale a été d'inventer le néologisme « peuccapoussa », qui, bien sûr, est légitimé à vivre au sein d'une pièce de théâtre, mais qui, qui sait, pourrait aussi devenir d'usage courant dans les foyers de nombreuses familles. Un deuxième exemple : pour « sedia a sdraio », traduisible par « chaise longue », j'ai proposé le néologisme « caèya di solèi ».
Cependant, chaque fois que je n'ai pas pu ou voulu proposer un néologisme, j'ai emprunté le terme français correspondant en l'écrivant en italique. En général, tous les mots provenant d'une autre langue (italien, français, anglais) ont été écrits en italique pour signaler au lecteur la présence d'un terme écrit avec une graphie différente (\textit{proscenium}, \textit{scène}, \textit{scénographie}).

Les noms propres, quelle que soit leur graphie originale, n'ont pas été écrits en italique : Ferrari, Champagne, Facebook, Orage, Pierre, Francesca... Dans ces cas, la première lettre majuscule identifie un nom propre qui sera lu dans sa graphie originelle, si elle est connue du lecteur.

La ponctuation, l'emploi des majuscules, les critères pour écrire les nombres en toutes lettres et les autres conventions linguistiques suivent, en principe, les règles de la langue française.

Un autre aspect très important pour lequel j'ai dû faire un choix concerne les variantes phonétiques du francoprovençal. Il est connu, du moins parmi ses locuteurs, que le patois varie selon l'endroit où il est parlé, le rendant très hétérogène. Pour des raisons évidentes, la variante utilisée dans \textit{Dji Cou Digourdì} est le patois de la Commune de Charvensod. Cependant, tout en restant dans cette limite linguistique, on y trouve de nombreuses variantes phonétiques : éve/ive, me plé/me pli, féyo/fio, aprestoù/aprestó (en général tous les participes passés en où/ó). Au milieu de cette très vaste richesse phonétique, j'ai cherché un compromis entre cohérence et valorisation des variantes qui, à mon modeste avis personnel, représentent une facette précieuse de l'identité de ses locuteurs. Par conséquent, au sein d'une même pièce de théâtre, le lecteur trouvera le même mot écrit de manière différente ; en même temps, j'ai cherché à éviter que deux variantes ne coexistent dans une même phrase (èitsa/èita, counte/conte).

Par principe, les variantes phonétiques héritées de la langue italienne n'ont pas été transcrites, afin de valoriser, dans la mesure du possible, l'origine française du terme.

Enfin, et ce n'est pas le moins important, la graphie à utiliser pour écrire le patois de Charvensod. J'ai toujours écrit le patois de manière approximative, en m'inspirant de quelques règles phonétiques de la langue française et en demandant conseil à des collègues plus expérimentés. Cependant, pour pouvoir publier \textit{Dji Cou Digourdì}, j'ai dû chercher quelqu'un qui était capable et avait le temps de corriger environ 500 pages de scripts. La seule institution en Vallée d'Aoste que j'ai réussi à identifier pour bénéficier d'un service gratuit de révision linguistique a été le Guichet linguistique francoprovençal de la Vallée d'Aoste, lequel adopte la graphie définie par le BREL\footnote{ Bureau Régional Ethnologie et Linguistique.}. Par conséquent, l'intégralité du texte de \textit{Dji Cou Digourdì}, ayant été révisée par le Guichet linguistique, suit les règles de graphie définies par le BREL. La logique qui imprègne cette graphie consiste à écrire chaque mot exactement comme le locuteur le prononce ; chaque lettre est propédeutique à la reconstruction du son de chaque mot.

Je n'entrerai pas dans le détail des avantages et des inconvénients, des qualités et des défauts de cette graphie plutôt qu'une autre, car ce n'est certainement pas le lieu pour disserter d'un sujet qui exigerait des connaissances et des compétences bien au-delà des miennes. L'objectif de ce bref paragraphe est de justifier le choix stylistique d'adopter cette graphie spécifique. Un choix qui, comme il a été précédemment déclaré, repose sur un aspect de pure praticité, à savoir la possibilité de publier un texte de 500 pages entièrement révisé.

\paragraph*{Un don}
Que représente pour moi \textit{Dji Cou Digourdì} ? Un don pour Le Digourdì, un acte d'amour, car cela m'a demandé une quantité de temps énorme, et le temps donné est amour.
\begin{center}
Il ne me reste plus qu'à vous souhaiter une\ldots\\\vfill bonne lecture !\\\vfill\#todzorpidigourdì
\end{center}




\mainmatter

%\title{L’OPETAILLE MODERNO}
\author{Pièse icrita pe Jo\"{e}l Albaney}
\date{Téatro Giacosa de Veulla, 15 mi 2009}

\maketitle

\fotocopertina{Foto/2009/gruppo.jpg}{Paola Lucianaz, Francesca Lucianaz, Jo\"{e}l Albaney, Jasmine Comé, Paolo Cima Sander, Giada Grivon, Paolo Pession, Pierre Savioz, Ilaria Linty}{Valeria Brunod, Serena Giorgi, Jo\"{e}lle Bollon, Simone Roveyaz, Laurent Chuc, Ester Bollon, Marco Ducly}{2009}\label{link}

\LinkPiese{L'opetaille moderno}{https://www.youtube.com/watch?v=lP22_oK_hws&list=PLBofM-NS_eLJUln45l7VH457fGak_Bk5O&index=10}{.5}

\souvenir{Mon souvenir de l'Opetaille moderno l'è, san doute, la sensach\'on que sayò eun tren de vivre eun sondzo, eugn émoch\'on euncrouayable. L'ie lo premì cou que noutra compagnì, djeusto nèisiya, pouyè si eun vrèi palque\ldots é lo fé que si palque l'ie fran lo Téatre Giacosa, \textit{symbole} di téatro populéro valdotèn, l'ie eun fouà de jouà é responsabilitoù. Dze nen n'i eun dzen souvenir pe la collaborach\'on que n'en i avouì lo Charaban, eun particulié avouì Vittorio Lupi, Mauro Rossi é Giovanni Neri. L'an fé-no découvrì comèn s'apreste a niv\'o professionel eun spectaclle téatral, sourtoù pe sen que regarde la \textit{scénographie} é le costume. Me rappelo avouì plèizì que Aldo Marrari l'ayè chouivi-no deun la produch\'on de totta la pièse.

Deun mon \textit{cœur} l'Opetaille moderno l'a cheur eun caro eumpourtàn, pequé l'a dimoutro-me que n'a ren de pi dzen que traillì avouì de dzi é de-z-amì que l'an la mima pach\'on de té.}{Jo\"el Albaney}

%
\queriaouzitou{
\begin{itemize}

\item[$\bullet$] Pe pourté lo pi poussiblo de dzi i Téatre Giacosa, vi que pe Le Digourdì l'ie lo débù si eun vrèi palque, caque dzor douàn lo spettacllo, son itoù stampoù eunna patél\'o de foillette publisitère é attatchà i pal de la lemie ou si lo vèyo di machin-e di Tsarvensolèn. Damadzo que lo lon de la nite l'a plouì détchise é le Tsarvensolèn son récha-se eun se divijèn eun dou partì: n'ayè qui pensè que sise foillette sayoon itoù betoù de la par de la Quemeua pe annonchì que l'ariàn gavoù ià l'électrisitoù; i contréo, n'ayè qui l'ie stra malechà pequé lo papì é l'entso l'ayoon eumpouertchà lo vèyo de la machin-a. \\

\item[$\bullet$] La \textit{scène} di maladdo \textit{psychiatrique} l'è itaye eumprovizaye deun la dériye proua jénérale sensa que Ester diise ren a gneun! Pouade imajiì la réach\'on é le riaye de totta la compagnì.\\

\item[$\bullet$] Lo Téatre Giacosa de Veulla l'a lo palque eunna mia dicllo ver lo pebleuque. Donque, tcheu le-z-objé avouì de raoue dèyon itre blocoù pe pa colaté. Malerezamente, d'eunna \textit{scène} avouì bièn de trimadzo, le fren di raoue de la coutse de Geromine son digantsa-se. Bièn aloù que noutro Paolo Cima Sander l'è arrevoù  a vardé lo pèise de la coutse que l'ie eun tren de colaté ver lo pebleuque! Queriaou de savèi pe queunta \textit{scène} capite? Alade la tsertchì deun la \textit{vidéo} de la pièse\footnote{ QR code a padze \pageref{link}}.\\

\item[$\bullet$] Pe la pièse l'Opetaille moderno le Digourdì l'an rejistroù leur premiye parodì d'eun \textit{refrain} publisitère: \og Di Congo pe travaill\fg. La publisit\'o orijinelle l'è seutta:

\begin{figure}[H]
%\vspace*{-5pt}
\centering
      \begin{subfigure}{.75\textwidth}
  \centering
    \video\hspace*{0.5mm} \textsc{\small Bollywood TVC - Rio Casa Mia}\hspace*{0.5mm} \video\\\vspace*{2mm}
    \qrcode[hyperlink, height=0.5in]{https://www.youtube.com/watch?v=EoVlYvr5JbI}
  \end{subfigure}%
\end{figure}

\item[$\bullet$] Sel\'on lo premì Prézidàn di Digourdì, Jerome Saccani la\-mè traillì derì le rid\'o pe pouèi veure le dzente feuille de la compagnì totte eun ganeuss\'on!
\end{itemize}
}

\Scenographie
\begin{itemize}
\item[$\bullet$] 2 coutse d'opetaille é 1 coutse a doe plase bièn comodda \lettodoppio ;
\item[$\bullet$] 3 \textit{tables de nuit};
\item[$\bullet$] 2 armouére;
\item[$\bullet$] 2 caèye nèye eun plasteucca ;
\item[$\bullet$] 1 pourtamantì;
\item[$\bullet$] 1 poltronna bièn comodda;
\item[$\bullet$] 1 fondal téatral avouì eunna pourta i mentèn pe laquella le-z-atteur pouon entré é chotre;
\item[$\bullet$] 1 bourset pe le grou sou é 1 pe le sou tri;
\item[$\bullet$] 1 campaneun \campanellino ;
\item[$\bullet$] 3 botèille de \textit{gatorade} ;
\item[$\bullet$] 1 flute ;
\item[$\bullet$] 2 cabaré d’ardzèn: 1 pe pourtì eun \textit{cracker} é l'atro pe la cachoula de la \textit{bourguignonne} ;
\item[$\bullet$] 1 per de motchaou;
\item[$\bullet$] 2 platte de papì, de fortsette é de caoutì \posate ;
\item[$\bullet$] 2 cachoule: eunna grousa pe fé la puré é l'atra pi pégna pe fé bolequé  l'ouillo de la \textit{bourguignonne};
\item[$\bullet$] 1 vèyo é 1 paille.
\end{itemize}

\setlength{\lengthchar}{4cm}

\Character[PIERRE]{PIERRE}{Pierre}{Premì prézentateur, \name{Pierre Savioz}}

\Character[LAURENT]{LAURENT}{Laurent}{Sec\'on prézentateur, \name{Laurent Chuc}}

\Character[GEROMINE]{GEROMINE}{Geromine}{Eunna viille madama\viille\ recovéraye a l'opetaille, \nameF{Jo\"{e}lle Bollon}}

\Character[CASIMIR]{CASIMIR}{Casimir}{Vétchot \viou\ campagnar  recoveroù a l'opetaille, \name{Jo\"{e}l Albaney}}

\Character[ARISTOCRATE]{ARISTOCRATE}{PersEmpourtanta}{Mesieu bièn eumpourtàn, noble é aristocrate, recoveroù a l'opetaille, \name{Marco Ducly}}

\Character[GASTON]{GASTON}{Eunfeurmi}{L’eunfermì personnel de l'Aristocrate, \name{Laurent Chuc}}

\Character[PRIE]{PRIE}{Prie}{Lo prie de l'opetaille \prete , \name{Pierre Savioz}}

\Character[FENNE DI POULISIE]{FENNA DI POULISIE}{Fennepulisie}{Femalle arbeillaye comme le méccanisièn de la \textit{Ferrari}, eunterprétaye pe \textsc{Jasmine Comé}, \textsc{Giada Grivon} é \textsc{Serena Giorgi}. Giada é Jasmine feràn eunc\'o le-z-eunfermie de la \textit{psychiatrie}.}

\Character[STARTER]{STARTER}{Starter}{Cape di fenne di poulisie, \name{Simone Roveyaz}}

\Character[EUNFERMIE]{EUNFEURMIE}{Eunfeurmie}{Eunfermie de l'opetaille avouì eun caratéo d'eun maréchal, \nameF{Francesca Lucianaz}}

\Character[JOSETTE]{JOSETTE}{Felie}{La feuille de Geromine, \nameF{Ilaria Linty}}

\Character[MEDESEUN MITCHO]{MED. MITCHO}{MedMitcho}{Lo \textit{Doctor House} de l'opetaille, \name{Paolo Cima Sander}}

\Character[FERNANDA]{FERNANDA}{Fernanda}{Lo Chef de l’opetaille, \nameF{Ester Bollon} Ester ferè eunc\'o lo maladdo \textit{psychiatrique}.}

\Character[TSAMBR\`I]{TSAMBR\`I}{Tsambri}{Tsambrì personnel de l'Aristocrate, \name{Pierre Savioz}}

\Character[]{FENNA DI POULISIE I}{FennepulisieA}{\hspace*{0cm}}
\Character[]{FENNA DI POULISIE II}{FennepulisieB}{\hspace*{0cm}}
\Character[]{FENNA DI POULISIE III}{FennepulisieC}{\hspace*{0cm}}

\DramPer

\act[\avanSpect\ Avanspettaclle \avanSpect]
\StageDir{\hspace*{2.5em}Se senton de vouése que veugnon de dérì la tèila.}

\begin{drama}

\Pierrespeaks Laurent entra té, mé n'i pouiye.

\Laurentspeaks Pouiye? Feulla!

\Pierrespeaks \direct{Ajitoù\ajitou} Soplé va té!

\Laurentspeaks T'i tan grou é t'a pouiye de cattro personne que son vie no vére?!

\Pierrespeaks Ad\'on fièn na baga? Baillèn dou creppe a la moura!

\StageDir{Le dou fan eunna man a la moura (``tchisse, cattro, totta man, satte\ldots '').}

\Laurentspeaks Ah! Pierre, deusteun! Feulla!

\Pierrespeaks\direct{Eun prégnèn coadzo} Ad\'on vou mé!

\StageDir{Lemie \lemieSi\ desù lo \textit{proscenium}. Pierre chor de dérì la tèila.}

\Pierrespeaks Bonsoir a tcheutte!

\StageDir{Entre eunc\'o Laurent é se plache a drèite de Pierre.}

\Pierrespeaks\direct{A Laurent} Mondjemé véo de dzi oueu lo nite! N'ayoù pa la fèi tan pai!

\Laurentspeaks\direct{Eun avèitsèn lo pebleucco} N'ayoù pa la fèi tan pouèi!

\Pierrespeaks A mé l'ay\'on deu-me que dèi seu desù lo palque te vèyave ren de sen que capitae déz\'o lo palque! \direct{Ajitoù} Mi me semble pa! Se vèi totte!

\Laurentspeaks Eh ouè! Mi se t'avèitse amoddo trèi car de la sala son pa de Tsarvensoù!

\Pierrespeaks T'a rèiz\'on! Can mimo\ldots a Fin-is l'an pi eunc\'o de dzente feuille! \`Eita salle do lé eun premì feulla!

\StageDir{Laurent seuble i feuille.}

\Pierrespeaks Mi va savèi péqué n'a pi de dzi de Fin-is oueu!

\Laurentspeaks Pierre! Avouì tcheu le tracasse que le Tsarvensoulèn l'an i dzor de oueu, fegueua-té se l'an lo ten de vin-ì sé pe avèitchì no!

\Pierrespeaks T'a rèiz\'on! Son tcheu pi greundzo\ldots pe deue\ldots té pensa que eun cou t'ayè praou te réchì a satte é demì pe alé eun Veulla a ouètt'aoue. Ara, avouì seutta dzenta arionda que l'an fé-no ba i Pon-Suà te fa te réchì a chouì é demì! Péqué eun cou que t'i  ba, te fa baillì la présédanse i Gressaèn; aprì pi de 50 an que n'en i no si drouette! \'E comme se bastise pa, can te feullon devàn l'an eunc\'o eun grou sourì a 32 di \sorrisone .

\Laurentspeaks T'a rèiz\'on! Aprì te fa eunc\'o chotre demì aoua devàn di travaille pe alì a l'icoula prende le botcha é alì i mitcho pe fé medjì!

\Pierrespeaks Mi na! Senque te me di? L'è pa lo travaille di-z-ommo sitta! Son le fenne que dèyon pensé i megnadzo!

\Laurentspeaks \direct{Ironique} Eh brao! Eun cou l'ie pouèi! Aya a Tsarvensoù, dèi can la betoù si do formach\'on di femalle di fiolet, leur pi grou tracasse l'è si d'aprestì l'èima, baillì la biacca i fiolet é prenotì le campe réjonal pe la demendze.

\Pierrespeaks Ouè, ouè\ldots sitta cou va fran tott'a bal\'on!

\Laurentspeaks Te pou pa resté avouì seutta vya! T'i stressoù! Aprì fenèi que eun se beutte maladdo!

\Pierrespeaks Queutta pédre! Te la sa la dériye?

\Laurentspeaks Di-mé.

\Pierrespeaks Noutro Senteucco, Ennio Subet, dèi can l'a si qué que lo Ministre Brunetta l'è fé-se eunna loué si lo travaille, l'a voulì eunc\'o llou fé la sin-a.

\Laurentspeaks N'ayò sentì de seutta loué: la \textit{Subettina}. Espleucca vèi comèn fonchoun-e!

\Pierrespeaks Ad\'on, te sa que Brunetta l'a desid\'o d'aoumenté l'oréo de la \textit{mutua}; donque, lo noutro Senteucco la betoù na riilla pe tcheut le sitouayèn! Dèi ara can eun l'è maladdo, l'è maladdo i $100\%$! Te pou gnenca pamì alì ba a la crotta pe prendre eunna boua botèille de veun ou alì de foua pe couillì le pomme \mela .

\Laurentspeaks Ah ara n'i comprèi péqué l'è na senâ que vèyo pa lo meun vezeun di mitcho é pe tèra n'at totte le pomme pouriye.

\Pierrespeaks \ldots é fa fé attench\'on! Péqué lèi  son le vallet de veulla que pasoun, dzor é nite, mitcho pe mitcho pe verifiì. An effè, l'an pamì lo ten de pasì i mentèn de la parotse pe baillì la multa i machine parquédjaye mal!

\Laurentspeaks \ldots é comme se n'isse panco praou, n'a eunc\'o d'atro mé cher-z-amì, péqué eun cou que t'a aprést\'o sin-a, t'a baillà medjì i mèinoù é t'a lav\'o le-z-éze\ldots te l'amerie pa t'itaoulé eun momàn desù lo chofà pe avèitchì lo \textit{Grande Fratello}?

\Pierrespeaks Ouè!

\Laurentspeaks Eh na! Péqué totte le souaré, can si preste m'achouaté si lo chofà lo tseun comenche a djapì é molle pa tanque lo pourto pa féye eun tor deun lo péi!

\Pierrespeaks Mi senque te me di? Itèn pa pe eunna grousa Veulla! T'a praou baillì campa a la tsèin-a é lo tor se lo fé da solette!

\Laurentspeaks A Tsarvensoù?! Mi t'i matte! Fenèi que lèi teurion desì avouì lo fezì!

\Pierrespeaks Me rappelao pamì de seutta counta! \'E t'a sentì que n'at eunc\'o la finanse é le carabignì di bouque que son eun tren de cayì to pe l'er lo péi pe tchertchì seutte-z-arme?

\Laurentspeaks N'ay\'o sentì de seutta counta; é say\'o tellemàn tracachà que n'i falì euntéré deun lo courtì le dou pistolet a éve: le dou \textit{Super Liquidator} di botcha!

\Pierrespeaks Acouta vèi seuilla.

\StageDir{Laurent s'aprotse.}

\Pierrespeaks \direct{Eun moutrèn quetsouza déz\'o la djacca} Te vèi lo pistolet a éve que n'i seuilla?
Me fa todzor lo pourté aprì\ldots l'è pa denonchà.

\Laurentspeaks Oh pouff, que viya! No n'en eunc\'o baillà lo non Digourdì a noutra compagnì!

\Pierrespeaks Ouè que coadzo!

\Laurentspeaks  Spéèn que le tréze que son séilla dérì \direct{moutre la tèila} teugnon pi ate lo non di groupe!

\Pierrespeaks Can mimo douàn can sen chortì di tend\'on n'ay\'on tan pouye di dzi\ldots ara n'en finque prèi tro de coadzo.

\Laurentspeaks N'en eunc\'o prao deu-nen! Spéèn can mimo que la quemeua é tcheutte le Tsarvensolèn contenisa no baillì na man; piatro sit an l'è lo premì é dérì an que fièn lo \textit{Printemps Thé\^{a}tral}.

\Pierrespeaks Ouè! \direct{Eun avèitsèn la moutra \orologio} Ara senque te nen di se prézentisan la pièse?

\Laurentspeaks \direct{Eun avèitsèn lo seun pouse sensa moutra \footnote{ Malerezamente Laurent l'è oublia-se de betì la moutra, mi lo pebleucco l'a bièn riette pe seutta gaffe.}} Ouè l'è l'aoua de comenchì, mi devàn vourio fé na présizach\'on pe tcheu le Tsarvensolèn: tcheu sise manifeste que n'en bet\'o ia pe la parotse, l'è pa pequé gavon l'éve ou l'élétrisit\'o, mi l'è pe lo spétaclle de seutta nite! Tracachade-v\'o pa!

\Pierrespeaks Amoddo! Mogà no sistémèn an miya pe la prézentach\'on?

\StageDir{Pierre é Laurent se cllouzon la djacca é drichon lo papillon \papillon . Eun pi, Laurent se beutte si le poueunte di pià é Pierre se plèye si le dzegnaou de fas\'on que sisan ate tcheut dou igale.}

\Pierrespeaks \textit{Mesdames} é \textit{Messieurs} a vo ``L'opetaille moderno''. Pise icrita pe Jo\"{e}l Albaney.

\Laurentspeaks Boun-a souaré a tcheut!

\StageDir{Pierre é Laurent chorton. Teuppe \lemieBa .}

\act[Acte I]

\ridoiver

\scene[-- L'Aristocrate é le pouo matasse]

\StageDir{Partèi la tsans\'on:}

\sound{https://www.youtube.com/watch?v=cntvEDbagAw}{A Message to You Rudy -
The Specials}

\StageDir{Eun \textit{scène} n’a trèi coutse. Doe coutse, salle iaou son itaouloù Casimir é Geromine, son bièn seumple é protso n'a maque doe caèye é doe \textit{tables de nuit}; i contréo, la trèjima coutse (a gotse) salla de la personna eumpourtanta l’è d’ott\'on, queusseun de sèya, \textit{couvre-lit} avouì le frandze. Acoutì n'at eunna comodda poltronna é eunna \textit{table de nuit}. Pe completì la \textit{scénographie}, i fon di palque n'a dou-z-armouére eun feur, eun pourtamantì é eun grou fondal avouì eunna pourta i mentèn.\\ L'aristocrate l’è catchà déz\'o le queverte é l'è eun tren de drimì. Geromine l’et eun tren de fé lo tsaous\'on é lo vioù l’et eun tren de lie lo journal avouì le lenette ba desì lo na.}

\StageDir{Se sen pamì la tsans\'on.}

\Gerominespeaks \direct{A Casemir} L’è na senâ que si seuilla dedeun é si pa senque atègnon a me fé chotre.

\Casimirspeaks T’atten pi eunc\'o té comme tcheutte!

\Gerominespeaks Si pi praou coudzia. Mi mé sensa lo meun courtì é le mie dzeleunne \gallina\ si perdiya!

\Casimirspeaks \direct{Euntre lli} Eunna pi, eunna mouèn\ldots

\Gerominespeaks N’i pa comprèi\ldots senque t’a deu?

\Casimirspeaks Ren, ren! Diavo maque que te pouave te le porté avouì té! Omouente n’ario magà pouì féye de nouo discour avouì caqueun. Le-z-aoue comenchon itre londze seuilla dedeun\ldots é lo spettacllo l’et todzor lo mimo \direct{avouì lo dèi moutre \deidreite\ Geromine} é ta viya la si perqueue: l’è chouì cou que te me la conte!

\Gerominespeaks Te pense que a mé fiyisse plèizì restì seuilla avouì eun rabadàn comme tè? Surtoù veure, can te tsandzon le fise, eun moustre patanì pi é ouse, blan comme lo lasì é sensa pamì ren de euntéressàn pe le femalle.

\Casimirspeaks Magara pe de femalle comme té! Se le femalle que travaillon seuilla son jantile maque avouì mé n’aret eunna rèizón! \'E té te fa pa itre dzalaouza! \'E aprì resta maque tranquilla, itsauda-té pa\ldots piatro te pren eugn atro \textit{infarto}!

\Gerominespeaks Mé ara resto adì bièn é si pa senque baillerio pe itre eun campagne.

\Casimirspeaks Ouè fa beun diye que s’en eun trèi malado eun tsambra é l’atro \direct{moutre la personna eumpourtanta} l’a panco t’an prédzà.

\Gerominespeaks Ouè ren! S’en gneunca lo non. Se n'i comprèi amodo dèi itre euna grousa tita.

\Casimirspeaks Eh vouè, t’ayè panco comprèi-lo? Mi te avèitse pa lo TG3? Sitte l'é tcheu le momàn eun télévij\'on \tv . Eun dzor pe an bagga é eun dzor pe eugn'atra.

\Gerominespeaks Na. Mé acouto maque la \textit{Voix de la Vallée} desù la radi\'o \radio .

\Casimirspeaks Ad\'on l’é l’aoura que te te modernizisse an miya!

\Gerominespeaks Mé reusto bièn pai: de ouette a onze se si i mitcho \textit{Radio Zeta} avouì Ciccetti é a maenda lo \textit{Gazzettino}, avouì Cesarino!

\StageDir{Se rèche l'aristocrate que l’ie eun tren de drimì. Se derèidì é aprì pren lo campaneun \campanellino\ pe querì lo seun valet.}

\PersEmpourtantaspeaks Gaston! Gastooon!

\StageDir{Entre Gaston.}

\Eunfeurmispeaks Bondzor mesieu! Sade dza réchà?

\PersEmpourtantaspeaks Djaque!  Queun dzor l'é oueu? 

\Eunfeurmispeaks Oueu l'è devendro 15 Mi. L'è onj'aoure é sinque di mateun, lo ten l'è séèn é eun Val d'Outa l'è pa capitoù ren d'eumpourtàn.

\PersEmpourtantaspeaks Amoddo, ad\'on refèicha-mè djeusto le queverte.

\Eunfeurmispeaks To de chouite!

\StageDir{Gaston gnouye refèichì le queverte.}

\PersEmpourtantaspeaks Tro itrèite! 

\Eunfeurmispeaks\direct{Eun refèichèn} Va mioù pouai?

\PersEmpourtantaspeaks Tro lardze!

\StageDir{Gaston moungouye eunc\'o eun momàn.}

\Eunfeurmispeaks Va bièn ara?

\PersEmpourtantaspeaks Ouè va bièn.

\StageDir{Lo vioù, que l'a vi la \textit{scène}, l’a i fon de la coutse le queverte beuttaye mal é lo toppon pa totte. Ad\'on pe pa trebelì proue a se fé èidjì.}

\Casimirspeaks Gaston! Squezade Gaston! 

\StageDir{L’eunfeurmi fé semblàn de ren, ad\'on lo vioù tourne lo criì eun vouaillèn.}

\Casimirspeaks Deh, Squezade! 

\Eunfeurmispeaks Ouè.

\Casimirspeaks Me refèichade la coutse i fon pe plèizì?

\Eunfeurmispeaks Mi squersade pa mesieu, pouì pa vegnì tanque lé! Aprì qui reste seuilla a avèitchì lli?

\Casimirspeaks Que demanda foula que n’i fé. L’é vrèi n’ayoù pa pensou-lèi.

\StageDir{Se viounde di coutì de la viille é sopatte la tita.}

\scene[-- Le sen-z-ouillo]

\StageDir{Entre, de la pourta i fon di palque, eun prie tot arbeillà de neur, avouì la Bible eun man.}

\Priespeaks\direct{Eun fièn la croueu} Bondzor a tcheutte.

\Casimirspeaks Bondzor Monseur.

\StageDir{Lo prée s'aprotse a Gaston, que l'ie eun tren de poulitì le-z-onlle de l'aristocrate.}

\Priespeaks Mesieu\ldots

\Eunfeurmispeaks Ouè.

\Priespeaks \ldots qui l'è lo pi vioù é pi maladdo seuilla?

\Eunfeurmispeaks \direct{Eun moutrèn Casimir}
L'è si lé!

\Priespeaks Mersì.

\StageDir{Lo prie s'aprotse a la coutse de Casimir é dousemàn, avouì bièn de pitié, lèi fé la croueu \croce\ si lo fron.}

\Casimirspeaks\direct{Eun branquèn lo bri di prée} \'E so?! Que fiade? Mi squersade pa!

\Priespeaks \direct{Eun lèi prédzèn comme fuse eun pouo matasse} Mesieu, comme la demando-me voutra fameuille si vin-ì vo saliì pe lo dérì cou.

\Casimirspeaks \direct{Eun se totsèn le-z-attribus} Mi na na! Que fiade? Si pa preste a moueure mé!

\Gerominespeaks Mi senque te te totse? N'a pamì ren lé \riye !

\Casimirspeaks Ah! souplì là!

\Priespeaks \direct{\'Eton-où \ouaou} Comèn?! Mi seutta l’è pa la tsambra numer\'o 17?

\Casimirspeaks Na, na seutta l’é la tsambra nimér\'o\ldots \direct{se veurie ver Geromine} nimér\'o?

\Gerominespeaks \direct{A basa vouése} 23.

\Casimirspeaks Seutta l’è la 33!

\Gerominespeaks \direct{Eun vouaillèn} 23!

\Priespeaks Boundjeu de la France! Squezade-mé tan! Si fran trompou-me! Ad\'on me fa belle partì, piatro me scappe finque lo mor!

\StageDir{Baille la bénédich\'on a la tsambra é can l'è preste a chotre tourne eun derì ver Casimir.}

\Priespeaks Mesieu, si pa se v'ouèide sentì que no fa refé lo tette de l’éillize. Se vouillade pouade fé can mimo eunna queilletta pe l’éillize\ldots

\Casimirspeaks Na mé n'i dza baillà proou a meun ten.

\Priespeaks\direct{Diplèizì} Ah. \direct{Ver Geromine} Vo madama?

\Gerominespeaks Me deplì, mi mé se dérì n'i pa lo boursette. Pi que de bot\'on n'i pa.

\Priespeaks Fé pa ren ad\'on. Fé pa ren. Tracachade vo pa.

\Casimirspeaks Acoutade na baga: \direct{eun moutrèn la personna eumportanta} deumandade a llou que nen n'a d’avanse!

\Priespeaks Ah mersì bièn.

\StageDir{Lo prie va ver l'aristocrate.}

\Priespeaks\direct{A Gaston} Bondzor mesieu. Poui-dz\'o lo réchì?

\Eunfeurmispeaks Eh na, l'è eun tren de drimì. L'è mioù que lo rècho mé.

\Priespeaks Va bièn. Vito que n'i prisa.

\StageDir{Gaston to todzèn tsertse de réchì lo seun mesieu.}

\Eunfeurmispeaks\direct{Eun totsèn la man a l'aristocrate} Mesieu. Mesieu\ldots

\StageDir{L'aristocrate se rèche pa.}

\Priespeaks \ldots squezade. Mi n'i fran prisa. Lèi penso mé.

\Eunfeurmispeaks Va bièn, comme vouyade.

\StageDir{Lo pri beutte eunna man si l'ipala de l'aristocrate.}

\Priespeaks\direct{Eun braillèn} Mesieu!

\StageDir{L'aristocrate se rèche épouvant\'o .}

\PersEmpourtantaspeaks Ah! Eh! Oh, senque?!

\Priespeaks Bondzor.

\PersEmpourtantaspeaks Salì!

\Priespeaks Mesieu l’an deu-me que v'ouèi de voye moustre de fé de bienfezanse pe l’éillize.

\PersEmpourtantaspeaks Oué\ldots \direct{Ver Gaston} va prende lo bourset.

\StageDir{Gaston pren lo bourset di mesieu é teurie foua bièn de sou \cash . Lo mesieu lèi di de na avouì la tita; ad\'on Gaston nen teurie foua eugn atro pi petchoù é lo mesieu tourne lèi diye de na.}

\Eunfeurmispeaks Monseur n'i pa de tri. Vou a vére pe lo bourset di pise.

\StageDir{Gaston tourne eun derì avouì eun per de sou tri eun man.}

\Eunfeurmispeaks N'i acap\'o so!

\StageDir{Gaston baille i prie le dou tri que l'a prèi pe lo bourset.}

\Priespeaks Oh mersì. \direct{Ironique} V'ouite fran itoù jantilo.

\Eunfeurmispeaks Mersì a vo.

\Priespeaks Oueu lo nite fiyo eunna prière pe tcheu vo é tourno vo troué demàn mateun. Repouzade-v\'o. Orvouar a tcheutte.

\StageDir{Lo prie fé la marca de la croueu é chor.}

\scene[-- Ferrari \ferrari\ poulisie]

\StageDir{Entron trèi femalle di poulisie: son arbeillaye avouì eun tone rodzo de la Ferrari. Eunna avouì l'icaoua \scopa , l’atra avouì de prodouì pe poulitì é la dérie avouì eun tchappapoussa.}

\Fennepulisiespeaks Bondzor a tcheutte!

\Casimirspeaks \direct{Eun fièn vère que l’è to contèn} Bondzor! Voualà l'è dza torna aoura di poulisie!

\FennepulisieAspeaks Bièn lévoù?

\Gerominespeaks Mi ouè, mersì, é vo?

\FennepulisieBspeaks To bièn, mersì madama.

\Gerominespeaks V'ouite dza torna preste pe eunna nouva galoppaye?

\FennepulisieBspeaks Eh no totse\ldots

\Casimirspeaks Ouè mi sen comèn l’an aprèi amoddo lo patoué seutte!

\FennepulisieCspeaks Sen itaye coudziye. Diyon que lo maladdo se te l’èi prèdze sin-a lenva se queutte bièn pi alé\ldots

\Casimirspeaks Ouè, ouè fran pèi!

\StageDir{A si poueun, entre lo cape avouì eun cronomètre \cronometro\ é eunna bandjèira a car\'on blan é ner (salla que emplèyon pe le compétich\'on di machine).}

\Starterspeaks Sah, sah, ad\'on v'ouite tcheu preste? Dai eh! Beuttade-v\'o totte eun reugga, eunc\'o té!

\FennepulisieAspeaks Mi vouè, resta tranquilo. Oueu si eun plèin-a forma!

\Starterspeaks No fa betì mouèn de do meneutte, piatro no payon pamì! Perqué lo cou passoù \direct{avèitse eunna fenna di poulisie} l'amia de té l’an spédi-la ià perqué l’a ralent\'o totte le-z-atre!

\FennepulisieBspeaks Ouè, fa beun diye que l’an beutt\'o pi de do meneutte!

\FennepulisieCspeaks Pensa que salla poura matassa perdave de ten a poulitì tcheut le cassette!

\FennepulisieAspeaks Bondàn ézajér\'o! L’ayè panco comprèi comèn martse seuilla dedeun. Djeusto la mandì i mitcho!

\Starterspeaks Bon perdèn pa de ten, prest? Trèi, dou, eun\ldots Via!

\StageDir{Fé partì lo cronomètre é sopatte la bandjèira \bandiera\ pe l’er. Le femalle totte a galoppe pouliton lo pi vito poussiblo; saouton d’eun coutì a l’atro; eun mimo ten tsanton é danchon totte eunsemblo, avouì finque le dou vioù, la tsans\'on:}

\sound{https://soundcloud.com/user-234168361/di-congo-pe-travailli}{Di Congo pe travaillì - Le Digourdì}

\StageDir{Can la mezeucca fenèi, se plachon protso i leur cape.}

\Starterspeaks Stop! Brave!

\FennepulisieBspeaks Si cou l’è fran bièn aloù.

\Starterspeaks Ouè, an meneutta é 36 seconde! Eh, can v'ouite vo trèi l’è totte eugn atro travaillì! 

\Fennepulisiespeaks Iuhuh!

\Starterspeaks Ara vo béyade maque eun bon \textit{Gatorade} pe eunna\ldots

\FennepulisieCspeaks Vito que n'en sèi!

\StageDir{Lo cape distribuèi lo bèye eun terièn foua de l'abrosaque trèi botèille de \textit{Gatorade}.}

\Starterspeaks \ldots é lo \textit{champagne} \champagne\ vo atèn pi de foua!

\Fennepulisiespeaks Iuhuh!

\StageDir{Chorton tcheu catro.}

\Gerominespeaks Son tcheu mat seuilla dedeun!

\Casimirspeaks Dériàn recoverì seutte dzi é pa no!

\Gerominespeaks Crèyo beun!

\scene[-- L'aoua di vezeutte]

\StageDir{Entre l’eunfermie.}

\Eunfeurmiespeaks \direct{Eurle avouì acsàn tédesque} L'oréo di vezeutte comenche ara!

\Eunfeurmispeaks Sht! Te vèi pa que lo meun Chef drime!

\Eunfeurmiespeaks \direct{Arroganta} Me euntéresse pa, sitte l'è lo meun répar é fio sen que n’i voya!

\Eunfeurmispeaks Avouì llou \direct{moutre lo seun Chef} séilla véyèn pi tanque can!

\Eunfeurmiespeaks Acouta bièn pégno eunfermì de prouva: mé n'i 27 an de servicho é de seutta plasse l’a jamì tramou-me gneun! A-te comprèi?

\Eunfeurmispeaks Mah, lèi dzouyério pa le mitcho.

\Eunfeurmiespeaks Can mimo torno repété: \direct{eun vouaillèn} L’oréo di vezeutte comenche ara!

\StageDir{L'eunfermie chor de \textit{scène}.}

\Gerominespeaks Sen pamì tan dzoun-euo mi n’en praou de sentì eun cou pe comprendre senque no dion.

\Casimirspeaks Ah se l’è comme totta la senâ vou cheur me lagnì avouì to si mondo que veun me trouvì.

\Gerominespeaks Compregno beun perqué t’a pa gneun. T’i gramo comme eun pet.

\Casimirspeaks \direct{Eun desuèn la vouése de Geromine} T'i gramo comme eun pet!

\PersEmpourtantaspeaks \direct{Ver l’eunfermì} Acoutta, n’i sèi.

\Eunfeurmispeaks Ouè,  vo pourto eun vèyo d'éve \bicchiere ?

\PersEmpourtantaspeaks Vèi té.

\Eunfeurmispeaks Vouillade salla di grousse boule ou salla di seunteucco?

\PersEmpourtantaspeaks Salla di boule va bièn.

\Eunfeurmispeaks Ouè to de chouite, mesieu.

\StageDir{L'eunfermì reumplèi eunna flute d'éve é la soum\'on a l'aristocrate (djeusto eunna pégna gotta). Aprì, reprèn lo vèyo é lo beutte eun caro. Dimèn entre eunna dzenta feuille blounda, Josette, la feuille de Geromine.}

\Feliespeaks Bondzor a tcheutte!

\Casimirspeaks \direct{Eun se terièn si de la coutse pe vère mioù} Bondzor!

\Gerominespeaks Oh! Avèitsa qui se vèi. Ad\'on te sa qué que t’a na mamma?

\Feliespeaks Mamma di pa pouèi. N’i pa i lo ten devàn é avouì tcheu le-z-eungadzemèn que n'i.

\Gerominespeaks Ouè, ouè, bailla-mé eun poteun.

\StageDir{Se baillon eun poteun. Josette s'achouatte si la quèya protso de la coutse.}

\Feliespeaks Can mimo to bièn? Comèn se reuste seuilla?

\Gerominespeaks \direct{D’eun ton ironique} Se reste bièn. L’è caze eun oberdze. Si poste seuilla ara l’a maque trèi-z-itèile \stella\ pequé la catrima l’an djeusto gavou-la. \direct{Avèitse Casimir ironiquemàn} Aprì la compagnì l’è fran dzenta! Fiyèn de salle riaye é de salle fite que avèitsa!

\Feliespeaks S’i fran contenta. Ad\'on t’a fran fata de ren ad\'on? Mé n'i pensoù de pourti-te an bouite de biscouì.

\Gerominespeaks Oh mersì Josette. \direct{Avèitse la conféch\'on} Lo pri te pouave lo gavì!

\Feliespeaks An ouè n’i oublià! Squeza-mé!

\Gerominespeaks Pe $0.80$ santime t’aré cheur atseuttou-me de biscouì \biscotto\ bièn bon!

\Feliespeaks Mamma n’ayet l’offre a la LIDL é pouèi n’i fé que spendre an mia de pi pe atsetì doze conféch\'on.

\Casimirspeaks Seutta l’è cheur tin-a feuille!

\Gerominespeaks Ita tchica quèi té! Pa fata d'acouté noutre conte!

\PersEmpourtantaspeaks \direct{Ver l’eunfermì} Gaston, n’i fan: quetsouza a medjì.

\Eunfeurmispeaks Vo baillo eun \textit{cracker}. 

\PersEmpourtantaspeaks Va bièn.

\Eunfeurmispeaks To de chouite, mesieu. Vouillade sise avouì la sa ou sensa sa?

\PersEmpourtantaspeaks Avouì la sa. 

\StageDir{L'eunfermì, avouì eun cabaré d'ardzèn, lèi pourte eun \textit{cracker}; l'aristocrate lèi baille eunna mordia é tourne lo pouzé desì lo cabaré. L'eunfermì beutte tourna totte eun caro.\\ Entre lo medeseun.}

\scene[-- Medeseun Mitcho \casa]

\MedMitchospeaks Bondzor a tcheutte!

\Eunfeurmispeaks  Bondzor mesieu! L'è arreuvoù Doctor \textit{House}!

\Casimirspeaks Qui?

\Eunfeurmispeaks Medeseun Mitcho!

\MedMitchospeaks Ad\'on comèn l'è, to bièn pe seuilla?

\Gerominespeaks Todzor bièn mesieu lo medeseun. Mé resto bièn eunc\'o oueu.

\MedMitchospeaks Véyèn pi ara, madama. Ad\'on, teuriade si la mandze é baillade-mé lo bri.

\StageDir{Lo medeseun mezeue la préch\'on.}

\MedMitchospeaks\direct{Eun s’apesissèn de la feuille \inamourou} \ldots é seutta dzenta demouazella de iaou chort?

\Gerominespeaks L’è ma feuille.

\MedMitchospeaks Ah n’i comprèi\ldots

\StageDir{Lo medeseun fenèi de prendre la préch\'on.}

\MedMitchospeaks Voualà lèi s’en caze madama! Dou où trèi dzor é vo mando i mitcho.

\Gerominespeaks Ouè l’è devendro é vo èi dza deu-me seutta baga demarse!

\MedMitchospeaks Me diplì mi la voutra préch\'on monte é bèiche tcheu le momàn; é aprì avouì an feuille pai penso que vo féyo pi restì eunc\'o seuilla doe ou trèi senâ.

\Gerominespeaks Ah! Restade maque tranquillo que vo èi vi-la oueu é la vèyade cheur pamé a me trouvì!

\Casimirspeaks Pa deutte\ldots l’a eunc\'o onze conféch\'on de biscouì a fére foura!

\Gerominespeaks Resta tchica quèi té!

\StageDir{Lo medeseun fé lo tor de la coutse é euncrije le joueu de Josette, laquelle se trame pe lo quetì traillì.}

\MedMitchospeaks\direct{A Geromine} Mé aya l'amerio\ldots teriàde-vo si\ldots

\StageDir{Geromine se beutte achouataye si la coutse avouì eunna mia de difficult\`o.}

\MedMitchospeaks Amoddo, teriade si le bri, si lo cou, cllouzade le joueu, sarade bièn la botse é ara sarade é cllouzade le man pe trèi car d'aoua de ten\ldots 

\StageDir{Geromine fé totte senque lèi di lo medeseun é, dimèn que ivre é cllou le man, lo medeseun nen profite pe fé lo fleungàn avouì Josette.}

\MedMitchospeaks\direct{A Josette, bièn malisieu \malisieu} Ad\'on squezade, mi queun l'è lo voutro non?

\Feliespeaks Si Josette, plèizì!

\MedMitchospeaks Oh Josette! Mé si \textit{House}! Medeseun \textit{House}, mi criàde-me maque Mitcho; é diade-mé, an fenna dzenta comme vo senque fé? Queun travaille fé?

\Feliespeaks Si dirijanta eun Réj\'on.

\Gerominespeaks\direct{Eunfastedjaye \malechaa pe itre itaye abandonaye} Va bièn pouèi medeseun!

\MedMitchospeaks Ouè madama! Trèi car d'aoua n'en deutte.

\Gerominespeaks Mi comencho a itre lagnaye!

\MedMitchospeaks Alé, conteniade.

\StageDir{Geromine conteneuvve a clloure é ivrì le man avouì le joueu clloujì. Lo medeseun Mitcho reprèn a martélé Josette.}

\MedMitchospeaks\direct{A Josette} Ad\'on diavo, an dzenta feuille comme vo l'è mariaye?

\Feliespeaks Na si \textit{single}\ldots

\MedMitchospeaks Ah v'ouite \textit{single}! Me deplì\ldots

\Gerominespeaks A mé na! Ad\'on va bièn pai medeseun?

\MedMitchospeaks Ouè madama! 

\StageDir{Lo medeseun tourne s'occupé de Geromine.}

\MedMitchospeaks Ad\'on madama, teriade foua le pià de la coutse é itade achouataye. Levade si le man é aprì pléyade l'itseun-a é portade le man tanque i fon di pià.

\StageDir{La poua Geromine se plèye si é ba eun per de cou avouì l'èidzo di medeseun que la cllou é l'ivre comme eun livro.}

\MedMitchospeaks Ba é si, si é ba\ldots tourna trèi car d'aoua pai.

\Gerominespeaks Todzor avouì le joueu cllouzì?

\MedMitchospeaks Ouè.

\StageDir{Lo medeseun tourne avouì Josette.}

\Casimirspeaks\direct{Eun desuèn Josette} Si! Ba. Si é ba! Si, si, si!

\StageDir{To d'eun cou, entre eun maladdo psychiatrique que, eun trambélèn, eurle:``Tchouf, tchouf! Tchouf, tchouf!''. Aprì eun per de seconde doe eunfermie entron i galoppe, fan eunna pouenteua i pasiàn ià de tita, l'eumbrancon é lo teurion foua de pèise. Dimèn medeseun \textit{House} tranquilize Josette é Geromine.}

\MedMitchospeaks\direct{Ver Casimir} Ara pasèn a vo. 

\Casimirspeaks Ouè, prest!

\StageDir{Medeseun \textit{House} s'aprotse a Casimir.}

\MedMitchospeaks Lo bri mersì é si la mandze.

\StageDir{Lo medeseun mezeue la préch\'on, mi dimèn avéitse Josette é sensa s’apesèivre countenie a saré la poumpetta de la préch\'on que sare todzor pi lo bri di pouo Casimir.}

\Casimirspeaks\direct{Eun souffràn \dolore} Ahi, argh, ahi!

\MedMitchospeaks Ouè mi que préch\'on ata! 

\Casimirspeaks Que drolo! La feuille que l’è seuilla protso l’è dza eun momàn que l’è arrevaye é penso fran que sise sen!

\MedMitchospeaks Souplé! Son pa de bague pe voutro éyadzo!

\Casimirspeaks Avèitsade que mé tanque ara le pastiille bleuve n’i maque vi-le eun publisitoù!

\MedMitchospeaks Lèi manque eunc\'o finque lo \textit{Viagra}, pai te me stchoppe comme eun pallontcheun! Sa, ara ivrade la botse\ldots

\StageDir{Avouì eunna lemie medeseun Mitcho avèitse dedeun la botse de Casimir, mi teurie vitto eun déri la tita.}

\MedMitchospeaks Ouè mi que bon flo de crotta! Senque vo baillon medjì seuilla dedeun lo mateun? \textit{Bagna cauda} é biscouì? 

\StageDir{Lo medeseun pren coadzo é avèitse tourna deun la botse de Casimir.}

\MedMitchospeaks Diade ouè\ldots

\Casimirspeaks\direct{Avouì la botse iverta} Ouè!

\MedMitchospeaks Diade na\ldots

\Casimirspeaks\direct{Avouì la botse iverta} Na!

\MedMitchospeaks Ouè\ldots Mogà oueu lo nite devàn que alì a drimì queuttade la dentchiye \dentiera\ dedeun la \textit{conegrina}!

\Casimirspeaks Va bièn, comme vouillade vo medeseun mitcho.

\MedMitchospeaks \ldots é demàn mateun lèi baliade eunc\'o eunna dzenta frotaye. Ara teriàde-vo si.

\StageDir{Casimir, avouì an miya de difficult\'o, se beutte achouatoù si la coutse.}

\MedMitchospeaks Teurriade si la flanella\ldots

\Casimirspeaks  Eh fiade vo mé arevo pa!

\MedMitchospeaks\direct{Eun terièn si la ``canotta''} Oh que bon flo de servadzo! Demàn mateun mogà frottade eunc\'o déz\'o le bri! Ara fiade de pégno crep de tosse.

\StageDir{Casemir fé fran de son pezàn é plen de catarre; lo medeseun dimèn que lo vezeutte avèitse to di lon la dzenta Josette, laquelle, de louèn, rep\'on avouì de sourì malisieu.}

\MedMitchospeaks N’i deu pégno!

\StageDir{Lo vioù le fé pi pégno.}

\MedMitchospeaks Lèi sen caze. Sel\'on mé, se vouillade eunna tosse tchica pi sètse vo consèillo de passé de trèi a dou paquet de \textit{trinciato} pe dzor!

\Casimirspeaks Voué, va bièn medeseun, comèn vouillade vo.

\MedMitchospeaks \ldots aya le tsambe é aprì n'en fenì. 

\StageDir{Lo medeseun fé lo tor de la coutse é se pourte devàn Casimir; avouì eun martelette \martello\ léi baille eun crep deur desì lo dzegnaou, mi Casimir levve eun bri; lo medeseun lèi baillie eugn atro crep mi, comme douàn, Casimir levve lo bri. Dimèn Josette avouì la squiza de beté eun plase la coutse de la mamma se fé avèitchì lo qui di medeseun, loquel, distré pe Josette, baille eun crep de martelette si la tsamba de Casimir: la réach\'on, si cou, l'è eun caouse i bale i medeseun que fenèi pe tèra. Josette lo èidze a se terì si.}

\Feliespeaks \direct{Eun galoppèn vitto ver lo medeseun} Medeseun, medeseun, v'ouite fé-vo de mou?

\MedMitchospeaks  Na, na, to a poste! Tracachade vo pa. 

\StageDir{Josette tourne protso a Geromine.}

\Casimirspeaks\direct{I medeseun} Eh squezade mé!

\MedMitchospeaks Ouè, ouè\ldots eun jénéral le bague semblon normalle. Ara passerio i noutro pi eumpourtàn cliàn!

\StageDir{Lo medeseun s'aprotse a l’eunfermì.}

\Eunfeurmispeaks Eh drime, me diplì tan!

\MedMitchospeaks Gneun problème. Passo pi mé pi tart. An djeusto eunna bagga: va-tì bièn la coutse que n’en beuttou-lèi protso a la fenitra? Sade, l’amen fére contèn le noutro pi eumportàn cliàn.

\Eunfeurmispeaks Ouè, fran amoddo! Aprì de seuilla vèi bièn lo Mont-\'Emilius que llu l’ame da mat.

\MedMitchospeaks Si fran contèn. Ad\'on tourno pi tar. Tanque!

\Eunfeurmispeaks Ouè vo remersio. Tanque!

\StageDir{Med. Mitcho torne avouì Casimir.}

\MedMitchospeaks\direct{Ver Casimir} Bon ara me nen vou. Mogà tourno pi tar pe eun dzen \textit{clistere}! \direct{Avèitse la dzenta feuille} \ldots é vo\ldots v'ouèide eunna beurta grima, vo vèyo an mia blaye. Pe mé v'ouèide fata de eunna pégna vezeutta!

\Gerominespeaks A mé semble pa: l’è todzor igalla.

\Feliespeaks Mamma mé vou vère. Vourriyo jamì que aprì me sento mal djeusto foura de l’opetaille. 

\Gerominespeaks Josette fé atench\'on!

\StageDir{Josette baille dou pouteun \bacino\ a la mamma é chor avouì lo medeseun.}

\Gerominespeaks \direct{Avèitsèn lo medeseun} Té pren-te varda!

\MedMitchospeaks Ouè. \direct{Eun chortèn avouì Josette} \textit{Au revoir} a tcheutte!

\StageDir{Josette é medeseun Mitcho chorton.}

\scene[-- Lo motchaou]

\StageDir{Lo vioù pren eun motchaou desù sin-a \textit{table de nuit} mi lèi tsi pe tèra. A si poueun pe lo prendre se dèi caze cayì pe tèra é se trèinì pe lo prendre. L’eunfermì, que l'a vi amoddo la \textit{scène}, baoudze pa eun dèi. Lo maladdo pren lo motchaou eun se lagnèn é a fose se souffle lo na.}

\StageDir{L'aristocrate iternèi \starnutire .}

\Eunfeurmispeaks Mondje prégnade pa de frette! Vo prègno to de chouite eun motchaou!

\StageDir{L'eunfermì galoppe pe prende eun dzen motchaou de tissù.}

\PersEmpourtantaspeaks Fiade vito!

\Eunfeurmispeaks Ouè, ara fiade eunna dzenta soufflaye\ldots

\StageDir{L'eunfermì pourte lo motchaou i na de Gaston, lequel souffle détchis.}

\Eunfeurmispeaks Oh\ldots fran eunna dzenta soufflaye! V'ouèide caze eumpli-lo!

\Gerominespeaks Na, mi se pou pa itre fagnàn pouèi?

\Casimirspeaks Ouè, l’atro l’at caze fé-me moueure pe coillì lo motchaou é avèitsa avouì sit ommo senque l’a fé? 

\StageDir{Dimèn l'eunfermì arendze l'atchaou di queseun a Gaston.}

\scene[-- L'aoua de maènda]

\StageDir{Entre l’eunfermie maréchal avouì  eun per de platte, vèyo é fourtsette de papì.}

\Eunfeurmiespeaks L’oréo di vezeutte fenèi ara! \direct{Ver Casimir é Geromine} La maenda l’è caze presta. A vo lo plat é le fortsette de papì.

\StageDir{L'eunfermie tappe le-z-éze si la coutse de Casimir é Geromine.}

\Gerominespeaks Mersì! Todzor de papì?

\Eunfeurmiespeaks Vouillade lavé vo le-z-éze \posate?

\Gerominespeaks La mia l'ie eunna demanda\ldots feyavo pe riye\ldots mondje!

\Casimirspeaks Qui areuvve no pourté a medjì?

\Eunfeurmiespeaks Areuvve ara Fernanda vo servì. \direct{Ver l'eunfermì} \ldots é pe llu \direct{avouì la tita moutre Gaston} areuvve eunc\'o lo menù spésial. \textit{Au revoir}!

\StageDir{L'eunfermie chor.}

\Casimirspeaks\direct{A Geromine} Mé medzì seuilla dedeun lamo ren!

\Gerominespeaks A qui te lo di!

\Casimirspeaks Me semble de itre a eunna de salle fìte di Pro Loco.

\Gerominespeaks Aprì fa totte pouzì desù le queverte pequé te te beurle le tsambe.

\Casimirspeaks Vrèi eunc\'o sen.

\StageDir{Entre Fernanda, lo Chef de l’ipetaille, eunna fenna grousa, avouì eun gran faoudè de la Pro Loco é eun grou sourì que moutre eunna groussa di nèye. Avouì eun bri pourte eunna grousa cachoula é avouì l'atra man teun eunna potse que eumplèye pe fé an miya de ravadzo eun la bouéchèn countre la cachoula.}

\StageDir{Partèi lo jénérique de:}

\sound{https://soundcloud.com/user-234168361/extrait-de-il-pranzo-e-servito}{Il pranzo è servito (Sigla) - Augusto Martelli}

\Fernandaspeaks V'ouite pa eunc\'o mor?

\Casimirspeaks Ouélla Nanda, t’areuvve pountuella eunc\'o oueu!

\Fernandaspeaks Comme todzor! T’a bièn fan oueu? Ou pa tan?

\Casimirspeaks Ouè, dièn que eunna dzenta \textit{bistecca} \bistecca\ lèi resterie.

\Gerominespeaks Sondza maque! Véyèn se eundovin-o: oueu n’a puré é straqueun! Na oubliavo, n’at eunc\'o la \textit{minestrina}.

\Fernandaspeaks Te sa que t’a belle eundovin-où! Oueu n'a lo menù spésial: polenta é fricand\'o ou tartiffle i beuro avouì lo lapeun!

\Casimirspeaks Mondje! Ad\'on beutta belle eun platte de polenta!

\StageDir{Fernanda teurie foua de pégne pastiille \pillola .}

\Fernandaspeaks Voualà! Polenta é fricandò! Seutte son le noue pillule iofilizaye: te le tappe ba avouì eunna gotta d'éve é t'i plen pe to lo dzor.

\Casimirspeaks Seutte pégne bague?

\Fernandaspeaks  Ouè.

\Gerominespeaks\direct{A Casimir} Fé vèi veure!

\StageDir{Casimir alondze lo bri pe pasé la pillule a Fernanda, mi malerezamente lèi tchi pe tèra. Fernanda lo reprodze eun lèi baillèn, avouì la potse, eun crep desì lo djise.}

\Fernandaspeaks Mi Casimir! T’a fé tchire la Polenta é fricandò! 

\Casimirspeaks Ouè é té t'a to rempli-me de puré!

\Fernandaspeaks \direct{Eun perdèn la pachense} Ouè, ouè. Ad\'on v'ouèide desidoù senque vouillade medzì? N'i pa de ten a pèdre!

\Casimirspeaks Ouè, ouè. N'i désidoù: vou desù la puré é straqueun. Te sa n’i pa l’ive de viya pe dijirì to si medjì.

\Fernandaspeaks Beutta eun sé lo plat ad\'on.

\StageDir{Fernanda beutte ba eunna bella potchaye, eun icretèn tchica desì la coutse.}

\Fernandaspeaks  Eunc\'o an mia?

\Casimirspeaks  Na, na, n'a prao!

\StageDir{Fernanda lèi beutte ba eugn'atra potchaye.}

\Fernandaspeaks Eunc\'o eunna potchà! N'i pa voya que aprì te va i mitcho é te di que Nanda te baille pa prao medjì!

\Casimirspeaks Na, na, seu n'a prao!

\Fernandaspeaks\direct{Eun s'aprotsèn a Geromine} Geromine te vou eunc\'o té?

\Gerominespeaks Mi ouè. Perqué se meudzo pezàn aprì resto rechaye totta l'éproù.

\StageDir{Comme douàn, Fernanda tappe ba eunna potchaye eun icretèn la poua Geromine.}

\Fernandaspeaks  Eunc\'o eunna potchà?

\Gerominespeaks  Na, na, n'a prao!

\StageDir{Sensa lèi baillì lo ten de fenì, Fernanda remplèi pe lo sec\'on cou lo platte de Geromine.}

\Fernandaspeaks\direct{Eun se poulitèn le mandze é le lenette} Mersì madama! Mersì!

\StageDir{Fernanda s'apersèi d'avèi to pouertchà Geromine. Pe rezoudre, pren lo panamàn, lo lètse é lo frotte countre lo vezadzo é le lenette \lenette\ de Geromine.}

\Fernandaspeaks Mondjemé, squezade n'i to pouertcha-vo!

\Gerominespeaks\direct{Eun tsertsèn de pa se fé totchì} Ouè, mersì madama, mersì!

\Fernandaspeaks Voualà é se v'ouèide eunc\'o fan criade-mé. A vo lo baillo!

\Gerominespeaks Na, na, pa fata.

\Fernandaspeaks Bon, ara vou di dentiste.

\Casimirspeaks N'i la fèi que l'è pouza que te remande.

\Fernandaspeaks  Eun per de-z-àn!

\StageDir{Fernanda chor.}

\Gerominespeaks L’et pi caze satte dzor que meudzo so é tcheu le dzor l’a caze eun gou nouo.

\Casimirspeaks Todzor a te lamenté! L'è beun bon so!

\PersEmpourtantaspeaks \direct{Ver l’eunfermì} Gaston!

\Eunfeurmispeaks Ouè mesieu!

\PersEmpourtantaspeaks N’i peucca a l’itsin-a\ldots

\Eunfeurmispeaks Ouè, ad\'on teriade-v\'o si que vo fiyo eun dzen servicho.

\StageDir{L'Aristocrate se teurie si é Gaston comenche a lèi froté l'isteun-a.}

\Eunfeurmispeaks Seuilla?

\PersEmpourtantaspeaks Na, pi aoutre.

\Eunfeurmispeaks Seuilla?

\PersEmpourtantaspeaks Na, tchica pi ba.

\Eunfeurmispeaks Ad\'on seuilla?

\PersEmpourtantaspeaks Ouè brao fran li!

\Eunfeurmispeaks Oh mesieu si bièn contèn de vo fé contèn!

\PersEmpourtantaspeaks Ouè si contèn can te me fé contèn!

\StageDir{Entre eun tsambrì avouì lo papillon \uomoelegante\ é va ver l'aristocrate que i mimo ten se beutte comoddo si la coutse. Gaston lèi refèiche le lenchoueu.}

\Tsambrispeaks Bondzor a tcheut! Bienvenù a l’opetaille moderno. Senque vouillade medjì oueu?

\PersEmpourtantaspeaks Senque v'ouèide?

\Tsambrispeaks \direct{Pren lo menù é li} Ad\'on lo chef vo propoze, comme \textit{entré}:

\begin{itemize}
\item[--]  \textit{moules à la francillon};
\item[--]  \textit{soufflé au fromage de montagne};
\item[--]  \textit{mousse de saumon avec crevettes};
\item[--]  \textit{salade et vinegrette}.
\end{itemize}

pe premì n’en:

\begin{itemize}
\item[--]  \textit{bouillabaisse marseillaise};
\item[--]  \textit{soupe aux legumes}.
\end{itemize}

pe sec\'on:

\begin{itemize}
\item[--] \textit{mérou au bresse bleu};
\item[--]  \textit{encornets a la biscaïenne};
\item[--]  \textit{ratatouille};
\item[--]  \textit{tartiflette}.
\end{itemize}

Pe lo bèye vo propozo Cabernet di 1932 ou Merlot 1927.

\StageDir{Le dou vioù s’avèitson amoddo é arreuvon pa a comprendre \confuso \boh.}

\PersEmpourtantaspeaks Na, na\ldots lèi sen pa. Mé vouillavo eunna \textit{bourguignonne}!

\Tsambrispeaks Eunna \textit{bourguignonne}, n'a gneun problème. Vo apresto totte!

\StageDir{Lo tsambrì chor.}

\PersEmpourtantaspeaks\direct{A Gaston} Que servicho!

\Gerominespeaks \direct{Eun desuèn Casimir} Que bon-a la puré!

\Casimirspeaks\direct{Eun tsertsèn de semblì a Geromine} Que bon-a la puré! Que nen sayoù mé?!

\Gerominespeaks Mi fi-te feun!

\StageDir{L’eunfeurmì dimèn apreste l'aristocrate eun lèi betèn  la servietta i cou.}

\Gerominespeaks Can mimo dedeun sit ipetaille n’en pi dza vi-nen de bague drole!

\Casimirspeaks Ah ouè, bondàn! Va savèi senque l’arèn eunc\'o da vère noutro dzouvin-o!

\Gerominespeaks Que l’opetaille l’ise eun tsandzemèn me vatte, mi moderno tanque a si poueun na!

\StageDir{Entre lo tsambrì avouì eun cabaré \cabaret .}

\Tsambrispeaks Mesieu la voutra \textit{bourguignonne}! Vouillade veure?

\StageDir{L'aristocrate di de ouè avouì la tita. Lo tsambrì levve lo topèn é s'aprotse a la coutse, mi malerezamente s'euntsambotte é vouidze to l'ouillo bolequèn adosse a l'aristocrate. Le dou vioù, dimèn que l'aristocrate braille di mou, s'avèitson é se fan an dzenta riaye.}

\ridocliou

\newpage

\scene[-- Se torne i mictho]

\StageDir{\Fv{Eun per de dzor aprì}}

\ridoiver

\StageDir{Eun \textit{scène} n’at a la coutse l'aristocrate to fèichà di pià a la tita. Tellamente que l'è fèichà, son eunfermì lèi baille bèye avouì eunna paille. De l'atro coutì, le dou vioù son eun pià eun tren de s'arbeillì é de presté le valize, vu que, finalemàn, chorton de l'opetaille. I mimo ten entre lo medeseun.}

\MedMitchospeaks Bondzor a tcheutte!

\Gerominespeaks Oh Bondzor!

\Casimirspeaks Bondzor medeseun!

\MedMitchospeaks\direct{A Geromine} Ad\'on oueu l’è lo dzor! Fiade le voutre valize que vo mando i mitcho perqué si cou l’è totte eun plase! Saliade-mé la feuille!

\Gerominespeaks\direct{Pa tro convencua} Ouè, ouè!

\StageDir{Medeseun Mitcho baille lo foillette de l'opetaille a Casimir é Geromine.}

\MedMitchospeaks\direct{A Casimir, itèn bièn louèn pe pa sentre lo seun flo} Itade-mé bièn eunc\'o vo.

\Casimirspeaks Ouè mersì, salì!

\MedMitchospeaks\direct{Eun chortèn} Salì  a tcheutte!

\StageDir{Lo medeseun Mitcho chor.}

\Casimirspeaks A forse pouì tornì eun cantin-a a bèye lo vèyo \bicchiererosso\ é  féye la belote!

\Gerominespeaks Ouè, fé maque lo feun, pouèi demàn ti tourna seuilla!

\Casimirspeaks\direct{Eun desuèn la vouése de Geromine} T'i torna seuilla! \direct{Avouì la sin-a vouése} Te tsandze pa té!

\StageDir{Casimir pren la valiza \valigia .}

\Gerominespeaks T'a prèi totte?

\Casimirspeaks Ouè, mé si prest\ldots

\Gerominespeaks La tita?

\Casimirspeaks \ldots te tsandze pa le batiye?

\Gerominespeaks \direct{Eun prégnèn sa valiza} Na! A propoù! Alèn saliì noutro vezeun?

\Casimirspeaks Ouè! \'Eita-l\'o ba lé!

\StageDir{Casimir é Geromine s'aproston a l'aristocrate.}

\Gerominespeaks\direct{A l'aristocrate} Te queutto la bouite di biscouì, te le meudze\ldots ouè, can t'areuve pi a le medjì!

\StageDir{Geromine baille le biscouì a l'eunfermì.}

\Eunfeurmispeaks Mersì madama!

\Casimirspeaks\direct{Eun se moquèn de l'aristocrate} Ad\'on! T'areuve pa a fé plin plin? \direct{Eun sopatèn la tita} T'areuve pa? Alé! Fé plin plin!

\Gerominespeaks\direct{Eun baillèn manforta a Casimir} Ouè betèn si amoddo le queverte é\ldots ah\ldots atèn, mogà dèi se soufflé lo na!

\StageDir{Geromine teurie foua eun motchaou é lo pourte i na de l'aristocrate.}

\Gerominespeaks Souffla lo na, ouè souffla, souffla!

\StageDir{L'aristocrate tsertse de diye de na é de se difendre.}

\PersEmpourtantaspeaks Mmm, mmm, mmm! 

\Gerominespeaks Mi brao!

\Casimirspeaks L'a gnenca eumpli-lo! Aprì la \textit{cannuccia}! Te bèi deun la \textit{cannuccia}\ldots é\ldots é t'a voya de me gratì l'itseun-a eunna miya? Na?\direct{A Geromine} Gratta té!

\Gerominespeaks\direct{Eun gratèn l'itseun-a de Casimir} Ah ah ah, ouè, ouè grato mé, grato mé!

\Casimirspeaks Bon, vo salièn! Conserva-té di gamolle!

\Gerominespeaks Ouè si cou l'è praou fèichà \bendato! Salì!

\PersEmpourtantaspeaks\direct{Malechà \malecha} Mmm, mmm, mmm! 

\Eunfeurmispeaks Salì!

\StageDir{Casimir é Geromine chorton.}

\ridocliou

\DeriLeRido

\RoleNoms{Collaborateur}{Flavio Albaney, Paola Lucianaz, Fabrizio Pession}
\RoleNoms{Costumiste}{Roberta Charrere, Ornella Grivon}
\RoleNoms{Souffleur}{Valeria Brunod}
\RoleNoms{Tramamoublo}{Jean-Pierre Albaney,  Jerome Saccani}

\end{drama}
%\souvenir{Pe mé Forum Valdotèn l'è itaye la pièse pi dzenta de totte. Lo 2010 l'è itoù l'an di premì cou: \textit{vidéo} pe eunna pièse, imitach\'on d'eun personadzo fameu (dzeudzo Santi Licheri), parodì d'eun programme télévizif. L'avanspettaclle l'ayè tsaoud\'o lo pebleucco, lo ritme de la résitach\'on martsè é dérì le rid\'o totte l'è aloù amoddo. Tsaqueun de sise petchoù boc\'on l'a fonchoun-où pe son contcho é betoù eunsemblo l'an cré\'o la pièse pi dzenta di Digourdì.
}{Francesca Lucianaz}

%Per me 2010 pièce più bella per il nostro gruppo -  l'anno delle prime volte: Primi video, prima  imitazione del giudice, ritmo alto, Prima pièce su programma televisivo,tutto è andato Perfetto (anche avan spectaclle) (è un insieme riuscito), prima volta danza sul palco
%\queriaouzitou{
\begin{itemize}
\item[$\bullet$] Waka Watse l'è itaye la premiye \textit{cover}\footnote{ Deun lo 2009, l'ie itaye djeusto réjistraye la parodì d'eun \textit{refrain} publisitère.} di Digourdì. Avouì seutta tsans\'on le Digourdì l'an eunc\'o réalizoù, pe lo premì cou de leur istouéré, eunna \textit{chorégraphie} que l'an danchà i mentèn de seutta pièse.

\item[$\bullet$] La \textit{chorégraphie} de Waka Watse l'è itaye repropouzaye i mariadzo de Pepe e Paoletta.

\item[$\bullet$] Pe la \textit{vidéo} Catalogue Boeuf n'at eun bitchoulì que pe eun per de seconde arrevè pa a comprendre comèn eun bou piémontèis pouè prédjì devàn a eunna machin-a di reprèize!
\end{itemize}
}

%\subsection*{Queriaouzitoù}
%\souvenir{Me rappello pocca bague di pièse, n'i pa eunna grousa mémouée (a l'universitoù itedzoo 100 padze deun dou dzor é lo dzor aprì n'ayoo dza totte oublià!).
Mi can mimo me rappello amoddo comèn l'è nèisiya la compagnì di Digourdì!

L'ie lo 2007 é sayoo eun tren de guedé pe bèichì ba eun Veulla. S'ayoo presque foua d'Ampaillan, can sento a la radi\'o eugn'annonse publisitère d'eunna compagnì téatrale é to de chouite diyo a mé mimo: \og mi pequé seu a Tsarvensoù n'en pa eunna compagnì de téatro?\fg. Areuvvo i Pon-Sià é acappo le trèi de l'Ave Maria: Giada, Ester é Jasmine. Teurio ba lo finestreun é, tot entuziaste, lèi diyo: \og Oh mi pequé fièn pa eunna compagnì téatrale? An matse de Quemeue nen an eunna. Pouèn pa pa èi la noutra!\fg. Dèi si dzor gnouo a pompé de dzi pe terì si eunna compagnì. A la feun lo \textit{zoccolo duro} que me baillè fose l'ie la classe di '84-'85: Jo\"elle, Jo\"el, Pierre, Ester, Jasmine, Ester, Simone é Francesca ('83). 

Caque ten aprì no-z-accapèn eun tchi Gidio (restoràn Monte Emilius) pe no-z-organizé é tsertchì d'atre dzouveunno. Sayoon eun tren de no-z-achouaté deun la pégna tsambretta a gotse di banc\'on, can Jo\"el se prézente avouì eun que n'ayoon jamì vi: botte blantse (que gneun betè), \textit{jeans} to strachà é polo de la Fred Perry avouì lo colette terià si! Me prézento: \og Salì, mé si Laurent\fg; \og \textit{Ciao, io sono Paolo}\fg. Ester m'avèitse comme pe diye \og mi sitte sa gneunca prédjì patoué\fg. Eun pi l'ie pi vioù de no!\\ \'E beun… sitte l'ie lo super Paolo Cima Sander (pa fata de counté véo de cou lo pebleucco l'a riette grase a llou!).

Dèi salla premiye réuni\'on n'en comenchà a no troué pe eunna sala de la viille bibliotèque (eun cou eunc\'o finque pe eunna sala dézò Mèiz\'on Quemeua), pe liye de teste téatral é fée de premiye proue. Eun teste que n'en prouoù mi jamì prézentoù l'è \og Piéreun lo stchapeun\fg\ldots ah! Deun salla pièse n'ayè eun directeur de si pa senque… é te sa qui lo fiyè? Paolo Cima Sander! Eugn éffé deun mon portable n'i eunc\'o ara salvoù son contacte comme \og Paolo directeur\fg!

Aprì fayè chédre eun Prézidàn. Can n'en fé la réuni\'on n'i to de chouite pompoù Jo\"el  \og Ouè té t'i adatte pe fé lo Prézidàn, t'i fran eun gamba, t'i lo top!\fg\ldots é pouèi n'en accapoù qui sariye aloù douàn lo dzeudzo d'eun lo case saoutisan foua de problème légal.

N'en pa moloù de no-z-accapé é de proué de teste téatral tanque lo dzor iaou sen senti-no preste pe prézenté eun premì spettaclle a totta la populach\'on de Tsarvensoù, lo 17 mi 2008 a la bibliotèque communale de Tsarvensoù.

Pouèi son nèisì le Digourdì… ah na! Pouèi l'è comèn l'è nèisiya la compagnì. Mi pequé \og Digourdì\fg é pa \og Le compagn\'on de la fiacca\fg ou \og Le botchassa de Tsarvensoù\fg?

Eunna nite mé é Simone no no reteriavon aprì eun mariadzo. L'ie beun tchica tar mi n'ayoon pa voya de no reterì. Adon désidèn d'ataqué si pe Cogne. Passèn donque i mitcho de Simone pe prende le cllo de la machina. Mé attégnao foua di mitcho é for probablo que Simone l'arè fé de cazeun eugn entrèn i mitcho, pequé, aprì eun per de seconde, sa mamma chor foua si la louye é no di: \og Iao alade?\fg. Simone lèi rep\'on: \og Alèn si eun Cogne\fg. Dispéraye la mamma no braille: \og Si eun Cogne?! Fiade pa tro le digourdì!\fg.
Attaquèn si pe Cogne é dimando a Simone: \og Mi senque vou deue digourdì?\fg; \og Boh, si pa, mamma me lo di todzor\fg. Dèi salla nite n'i mémorizoù lo numér\'o de téléfonne de Simone comme \og Simone Digourdì\fg.

Aprì tchica de ten, lo lon d'eunna proua, sen deu-no: \og Mi queun nom no no baillèn?\fg. Aprì trèi meneutte propouzo: \og Le Digourdì!\fg, \og Mi senque vou diye?\fg; avèitsèn sen que di lo dichonnéo: \og Abile, actif, euntelledzèn\fg\ldots \og Sen no!\fg.

Voualà comèn son nèisì Le Digourdì!}{Laurent Chuc}

%\queriaouzitou{
\begin{itemize}
\item[$\bullet$] Pierre Savioz l'a i eun problème tecnique avouì son microfonne. Can l'a fallì chotre pe tsandji-lo, l'a désid\'o de resté foua \textit{scène} pe eun bon momàn, eun fièn eunna mia tracachì Jordy Bollon (que l'ie maque lo sec\'on cou que resitè si lo palque). Mi can l'è entroù l'a bièn esplecoù i pebleuque pequé l'ayè fallì scapé foua!\\

\item[$\bullet$] Pe lo premì cou, la prézentach\'on di-z-atteur l'è itaye féta de fas\'on diffienta reuspé a la tradichonella magniye de querì eun aprì l'atro tcheu le-z-atteur a la feun di spettaclle. Tcheu le-z-atteur son it\'o prézent\'o a l'entérieur de la danse \textit{chorégraphique} finale.
\end{itemize}
}
%\queriaouzitou{
\begin{itemize}
\item[$\bullet$] Lo titre \og Tanta betsii \fg l'è eun djouà de paolle avouì bièn de eunterprétach\'on: 1) la tanta responsabla de la betsii; 2) la tanta l'è la tseur de la betsii; 3) a l'italienne lo titre semble a \og an matse de betsii \fg{};
\item[$\bullet$] Pe le digourdì, \og Tanta betsii \fg l'è itaye la premie pièse eunterprétaye i Téatro Splendor de Veulla. 

\item[$\bullet$] \og Tanta betsii \fg l'è la pièse pi queurta de totta l'istouére di Digourdì: 30 meneutte!

\item[$\bullet$] Ver mèitchà piése, Paolo Cima Sander s'oubliè todzor eunna battiya a propoù di bouì. Pe tsertchì de lèi la fé rappelì, Pierre, can comprégnè que Paolo l'ie oubliase la battiya, dijè \og Oh flotte! \fg. Naturellamente, eunc\'o lo dzor di spettaclle Paolo l'è oublia-se  la battiya\ldots vo queuttèn la queriaouzitoù d'accapé si momàn.

\item[$\bullet$] Qui avèitserè la \textit{vidéo}, for probablo s'apesèiserè d'eun pégno \textit{vaff****lo} que noutro Paolo l'è queutto-se scapé!
\end{itemize}
}
%\queriaouzitou{
\begin{itemize}
\item[$\bullet$] \og Disco flama\fg l'è itaye la premiye collaborach\'on avouì eugn'atra compagnì téatralla pe réalizé eunna pièse. For probablo l'è it\'o eunc\'o finque lo premì cou deun l'istouére di Printemps Thé\^atral.\\

\item[$\bullet$] Pe Marlène Jorrioz, future Prézidanta di Digourdì, \og Disco flama \fg reprézente son débù deun la compagnì.\\

\item[$\bullet$] Eunna particularitoù de \og Disco flama\fg l'è la forta contrapozich\'on euntre le dou patoué de Tsarvensoù é Pollein. Doe Quemeue eunna atatchaye a l'atra, mi avouì eunna foneteuca diffienta di patoué: mogà/magà; iaou/ioù; bon/boun.  
\end{itemize}
}


%\queriaouzitou{
\begin{itemize}
\item[$\bullet$] Margot, lo petchoù tseun protagoniste de la \textit{scène} finale, l'è itaye vardaye dérì le rid\'o pe lo pappa de Paolo Cima Sander. Pe la vardé tranquila, l'a faillì lèi baillì eunna matse de croquette\ldots for probablo que l'è dèi salla nite lé que l'a attacoù a èi tchica de problème de santé!

%margot "cagnolino" -> c'era papà dietro le quinte a darle dei croccantini.. da lì infatti ha iniziato a star male XD.. l'ha riempita come un'uovo!

%\item[$\bullet$]-> 2016 "scherzo di roveayz o joel... mi hanno legato le scarpe da cacciatore

\item[$\bullet$] La pièse ``N'en pa lo ten" counte de fas\'on bièn ironique lo \textit{laxisme} di-z-uficho pebleuque. Deun lo 2016, pe le Digourdì n'ayè bièn de-z-atteur que trailloon pe de-z-uficho réjonal, mi pe seutta pièse gneun de leur l'a partésipoù. Gneun de leur l'ie refezose; l'ie fran it\'o eun case. Anse\ldots aprì la pièse sayòn tcheu triste de pa èi partésipoù a l'icriteua di tecste, vi que l'arian i bièn de-z-idoù pe rendre eunc\'o pi comique la counta.

\item[$\bullet$] ``N'en pa lo ten" reprézante lo débù deun la compagnì de Richard Cunéaz, lo premì atteur itrandjì di Digourdì. Sa estraordinéa capasitoù, malgré eunc\'o dzoueun-o, d'eunterpretì de personadzo vioù ou tipique de la tradech\'on valdoténa, l'a pouchà lo directif a lo térì dedeun la compagnì, eun cllouzèn eun joueu pe le-z-orijine pompiolentse! Pe seutta pièse Richard pouye pa si lo palque, mi prite sa vouése comme vouése foua campe.

\item[$\bullet$] Pe diye ``$\#$" (\textit{hastag} ou \textit{cancelletto}), lo sembole eumpléyà si Instagram pe décriye le fotografie, la compagnì l'a propozoù eun néolojisme: petchoudacllenda.

\item[$\bullet$] Eunna partiya di spettateur l'a pa tan appréchà ou l'è senti-se offenchaye pe la parodì si lo \textit{laxisme} di-z-uficho pebleuque. Naturellamente, seutte creteuque  son itaye féte pe eunna partiya de sise que traillòn deun lo pebleuque. Mi, can mimo, recougnisèn que, a niv\'o de réalizach\'on, l'è cheur pa itaye eunna di pise pi dzente.
\end{itemize}
}

%Curiosità:
%Cape Treisou : dal "capo" CVA Trisoldi... Magari non scriverlo
%\queriaouzitou{
\begin{itemize}
\item[$\bullet$]  \og La vèille di grand dz\^o \fg{} l'è it\'o lo premì \textit{court-métrage} de la compagnì Le Digourdì.

\item[$\bullet$] \og La vèille di grand dz\^o \fg{} l'è eunc\'o lo débù de Aimé Squinabol deun Le Digourdì comme acteur. L'ie dza pouyà si lo palque deun l'an 2012, mi maque avouì eunna pégna apparich\'on sensa battiye.

\item[$\bullet$] Renzo, que eunterprète eun tsantre de la parotse, sayè pa qué que dèijé fé de reprèize. Jordy l'ayè maque deu-lei que dèijè baillì eunna man pe tramé de moublo.
\end{itemize}
}
%\title{TODZO PI DIGOURDÌ}
\author{Pièse icrita pe Le Digourdì}
\date{Téatro Splendor de Veulla, 10 marse 2018}

\maketitle

\fotocopertina{Foto/2018/gruppo_red.jpg}{Sophie Comé, Ronny Borbey, Francesca Lucianaz, Paolo Cima Sander, Julie Squinabol, Jo\"{e}l Albaney, Michel Comé, Marco Ducly, Stephanie Albaney, Thierry Jorrioz, Simone Roveyaz, Aimé squinabol, Marlène Jorrioz, Paolo Dall'Ara, Richard Cunéaz}{Evi Garbolino, Jo\"{e}lle Bollon, Pierre Savioz, Jordy Bollon, Laurent Chuc, André Comé}{2018}

\LinkPiese{Avanspettaclle, Todzo pi Digourdì}{https://www.youtube.com/watch?v=IaFisR_YuYo&list=PLBofM-NS_eLJUln45l7VH457fGak_Bk5O&index=4, https://www.youtube.com/watch?v=MiAzDgBoOTY}{.45}

\souvenir{La pièse \og Todzo pi Digourdì \fg l'è itaye icrita eugn'occaj\'on di dji-z-àn de la Compagnì. Lo sujé di teste l'è la \textit{biographie ironique} di Digourdì.
Personellamente \og Todzo pi Digourdì\fg resterè todzor deun mon queur, pe bièn de rèizón.

L'è itaye la dériye pièse comme Prézidàn di Digourdì; l'è eunna pièse avouì laquella eugn  itrandjì pou comprendre amod\-do sen que vou deue itre Digourdì, di momàn que lo \textit{fil rouge} de la counta l'è l'esprì comique de totta la compagnì, avouì voya de riye pe fée riye; l'è itoù lo premì cou iaou la Compagnì l'a prooù a tsandjì \textit{style} téatral: \textit{scénographie} redouite i \textit{minimum}, \textit{scène} reprézentaye deun difièn conteste \textit{espace-temps}\footnote{ Eun premì tentatif l'ie itoù fé avouì la pièse \og Matte\ldots sen tcheut matte\fg{} deun lo 2013.}, \textit{mimétisme} é surtoù bièn de \textit{métathéâtre}.}{Jordy Bollon}
\queriaouzitou{
\begin{itemize}
\item[$\bullet$] La grafie di titre de la pièse l'è pa correcte:  \og Todzo\fg{} se devriye icriye \og Todzor\fg{}. Dièn pa lo non di/de la responsablo/a de seutta erreur pe pa provoqué de discuch\'on inutile. Can mimo, va la pèin-a de soulignì pequé n’en desidoù de vardé lo titre trompoù. Pe doe rèiz\'on: premì, pe no rappelé de fé pi attenchón avouì la grafiye di teste que no publièn; secón, di momàn que la pièse counte de fasón \textit{autoironique} sen que vou deu itre Digourdì, eun titre avouì eunna petchouda erreur l'ie bièn reprézentatif di sujé de la pièse mima.\\

\item[$\bullet$] Deun la scène X - La Nite di-z-Oscar, noutro Paolo Cima Sander repette, deun eunna meneutta é demì, nou cou \og can mimo\fg{} é chouì cou \og eumpourtàn\fg{}!

\refstepcounter{videos}
\begin{figure}[H]
\vspace*{-5pt}
\centering
\begin{subfigure}{.75\textwidth}
\centering
\video\hspace*{0.5mm} \textsc{\small Can mimo vs Eumpourtan}\hspace*{0.5mm} \video\\\vspace*{1mm}
    \qrcode[hyperlink, height=0.5in]{https://www.youtube.com/watch?v=tCJzWhbbfCQ}
\end{subfigure}%
\addcontentsline{vds}{section}{Canmimo vs Eumpourtàn}
\end{figure}

\item[$\bullet$] Deun l'avanspettaclle, le dou prézentateur son réel\-la\-men\-te lo Seunteucco é lo Vise Seunteucco de la Quemeua de Tsarvensoù. 
\end{itemize}
}

\Scenographie
\begin{itemize}
\item[$\bullet$] 1 poltronna moderna;
\item[$\bullet$] 1 chofà;
\item[$\bullet$] 1 peuccapoussa \aspirapolvere\ \textit{automatique};
\item[$\bullet$] 1 \textit{hoverboard};
\item[$\bullet$] 1 \textit{album} di foto;
\item[$\bullet$] 1 ban de eunna cantin-a iaou aprésté lo bèye pe le clliàn;
\item[$\bullet$] 2 pégne table, comme salle di cantin-e ;
\item[$\bullet$] 15 caèye que se pouon pléì, comodde pe itre tramaye eun pocca ten;
\item[$\bullet$] 1 volàn di poulmeun;
\item[$\bullet$] 1 \textit{pupitre};
\item[$\bullet$] 1 busta;
\item[$\bullet$]1 Oscar.
\end{itemize}

\setlength{\lengthchar}{3.5cm}

\Character[LAURENT]{LAURENT}{Laurent}{Premì prézentateur, lo Vise Seunteucco de Tsarvensoù, \name{Laurent Chuc}}

\Character[RONNY]{RONNY}{Ronny}{Sec\'on prézentateur, lo Seunteucco de Tsarvensoù, \name{Ronny Borbey}}

\Character[MAGÀN]{MAGÀN}{Magan}{Magàn Sophie \viille, sa eumpléì amoddo lo téléfonne é la tecnolojì, \nameF{Sophie Comé}}

\Character[PAGÀN]{PAGÀN}{Pagan}{Pagàn Paul \viou, euncó llou bièn tecnolojique, \name{Paolo Dall'Ara}}

\Character[NEVAOU]{NEVAOU}{Nevaou}{Nevaou de pagàn Paul é magàn Sophie, \name{Aimé Squinabol}}

\Character[NEVAOUZA]{NEVAOUZA}{Nevaousa}{Nevaouza de pagàn Paul é magàn Sophie, \nameF{Julie Squinabol}}

\Character[SERVENTA]{SERVENTA}{Serventa}{Serventa d'eunna cantin-a de Tsarvensoù,  \nameF{Stéphanie Albaney}}

\Character[FRANCESCA]{FRANCESCA}{Francesca}{La vrèya Francesca Lucianaz di Chef-Lieu, \nameF{Marlène Jorrioz}}

\Character[PIERRE]{PIERRE}{Pierre}{Lo vrèi Pierre Savioz di Tsatì, \name{Pierre Savioz}}

\Character[JO\"{E}LLE]{JO\"{E}LLE}{Joelle}{La vrèya Jo\"{e}lle Bollon di Chef-Lieu, \nameF{Jo\"{e}lle Bollon}}

\Character[JO\"{E}L]{JO\"{E}L}{Joel}{Lo vrèi Jo\"{e}l Albaney d'Ampaillan, \name{Jo\"{e}l Albaney}}

\Character[VÉTCHOT]{VÉTCHOT}{Vetchot}{Eun vétchot \viou\ di péì, tchica boutro, \name{Richard Cunéaz}}

\Character[CIMA]{CIMA}{Cima}{Lo vrèi Paolo Cima Sander de Feleunna, \name{Paolo Cima Sander}}

\Character[MARCO]{MARCO}{Marco}{Lo vrèi Marco Ducly de Feleunna, \name{Marco Ducly}}

\Character[SIMONE]{SIMONE}{Simone}{Lo vrèi Simone Roveyaz di Pon-Sià, \name{Simone Roveyaz}}

\Character[ÉLEVEUR]{ÉLEVEUR}{Spectateur}{\'Eleveur tsarvensolèn, \name{Thierry Jorrioz}}

\Character[MAMMA]{MAMMA}{Mamma}{Eunna mamma d'eun atteur di Digourdì, \nameF{Stéphanie Albaney}}

\Character[TSACHAOU I]{TSACHAOU I}{TsachaouI}{Eun tsachaou de Tsarvensoù, \name{Jordy Bollon}}

\Character[TSACHAOU II]{TSACHAOU II}{TsachaouII}{Eungn atro tsachaou de Tsarvensoù, \name{André Comé}}

\Character[FOTOGRAFE]{FOTOGRAFE}{Fotografe}{Vioù fotografe di Printemps Thé\^atral, an mia trambelìn, \name{Thierry Jorrioz}}

\Character[CHAUFFEUR]{CHAUFFEUR}{Chauffeur}{\textit{Chauffeur} di poulmeun pe la chortia annuella di Digourdì, \name{André Comé}}

\Character[JORDY]{JORDY}{Jordy}{Lo vrèi Jordy Bollon di Chef-Lieu, \name{Jordy Bollon}}

\Character[THIERRY]{THIERRY}{Thierry}{Lo vrèi Thierry Jorrioz di Chef-Lieu, \name{Thierry Jorrioz}}

\Character[STEPHANIE]{STEPHANIE}{Stephanie}{La vrèya Stephanie Albaney d'Ampaillan, \nameF{Stéphanie Albaney}}

\Character[CONDUCTEUR]{CONDUCTEUR}{Conducteur}{Conducteur télévizif de la nite di-z-Oscar, \name{Richard Cunéaz}}

\Character[ASSISTANTA]{ASSISTANTA}{Valletta}{Assistanta di conducteur télévizif, \nameF{Stéphanie Albaney}}

\Character[]{LE DOU NEVAOU}{Ledou}{\quad}

\Character[]{TCHEUTTE}{Tcheutte}{\quad}

\DramPer

\act[\avanSpect\ Avanspettaclle   \avanSpect]

\StageDir{\hspace*{2.5em}Lemie \lemieSi.}

\StageDir{\hspace*{2.5em}Laurent l'è si \textit{proscenium} que avèitse lo pebleuque.}

\begin{drama}

\Laurentspeaks Bonsouar é bienvenù a tcheutte a seutta souaré di Printemps Thé$\hat{a}$tral avouì le Digourdì de Tsarvensoù. Me semble ieur can no sen acapó lo premì cou comme compagnì é n'en désidoù de beté si le Digourdì. Dèi si dzor lé n'en fé tan de souaré é de chortie, todzor eun vardèn lo mimo esprì, la mima voya de riye é de fé riye. Donque, eugn'occajón de noutro anniverséo, n’ayòn désidó de pourté inque si lo palque eun momàn bièn tipique, é sourtoù bièn téatral, de sen que l'iye la viya valdoténa: lo Consèille réjonal. Mi aprì no no sen deu: mi le dzi veugnon sé pe riye ou pe plaoué? Pe riye! É ad\'on no no sen deu: na, na, na! L'è vrèi que lo téatro é la poleteucca son bièn gropoù; bièn de cou te comprèn pa qui fé la poleteucca é qui fé lo téatro. Mi i mimo ten, euncó no Digourdì pouèn pa tan cretequì la poleteucca; euncó no n'en noutro pégno euntsardzo eun poleteucca. La Joueunte comunalla de Tsarvensoù l'è compozaye pe cattro Digourdì si sinque! N'en la majoranse! \'E l'unique que l'è pa eun Digourdì, lo seul que l'a jamì resitó desì si palque (é de so dèi se baillì lagne)\ldots l'è noutro Seunteucco! Seunteucco, que comme tcheu le vrèi politisièn, l'è vin-ì eun salle de téatro, l'a saroù la man a tcheutte, l'è fé-se vére, l'a fé de grou sourì\ldots é ara iaou l'è aloù? L'è aloù ià! Mi l'a pa comprèi que sise que l'an resitoù devàn sayòn sise de Brutchón! Pa de Tsarvensoù! Mi se pou beté Seunteucco eun que prèdze gnenca noutro patoué? Mé ara dimando a tcheu vo\ldots mi se pou voté eun Seunteucco que\ldots

\StageDir{Ronny Borbey, Seunteucco de Tsarvensoù, entre eun \textit{scène}.}

\Ronnyspeaks Laurent! Avèitsa que\ldots se te vou conten-ì a fé lo Vise Seunteucco tanque lo 2020\ldots fé attench\'on a sen que te di!

\Laurentspeaks Mi té de iaou t’i entroù?

\Ronnyspeaks N'a pa d'eumpourtanse de iaou si entroù! Te semble lo case  de diye de bague di janre?

\Laurentspeaks T'a rèiz\'on\ldots

\Ronnyspeaks Ouè que n'i rèiz\'on!

\Laurentspeaks Mi ara que sanse l'a deusqueté douàn a totte seutte personne?

\Ronnyspeaks Na l'a pa de sanse. Surtoù pequé fa planté-la lé de deusqueté comme eun Consèille réjonal. Se comprèn jamì ren: eun chor de la majoranse, l'atro entre deun la majoranse, aprì l'atro euncoa chor de la minoranse, si de douàn entre eun minoranse\ldots se comprèn pamì ren! Mi prao, fa avèitchì i futur, djeusto?

\Laurentspeaks Ouè! Mé diyo: pequé profitèn pa de totte seutte personne pe comenché a icriye lo programme di prochèn-e-z-éléchón?

\Ronnyspeaks Djeusto! Mi te diyo eungn atra baga: fièn comme fan le Grilleun!

\Laurentspeaks Ah ouè le sinque-z-itèile!

\Ronnyspeaks Djeusto! Leu senque fan? Demandon i dzi se son d’acor, beutton eun votach\'on, fan le \textit{parlamentarie}. Comme lo PD que fé le \textit{primarie}.

\Laurentspeaks \ldots é ad\'on no senque fièn?

\Ronnyspeaks Fièn le tsarvensarie! Demandèn i dzi se son d’acor.

\Laurentspeaks Va bièn, comenchèn!

\Ronnyspeaks Té pensa a doe bague: i spor é a l'imondicha\ldots

\Laurentspeaks Ah l'imondicha! N'i dza deu-lo eun Consèille, a la majoranse é a la minoranse que l'imondicha l'è pa eunna compétanse de la Quemeua, mi de l'Unité!

\Ronnyspeaks \ldots é ad\'on pensa i tourisme. Va bièn?

\Laurentspeaks Ouè; é té pensa a la queulteua é i travó poubleucco.

\Ronnyspeaks Va bièn. Ad\'on atacca té!

\Laurentspeaks Atacco\ldots ad\'on\ldots spor\ldots n'i eunna idé dèi can sayò ate pouèi \direct{avouì la man moutre l'atchaou}, dièn dèi can sayò pégno: eunna grousa manifestach\'on, bièn eumpourtanta\ldots

\Ronnyspeaks \ldots pa la Becca!

\Laurentspeaks Na bièn pi grou! Vouillo pourté a Tsarvensoù le Jeux Olympiques!

\Ronnyspeaks A Tsarvensoù? Mi se pou pa! N'en lo caro pe totte le streutteue?

\Laurentspeaks Ouè! Fé-me esplequì! N'en tot: ba pe la plan-a, protso de la grandze, protso di mitcho de Franco Lucianaz fièn eunna grousa streutteua pe saouté avouì le-z-esquì; aprì fièn lo restoràn iaou betèn traillì tcheu sise de Valpettaz!

\Ronnyspeaks Ah pouèi areuvvon le vouése de Valpettaz!

\Laurentspeaks Ouè,  30 ou 40 vouése! Aprì lo fondo\ldots lo fièn si a Tsan-Plan; betèn eunna grousa cabouetta pe le beillette, eunna dzenta cantin-a iaou fièn traillì sise di Combes é di Tsatì!

\Ronnyspeaks \ldots é lé d'atre vouése!

\Laurentspeaks Son 200 vouése! Aprì lo bob: lo fièn partì si i rascar de Combatechiye\ldots

\Ronnyspeaks Mi gneun reste si pe de lé! Queunte vouése prégnèn?

\Laurentspeaks Ouè mi la feun de la pista di bob la fièn devàn lo mitcho de Diego Bollon! Countèn Diego countèn le Bollon, countèn le Bollon countèn Sen-Sal\'o !

\Ronnyspeaks Le Bollon son tan é no voton tcheutte!

\Laurentspeaks Mi pe gagnì le-z-éléch\'on fa euncó avèitchì la partia basa. Lé fièn eunna grousa \textit{patinoire} é la fièn jéré a Franco Lombardo!

\Ronnyspeaks Lé son 300 vouése! Sen a poste! 

\Laurentspeaks Garantì,  300 vouése!

\Ronnyspeaks Me plé!

\Laurentspeaks Mé avouì lo spor si a poste. Ara té pensa a la queulteua.

\Ronnyspeaks Acouta-mé: no n'en cherdì le Digourdì pe fé lo \textit{court-métrage} si la Reconstituch\'on de la Quemeua; le DVD l'an i eun gran sucsé! Ad\'on senque fièn? Pourtèn a Tsarvensoù lo festival di \textit{cinéma} de Cannes é de Venize! Imajina-té le journal:\fg Festival di \textit{cinéma} de Tsarvensoù\og{}!

\Laurentspeaks Wow!

\Ronnyspeaks Imajina le \textit{star} que se fan le \textit{selfie} si la promenade!

\Laurentspeaks Ah ah ah! Mi iaou sarie noutra promenade?

\Ronnyspeaks Mi n'en euncó no la promenade: lo tsemeun de Veulla! Imajina-té le \textit{star} que van si é ba pe lo tsemeun de Veulla!

\Laurentspeaks Si é ba, ba é si!

\Ronnyspeaks Se fan le \textit{selfie} é aprì fièn finque la directe avouì Barbara d'Urso, avouì lo programme \og Iproù 5\fg{}!

\Laurentspeaks Pa mal!

\Ronnyspeaks Pouèi acapèn euncó le vouése di fenne que steurion!

\Laurentspeaks Ouè, djeusto, salle son de tsaplette.

\Ronnyspeaks Ara té pensa i tourisme\ldots

\Laurentspeaks Tourisme\ldots sel\'on mé fa fé eunna téléférique que partèi de Plase Chanoux, areuvve si i Chef-Lieu, aprì pase euncó protso sen di Borbey, conteneuvve tanque si i val\'on de Combouì é a la feun s'arite si i refuje Arbolle.

\Ronnyspeaks Pa mal!

\Laurentspeaks Ouè, pouèi le portèn tcheutte si eugn Arbolle, bèyon lo cafì pi bon i mondo (lo Cafì Ollietti), mi surtoù\ldots payon! Le fièn paì la tasse de \textit{soggiorno}!

\Ronnyspeaks Pequé sise no voton pa, pequé iton pa a Tsarvensoù.

\Laurentspeaks Ouè, no voton pa mi payon! Baillon de sou a la Quemeua, la Quemeua baille de sou i sitouayèn, le sitouayèn son contèn é le sitouayèn contèn no voton!

\Ronnyspeaks Sen a poste.

\Laurentspeaks Ouè! Mi pe gagnì fa todzor prendre eun considérach\'on eunna baga: le trav\'o poubleucco. Si so, queutto la paolla a té.

\Ronnyspeaks Donque fé-me pensì\ldots n'i pensoù. Ad\'on té te cougnì amoddo la Becca, mi te feré pi eugn'atra manifestach\'on: Veulla - Tsarvensoù - Mont-Emilius, pouèi te va euncó pi si!

\Laurentspeaks Na si pa d'acor!

\Ronnyspeaks Atèn que t'euspleucco. Te sa senque fièn de la poueunte de la Becca?  La copèn!

\Laurentspeaks Naaa! Comèn copé la Becca?

\Ronnyspeaks Ouè!

\Laurentspeaks \direct{Dispéoù} Na! Copé la Becca! Na votade pa! Na!

\Ronnyspeaks Acoutta! Copèn la Becca pequé pouèi portèn lo solèi a Roulaz é prégnèn totte le vouése de Roulaz; no vote finque Anselmino que la jamì votou-no! Portèn aprì lo solèi a Feleunna! Counta véo de vouése a Feleunna!

\Laurentspeaks Mondjemé! Eunna sentèin-a!

\Ronnyspeaks \ldots é pourtèn finque lo solèi a Plan-Feleunna! Counta le vouése\ldots

\Laurentspeaks N'i pa praou de dèi! Va bièn, si caze d'acor; mi la Becca iaou la betèn?

\Ronnyspeaks N'i pensoù euncó a so: lo pro de la grandze de Plan-Feleunna. Atsetèn lo pro (le tsanéno l'an fata de sou pe se paì la \textit{badante}), plachèn la Becca é fièn Charvland, lo Gardaland de Tsarvensoù! L'è pa eunna dzenta idoù?

\Laurentspeaks L'è eunna idoù euncrouayabla! A Charvland lèi betèn a traillì sise de la Giradaz, sise d'Ampaillan, sise di Pont-Suaz. Aprì totte le-z-attivitoù di Pon travaillon de pi é se travaillon no voton!

\Ronnyspeaks Eh ouè!

\Laurentspeaks Mé diyo que sel\'on mé n'a la caèya pe 30 an!

\Ronnyspeaks Cheur\ldots mi n'en deu devàn que fa demandé i dzi se son d'acor: fa fé le Tsarvensarie.

\Laurentspeaks Mi n'a pa fata avouì si programme!

\Ronnyspeaks Fien-leu pequé no sen comme le Grilleun démocratique\ldots é propozèn eun programme fattiblo!

\Laurentspeaks No sen de Valdotèn, sen pa sé pe counté de counte foule! Sen que dièn lo fièn!

\Ronnyspeaks Se pou fiye! Ad\'on mé beutto eun votach\'on: countréo?

\StageDir{Ronny é Laurent avèitson lo poubleucco.}

\Laurentspeaks Me semble gneun\ldots té avèitsa si, que mé avèitso ba.

\Ronnyspeaks Gneun! Amoddo! Ara demando: favorable?

\StageDir{Ronny é Laurent avèitson tourna lo poubleucco.}

\Laurentspeaks Avèitsa qui la votou-no! La votou-no finque Siro Viérin! Mi demàn nèi!

\Ronnyspeaks N'en l'unanimitoù. Saren-no la man pe la fotografie avouì le journaliste\ldots

\StageDir{Ronny é Laurent se saron la man é sourion pe se fé fé eunna foto. Aprì chorton.}

\Laurentspeaks Ah na!

\StageDir{Ronny é Laurent tornon i mentèn di palque.}

\Ronnyspeaks Na sen oublia-no di Digourdì!

\Laurentspeaks Donque vo quetèn avouì lo spettaclle di Digourdì icrì eun occaj\'on di djijimo anniverséo de la compagnì! A vo\ldots

\Ronnyspeaks Todzor pi\ldots

\Tcheuttespeaks \ldots Digourdì!

\act[Acte I]

\ridoiver

\scene[-- L'\textit{album} di Digourdì]

\StageDir{Lemie \lemieSi\ a drèite di palque, iaou acapèn eun chofà é magàn protso d'eunna poltronna.\\ \Fv{Tsarvensoù, 4 janvieu 2050}
}

\Maganspeaks \direct{Eun se achouatèn si la poltronna} Oh\ldots si belle lagnaye! Ara m'achouatto eun momàn. Ah na! Me fa euncó poulité lé pe tèra. Ah! Mi n'i la soluch\'on! Lola!

\StageDir{Eun peuccapoussa \textit{automatique} chor foua di chofà é poulite sel\'on sen que comande Magàn.}

\Maganspeaks Oh saye! Voualà, poulita an mia pi a drèite\ldots pi a gotse\ldots Stop! Ara tourna i teun poste, a coutcha!

\StageDir{Lo peuccapoussa se bloque.}

\Maganspeaks Seutte baradziye, martson pa. Le beutto a poste mé.

\StageDir{Magàn se levve é catse lo peuccapoussa dézò lo chofà. Aprì s'achouatte tourna si la poltronna é teurie foua lo portable.}

\Maganspeaks\direct{Eun lièn} Véyèn sen que conte-tì lo moundo oueu\ldots

\StageDir{Entre pagàn avouì l’\textit{hoverboard}.}

\Paganspeaks Sophie! Si arevoù!

\Maganspeaks L'iye l’aoua bon!

\Paganspeaks \direct{Eun se boudzèn avouì l’\textit{hoverboard}} N'i fé eunna dzenta promin-ada oueu\ldots avouì la dzenta mezeucca de l’Orage n'i fé belle dou \textit{kilomètre}!

\Maganspeaks \ldots é t'i nienca tsizì! T'i fran eun gamba!

\StageDir{Pagàn fé dou tor si lo poste é aprì bèiche ba de l’\textit{hoverboard}.}

\Paganspeaks \`Eitsa que mé si eungn esper de seutte machine! Pa comme té, todzor dérì a si portablo é a Facebook é iPhone!

\Maganspeaks \`Eitsa que mé oueu n'i dza to fé, n'i dza to poulit\'o !

\StageDir{Pagàn s'achouatte si lo chofà.}

\Paganspeaks Ouè, poulit\'o avouì salle machine que t'a pe lo mitcho\ldots que nen si mé sen que te combeun-e pe lo mitcho! \direct{Eun terièn foua lo portablo} Ara fé-me lie lo journal.

\Paganspeaks \`Eitsa Sophie! Amélie Viérin noua Prézidante de la Réj\'on\ldots eh, Sophie, dza lo pappa Laurent l'ayè fé lo Prézidàn 30 an fé!

\Maganspeaks \ldots é lo pappagràn l'ayè fé-lo dza i seun ten!

\Paganspeaks Ouè, que dzen veure comèn le mitchì se tramandon, l'è dzen so!

\Maganspeaks T'a fran rèiz\'on!

\StageDir{Le dou nevaou entron eun galopèn avouì eun man eun grou album di fotografie. S'achouatton protso de pagàn.}

\Ledouspeaks Pagàn, pagàn! Senque l'è si grou livro?

\Nevaousaspeaks L'è icrì: \og Le Digourdì! \fg{}

\Paganspeaks Aimé, Julie! Mi iaou v'ouèide acapoù so?  

\StageDir{Pagàn pren lo livro eun man é lèi souffle desì pe lèi gavé ià la poussa\footnote{ Commé éffé spésial n'en betoù bièn de \textit{borotalco} deun l'\textit{album}, pe fé vère la poussa can pagàn lèi soufflè desì.}.}

\Paganspeaks\direct{Eton-où \ouaou} Mi sit\ldots \direct{avèitse Magàn} té t'a catcha-lo!

\Maganspeaks Na!

\Paganspeaks Si l'è lo livro avouì totte le foto di Digourdì! La pi renoumaye compagnì de téatro de la Val d’Ousta!

\Nevaouspeaks Mi Digourdì qui?!

\Paganspeaks Deh Aimé! Pourta de reuspé pe Le Digourdì! Le Digourdì l'an port\'o lo nom de Tsarvensoù ià pe to lo moundo!

\Nevaousaspeaks Ah ouè? Comèn l'arian fé? Ara éziston-tì eunc\'o ? Can son nèisì?

\Paganspeaks\direct{Bièn fier \tipoconocchiali} Ad\'on\ldots vegnade seu\ldots

\StageDir{Pagàn fé de caro si lo chofà i dou nevaou. Le nevaou s'achouatton.}

\Paganspeaks \ldots ara vo fiyo la conta\ldots l'iye l'an 2000 é\ldots ouette é\ldots

\StageDir{Teuppe \lemieBa\ si la drèite di palque. Lemie \lemieSi\ a gotse é i mentèn di palque.}

\scene[-- La nésanse di Digourdì]

\StageDir{Eun \textit{scène} n'at eun banc\'on é doe pégne table avouì eun per de caèye. N'a la serventa que poulite lo banc\'on é eun vétchot achouatoù a gotse que li lo journal pe son contcho.\\ Entron deun la cantin-a  Jo\"{e}l, Jo\"{e}lle, Pierre é Francesca.}

\Francescaspeaks\direct{A la serventa} Bondzor! T'a eunna plase pe cattro?

\Serventaspeaks Ouè, salla tabla \direct{eun moutrèn la tabla a drèite} va-tì amoddo?

\Francescaspeaks Ouè.

\StageDir{Le cattro-z-amì s'achouatton.}

\Serventaspeaks Sade dza senque bèye?

\Pierrespeaks Féo mé pe tcheutte, fièn cattro bire!

\Francescaspeaks  Na, pe mé te me fé eun ju de frouì i marteun sèque!

\Joellespeaks \ldots é pe mé i-z-ambrocalle!

\Joelspeaks\direct{Eun rièn} I marteun sèque, i-z-ambrocalle! Desandro bèyavade pa tan de ju de frouì\ldots ou mioù: ambrocalle é marteun sèque ouè, mi l'ian dedeun l'éve de viya!

\StageDir{Pierre ri eunsemblo a Jo\"{e}l.}

\Francescaspeaks Euh Jo\"{e}l Albaney! Fa-tì te rapélé que n'i portó té é llou \direct{eun moutrèn Pierre} i mitcho? \'E pe terì si sitta tanque si la Basteuille n'en finque rechà Viro!

\Vetchotspeaks Ouè! Mé alao si i fèye é sise se retèriaon!

\Pierrespeaks\direct{I vétchot} Conteneuvva a lie lo journal! Euncó té can t'ie dzoueun-o t'a fé de fite!

\Vetchotspeaks Penso beun! Mi l'an jamì pourto-me i mitcho le femalle! Todzor arrevó i mitcho avouì l’ape; é pe to diye, i ten de mé le femalle servichaoun pe d’atro!

\Joelspeaks Mi soplé! Li lo journal!

\StageDir{La serventa pourte lo bèye pe le-z-amì.}

\Serventaspeaks Ah ouè \ldots me fiade fran riye: acouté vo me semble de vère lo Charaban!

\StageDir{Silanse. Le cattro-z-amì penson.}

\Joellespeaks Ah lo Charaban\ldots sarie pa mal beté si eunna compagnì de téatro a Tsarvensoù!

\Joelspeaks Pierre! Pourian beté si eunna compagnì de téatro!
 
\Pierrespeaks Pequé pa? Pa mal comme idoù. 

\Joelspeaks\direct{Todzor a Pierre} N'arian djeusto fata de fenne!

\Pierrespeaks Eh ouè sen maque mé é té!

\StageDir{Le dou penson eun momàn dimèn que Jo\"{e}lle é Marlène comenchon a se amalechì pe pa itre considéraye.}

\Pierrespeaks\direct{A la serventa} Squeuza! Té te prèdze beun tchica patoué\ldots

\Serventaspeaks  Pa pi tan\ldots

\Joelspeaks Na mi te lo prèdze bièn! Amoddo!

\Pierrespeaks \ldots é sen a trèi! 

\Joelspeaks Mi n'a pa praou!

\Pierrespeaks\direct{I vetchot} A propoù, té que le femalle te le-z-eumpléyae pe d'atro\ldots

\Vetchotspeaks  Senque?

\Pierrespeaks\direct{I vetchot} Té que t'i plen de femalle, te sarie no deu se n'a de fenne que pouon itre euntéressaye a fé téatro?

\Vetchotspeaks  Argh! Que counte foule, n'i pa lo ten, n'i d’atro pe la tita. Aprì vo senque vouillade fiye? V'ouite cattro rabadàn nèisì ieur, senque nen sade vo de téatro? Soplé!
 
\Joelspeaks  Rappella-té que no dzouin-o fièn sen que n'en voya se n'en la pach\'on! \direct{A Pierre} Mé é té sen prest, n'arian fran voya de comenchì!

\Francescaspeaks Can mimo v'ouite fran pa seumpateucco! Lèi sen mé é Jo\"{e}lle! 

\Pierrespeaks  Mi ouè, té é Jo\"{e}lle! Djeusto, Jo\"{e}lle! Pe case te cougnì pa caque feuille que l'a voya de fé téatro?
 
\Joellespeaks \direct{Malechaye \malechaa} Mi fé-té feun! Mé é Francesca pouèn pa resité avouì vo?
  
\Joelspeaks Say\'on eun tren de squersé!

\Pierrespeaks L'iye pe riye.

\Francescaspeaks\direct{Ironique} Que riye!
\Joellespeaks\direct{Offenchaye} Fran seumpateucco! 

\Pierrespeaks Ad\'on fiade-mé fiye le contcho. \direct{Counte le-z-atteur  de la compagnì} Eun, dou, trèi, cattro é sinque avouì la serventa\ldots pe fé eunna compagnì no manque euncó caqueun.

\Joelspeaks Sen pocca pe ara\ldots

\Pierrespeaks  Ah! L'è vin-i-me eunna idoù! Le premì trèi que entron eun cantin-a le térièn dedeun!

\Joelspeaks Na Pierre n'i pouiye! Avèitsa que reusquèn!

\Pierrespeaks Mi ouè! Sen que capite, capite! A Tsarvensoù sen tcheut amoddo!

\Joelspeaks Na, na, na! \direct{Ver le feuille} Vo v'ouite chiye?

\Joellespeaks Si pa, prouèn.

\Pierrespeaks\direct{A tcheutte} Vo fiade de mé?

\Francescaspeaks Renque pe si cou.

\Joelspeaks Prouèn!

\StageDir{Entre Cima. Le cattro-z-amì se beutton le man pe le pèi.}

\Cimaspeaks \direct{Ver la serventa avouì pach\'on} Bonsouar! Tot amoddo? Te me fé eun dzén-epì tsa? Mi tsa! Maque comme t'i boun-a té a lo fiye \malisieu .

\Serventaspeaks Ouè va bièn.

\Francescaspeaks  Ouè mi sitte l'è de Feleunna é sa gnenca lo patoué!

\Cimaspeaks Feleunna? Feleunna! Mi té can te di Feleunna te di Paolo Cima Sander! Mi gars\'on, chourtade eunna mia de sise \textit{schemi}! Chortade de Tsarvensoù, bèichade ba de salla montagne, ivrade le joueu, végnade avouì no que n’en lo solèi to l’an!

\Joellespeaks\direct{Eun rièn} Na, na\ldots sitte l'è tro rabadàn, l'è eun de no!

\Cimaspeaks Rabadàn? Jo\"{e}llina, mé que n'i pasoù totta la viya a te prédjì, mi reusta quèya\ldots can mimo\ldots

\Pierrespeaks\direct{I seun-z-amì} Vo dimando squiza mi si cheur que le prochèn dou\ldots

\Joelspeaks Na Pierre! Mé n'i tourna  pouiye!

\StageDir{Entron Simone é Marco eun rièn é squersèn. Pierre l'è dispéoù.}

\Joelspeaks\direct{Eun rièn pe pa plaoué} Ah ébeun! L'è pa pi que t'i aloù eun meillorèn!

\Marcospeaks\direct{A Simone} \ldots é desando nite iaou t'i sparì?

\Simonespeaks Ouè Marco, n’ayoù mioù a fiye que itì avouì vo!

\Marcospeaks  Ouè va bièn mi t'a quetou-me a pià a l’Inside \malecha !

\Simonespeaks Veullanoua-Tsarvensoù a pià l'è pa que te lèi beutte tan\ldots

\Marcospeaks Ouè doe meneutte!

\Serventaspeaks Senque béyade?

\Simonespeaks Beutta maque doe bire.

\Pierrespeaks\direct{Ver lo banc\'on} Marco, Simone! Vo laméria fé téatro avouì no?

\Marcospeaks Acoutta mi de feuille nen v'ouèide?

\Joelspeaks Plen pèi!

\Simonespeaks Eunna baga eumpourtanta\ldots de chortie foua Val d'Ousta nen fièn?

\Pierrespeaks Ouè cheur! Tcheu le mèis!

\Marcospeaks Pe mé se pou fiye!

\Pierrespeaks Ad\'on gars\'on, vegnade tcheu seuilla!

\StageDir{Tcheutte se plachon a l'entor de la tabla avouì lo vèyo levoù.}

\Joelspeaks Fièn eun santé a la noua Compagnì de Tsarvensoù!

\Tcheuttespeaks Santé!

\StageDir{Teuppe \lemieBa\ a gotse é i mentèn di palque.}

\StageDir{Lemie \lemieSi\ a drèite di palque.}

\scene[-- Pe to lo moundo \moundo!]

\Paganspeaks Véo d'émoch\'on; 4son dza belle pasoù 40 an!

\Nevaouspeaks\direct{Eunna mia tro savèn} Seutta l'iye pa eunna compagnì de téatre! V'ouyavade eunna compagnì de la crotta!

\Paganspeaks Grama lenva que t'a té!  Pourta reuspé pe le Digourdì! Totsa-meu pa le Digourdì!

\Nevaousaspeaks Squeza-mé pagàn, poui-dze te dimandé eunna baga? Péqué lo nom Digourdì? 

\Paganspeaks\direct{Fier de seutta dimanda} Ah!  Seutta l'è eunna dzenta counta: ad\'on, eun dzor\ldots \direct{comenche a borboté, vague} dou de no son partì ià pe eun mariadzo; aprì can son tournoù i mitcho l'iye tar é l'ay\'on fata de pantchì d'éve\ldots se son betoù protso d'eun meur, mi l'è chortia la mamma de eun de sise dou\ldots la mamma l'a vouaillà : \og Deh! Me raccomando\ldots fiade pa tro le Digourdì! \fg{} . Voualà! De si momàn la Compagnì l'ayè son nom!

\Nevaouspeaks Ouè mi pagàn, se sitte l'è lo nom, vouillo pa imajin-ì le pièse que v'ouèide fé!

\Paganspeaks Aimé! Pourta de reuspé pe le Digourdì é rappella-té que le Digourdì l'an pourtoù lo nom de Tsarvensoù ià pe to lo moundo!

\StageDir{Pagàn se levve.}

\Paganspeaks Té pensa que can n'en fé la premiye pièse,  n’ayè plen de dzi, n’ayè de machine tanque i mentèn de Veulla, n'ayè finque de pompì volontéo de Cogne é Perloz, n’ayè 600 volontéo de totta la Val d'Ousta que l'an édja-no pe fé la premiye! Tourno diye: la premiye!

\Ledouspeaks \direct{Sourprì \ouaou} Pe la premiye 600 volontéo? 

\Maganspeaks\direct{Eun se levèn} Mi senque te counte? \direct{Eun prégnèn l’album é eun s'achouatèn} Acoutade mèinoù, de sen que counte pagàn coppade la mèitchà, vardade eun car é tsapotade euncó eunna mia! Vegnade seuilla  que vo counto mé\ldots

\StageDir{Teuppe \lemieBa\ a drèite di palque.}

\StageDir{Lemie \lemieSi\ a gotse é i mentèn di palque.}

\scene[-- Catro tsatte \gatto\ \gatto\ \gatto\ \gatto]

\StageDir{Si lo palque n'a 15 caèye plachaye si trèi feulle. Eun vétchot é eun éleveur bièn annouiyà son achouatoù si la secounda feulla.}

\StageDir{Eunna mamma entre to de coursa.}

\Mammaspeaks\direct{I vétchot, avouì lo flo queur} Squizade-mé! L'è-tì dza comenchà? 

\Spectateurspeaks Comme tcheu le demicro ataque a ouette é demì!

\Vetchotspeaks  N'a pa de prisa!

\Mammaspeaks Amoddo! Mé ara dèyo vardé eun per de plase pe de paèn.

\StageDir{La mamma comenche a se dizarbeillì pe vardé la plase i seun paèn. Se gave lo palt\'o, eunna maille, eunna botta é le plache si le caèye.}

\Mammaspeaks So pe tanta Filoméne, la siaou, lo botcha\ldots é l'onclle Gene! \direct{Eun gavèn ià eunna caèya é eun se gavèn eunna botta} Pe llou  que areuvve avouì la caèya avouì le raoue queutto eunna botta! 

\StageDir{La mamma finalemàn s'achouatte si la premì  feulla, totta émochon-aye. I contréo, lo vétchot é l'allévateur avèitson torse la mamma.\\ Silanse.}

\Mammaspeaks  Bon mancon seun meneutte! Spéèn que arrevisan tcheutte!

\StageDir{Silanse. Entron dou tsachaou. Eun de leur seuble eunna tsans\'on.}

\TsachaouIspeaks\direct{Eun quetèn  de seblé} N'i la fèi que n'en trompoù poste! \direct{I vétchot} Squezade, l'è-tì seuilla la réuni\'on pe la tsasse? 

\Vetchotspeaks Véyade-tì pa que l'è la réuni\'on di Comité di Bataille?

\TsachaouIIspeaks Mi v'ouèide robo-no la plase? Oueu totse a no, vo v'ouyavade devàn ieur, oueu totse a no! Lo devendro totse a no!

\Vetchotspeaks Na, na, na.

\Spectateurspeaks \ldots é oueu que dzor l'et?

\TsachaouIIspeaks Oueu l’è devendro.

\Vetchotspeaks Oh la fèi n'i pa verià lo calandrì!

\TsachaouIIspeaks T’ou veure que n'en trompoù tcheu dou é sen vin-ì de desandro!

\Mammaspeaks\direct{I tsachaou} Sssht! Te vèi pa que sen a téatro!

\TsachaouIIspeaks Téatro? A Tsarvensoù? Sarè 20 an que n'a pamì lo téatro a Tsarvensoù!

\Mammaspeaks Ouè mi seutta l'è totta eunna noua Compagnì! Le dzoueunno de ara l'an désidoù de beté torna si lo téatro a Tsarvensoù.

\StageDir{Le tsachaou, lo vétchot é l'allévateur son sensa paolle.}

\Mammaspeaks Lèi crèyo pa que sade ren de to sen!

\TsachaouIspeaks Na.

\Mammaspeaks L'an reumplì lo veladzo de manifeste, de volanteun, ià pe totta la parotse!

\TsachaouIspeaks Pa vi ren.

\Mammaspeaks L'an finque betoù-le si le-z-annonse di mor!

\TsachaouIspeaks\direct{\'Eton-où , ver l'atro tsachaou} Ah t'a comprèi ara qui l'iye si Digourdì! Mé pensò que l'iye eun rabadàn que l'iye mancoù.

\TsachaouIIspeaks Mi son ià de tita! Mé n'i vi icrì ouette é demì di nite é n'i pensoù: \og mi sarè-tì pa seutta l'aoua de fé le sépolteue! \fg .

\TsachaouIspeaks Ad\'on l'è nèisia eunna noua Compagnì\ldots ad\'on avèitsen-la!

\Vetchotspeaks Na, na, na, mé vou belle que i mitcho.

\TsachaouIspeaks\direct{Eugn aritèn lo vétchot} Mi na fièn eunna mia de prézanse, n'a gneun!

\StageDir{Le dou tsachaou s'achouatton si la trèijima feulla.\\ Bèichon eunna mia le lemie.}

\Mammaspeaks Oh comenchon!

\StageDir{La mamma, le tsachaou, lo vétchot é l'allévateur boueuchon di man \bouechiman .}

\StageDir{Silanse. Tcheut avèitson lo spettaclle.}

\Vetchotspeaks\direct{A l'allévateur} Mi senque l'è-tì salla beurta baga lé?

\StageDir{Lo vétchot moutre avouì lo dèi eunna personna douàn llou.}

\Spectateurspeaks Que rotobala!

\Vetchotspeaks Se comprèn pa se l'è eugn ommo ou eunna fenna!

\Spectateurspeaks Na.

\Vetchotspeaks Eunna baga pai\ldots clloure i mitcho é templì ià la cllo!

\Mammaspeaks\direct{Inervaye, ver le dou que son eun tren de prédjì} Deh! \`Eitsade que v'ouite eun tren de prédjì de la feuille de mé! Itade quèi! Grame lenve!

\StageDir{Silanse. Lo secón tsachaou comenche a tossèi for, caze s'apeutre. Son compagnón lèi baille de patèle si l'itseun-a, mi servèison a ren.}

\Spectateurspeaks\direct{Eun tapèn eun bombón i tsachaou} Tchappa eun bombón di prée é quèi!

\TsachaouIIspeaks Mersì!

\StageDir{Lo tsachaou meudze lo bombón é se calme. Silanse.}

\StageDir{Lo premì tsachaou tchoppe a riye bièn for. Tcheu le-z-atre riyon pa é avèitson mal lo tsachaou que molle pa de riye. Silanse.}

\Vetchotspeaks\direct{Eun sarèn le joueu} Mi! \direct{Ver lo spettateur} L'è-tì pa de Feleunna si lé?

\Spectateurspeaks\direct{Eun sarèn le joueu} Orco can ouè!

\Vetchotspeaks\direct{Digout\'o} San-tì pa que son tcheu de-z-itrandjì?

\Spectateurspeaks L'an fran recoilla-le tcheutte!

\Vetchotspeaks Se san dza l'italièn l'è dza eun méacllo! Prétègnon de vin-ì inque prédjì patoué!

\TsachaouIIspeaks Sssht! Mal polì que v'ouite pa d'atro!

\StageDir{Silanse. Tcheu stchoppon a riye mouèn que lo premì tsachaou, que se sen eunna mia foua plase. Silanse.}

\TsachaouIIspeaks \direct{A l'atro tsachaou} Salla lé i caro\ldots

\TsachaouIspeaks Sssht prèdza plan.

\TsachaouIIspeaks \ldots l'è-tì pa la feuille de Diego? La blonda aoutre lé!

\TsachaouIspeaks Ouè, ouè, l'è lleu\ldots que couése!\ok

\StageDir{Silanse. Lo vétchot s'eundrime. Se avion le lemie \lemieSi .}

\Mammaspeaks\direct{Eun se levèn é eun bouéchèn le man} L'è fenì! \direct{Ver l'allévateur} Oh l'è fenì!

\StageDir{L'allévateur l'è eunna mia dizorientoù, mi can mimo se levve é eun bouéchèn di man rèche lo vétchot.}

\Spectateurspeaks\direct{I tsachaou} L'è fenì! 

\StageDir{Lo vétchot se rèche. Le tsachaou se levvon é tcheut eunsemblo boueuchon di man pe fé compagnì a la mamma, totta euntuziasta de la pièse.}

\StageDir{Teuppe \lemieBa\ a gotse é i mentèn di palque.}

\StageDir{Lemie \lemieSi\ a drèite di palque.}

\scene[-- Fran de-z-atteur proféssionel!]

\Nevaousaspeaks Magàn, n'ayè fran eun pebleuque mal polì\ldots

\Maganspeaks Ouè pebleuque! Se te vou queri-lo pebleuque! N'ayè catro tsatte!

\Paganspeaks Mi senque te nen sa té? Sophie tourna aoutre a moungouì avouì salla baga lé, lo portable, Facebook, Amazon\ldots

\StageDir{Pagàn gave ià di man l'album a magàn. Magàn se levve é tourne s'achouatté si la poltronna. Pagàn s'achouatte si le chofà, todzor i mentèn di dou nevaou.}

\Maganspeaks \`Eitsa, pitoù d'acouté salle counte foule de té, ito pi belle achouataye inque tranquila.

\Paganspeaks\direct{I nevaou} Magàn l'è an mia viille é ad\'on se rappelle pa tan comèn son alaye le bague. Aprì sitte l'iye maque lo comenchemèn! Alèn eun devàn\ldots

\StageDir{Pagàn vionde le padze de l'album.}

\Nevaouspeaks\direct{Eun moutrèn eunna fotografie}  Seuilla? Senque fiavade?

\Paganspeaks Seuilla? Sé l'iye lo premì Printemps Thé\^atral, i téatro Giacosa de Veulla! Que sucsé! Que bouéchì di man! N'ayoon fé fran eunna dzenta pièse.

\Nevaousaspeaks Wow! Vo douàn d'entré si lo palque comèn vo aprèstavade? Senque fiavade douàn d'entrì?

\Paganspeaks\direct{Eunna mia surprì de seutta dimanda} No sayon de professioniste! Tsaqueun s'aprestave, prouave sa partiya, aprì n'ayè de \textit{coiffeur}, caqueun que fiè an mia de méditach\'on pe\ldots

\Maganspeaks \ldots senque t'i eun tren de countì? Todzor de counte foule. Mi soplè! Ara vo counto mé  sen que capitave dérì le rid\'o .

\StageDir{Teuppe \lemieBa\ a drèite di palque.}

\StageDir{Lemie \lemieSi\ a gotse é i mentèn di palque.}

\scene[-- Dérì le rid\'o]

\StageDir{No no trouèn dérì le rid\'o di Téatro Giacosa. Deun seutta scène tcheu le-z-atteur fan semblàn que lo rid\'o di fon di palque sisa lo rid\'o rodzo di téatro, de fas\'on que lo pebleuque sisa dérì le rid\'o eunsemblo i-z-atteur mimo.\\ Tcheu l'an lo tecste de la pièse eun man, caqueun l'è ajitoù, d'atre relassoù é caqueun d'atro finque tro relassoù!\\ Eun scène n'a Cima achouat\'o si eunna caèya.}

\Cimaspeaks \direct{Eun se levèn, bièn ajitoù, eun troulèn comme eun pedzeun} Na, na me rappello pa la battiya, pouì pa! Ad\'on l'iye seumpla: \og Mé te diyo qu'areuvvo de Feleunna\ldots caco, peucho, avèitso la leunna\fg\ldots na, na, na, si tro ajitoù!

\StageDir{Entre Pierre arbeillà comme eun prie, eun béyèn eun ju de frouite. Cima s'achouatte.}

\Pierrespeaks\direct{Bièn relassoù} Paolo! Senque t'a? Te me semble preste a partì eun guèra!

\Cimaspeaks Na, na, na, si tro ajitoù!

\Pierrespeaks Ita tranquilo.

\Cimaspeaks\direct{Eun tremblèn} Na, na, na, si tro ajitoù!

\Pierrespeaks Te la sa la partia?

\Cimaspeaks Na! N'i eunna battiya seumpla: \og Mé dze si de Feleunna\ldots \fg voualà n'i oublia-la.

\StageDir{Entron Marco é Jo\"{e}l. Eunna mia pi\'on \pion, se teugnon si eun avouì l'atro.}

\Joelspeaks\direct{Eun tsemièn torse avouì Marco} Marco! Tourna fiye salla de douàn que te fiyè bièn.

\Marcospeaks\direct{Eun se medzèn le paolle} Queunta? Ah ouè, \direct{eun tsantèn} \og eun desì meulle lèi l'a fi!\fg.

\Joelspeaks Mi brao! Mi t'i meilloroù! Ara te boueucho lo ten\ldots

\StageDir{Jo\"{e}l se baille de patèle si la tsamba pe bouéchì lo ten. Paolo é Pierre avèiston la scène dispéroù.}

\Marcospeaks\direct{Eun tsantèn (caze) a ten avouì Jo\"{e}l} \og Eun desì meulle lèi l'a fi!\fg .

\Joelspeaks Mi brao!

\Marcospeaks Mersì, mi ara fièn eunna baga? Prouèn la pièse.

\Joelspeaks Ouè!

\Cimaspeaks Mi senque prouèn?! Viade pa comèn v'ouite combinoù?

\Joelspeaks\direct{Eun se medzèn le paolle} Senque n'at? N'a quetsouza que va pa?

\Cimaspeaks Te vèi pa comèn ti combinoù?

\Joelspeaks\direct{Comme douàn} Pequé té t'i combinoù mioù de mé?

\Cimaspeaks Quetèn pédre\ldots \direct{eun braillèn} n'en lo spettaclle seutta nite!

\Joelspeaks Bon sen beun a poste; sen aloù bèye eunna goloù é sen arrevoù lo pi vito poussiblo é sen arevoù ara\ldots é ara prouèn\ldots

\Cimaspeaks Mi v'ouite deun eun stat! Iaou v'ouite aloù? Di Moldave?

\Marcospeaks Na, sen aloù i Café du Téatro, que l'è pi protso!

\Cimaspeaks Quetèn pédre. Na, na, na!

\Joelspeaks Te comprègno pa. Tcheu le cou t'i ajitoù. Lé de foua \direct{moutre lo rid\'o di fon} lèi seràn catro tsatte.

\Cimaspeaks Catro tsatte?! Mi té t'i ià de tita! T'aré bi eunna \textit{cisterna} de rodzo!

\Marcospeaks\direct{A Jo\"{e}l} Va vère véo de dzi n'at.

\Joelspeaks\direct{Eun s'aprotsèn i rid\'o di fon} N'aré catro dzi\ldots

\StageDir{Jo\"{e}l se beutte a dziill\'on é avèitse déz\'o lo rid\'o .}

\Joelspeaks\direct{Eun rièn} Mondjeu! Mondjeu! L'è plen de dzi!

\Cimaspeaks\direct{Malechà} Mi t'i té plen!

\Joelspeaks N'i jamì vi tan de dzi pouèi! Gnenca a la fèira de Sent-Or.

\Cimaspeaks Oh mondjeu, na, na, na!

\Marcospeaks\direct{A Cima} Ita tranquilo! Acoutta, fièn an baga\ldots vou mé vère véo de dzi n'at\ldots llou \direct{moutre Jo\"{e}l} l'è normal que l'a vi-nèn an matse: vèi doblo!

\StageDir{Marco s'aprotse i rid\'o di fon é, can beutte la tita i mentèn di rid\'o, Jo\"{e}l lo pouche. Marco fenèi foua di rid\'o , mi to de chouite tourne dedeun, iao acappe Jo\"{e}l eun tren de riye.}

\Marcospeaks\direct{Malechà} Mi t'i mar\'on! Mi queunte fegueue te me fi fiye?!

\Pierrespeaks Sssht!

\Joelspeaks Mi se squerse!

\StageDir{Entre Francesca.}

\Francescaspeaks Ad\'on, mi senque l'è to si vacarno? Vo siade panco tsandjà? Senque v'ouite eun tren de fiye? Dji meneutte é comenchèn! I galoppe, vito vo tsandjì!

\Marcospeaks Ouè Francesca, eunna maille é eun per de pantal\'on é sen preste.

\Joelspeaks Eunna mailletta é si preste!

\StageDir{Marco, Pierre é Jo\"{e}l chorton.}

\Francescaspeaks \ldots é tcheu le-z-atre iaou son?

\Cimaspeaks Mi que nen si. Oh na, na, lèi la fièn pa! Na, na, na!

\Francescaspeaks Oh Cima! Calma té\ldots ouè\ldots sit l'è \textit{andait}! 

\StageDir{Entre Simone eun saoutèn é avouì eun vazet de \textit{zuccherini} alcolique.}

\Simonespeaks\direct{To contèn} Ouélla Franci! Agouta euncó té eun \textit{zuccherino}!

\StageDir{Simone soum\'on eun \textit{zuccherino} a Francesca.}

\Cimaspeaks\direct{Todzor pi ajitoù} Ouè senque l'è oueu? L'è fita?

\Francescaspeaks Euh Simone Roveyaz! Soplé, dji meneutte é comenchèn!

\Simonespeaks\direct{Eun moutrèn Cima} \ldots é sitte? T'a pouiye eh?

\StageDir{Simone, comme se Cima fuche eun petchoù tseun, lèi tappe i vol de \textit{zuccherini}. Cima tsertse de le medjì i vol.}

\StageDir{Entre Jo\"{e}lle eun galopèn, an mia inervaye.}

\Joellespeaks Sssht! Oh mi prédzade plan! Se sen totte lé dérì! \direct{A Francesca}  A propoù, Jo\"{e}l iaou l'è? Dièn proué eun per de bague.

\Francescaspeaks N'i spedi-lo se tsandjì eunsemblo i-z-atre é l'a deu-me \og eunna mailletta é si preste\fg \ldots panco vi-lo!

\Joellespeaks Oh mondjeu!

\Cimaspeaks\direct{Eun tremblèn todzor si la caèya} Na, na, na, la spountèn pa si cou!

\Francescaspeaks Sen stra eun retar!

\Joellespeaks Lo si\ldots é aprì fa co fé la foto comme souvenir de la première i Giacosa!

\Francescaspeaks Oue la ferèn se n'en lo ten!

 \Joellespeaks Pouèn euncó pa la fiye! Pe comèn pouriye chotre seutta foto: sitta \direct{eun moutrèn Cima} que tremble comme eun pedzeun, l’atro lé \direct{moutre Simone} que areuvve pa a resté rito, l'atro que molle pa de bèye!
 
 \Francescaspeaks Si pa sen que te diye. Euncó mé me fa lie lo boc\'on de mé\ldots
 
\StageDir{Entre Jo\"{e}l.}

\Joelspeaks\direct{Eun fièn eunna pirouette pe se moutré} Me voualà preste pe lo \textit{referendum}!

\Joellespeaks Ouè mi Jo\"{e}l, mi comèn t'i arbeillà?

\Joelspeaks Pequé? Pe la pièse di \textit{referendum}.

\Joellespeaks T'i arbeilla-te pe la pièse de l'an passoù! Seutta l'è la noua pièse, l’opetaillo moderno.

\Joelspeaks Isto! La noua pièse! L'opetaillo moderno.

\Joellespeaks Rècha-té va! Soplé, galoppa te betì eun pijama ou quetsouza que semblise a eun pijama!

\Joelspeaks Mi so semble eun pijama!

\StageDir{Jo\"{e}l comenche a chotre di palque.}

\Francescaspeaks\direct{Ver Jo\"{e}l} Na, Jo\"{e}l iaou t'i eun tren d'alé? Ara restèn tcheutte seu que de sé a dji meneutte no fa fé la fotografie. 

\StageDir{Entre Pierre.}

\Francescaspeaks Boneur que l'è arrevoù euncó Pierre. \direct{Ver Pierre} A propoù Pierre Savioz: t'a pensoù té i souffleur?

\Pierrespeaks Ouè le joueur. Aprì areuvvon Marco é Simon.

\Francescaspeaks Na le souffleur!
 
\Pierrespeaks Ah le souffleur. Mi Francesca! Sel\'on té no n'en fata di souffleur? Mi na, n'i pa crià gneun, no n'en pa fata!
 
\Cimaspeaks Senque? N'en pa le souffleur, voualà l'è feniya. T'ayè eunna baga a fiye, eunna! \direct{Tsache ià Pierre} Mi va ià soplé! \direct{Ver Jo\"{e}lle} Euh Giada, na euh Esterina, na Jasmine\ldots

\Joellespeaks Jo\"{e}lle.

\Cimaspeaks Soplé va querì Pepe é Paoletta, dimanda-lèi se pouon fé le souffleur vi que leur cougnisson la pièse.

\Joellespeaks Ouè, ouè mi té achoutta-té é calma-té\ldots \direct{Ver tcheutte} mé vou tsertchì Pepe é Paoletta é tcheu le-z-atre pe la foto\ldots \direct{Ver Simone} é té Simone planta-la-lé de tsemin-ì é de rempleure le dzi de seucro que aprì san pa iaou van!

\StageDir{Jo\"{e}lle chor é entre Richard, lo fotografe avouì eunna grousa machina di foto de la Kodak.}

\Fotografespeaks Voué gars\'on, v'ouite preste pe la fotografie?

\Joelspeaks Oh l'è arrevoù Richard pe la fotografie! Veun maque.

\Joellespeaks \direct{Dèi foua scène} Arrevèn euncó no!

\StageDir{Avouì Jo\"{e}lle entron d'atre atteur de la pièse: Francesca Lucianaz, Giada Grivon, Jasmine Comé, Laurent Chuc, Ester Bollon, Ilaria Linty. Tcheutte se beutton eun pozech\'on: se plachon eun demisercllo eun baillèn le-z-ipale i pebleuque. Richard l'et i fon di palque.}

\Fotografespeaks Le pi ate protso i pi base.

\Joelspeaks Senque vou diye?

\Fotografespeaks I meun trèi diade fromadzo! Eun, dou\ldots trèi!

\Tcheuttespeaks \direct{Eugn eurlèn} Fromadzo!

\Fotografespeaks Na l'a pa martchà! Fa refiye.

\Cimaspeaks Mi eun pi feun pouavade pa lo acapé?

\Fotografespeaks I meun trèi: eun, dou\ldots trèi!

\Tcheuttespeaks \direct{Eugn eurlèn} Fromadzo!

\Fotografespeaks A poste!

\StageDir{Jo\"{e}l accompagne foua Richard. Dimèn tcheu le-z-atre se beutton eun demisercllo ver lo pebleuque é s'apreston pe lo motte final.}

\Joelspeaks Ad\'on Richard mersì de totte é a la prochène.

\StageDir{Richard chor é Jo\"{e}l rejouèn le-z-atre.}

\Joelspeaks Alé Gars\'on, ara si reprèi-me eunna mietta! Mondjeu! V'ouite preste? I meun trèi, noutre motte\ldots eun, dou\ldots

\StageDir{Dimèn entre eun scène Richard é se tappe i mentèn di demisercllo.}

\Fotografespeaks\direct{Eun braillèn} Fromadzo!

\Tcheuttespeaks \direct{Bièn malechà} Mi va ià! Feulla ià!

\StageDir{Richard veun tchachà foua. Lo groupe tourne se betì eun demisercllo, tcheutte avouì eunna man pouzaye desì salla de Jo\"{e}l.}

\Joelspeaks Eun, dou, trèi\ldots

\Tcheuttespeaks\direct{Eun braillèn} Todzor pi Digourdì!

\StageDir{Tcheutte chorton a gotse, mouèn que Jo\"{e}l que chor ver lo fon, dérì lo rido, pe baillì lo bienvenù i pebleuque é pe prézenté la pièse l'\og Opetaillo Moderno\fg{}.}

\StageDir{Teuppe \lemieBa\ a gotse é i mentèn di palque.}

\Joelspeaks Bonsoir a tcheutte. Oueu n'en la Compagnì di Digourdì de Tsarvensoù que pe lo premì cou deun l'istouére pouye si lo palque di Giacosa. Vo queutto donque a la pièse icrita pe leur \og L'opetaillo moderno\fg{}!

\StageDir{ Lemie \lemieSi\ a drèite.}

\scene[-- Plen de sou!]

\Maganspeaks Voualà: sen l'è sen que l'è capitoù dérì le rid\'o di premì Printemps!

\Nevaouspeaks\direct{Ver pagàn} Ouè, v'ouyavade cheur pa de esper de téatro!

\Maganspeaks Na! Digourdì pa pe ren!

\Paganspeaks\direct{Ver magàn}  Mi té te dèi todzor betì lo bèque i mentèn! Acoutta, va aoutre a lavé le-z-éze.

\Maganspeaks Le-z-éze? Mi te sa pa que le-z-éze se lavon pamì a man?

\Paganspeaks Comèn na? Comèn se fé?

\Maganspeaks Te féo vère. \direct{Eun terièn foua son téléfonne} N'i djeusto ditsardjà eunna applicach\'on.

\StageDir{Magàn gnaque eun bot\'on si son téléfonne.}

\Maganspeaks\direct{A vouése ata} Fuffi!

\StageDir{\Fv{Ouè Sophie}}

\Maganspeaks\direct{Comme douàn} Lava le-z-éze!

\StageDir{\Fv{Va bièn}}

\Maganspeaks\direct{A pagàn} Té te sen de tapadzo di platte que se lavon?

\Paganspeaks Mi sentì senque? Sento té que t'i matta!

\Nevaousaspeaks Magàn! Ouè, la lavaplatte l'è eun martse.

\Maganspeaks Vou aoutre vère. Si pa tan chiya de seutta modernitoù.

\StageDir{Magàn chor.}

\Paganspeaks Mondjemé! Sen mal betoù. \direct{I nevaou} Acoutade, magàn l'è eunna mia ià de tita. Can mimo, Aimé é Julie\ldots sitte l'iye maque lo comenchemèn. Aprì n'en fé eunna matse de spettaclle, sen fé-no eunna matse de sou é avouì sise sou n'en fé de fite, de maende, de sin-e é pi finque de dzente chortiye!

\Nevaousaspeaks De chortiye? Wow! Iaou v'ouite aloù?

\Paganspeaks Sen aloù ba a Pollein\ldots na\ldots Polonia! Aprì Londra, San Francisco,  Mosca, Dubai é pi finque Rio! Mi ara vo counto de can sen aloù eun Piém\'on!

\Nevaouspeaks Na, na, counta-no vèi de can t'i aloù a Dubai ou a Rio.

\StageDir{Entre magàn.}
 
\Maganspeaks Ouè Paul, counta vèi de can t'i aloù a Rio, té que t'a tan pouiye de l’avi\'on.

\Paganspeaks\direct{Malechà} Ara itade quèi tcheutte! Ara vo counto de can sen aloù i Ca' Brusà ba pe Alba! 

\StageDir{Teuppe \lemieBa\ a drèite.}

\StageDir{Lemie \lemieSi\ a gotse é i mentèn di palque.}

\scene[-- Chortiya bièn digourdia]

\StageDir{Acapèn nou caèye plachaye pe reprézenté eun poulmeun. Douàn lo sédil di chauffeur, l'iè  eun volàn. Totte le-z-ach\'on di-z-atteur que dèyon euntérajì avouì lo poulmeun (iverteua di pourte ou di finestreun) seràn mimaye.}

\StageDir{Lo chauffeur l'è achouatoù deun lo poulmeun to solette, eunna mia eunfastedjà.}

\Chauffeurspeaks Sise de Tsarvensoù son todzor eun retar: pe la chortiya di For\ldots eun car d'aoua, le pompì volontéo demì aoua\ldots é ara sise di téatro l'an dza trèi car d'aoua bondàn! Atégnèn euncoa. Dimèn controlèn que sise totte a poste: le lemie lèi son, lo clacson martse é le panavèyo martson\ldots se fé lo temporal sen a poste. Ara atégnèn.

\StageDir{Entron Jordy é Simone. S'aprotson i finestreun, prouon a prédjì i chauffeur mi se sen ren.}

\Chauffeurspeaks Sento pa!

\StageDir{Lo chauffeur teurie ba lo finestreun.}

\Jordyspeaks Mé é Simone sen arrevoù! Atégnèn seuilla de foua le-z-atre.

\Chauffeurspeaks Na, na vegnade si! Pouyade maque si que aprì alade a bèye eun cafì é vo vèyo pamì; si beun comèn martson le bague.

\StageDir{Simone ivre lo:}

\effet{https://soundcloud.com/user-234168361/portellone-posteriore}{Portell\'on}

\StageDir{Simone é Jordy entron deun lo poulmeun. Jordy cllou lo portell\'on\footnote{ Dèi-z-ara tcheu le cou que lo portell\'on se ivre ou se cllou, se sen lo son di portell\'on.}. Entron eun \textit{scène} Thierry é Jo\"{e}l. Jo\"{e}l l'at eun petchoù saque pe la mordiya.}

\Chauffeurspeaks\direct{Eugn avèitsèn Thierry é Jo\"{e}l} Sise? Sarèn pa de vo sise?

\Jordyspeaks Ouè, son de no!

\StageDir{Thierry ivre lo portell\'on.}

\Joelspeaks Bondzor, pouèn entré?

\Chauffeurspeaks Vo v'ouèide pa comprèi. Vo v'achouatade devàn.

\Joelspeaks Pequé?

\Chauffeurspeaks Pequé se itade mal, se vo tourne si to sen que v'ouèide bi ieur, oumouèn n'a lo finestreun! 

\Joelspeaks Que stoufiàn!

\StageDir{Thierry cllou lo portell\'on.}

\Thierryspeaks Que boutro!

\StageDir{Jo\"{e}l ivre la pourta di poulmeun. Thierry entre pe premì pe se plachì protso di finestreun. Eun passèn desì Thierry, Jo\"{e}l entre  é se plache i mentèn. Thierry cllou la pourta.}

\Chauffeurspeaks\direct{Eun moutrèn lo saque de Jo\"{e}l} So senque l'è?

\Joelspeaks So l'è la mordiya.

\Chauffeurspeaks Ad\'on la mordiya va dérì!

\StageDir{Lo chauffeur tchappe lo saque é lo tappe dérì.}

\Joelspeaks Mi pequé?

\Chauffeurspeaks Pequé se no ariton fan de counte!

\Thierryspeaks\direct{Dezò vouése a Jo\"{e}l} Que malpolì si chauffeur!

\Joelspeaks Ouè t'a rèiz\'on.

\StageDir{Entron eun scène trèi feuille: Marlène, Jo\"{e}lle é Stephanie. Se plachon douàn lo poulmeun pe se fé eun selfie.}

\Joelspeaks\direct{I chauffeur} Seutte son avouì no.

\Chauffeurspeaks Bièn, eunna mia de feuille fa beun!

\Joelspeaks Ouè nen n'en maque trèi. Fan todzor de foto! Ara pouèn diye sen que n'en voya, mi can entron fa fé attench\'on.

\StageDir{Le feuille prouon a ivrì lo portell\'on, mi areuvvon pa. Simone se levve é lo ivre. Le trèi feuille entron é salion tcheutte.}

\Stephaniespeaks Squezade lo retar, mi n'ayé lo passadzo a livel d’Ampallian ba\ldots

\Simonespeaks Ouè, fa clloure lo portell\'on!

\StageDir{Simone se levve é comme douàn cllou lo portell\'on.}

\Chauffeurspeaks\direct{Ironique ver le feuille} Ouè é euncó finque lo \textit{semaforo} di Bournì! Can mimo, pouèn partì? N'at euncó eun poste vouido\ldots

\Simonespeaks Ouè pequé fa passé prendre Marco a Feleunna.

\Chauffeurspeaks Ah, va bièn.

\Joelspeaks\direct{I chauffeur} Feleunna l'è djeusto douàn d'arrevé a Pollein.

\Chauffeurspeaks Si beun iaou l'è Feleunna!

\StageDir{Lo chauffeur aleumme lo poulmeun, mi lo moteur vou pa nen sèi de partì.}

\effet{https://soundcloud.com/user-234168361/motorino-davviamento}{Tentatif aviemàn moteur}

\Joelspeaks\direct{Eun rièn} Comenchèn bièn!

\StageDir{Lo chauffeur reproue.}

\effet{https://soundcloud.com/user-234168361/accensione}{Aviemàn moteur}

\StageDir{Finalemàn lo moteur s'aleumme.}

\StageDir{Lo chauffeur beutte la rétro.}

\Chauffeurspeaks Vo que v'ouite ba i fon, diade-mé se n'i de caro dérì.

\Jordyspeaks Ouè te diyo mé. Veun maque! Veun, veun, veun\ldots

\StageDir{Lo poulmeun totse contre lo meur é tcheutte boueuchon la tita countre lo sédil.}

\Jordyspeaks Praou pouèi!

\Chauffeurspeaks Ouè n'i praou sentì! Tacaleun!

\StageDir{Lo chauffeur beutte la premiye é partèi.}

\Joelspeaks Sen belle partì, alèn ba pe lo Piém\'on, medjì de tseur\ldots 

\StageDir{Silanse.}

\Joelspeaks N'at an mia de silanse desì si poulmeun\ldots

\Thierryspeaks Beteun de mezeucca!

\StageDir{Thierry alondze la man pe avì la radi\'o, mi lo chauffeur lèi baille eunna dzifla.}

\Chauffeur Na, beutto mé!

\StageDir{Lo chauffeur tsertse eun per de canal mi n'a maque de publisit\'o.}

\Chauffeurspeaks Tchouéyèn, maque de publisitò.

\Joelspeaks Ad\'on fièn eunna tsans\'on no! Jo\"{e}lle attaca-là té!

\Joellespeaks Va bièn. \direct{Eun tsantèn} Le Digourdì\ldots

\Tcheuttespeaks\direct{Eun tsantèn} \ldots le Digourdì, le Digourdì de Tsarvensoù, le Digourdì de Tsarvensoù, le Digourdì de Tsarvensoù.

\Joelspeaks Seutta l'è fran dzenta, la fièn euncó can vegnèn si.

\Chauffeurspeaks Va bièn. Can mimo no aprotsèn\ldots èitade: lé n'a lo premì sate de Feleunna. Ouè l'an fé-lo fran ate\ldots

\StageDir{Tcheutte (douàn la premiye feulla, aprì la secounda é a la feun la trèijima) sauton si can lo poulmeun pase desì lo \textit{dosso}.}

\Joelspeaks Ouè son fran ate, sourtoù ara que l'an levou-lo. Son beun dzen é comoddo; mi n'a todzor de dzi que se lamenton\ldots

\Chauffeurspeaks Djèique! Baillon maque de problème\ldots

\Joelspeaks\direct{Eun contegnèn son discoù} \ldots pren eunna Jeep se te ite eun Val d'Outa.

\Chauffeurspeaks\direct{Eun contegnèn son discoù} \ldots ou te va a 15 a l'aoua ou te difé la machina\ldots

\Joelspeaks Djèique.

\Chauffeurspeaks Nen fan eun tsaque 100 mètre!

\Joelspeaks Vrèi! Djeusto pouèi. Ah! Arrevèn protso de Feleunna.

\StageDir{Dimèn, tcheu le-z-atre s'arbeuillon pe se toppé di frette que comenche a lèi itre.}

\Chauffeurspeaks Iaou ite Marco?

\Joelspeaks Ba lé, 50 mètre a drèite.

\Thierryspeaks\direct{A Jo\"{e}l} Toppa-té que fé frette.

\StageDir{Jo\"{e}l, eunsemblo a tcheu le-z-atre, teurie foua di saque de palt\'o, de gan, de bounet pe se toppé di frette. Dimèn lo polmeun s'arite.}

\Chauffeurspeaks Mondjeu que frette! Tsandze fran l'er! Mé n'i pa prèi ren dérì.

\StageDir{Caqueun baille i chauffeur eun bounet.}

\Joelspeaks\direct{Eun fenissèn de s'arbeillì} Mé si organizou-me bièn, pequé avouì lo frette que fé seu i mèis d'avrì fa pa squersé!

\Chauffeurspeaks Ouè pensade de alé lo querì? Seu fé frette!

\Joelspeaks Ouè, ouè vou mé! Vo itade seuilla!

\StageDir{Jo\"{e}l bèiche ba di poulmeun. Se sen eugn:}

\effet{https://on.soundcloud.com/2ZWj9FGPMw3ZSJAD6}{Oua frèide}

\Joelspeaks\direct{Eun braillèn} Marco! Marco!

\StageDir{Entre Marco eun mailletta.}

\Marcospeaks\direct{A Jo\"{e}l} T'a fenì de fé lo mar\'on?

\Joelspeaks Mi comèn te fé sensa eun palt\'o? Beutta eun palt\'o que fé frette!

\Marcospeaks Ah, ah! Tcheu le-z-àn la mima battiya\ldots sen a avrì mar\'on!

\Joelspeaks Ouè, pouya si que fé frette!

\StageDir{Marco ivre lo portell\'on é pouye si lo poulmeun. Jo\"{e}l, tot i galoppe, cllou lo portell\'on, ivre la pourta é se tappe desì a Thierry pe entré.}

\Thierryspeaks\direct{A Jo\"{e}l} Ouè mi que couése dzalaye!

\Joelspeaks T'a sentì?

\StageDir{Thierry cllou la pourta. Se sen pamì lo ravadzo de l'oua dzalaye que soufflè foua di poulmeun. Le chauffeur partèi. Tcheutte se dizarbeuillon.}

\Joelspeaks Comenche l'Adret. Fenì Feleunna comenche a fé tsa. Te pase lo pon é te reste dza mioù.

\Chauffeurspeaks\direct{Eun avèitsèn douàn} Grousa seutta rionda! L'è noua?

\Joelspeaks Na l'è tchica que l'an fé-la.

\StageDir{Lo chauffeur entre deun la rionda avouì bièn de vélosité. Vionde lo volàn é tcheutte se plèyon ba a leur drèite.}

\Thierryspeaks Plan, va plan!

\Joelspeaks Ara praou chor foua!

\Chauffeurspeaks Fièn euncó eun tor!

\Tcheuttespeaks Na, na, na!

\Thierryspeaks Plan, va plan!

\StageDir{Lo chauffeur vionde de l'atro coutì lo volàn é tcheutte tournon a itre achouatoù drette.}

\Joelspeaks\direct{I chauffeur} Mi comèn se fé a prendre de rionde pouai! Can te entre te rallente!

\Chauffeurspeaks Mi n'i pa fé-la mé seutta rionda!

\Joelspeaks Mi soplé di pa de counte foule!

\Thierryspeaks Penson de itre i couscrì!

\Chauffeurspeaks Ouè avouì vo desì semble beun de itre i couscrì. Can mimo ara tanque a Muntal Deura no ariterè pa gneun.

\Joelspeaks Spéèn.

\StageDir{Silanse.}

\StageDir{To d'eun creppe, le Digourdì achouatoù dérì lo chauffeur sauton si pe l'er comme se quetsouza sise pasoù dézò le ràoue di poulmeun. Lo chauffeur arite to de chouite lo poulmeun.}

\Joelspeaks Mi senque l'è capitoù?

\Chauffeurspeaks Si pa. Avèitsèn dérì se n'a quetsouza.

\StageDir{Tcheutte se viondon pe avèitchì. Gneun vèi ren.}

\Joelspeaks Mi avèitsèn euncó eun cou.

\StageDir{Tcheutte eunsemblo se viondon pe avèitchì. Gneun vèi quetsouza.}

\Chauffeurspeaks Ad\'on bèicho ba pe vère.

\StageDir{Lo chauffeur ivre la pourta é bèiche ba di poulmeun. Can cllou la pourta Jo\"{e}l ataque a prédjì. Dimèn lo chauffeur mioule sen que n'a dérì lo poulmeun.}

\Joelspeaks Jordy eun \textit{chauffeur} pouèi te pouè accapi-lo maque té!

\Jordyspeaks N'en voulì reusparmì pe pa prendre Pelanda? 

\Joelspeaks Ouè mi se pou pa prendre de \textit{chauffeur} comme sitte!

\Thierryspeaks\direct{A Jo\"{e}l} Ah can t'ie té Prézidàn!

\StageDir{Lo chauffeur ivre la pourta.}

\Chauffeurspeaks Gars\'on\ldots n'i sétchà eun tèiss\'on.

\Joelspeaks Eun tèiss\'on? N'at eun deun to lo Piém\'on é t'a tchapou-lo té?

\Chauffeurspeaks Eh ouè.

\Joelspeaks Ara vèi té sen que fa fiye.

\Chauffeurspeaks Ara l'è mioù querì le Garde Forestier pe pa èi de counte.

\StageDir{Can lo chauffeur cllou la pourta Jo\"{e}l ataque a prédjì.}

\Joelspeaks Mi l'è pa poussiblo. A té, Jordy, te baillèn pamì de-z-entsardzo.

\Thierryspeaks\direct{A Jo\"{e}l} Pregnèn no lo moublo?

\Joelspeaks Te di?

\Thierryspeaks  Mi ouè tante sitte l'è pa bon a gueudé.

\Jordyspeaks Senque?

\Joelspeaks\direct{Ver tcheutte} Mé é Thierry n'en i eunna idou! Se prégnèn no lo poulmeun é ataquèn ba?

\StageDir{Tcheutte, mouèn que Jo\"{e}lle, son d'acor é entuziaste. Lo chauffeur l'è euncó eun tren de prédjì i téléfonne. Jo\"{e}l pouye si lo poste di chauffeur.}

\Joellespeaks\direct{Dispéraye} Jo\"{e}l soplé, fièn pa le Digourdì!

\StageDir{Jo\"{e}l beutte eun martse. I mimo ten areuvve lo chauffeur.}

\Chauffeurspeaks Mi senque v'ouite eun tren de\ldots

\StageDir{Partèi lo refrain de:}

\sound{https://www.youtube.com/watch?v=tgw1yEcWpTU}{Ruda tańczy jak szalona - Czadoman}

\StageDir{Lo poulmeun partèi ià a flama. Lo chauffeur lèi galoppe dérì.}

\StageDir{Aprì eun per de seconde a totta vélositoù, la mezeucca se tchoué, totta l'ach\'on rallente.}

\Joellespeaks\direct{Lentamente, a Jo\"{e}l} Attench\'on, la viille!

\StageDir{A drèite entre eunna viille madama. L'et eun tren de traversì la rotta. Tot a rallentateur, Jo\"{e}l frène, tcheutte boueuchon la tita countre le véyo ou le sédil. Dimèn, areuvve euncó lo chauffeur tot a galoppe, mi boueuche la tita countre lo dérì di poulmeun.}

\StageDir{Teuppe \lemieBa\ a gotse é i mentèn di palque.}

\StageDir{Lemie \lemieSi\ a drèite.}

\scene[-- Hollywood]

\Nevaouspeaks Ad\'on te vèi que sise Digourdì son pa itò de super star. V'ouite nèisì pe eunna cantin-a, lo voutro nom l'è chortì pe eunna souaré de fita, pe la premiye pièse n'ayè catro tsatte, i premì Printemps fiavade  maque de confuj\'on é di chortie\ldots prédzen-nen pa!

\Paganspeaks \'Eitsa que grama lenva que t'a! Fiade attench\'on: pourtade reuspé pe le Digourdì! Rappelade-v\'o  que le Digourdì l'an\ldots

\Ledouspeaks \ldots pourtoù lo nom de Tsarvensoù pe to lo moundo!

\Paganspeaks Ouè fran pai!\direct{Eun viondèn la padze de l'album} Avèitsade vèi sen que l'è capitoù aprì. Aprì eun per de-z-àn n'en fé eunna matse de sou é la Val d’Outa l'iye viin-a tro petchouda pe no é sen partì aoutre pe l’Amérique!

\Nevaousaspeaks Comèn v'ouite aloù aoutre pe l’Amérique?

\Paganspeaks  Totte l'è comenchà mersì a eun \textit{court-métrage} que n'en fé avouì Alessandro Stevanon. To lo moundo l'a vu si \textit{court-métrage} é l'an euncó finque queria-no a Hollywood; mi ara lamerio sentì la vouése de magàn\ldots di vèi se l'è pa vrèya seutta counta?

\Maganspeaks Ouè, pe eun cou me totse bailli-te rèiz\'on. Mi can mimo n'i pa comprèi sen que l'an vi eun vo sise de Hollywood pe finque vo baillì eugn Oscar!

\Nevaousaspeaks Eugn Oscar? Qui de vo la gagnà l’Oscar?

\Paganspeaks Se me rappello amoddo\ldots

\StageDir{Teuppe \lemieBa\ a drèite.}

\StageDir{Lemie \lemieSi\ a gotse é i mentèn di palque.}

\scene[-- La nite di-z-Oscar \oscar]

\StageDir{Eun scène, arbeillà bièn élégàn, n'a lo prézentateur de la nite di-z-Oscar. L'a douàn llou eun pupitre.}

\Conducteurspeaks Bonsouar, mersì a tcheutte, voutro \textit{acceuil} toujour si chalereu m'émochoun-e todzor. Mi l'è pa pe mé que v'ouite inque, tellamente nombreu pe reumplire totte le caèye di si magnifique téatro. V'ouite inque, \textit{mesdames} é \textit{messieurs}, pequé l'è arrevoù lo momàn que tcheu no, que to lo mondo l'è eun tren d'attendre: l'asségnach\'on de l'Oscar pe lo meillaou atteur protagoniste. L'è itaye bièn bataillaye. Vo rappello ara le sinque finaliste, euncó se vo sade bièn que maque eun de leur gagneré, maque eun nom l'è icrì deun la busta, maque eun de leur entrerè deun l'istouére. Le sinque finaliste son:
\begin{enumerate}
\item Leonardo DiCaprio;
\item Julia Roberts;
\item George Clooney;
\item Angelina Jolie;
\item Paolo Cima Sander.
\end{enumerate}

Complemèn a tcheu leur. De caque momàn devrie arrevé la busta é finalemèn noutra queriaouzitoù serè satisfète.

\StageDir{Entre eunna feuille élégante avouì eunna busta rodze é l'Oscar. Lo conducteur can la vèi se émochoun-e é la bèije eun per de cou. Aprì pren la busta.}

\Conducteurspeaks\direct{A la feuille} L'è salla djeusta? Pa comme l'atro an que l'iye eugn'atra busta?

\Vallettaspeaks Spéèn!

\Conducteurspeaks L'Oscar pe lo meillaou atteur protagoniste va a\ldots

\StageDir{Lo conducteur ivre la busta é teurie foua eun papì.}

\Conducteurspeaks \ldots Paolo Cima Sander!

\StageDir{Paolo Cima Sander l'è achouatoù i mentèn di pebleuque. Can sen son nom se levve eun pià tot émochon-où. Eunna lemie s'aleumme fran iaou l'è achouatoù.}

\Cimaspeaks\direct{Bièn émochon-où} Oh mondjeu, mi na, lèi crèyo pa, mi fran mé. Attégnade que areuvvo ba. 

\StageDir{Partèi la tsans\'on:}

\sound{https://www.youtube.com/watch?v=Ycg5oOSdpPQ}{Katchi - Ofenbach vs. Nick Waterhouse}\label{katchi}

\StageDir{Paolo, tellamente ajitoù, miclle eunna mia d'anglé avouì lo patoué.}

\Cimaspeaks Areuvvo! \textit{Oh my God}! Mersì a \textit{everybody}!

\Conducteurspeaks Ouè t'attégnèn!

\StageDir{Paolo s'aprotse i palque.}

\Cimaspeaks Me l'attégnavo pa!

\StageDir{A couza de l'émoch\'on, Paolo lèi beutte bièn de ten pe arrevé . Lo conducteur, ad\'on, tsertse de fé passé lo ten.}

\Conducteurspeaks Ad\'on lo attégnèn. Qui se l'attégnave? Complemèn a Paolo, mi euncó a tcheu le-z-atre que se refàn pi eugn atro an\ldots

\Cimaspeaks\direct{Eun braillèn} N'i sbaillà chortiya! Areuvvo!

\Conducteurspeaks Ouè t'attégnèn. Eugn éffé seutta l'è sa souaré; l'attégnèn volontchì, n'en pa de prisa.

\Cimaspeaks\direct{Comme douàn} Areuvvo!

\Conducteurspeaks\direct{A la feuille, pe fé pasé lo ten} Aprì pensa que dzen: pe eunna personna, que tanque ieur itave deun petchoù veladzo de montagne, arrevé inque a Hollywood, arrevé i \textit{temple}, i Mont Olympe, a l'éillize di \textit{cinéma} mondial.

\Vallettaspeaks Euncrouayablo!

\Conducteurspeaks Ah, voualà Paolo Cima Sander!

\StageDir{Paolo areuvve dézò lo palque é comenche a saliì le dzi. La mezeucca s'arite.}

\Cimaspeaks\direct{Avouì la ranfan-a} Me voualà! \textit{Of course, ok, ok}. \textit{I don't} me l'attégnavo pa. \direct{Eugn eumbrachèn eunna madama} Oh madama moratsade-me vèi!

\StageDir{Paolo conteneuvve a saliì le dzi.}

\Cimaspeaks Mersì, mersì, \textit{one, two, three}\ldots \textit{don't worry be happy}!

\Conducteurspeaks Prèdza maque patoué Paolo!

\Cimaspeaks\direct{Eun calabrot} \textit{Ni vidimu ndu giardinu}! Mersì! \direct{Ver lo conducteur} Iaou son le-z-itchilì?

\StageDir{Lo conducteur moutre iaou son.}

\Cimaspeaks\direct{Eun pouyèn si lo palque} Mondjeu di paadì!

\StageDir{Paolo areuvve si lo palque é salie lo conducteur é la feuille que lèi baille l'Oscar.}

\Cimaspeaks Oh mondjeu! Mi comèn diade seu a Hollywood \og \textit{tutto bagnato di caldo}\fg{}?

\Conducteurspeaks Blet de tsa!

\Cimaspeaks Ad\'on si fran blet de tsa! \direct{Eugn avèitsèn l'Oscar} Mi eunc\'o finque Oscar\ldots pensao pa a eunna baga pouai aprì itre itoù Secretéo di Consor de l'\'Eve de Feleunna.

\StageDir{La feuille chor.}

\Cimaspeaks Ara senque fa diye?

\Conducteurspeaks Sen que t'a voya, lo palque l'è de té.

\Cimaspeaks\direct{Todzor émochon-où} Na, na, trop eumpourtàn si Oscar. Fa remersì la fameuille: Mami Roby é Papi Tony\ldots 

\StageDir{La feuille entre é dézò vouése baille eun mésadzo i conducteur, dimèn que Paolo conteneuvve avouì le remersiemèn.}

\Cimaspeaks \ldots trop eumpourtàn Oscar; se sit an l'è l'Oscar eugn atro an seré lo Nobel!

\Conducteurspeaks Ouè cheur!

 \Cimaspeaks\direct{Todzor bièn émù é ajitoù} Ad\'on fa remersì euncó le vezeun! La vezin-a Miranda, Armando é Lucia que son bièn eumpourtàn; mi sourtoù comèn pa remersì le-z-amì de mé de Feleunna é Plan-Feleunna, le-z-amì valdotèn crèisì avouì mé: Rocco, Turi, Mimmu \textit{u pulici}, Tommaso Buscetta, Salvatore Savastano, Tommaso \textit{u pistulero} é Johnny \textit{a munnezza}. Mi can mimo\ldots Oscar! Que dzen! Can mimo si arrevoù seuilla avouì la fameuille, le-z-amì\ldots mi comèn pa remersì le-z-amì Digourdì! L'an édja-me, son restou-me protso, l'an pourto-me seuilla a Hollywood, djeusto?

\Conducteurspeaks Ouè Hollywood!

\Cimaspeaks Le Digourdì que son crèisì avouì mé!

\Conducteurspeaks L'è to vrèi Paolo é te pou le remersì, te pou lo braillì a tcheutte! Pequé me diyon ara de la réjì que n'en eunna sourprèiza pe té: Paolo, \textit{mesdames} é \textit{messieurs}, n'i fran l'onneur de querì inque si lo palque tcheu le Digourdì, totta la compagnì deun laquelle Paolo l'è crèisì é que la fé-lo vin-ì lo bon atteur que ara l'è douàn no! Réjì! Mezeucca!

\sound{https://www.youtube.com/watch?v=Ycg5oOSdpPQ}{Katchi - Ofenbach vs. Nick Waterhouse}[false]

\StageDir{Lo conducteur queurie, eun aprì l'atro, tcheu le Digourdì si lo palque. Tsaque Digourdì, arbeillà élégàn pe la nite di-z-oscar, attraverse lo parterre eun salièn lo pebleuque é pouye si lo palque pe saliì Paolo.}

\Conducteurspeaks N'i lo plèizì de querì si lo palque:
\begin{itemize}
\item[$\bullet$] Pierre Savioz;
\item[$\bullet$] Ester Bollon;
\item[$\bullet$] Giada Grivon;
\item[$\bullet$] Laurent Chuc;
\item[$\bullet$] Jasmine Comé;
\item[$\bullet$] Francesca Lucianaz;
\item[$\bullet$] Marco Ducly;
\item[$\bullet$] Ilaria Linty;
\item[$\bullet$] Marlène Jorrioz;
\item[$\bullet$] André Comé;
\item[$\bullet$] Jo\"{e}lle Bollon;
\item[$\bullet$] Jo\"{e}l Albaney;
\item[$\bullet$] Jordy Bollon;
\item[$\bullet$] Simone Roveyaz;
\item[$\bullet$] Thierry Jorrioz.
\end{itemize}

\StageDir{La mezeucca s'arite.}

\Conducteurspeaks Voualà son tcheue inque. \textit{Mesdames} é \textit{messieurs}, a vo le Digourdì!

\StageDir{Teuppe \lemieBa\ a gotse é i mentèn di palque.}

\StageDir{Lemie \lemieSi\ a drèite.}

\scene[-- An matse de DVD]

\Nevaouspeaks \ldots é queun l'iye lo titre de si \textit{court-métrage} que l'a fé gagnì l’Oscar a Cima?

\Paganspeaks L'iye\ldots \og RECONSTITUTION – La vèille di gran dz\^o\fg.

\Nevaousaspeaks \ldots é senque counte?

\Paganspeaks L'iye la counta\ldots \direct{Ver Sophie} de senque counte?

\Maganspeaks L'iye la counta de la \textit{reconstitution} de noutra Quemeua: sinque Tsarvensolèn, deun lo 1946, l'an preì eun man Tsarvensoù.

\Nevaouspeaks Se pou veure?

\Paganspeaks Ouè! N'en euncó lo DVD!\direct{A Sophie} Iaou t'a catcha-lo?

\Maganspeaks Lé avouì le bague de té!

\Paganspeaks Ouè, t'aré caya-lo ià!

\Nevaousaspeaks Eun DVD? Ad\'on to lo moundo l'arè vi-lo!

\Paganspeaks To lo moundo! L'atsetavon si Amazon, l'avèitsavon si l'iPhone, l'atsetavon euntchì no; mi lo cou que n'en vendi-nen de pi l'iye l'an que n'en fé la pièse pe le dji-z-àn di Digourdì! L'iye lo 10 marse 2018 i Téatro Splendor!

\Nevaouspeaks Mi iaou i Splendor?

\Paganspeaks Iaou? \direct{Eugn avèitsèn amoddo lo pebleuque} Lé! Can te chor a gotse! Me rappello que l'an bailla-no eunna matse de sou! N'en to vendì é vo sade senque l'è capitoù? L'è arrevoù eugn ommo que l'a bailla-no pe eun DVD 100 euro!

\Nevaousaspeaks Mi qui l'iye si matte?

\Maganspeaks L'iye lo Seunteucco deun cou: Ronny Borbey!

\Paganspeaks Que matte Ronny! L'è belle aloù eun pench\'on euncó llou ara\ldots

\Maganspeaks Ouè!

\Nevaouspeaks Ah ouè. V'ouyavade fran feun vo Digourdì, v'ouite itoù todzor de Digourdì!

\Paganspeaks Ouè. Aimè, te sa pequé?\direct{Eun se levèn} Pequé itre Digourdì vou deu itre: abile, actif é euntelledzèn! \'E no lo sen todzor itoù, dèi can sen nèisì! No sen todzor\ldots 

\StageDir{Se avion le lemie \lemieSi\ si to lo palque. A gotse n'a tcheu le Digourdì, que, eunsemblo avouì pagàn, magàn é le dou nevaou, eurlon ver lo pebleuque:\
\begin{center}
\ldots pi Digourdì!
\end{center}
}

\ridocliou

\DeriLeRido

\RoleNoms{Mezeucca, éffé sonore}{Michel Comé}

\RoleNoms{Lemie}{Alessandro Chiono}

\RoleNoms{Mijì}{Renzo Bollon}

\end{drama}
 
\queriaouzitou{
\begin{itemize}
\item[$\bullet$] Lo titre\footnote{ Pe la grafie di BREL lo titre sarie ``SÈIDEPOU(VO)ER". Lo dérì acsàn l'è considéroù pa nésesséo, pa an veetabla erreur; n'en donque désidoù de vardì la grafie orijinella di titre.} de la pièse l'è eun djouà de paolle que souligne dou élémèn di mal gouvernemèn de la classe dirijante: la sèi de pouvoer (SÈI DE POUVOÈR) é la corruch\'on, l'itre pouer (SÈIDE POUÈR).\\

\item[$\bullet$] Lo final de la pièse l'è it\'o icrì é ricrì pi de dji cou\ldots é prooù pi de 100 cou! L'è itaye la \textit{scène} pi traillaye de totta l'istouére di Digourdì.\\

\item[$\bullet$] Lo dzor di spettaclle, la Compagnì l'è itaye prémiaye pe la seconda plase di Prix Magui Bétemps pe la pièse di 2018 \og Todzo pi Digourdì\fg! Eunna grousa soddisfach\'on pe le Digourdì, di momàn que sayoon jamì it\'o prémià pe si Concour.\\

\item[$\bullet$] Pe lo premì cou l'è pouyà si lo palque eun noutro cher collaborateur é \textit{auteur}. No spéèn todzor que tournise si lo palque, mi llou préfée partisipé foua di téatre, eun no baillen de-z-idoù jénialle pe réalizé de counte comique.\\

\item[$\bullet$] L'élémèn lo pi métatéatral de SÈIDEPOUVOÈR l'è, sensa doute, la campagne élettorale que tcheu le-z-atteur l'an fé lo lon di senà devàn de la pièse. Le doe liste l'an pa mol\'o de baillì de beillette élettoral ià pe la parotse, pe convencre le dzi a vin-ì ba i téatre pe voté lo nouo Prézidàn di Digourdì. A témouagnadzo de la pressante propagande féta pe la lista numér\'o dou, n'a seutta \textit{vidéo}:

\begin{figure}[H]
\centering
\video\hspace*{0.5mm} \textsc{\small Vota Jordy di Digourdì}\hspace*{0.5mm} \video\\\vspace*{2mm}
\qrcode[hyperlink, height=0.5in]{https://www.instagram.com/p/Bul3B3loADp/}
\addcontentsline{vds}{section}{Vota Jordy di Digourdì}
\end{figure}

\end{itemize}
}

\backmatter
\chapter*{Appendix}
\pagestyle{plain}
\subsection*{Versione digitale}
Nel seguente \textit{repository} :
\begin{center}
\centering
\github\ \hspace*{0.5mm} \href{\detokenize{https://github.com/jbollon/Dji-cou-Digourdi}}{\color{RawSienna}{\textsc{GitHub}}} \hspace*{0.5mm} \github\\
%\vspace{-5pt}
 \vspace*{2mm}
\qrcode[hyperlink, height=0.5in]{\detokenize{https://github.com/jbollon/Dji-cou-Digourdi}}
\end{center}
\noindent è archiviato tutto il codice utilizzato per realizzare Dji cou Digourdì, nonché la sua versione in pdf. Dalla versione digitale è possibile accedere ai file originali utilizzati dalla Digourdì per riprodurre gli effetti sonori e i brani musicali all'interno delle pièce. Sarà sufficiente cliccare sul titolo dell'effetto sonoro o del brano.
Inoltre, la versione digitale ci consentirà di  tenere aggiornati i link e i qrcode di tutti i contenuti multimediali, qualora venissero modificati per qualsivoglia ragione.

\newpage
\subsection*{Statistique}
%\begin{minipage}{0.48\textwidth}
\paragraph*{L’OPETAILLE MODERNO}
\begin{tabular}{lc}
\toprule
\textbf{Paolla} & \textbf{Contcho} \\
\midrule
mesieu & 17 \\
bondzor & 15 \\
mersì & 15 \\
mitcho & 14 \\
medeseun & 13 \\
trèi & 12 \\
dzor & 11 \\
ten & 10 \\
madama & 10 \\
dzenta & 9 \\
\bottomrule
\end{tabular}
\end{minipage}
\hfill
\begin{minipage}{0.48\textwidth}
\subsection*{FORUM VALDOT\`EN}
\begin{tabular}{lc}
\toprule
\textbf{Paolla} & \textbf{Contcho} \\
\midrule
mesieu & 46 \\
dzeudzo & 36 \\
rita & 18 \\
dzor & 17 \\
frottapanse & 14 \\
forum & 11 \\
madama & 11 \\
cas & 11 \\
premì & 10 \\
réjón & 10 \\
\bottomrule
\end{tabular}
\end{minipage}

\vspace{1em}
\begin{minipage}{0.48\textwidth}
\subsection*{LA VATSE DE L’UNIVERSITO\`U}
\begin{tabular}{lc}
\toprule
\textbf{Paolla} & \textbf{Contcho} \\
\midrule
pappa & 29 \\
vatse & 25 \\
vétérinéro & 23 \\
dzor & 14 \\
lasì & 14 \\
bou & 11 \\
bague & 9 \\
seutte & 9 \\
pouì & 9 \\
problème & 9 \\
\bottomrule
\end{tabular}
\end{minipage}
\hfill
\begin{minipage}{0.48\textwidth}
\subsection*{EUN DROLO DE DISTRIBUTEUR}
\begin{tabular}{lc}
\toprule
\textbf{Paolla} & \textbf{Contcho} \\
\midrule
éve & 64 \\
litre & 28 \\
botèille & 20 \\
mesieu & 20 \\
mersì & 17 \\
machina & 12 \\
poste & 11 \\
documàn & 11 \\
aoue & 10 \\
pouì & 10 \\
\bottomrule
\end{tabular}
\end{minipage}

\vspace{1em}
\newpage
\begin{minipage}{0.48\textwidth}
\subsection*{MATTE\ldots SEN TCHEUTTE MATTE}
\begin{tabular}{lc}
\toprule
\textbf{Paolla} & \textbf{Contcho} \\
\midrule
marie & 20 \\
bague & 19 \\
bonsouar & 17 \\
ommo & 14 \\
tanta & 13 \\
non & 12 \\
trasmechón & 12 \\
pappa & 12 \\
avèitsa & 11 \\
diye & 11 \\
\bottomrule
\end{tabular}
\end{minipage}
\hfill
\begin{minipage}{0.48\textwidth}
\subsection*{TANTA BETSII}
\begin{tabular}{lc}
\toprule
\textbf{Paolla} & \textbf{Contcho} \\
\midrule
tanta & 21 \\
hermann & 18 \\
trèi & 14 \\
mamma & 12 \\
cesar & 12 \\
betsii & 12 \\
dzor & 12 \\
quése & 12 \\
maleur & 10 \\
pappa & 10 \\
\bottomrule
\end{tabular}
\end{minipage}





\vspace{1em}
\begin{minipage}{0.48\textwidth}
\subsection*{DISCO FLAMA}
\begin{tabular}{lc}
\toprule
\textbf{Paolla} & \textbf{Contcho} \\
\midrule
giulio & 32 \\
pollein & 31 \\
vèyo & 25 \\
doe & 21 \\
local & 20 \\
beutta & 19 \\
man & 18 \\
dzenta & 16 \\
tsarvensoù & 16 \\
pappa & 16 \\
\bottomrule
\end{tabular}
\end{minipage}
\hfill
\begin{minipage}{0.48\textwidth}
\subsection*{N'EN PA LO TEN}
\begin{tabular}{lc}
\toprule
\textbf{Paolla} & \textbf{Contcho} \\
\midrule
touéno & 20 \\
sandrino & 19 \\
ten & 16 \\
bondzor & 15 \\
amoddo & 10 \\
mesieu & 10 \\
baga & 10 \\
man & 10 \\
voualà & 9 \\
tourna & 9 \\
\bottomrule
\end{tabular}
\end{minipage}

\vspace{1em}
\newpage

\begin{minipage}{0.48\textwidth}
\subsection*{TODZO PI DIGOURDÌ}
\begin{tabular}{lc}
\toprule
\textbf{Paolla} & \textbf{Contcho} \\
\midrule
digourdì & 45 \\
tsarvensoù & 27 \\
téatro & 20 \\
feleunna & 20 \\
compagnì & 14 \\
aloù & 14 \\
pièse & 14 \\
dzi & 12 \\
atro & 12 \\
vouése & 12 \\
\bottomrule
\end{tabular}
\end{minipage}

\vspace{1em}
\begin{minipage}{0.48\textwidth}
\subsection*{SÈIDEPOU(VO)ÈR}
\begin{tabular}{lc}
\toprule
\textbf{Paolla} & \textbf{Contcho} \\
\midrule
digourdì & 39 \\
marlène & 27 \\
prézidàn & 25 \\
madama & 23 \\
mersì & 18 \\
bague & 17 \\
prédjì & 17 \\
mesieu & 16 \\
beillette & 15 \\
voté & 14 \\
\bottomrule
\end{tabular}
\end{minipage}
\hfill






\vspace{1em}

\newpage
\section*{Le dji paolle pi eumpléyaye}
\begin{tabular}{lc}
\toprule
\textbf{Paolla} & \textbf{Contcho} \\
\midrule
mesieu & 115 \\
dzor & 100 \\
digourdì & 91 \\
mersì & 88 \\
man & 84 \\
éve & 83 \\
bague & 83 \\
dzenta & 74 \\
ten & 72 \\
seutte & 71 \\
\bottomrule
\end{tabular}


\section*{Le dji paolle pi londze}
\begin{tabular}{ll}
\toprule
\textbf{Paolla} & \textbf{Londjaou} \\
\midrule
peuccaparmidjàn & 15 \\
tsarvensolentse & 15 \\
tranquillamente & 15 \\
professioniste & 14 \\
ecstraordinére & 14 \\
gratouitamente & 14 \\
avanspettaclle & 14 \\
reconstitution & 14 \\
ecstraordinéra & 14 \\
reconstituchón & 14 \\
\bottomrule
\end{tabular}


\begin{table}[]
\centering
\begin{tabular}{lr}
\multicolumn{2}{c}{Tot} \\
    \toprule
Tot parole & 42588 \\
\bottomrule
\end{tabular}%
\end{table}
\begin{table}[]
\centering
\begin{tabular}{lr}
\multicolumn{2}{c}{Les 10 mots les plus dits} \\
    \toprule
\multicolumn{1}{l}{\textbf{Mot}} & \textbf{N} \\
    \midrule
\multicolumn{1}{l}{Mesieu} &107\\
\multicolumn{1}{l}{Dzor} &93\\
\multicolumn{1}{l}{Mersì} &85\\
\multicolumn{1}{l}{Djeusto} &76\\
\multicolumn{1}{l}{Bague} &76\\
\multicolumn{1}{l}{Dzen} &75\\
\multicolumn{1}{l}{Deu} &70\\
\multicolumn{1}{l}{Vou} &69\\
\multicolumn{1}{l}{Pappa} &69\\
\multicolumn{1}{l}{Éve} &67\\
\multicolumn{1}{l}{Veun} &66\\
\bottomrule
\end{tabular}%
\end{table}
\newpage
\scriptsize
\begin{longtable}{llrrr}
\caption{Tous les acteurs}\\
\toprule
\textbf{Acteur} & \textbf{Top 3 mots} & \textbf{Pièce} & \textbf{Lignes} & \textbf{Mots} \\
    \hline
Albaney Joël &Fiade, Djeusto, Vère & 10 & 518 & 5725\\
Albaney Stéphanie &Fenì, Touéno & 5 & 109 & 1169\\
Bollon Ester &Dzor & 4 & 64 & 656\\
Bollon Jordy &Prézidàn, Veun & 7 & 212 & 2475\\
Bollon Joëlle &Mersì, Mesieu, Twitter & 8 & 243 & 2533\\
Borbey Denise &Hélène, Giulio, Payo & 1 & 63 & 933\\
Borbey Ronny & & 1 & 1 & 9\\
Brunod Christian &Vilma & 1 & 43 & 335\\
Chuc Laurent &Pappa, Mesieu, Pollein & 7 & 170 & 2621\\
Cima Sander Paolo &Bague, Dzor, Amoddo & 10 & 441 & 4893\\
Comé André &Dérì, Aoua & 7 & 150 & 1777\\
Comé Jasmine &Blan & 2 & 25 & 290\\
Comé Michel &Pizze, Bruno & 1 & 5 & 76\\
Comé Sophie &Tanteun, Mesieu, Busta & 6 & 108 & 1419\\
Cunéaz Richard &Paolo, Comprèi, Mitcho & 2 & 68 & 1008\\
Dalbard Aline & & 1 & 16 & 120\\
Dall'Ara Paolo &Digourdì & 2 & 88 & 1669\\
Ducly Marco &Mesieu, Djeusto, Bou & 10 & 251 & 2277\\
Giannino Tania &Pollein, Tsarvensoù & 1 & 16 & 414\\
Grivon Giada &Vatse & 2 & 15 & 154\\
Jorrioz Marlène &Petchoudacllenda, Mersì & 4 & 104 & 1458\\
Jorrioz Thierry &Plan & 2 & 35 & 230\\
Linty Ilaria &Documàn, Vétérinéro & 5 & 55 & 781\\
Lucianaz Eyvia & & 1 & 2 & 37\\
Lucianaz Fabien &Dzor, Prézidàn & 1 & 12 & 200\\
Lucianaz Francesca &Éve, Veure, Mersì & 8 & 155 & 2329\\
Marguerettaz Valérie &Vèyo & 1 & 33 & 354\\
Roveyaz Simone &Dzor, Pollein & 8 & 79 & 943\\
Savioz Pierre &Mesieu, Dzor & 8 & 356 & 4372\\
Squinabol Aimé &Digourdì & 1 & 11 & 153\\
Squinabol Federico &Gène, Beutta & 1 & 14 & 148\\
Squinabol Ilenia &Mitcho, Mamma, Deu & 1 & 3 & 63\\
Squinabol Julie & & 1 & 12 & 106\\
Squinabol Michel &Giulio, Atro & 1 & 13 & 115\\
Ventrice Michel & & 1 & 3 & 20\\
Vermillon Stefano & & 1 & 19 & 126\\
Viérin Elisa &Lavave, Vèyo & 1 & 35 & 193\\
Yeuillaz Alain &Selmo, Bèye & 1 & 39 & 407\\
\bottomrule
\end{longtable}
\begin{table}[]
\centering
\caption{Les 3 acteurs avec plus de lignes}
\begin{tabular}{l|r}
\toprule
\multicolumn{1}{l}{\textbf{Acteur}} & \textbf{Lignes} \\
\midrule

\multicolumn{1}{l}{Joël Albaney} &518\\
\multicolumn{1}{l}{Paolo Cima Sander} &441\\
\multicolumn{1}{l}{Pierre Savioz} &356\\
\multicolumn{1}{l}{Marco Ducly} &251\\
\bottomrule
\end{tabular}%
\end{table}
\begin{table}[]
\centering
\caption{Les 3 acteurs avec plus de mots}
\begin{tabular}{l|r}
    \toprule
\multicolumn{1}{l}{\textbf{Acteurs}} & \textbf{Mots} \\
    \midrule
\multicolumn{1}{l}{Joël Albaney} &5725\\
\multicolumn{1}{l}{Paolo Cima Sander} &4893\\
\multicolumn{1}{l}{Pierre Savioz} &4372\\
\multicolumn{1}{l}{Laurent Chuc} &2621\\
\bottomrule
\end{tabular}%
\end{table}
\begin{table}[]
\centering
\caption{Les 3 acteurs plus présents}
\begin{tabular}{lr}
    \toprule
\multicolumn{1}{l}{\textbf{Acteur}} & \textbf{Pièce} \\
    \midrule
\multicolumn{1}{l}{Albaney Joël} &10\\
\multicolumn{1}{l}{Ducly Marco} &10\\
\multicolumn{1}{l}{Cima Sander Paolo} &10\\
\multicolumn{1}{l}{Bollon Joëlle} &8\\
\bottomrule
\end{tabular}%
\end{table}
    \begin{table}[]
    \centering
    %\caption{desc...}
    \begin{tabular}{lr}\toprule\multicolumn{2}{c}{L’OPETAILLE MODERNO} \\\midrule
\multicolumn{1}{l}{Nombre d'acteur}&10\\
\multicolumn{1}{l}{Numero totale di parole}&2758\\
\multicolumn{1}{l}{Numero totale di battute}&302\\
\multicolumn{1}{l}{Attore con più parole}&Albaney Joël (666)\\
\multicolumn{1}{l}{Attore con più battute}&Albaney Joël (67)\\
\multicolumn{1}{l}{Parole più usate}&Mesieu, Bondzor, Mersì\\
    \bottomrule
    \end{tabular}%
    \end{table}
    \begin{table}[]
    \centering
    %\caption{desc...}
    \begin{tabular}{lr}\toprule\multicolumn{2}{c}{FORUM VALDOTÈN} \\\midrule
\multicolumn{1}{l}{Nombre d'acteur}&10\\
\multicolumn{1}{l}{Numero totale di parole}&3981\\
\multicolumn{1}{l}{Numero totale di battute}&219\\
\multicolumn{1}{l}{Attore con più parole}&Chuc Laurent (716)\\
\multicolumn{1}{l}{Attore con più battute}&Lucianaz Francesca (41)\\
\multicolumn{1}{l}{Parole più usate}&Mesieu, Dzeudzo, Dzor\\
    \bottomrule
    \end{tabular}%
    \end{table}
    \begin{table}[]
    \centering
    %\caption{desc...}
    \begin{tabular}{lr}\toprule\multicolumn{2}{c}{LA VATSE DE L’UNIVERSITOÙ} \\\midrule
\multicolumn{1}{l}{Nombre d'acteur}&12\\
\multicolumn{1}{l}{Numero totale di parole}&4423\\
\multicolumn{1}{l}{Numero totale di battute}&334\\
\multicolumn{1}{l}{Attore con più parole}&Savioz Pierre (1271)\\
\multicolumn{1}{l}{Attore con più battute}&Savioz Pierre (97)\\
\multicolumn{1}{l}{Parole più usate}&Pappa, Vatse, Vétérinéro\\
    \bottomrule
    \end{tabular}%
    \end{table}
    \begin{table}[]
    \centering
    %\caption{desc...}
    \begin{tabular}{lr}\toprule\multicolumn{2}{c}{EUN DROLO DE DISTRIBUTEUR} \\\midrule
\multicolumn{1}{l}{Nombre d'acteur}&16\\
\multicolumn{1}{l}{Numero totale di parole}&3718\\
\multicolumn{1}{l}{Numero totale di battute}&280\\
\multicolumn{1}{l}{Attore con più parole}&Albaney Joël (742)\\
\multicolumn{1}{l}{Attore con più battute}&Albaney Joël (63)\\
\multicolumn{1}{l}{Parole più usate}&Éve, Litre, Mesieu\\
    \bottomrule
    \end{tabular}%
    \end{table}
    \begin{table}[]
    \centering
    %\caption{desc...}
    \begin{tabular}{lr}\toprule\multicolumn{2}{c}{MATTE\ldots SEN TCHEUTTE MATTE} \\\midrule
\multicolumn{1}{l}{Nombre d'acteur}&13\\
\multicolumn{1}{l}{Numero totale di parole}&4746\\
\multicolumn{1}{l}{Numero totale di battute}&405\\
\multicolumn{1}{l}{Attore con più parole}&Bollon Jordy (898)\\
\multicolumn{1}{l}{Attore con più battute}&Cima Sander Paolo (71)\\
\multicolumn{1}{l}{Parole più usate}&Tanteun, Marie, Bague\\
    \bottomrule
    \end{tabular}%
    \end{table}
    \begin{table}[]
    \centering
    %\caption{desc...}
    \begin{tabular}{lr}\toprule\multicolumn{2}{c}{TANTA BETSII} \\\midrule
\multicolumn{1}{l}{Nombre d'acteur}&10\\
\multicolumn{1}{l}{Numero totale di parole}&3179\\
\multicolumn{1}{l}{Numero totale di battute}&290\\
\multicolumn{1}{l}{Attore con più parole}&Savioz Pierre (1214)\\
\multicolumn{1}{l}{Attore con più battute}&Savioz Pierre (100)\\
\multicolumn{1}{l}{Parole più usate}&Tanta, Rémy, Hermann\\
    \bottomrule
    \end{tabular}%
    \end{table}
    \begin{table}[]
    \centering
    %\caption{desc...}
    \begin{tabular}{lr}\toprule\multicolumn{2}{c}{DISCO FLAMA} \\\midrule
\multicolumn{1}{l}{Nombre d'acteur}&22\\
\multicolumn{1}{l}{Numero totale di parole}&5369\\
\multicolumn{1}{l}{Numero totale di battute}&514\\
\multicolumn{1}{l}{Attore con più parole}&Borbey Denise (933)\\
\multicolumn{1}{l}{Attore con più battute}&Borbey Denise (63)\\
\multicolumn{1}{l}{Parole più usate}&Pollein, Giulio, Vèyo\\
    \bottomrule
    \end{tabular}%
    \end{table}
    \begin{table}[]
    \centering
    %\caption{desc...}
    \begin{tabular}{lr}\toprule\multicolumn{2}{c}{N'EN PA LO TEN} \\\midrule
\multicolumn{1}{l}{Nombre d'acteur}&11\\
\multicolumn{1}{l}{Numero totale di parole}&4049\\
\multicolumn{1}{l}{Numero totale di battute}&424\\
\multicolumn{1}{l}{Attore con più parole}&Albaney Joël (942)\\
\multicolumn{1}{l}{Attore con più battute}&Cima Sander Paolo (86)\\
\multicolumn{1}{l}{Parole più usate}&Touéno, Fièn, Bondzor\\
    \bottomrule
    \end{tabular}%
    \end{table}
    \begin{table}[]
    \centering
    %\caption{desc...}
    \begin{tabular}{lr}\toprule\multicolumn{2}{c}{TODZO PI DIGOURDÌ} \\\midrule
\multicolumn{1}{l}{Nombre d'acteur}&16\\
\multicolumn{1}{l}{Numero totale di parole}&4865\\
\multicolumn{1}{l}{Numero totale di battute}&423\\
\multicolumn{1}{l}{Attore con più parole}&Dall'Ara Paolo (884)\\
\multicolumn{1}{l}{Attore con più battute}&Albaney Joël (79)\\
\multicolumn{1}{l}{Parole più usate}&Digourdì, Téatro, Feleunna\\
    \bottomrule
    \end{tabular}%
    \end{table}
    \begin{table}[]
    \centering
    %\caption{desc...}
    \begin{tabular}{lr}\toprule\multicolumn{2}{c}{SÈIDEPOU(VO)ÈR} \\\midrule
\multicolumn{1}{l}{Nombre d'acteur}&15\\
\multicolumn{1}{l}{Numero totale di parole}&5500\\
\multicolumn{1}{l}{Numero totale di battute}&395\\
\multicolumn{1}{l}{Attore con più parole}&Albaney Joël (878)\\
\multicolumn{1}{l}{Attore con più battute}&Albaney Joël (67)\\
\multicolumn{1}{l}{Parole più usate}&Digourdì, Dzen, Prézidàn\\
    \bottomrule
    \end{tabular}%
    \end{table}

%AGGIUNGERE STAT per attore (quello con più batture, con più parole, con meno parole, con meno battute) Gli attori con più presenze

\end{document}