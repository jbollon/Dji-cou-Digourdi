\title{TANTA BETSII}

\author{Pe eugn idoù de Paolo Cima Sander,\\ pièse icrita pe Flavio Albaney\\  revezetaye pe Joël Albaney}
\date{Téatro Splendor de Veulla, 26 avrì 2014}

\maketitle

\fotocopertina{Foto/2014/gruppo.jpg}{Francesca Lucianaz, André Comé, Sophie Comé, Ilaria Linty, Paolo Cima Sander, Marco Ducly, Pierre Savioz}{Jo\"{e}lle Bollon, Laurent Chuc, Stéphanie Albaney, Jordy Bollon, Jo\"{e}l Albaney, Simone Roveyaz}{2014}

\LinkPiese{Tanta Betsii}{https://www.youtube.com/watch?v=XT5qP9hjWy4&t=15s}{.5}

\souvenir{Sayòn mé, pappa é Peppino Camandona (bonanima), lé a medjì la trippa a la Caritas de la Pagoda; eun momàn tchica particulié, i mentèn di-z-ansièn, di pouo, di maladdo.
\\ L'ion eun trèn de fe la counta can Peppino ataque a no countì de si cou que l'an faillì pourtì si pe Cogne an quése di mor avouì eunna Fiat. Can son arrevoù si, son entr\'o pe lo mitcho é l'an accapoù, a gotse, lo mor deun la coutse é a drèite, deun l'atra tsambra, n'ayè l'abeuque que teriè ba de grappa!
\\ Ver mèitchà matin-où, cllouzon lo mor deun la quése é passon l'atra mèitchà de la matin-où eun compagnì di paèn, di-z-amì di mor é\ldots de l'abeuque! 
\\ Dimèn l'a attacoù a nèire détchis; naturellamente gneun l'è rendi-se contcho de ren é pe l'aoua de la sepolteua n'ayè belle demì metre de nèi. La Fiat l'ie pa 4x4, mi sitte l'ie pa lo vrèi problème. Lo vrèi défì l'è it\'o de pourté foua si le-z-ipale la quése sensa se euntsabotté pe la nèi\ldots l'an beun i tchica de difficult\'o, mi a la feun totte l'è aloù amoddo.
\\ Seutta donque l'ie l'ispirach\'on que l'a fé nèitre \og Tanta betsii\fg. Lo dzor aprì de la trippa de la Caritas, n'i terià ba le personadzo, eun per de-z-ach\'on é avouì Jo\"el Albaney n'en désidoù de betì la betsii i poste de l'abeuque. Aprì avouì to lo reste di groupe n'en djount\'o de-z-id\'o comique, comme le Tirolèis é la danse tsaque cou la paolla \textit{kartoffen} l'ie prononchaye.
\\ \'E naturellamente lo final\ldots to lo Splendor l'ie pléyà eun dou di riye. Eun \textit{crescendo} de ritme, \og é ara comèn fièn, maladetto n'en pa la quése, comèn fièn, iaou la betèn, l'è pa poussiblo sise dzouveun-o é l'internet, é ara comèn fièn\ldots \fg{}; lé mé é Pierre no avèitsèn, aprì avèitsèn la m\"e\ldots doe secounde aprì sen foua di palque dézò lo bouéchì di man di pebleuque.}{Paolo Cima Sander}
%Con Joel poi mi sono visto e ho detto prendiamo spunto da questo fatto. Io tiro giù una scaletta di personaggi e azioni. Con joel decidiamo di mettere la betsii al posto dell'alambicco. Poi con la vecchia squadra (franci, pierre...) abbiamo scritto il resto (i tirolesi betsii). Mi ricordo bene del balletto kartoffen.
%Poi geniale la chiusura... climax pazzesco! Guardato la mii e siamo usciti trionfanti!
%Era un tripudio di idee


%idea nata alla mensa della caritas della pagoda (Peppino camandona), c'era la trippa, (eravamo io papà e peppino)... contesto particolar (anziani, senza tetto).

%Stavamo mangiando... tra una cosa e l'altra... peppino racconta di una macchina particolare (fiat) per un funerale vanno a Cogne (con cassa da morto nella macchina). Entrano in casa, a sx hanno il morto e a dx hanno l'alambicco in funzione nell'altra stanza.
%Chiudono il morto verso metà mattinana.
%poi si mettono a fare la counta con gli altri e anche a bere... per la sepoltura erano caldissimi!
%Ma nel mentre ha nevicato di botto ed erano un po' alticci. Dovevano reggere la cassa sulla neve con un po' di difficoltà... e poi sono andati al funerale.
\queriaouzitou{
\begin{itemize}
\item[$\bullet$] Lo titre \og Tanta betsii \fg l'è eun djouà de paolle avouì bièn d'eunterprétach\'on: 1) la tanta responsabla de la betsii; 2) la tanta l'è la tseur de la betsii; 3) a l'italienne lo titre semble a \og an matse de betsii \fg{}.\\

\item[$\bullet$] Pe le digourdì, \og Tanta betsii \fg l'è itaye la premiye pièse eunterprétaye i Téatro Splendor de Veulla.\\

\item[$\bullet$] \og Tanta betsii \fg l'è la pièse pi queurta de totta l'istouére di Digourdì: 34 meneutte!\\

\item[$\bullet$] Ver mèitchà piése, Paolo Cima Sander s'oubliè todzor eun\-na battiya a propoù di bouì. Pe tsertchì de lèi la fé rappelì, Pierre, can comprégnè que Paolo l'ie eun tren de s'oubliì la battiya, dijè \og Oh flotte! \fg. Naturellamente, eunc\'o lo dzor di spettaclle Paolo l'è oublia-se  la battiya\ldots vo queuttèn la queriaouzitoù d'accapé si momàn.\\

\item[$\bullet$] Qui avèitserè la \textit{vidéo}, for probablo s'apesèiserè d'eun pégno \textit{vaff****lo} que noutro Paolo l'è queutto-se scapé! Paolo se squeuze avouì vo tcheutte!
\end{itemize}
}


\Scenographie
\begin{itemize}
\item[$\bullet$] 3 fondal gri, ate 4 mètre pe reprézanté lo meur d'eun grou \textit{garage};
\item[$\bullet$] 1 viille crédense de bouque;
\item[$\bullet$] 2 table de la Pro Loco;
\item[$\bullet$] 1 \textit{bombola} di gas \gas\ é 1 fornelleun;
\item[$\bullet$] 1 më pe fé betsii;
\item[$\bullet$] 3 caèye;
\item[$\bullet$] Eunna mia de cart\'on pe pa pouertchì pe tèra;
\item[$\bullet$] De matériel pe fé betsii: caoutì, platì, Scottex, sidel, cachoule, eumbouéleuza\ldots ;
\item[$\bullet$] De gou é de-z-engrédiàn pe la betsii: so, pèivro, aille, romaneun,  sarrieula, veun rodzo, cllou de garoffe, gnoué moscatta, trifolle, carote rodze.
\end{itemize}

\setlength{\lengthchar}{2.5 cm}

\Character[CESAR]{CESAR}{Cesar}{Ommo de Lucie que l'a jamì fé ren pe le mitcho, \name{Pierre Savioz}}

\Character[RÉMY]{RÉMY}{Remy}{Gars\'on \garson\ de Cesar é Lucie, 25 an é djeusto laouréoù, man blantse sensa caille é arbeillà avouì de pateun firmoù, \name{Jordy Bollon}}

\Character[HÉLÈNE]{HÉLÈNE}{Helene}{Feuille \feuille\ de Cesar é Lucie, 25 an, arbeillaye \textit{punk}; eunc\'o lleu l’a djeusto fenì l’universit\'o, donque l’a  jamì traillà, \nameF{Stéphanie Albaney}}

\Character[HERMANN]{HERMANN}{Hermann}{Betchì tirolèis, \name{Paolo Cima Sander}}

\Character[PETER]{PETER}{Peter}{Dzouvin-o tirolèis, betchì, sec\'on de Hermann, \name{Jo\"{e}l Albaney}}

\Character[KATRIN]{Katrin}{Katrin}{Siaou de Peter avouì eun for caratéo é de magniye de fé grouchie, \nameF{Francesca Lucianaz}}

\Character[LUCIE]{LUCIE}{Lucie}{Fenna de Cesar é nevaouza de tanta Melanie, \nameF{Ilaria Linty}}

\Character[GERMENE]{GERMENE}{Germene}{Viille fenna de Battista, 80 an \viille , vezeun-a di mitcho é bièn queriaouza, \nameF{Sophie Comé}}

\Character[BATTISTA]{BATTISTA}{Battista}{Eun mastoque d’ommo de 80 an  \viou , vezeun di mitcho é ommo de Germene,  \name{André Comé}}

\Character[PRÉYE]{PRÉYE}{Preye}{Monseur de la parotse, \name{Marco Ducly}}

\Character[]{LE DOU}{Ledou}{\quad}

\Character[]{LE DOU MEINOU}{Ledoumeinou}{\quad}

\Character[]{TCHEUT}{Tcheut}{\quad}

\DramPer

\act[Acte I]

\StageDir{\Fv{Comme d’abetide, tcheu le-z-àn i mitcho de tanta Melanie, eunna viille fenna pa mariaye é acotemaye a comandé, son eun tren de féye la tradichonella betsii. A la betsii, tcheu le parèn de lleu son obledjà d'itre prézèn pe prende le seun odre é féye tcheu le travaille nésesséo. Sit an le bague son diffiente: tanta Melanie pou pa lèi itre pequé l’è maladda é l'è platta a la coutse \maladdo. L’a donque faillì prende tot eun man Cesar que, vu que l’è jamì itoù eun tita a la bagga, tsertse de organizé le dzornaye comme l’è bon. Cesar l’a marià Lucie, niése de tanta Melanie, é l’a dou mèinoù. Oueu totta la formach\'on l’è desì lo poste de travaille é l’a dza gnou\'o la premiye dzornoù}}

\ridoiver

\scene[-- Apresten-n\'o]

\StageDir{Lemie \lemieSi .}

\StageDir{I mentèn di palque n'a doe table de la Pro Loco avouì de matériel é de-z-engrédiàn pe fé betsii. Déz\'o la tabla l'è plachaye la më. I fon di palque n'a trèi fondal: si di mentèn l'at eunna pourta que permé i-z-atteur d'alì eun tsambra de tanta Melanie. Dérì la tabla n'at eunc\'o eunna viille crédense iaou n'a de botèille é d'atre baradziye.}

\StageDir{Cesar é le dou mèinoù son eun tren d'apresté to sen que servèi pe la betsii: Cesar, i mentèn di palque, euncolle le cart\'on pe tèra pe pa eumpouertchì lo solàn, Rémy, a gotse, tsertse d'atatchì la bombola di gas i fornelleun, Hélène, achouataye a drèite, pleumme le trifolle é le carote rodze.}

\begin{drama}

\Cesarspeaks\direct{Bièn ajitoù \ajitou\ é pa cheur de sen que dèi féye} Gars\'on! A queun poueun v'ouite? Maleur! Mi l'è dza nou aoue, de sé a eun momàn areuvve Hermann di Tirol é no sen seu to pe l'er!

\Remyspeaks Pappa veun seuilla! Chor pa lo gas! 

\Cesarspeaks\direct{I galoppe ver Rémy} Trama-té\ldots mi pe fose que va pa! T'a pa praou sar\'o\ldots soplé bailla-mé le cllo!

\Remyspeaks Salle di mitcho?

\Cesarspeaks\direct{Dispéoù} Mi na, na Rémy! Le cllo pe saré lo tibo di gas!

\Remyspeaks N'a de cllo pe saré?

\Cesarspeaks Maleur! Queutta pédre. Fiyo mé que pappa l'a de dèi que son de trequèize!

\StageDir{Avouì bièn de forse, Cesar sare lo tibo di gas.}

\Cesarspeaks \ldots chor ren\ldots

\StageDir{Cesar levve la bombola.}

\Cesarspeaks  Pe fose Rémy! La \textit{bombola} l’è vouida. 

\Remyspeaks Donque fa l'eumpleure!

\Cesarspeaks\direct{Ironique} Que drolo! Mi te l'eumplèi pa té, te l'atseutte dza plèin-a.

\Remyspeaks Ad\'on vou ba i LIDL\ldots

\Cesarspeaks Iaou?

\Remyspeaks I LIDL! Vendon-tì pa le \textit{bombole}?

\Cesarspeaks Mi soplé Rémy! Pitoù\ldots t'a lo téléfonne?

\Remyspeaks Ouè.

\Cesarspeaks Pren lo téléfonne é te sen mamma que l'è ba eun Veulla pe fé de commich\'on pe tanta; te lèi di de passì eun tchi Paolina aoutre a Gressan pe prendre an \textit{bombola} di gas.

\Helenespeaks L’an passoù avouì tanta Melanie l’ie to pi comoddo. Lleu pensave a totte é n'ayé pa de fastide.

\Cesarspeaks \direct{Eun trebelèn pe euncollé le cart\'on pe tèra}\\ Hélène soplé\ldots Tanta Melanie l’è platta a la coutse avouì eunna fivra da caval é fa no dibroillì. \direct{Eun s'avèitsèn a l'entor} É ara comèn apeuillo sise cart\'on pe tèra? Danchèn lo \textit{boogy-boogy} se le ficho pa. \direct{A Rémy} T’a de \textit{nastro} pe le secotse?

\StageDir{Rémy rep\'on i téléfonne é avouì eun jeste di a pappa de restì quèi.}

\Remyspeaks Ouè mamma, iaou t'i? Ad\'on te torne eun dérì é te va eun tchi Paolina. Te fa prendre eunna \textit{bombola} di gas. Véo de quilo? \direct{I pappa} Mamma dimande véo de quilo fa la prendre\ldots

\Cesarspeaks\direct{Ajitoù} Que fisa lleu, pe itsaoudé l'éve pe le trifolle! Que nen si mé\ldots trèi quilo!

\Remyspeaks \direct{Perplecse \contraria , todzor i téléfonne} Trèi quilo! Ah, senque nen fièn de trèi quilo? É ad\'on mamma, fé té!

\StageDir{Rémy cllou la téléfonaye.}

\Cesarspeaks\direct{A Rémy}	Acoutta Rémy t'a de \textit{nastro} que me fa fichì le cart\'on?

\Remyspeaks\direct{Ironique} N'i plen le secotse de \textit{nastro}! 

\Cesarspeaks \direct{A Hélène} Hélène té t'a de \textit{nastro}?

\Helenespeaks  \direct{Ironique} Ouè que nen n'i: lo vou-te de cot\'on , de velì, dzano, rodzo, trasparèn?

\Cesarspeaks\direct{Malechà} Hélène! Avouì tcheu le problème que n'en, te semble-tì lo cas de me prendre eun bala? Pitoù de diye de counte foule,  téléfonna a mamma que passèye i bazar pe atseté eunc\'o de \textit{nastro}.

\StageDir{Dimèn Rémy reumplèi d'éve la cachoula pe le trifolle. Hélène pouze lo caoutì \caouti, digoutaye \digouto\ se gave le gan, pren lo téléfonne é queurie mamma.}

\Helenespeaks Mamma que acro fé si travaille! Acoutta, pappa l'a fata de \textit{nastro}. \direct{Ver pappa} Mamma dimande comèn te lo vou\ldots

\Cesarspeaks\direct{Todzor pi inervoù} Mi que apeillise!

\Helenespeaks \direct{I téléfonne} L'a deu-me que apeillise\ldots si pa fi té.

\StageDir{Hélène beutte ba lo téléfonne.}

\Cesarspeaks Senque l'a deu?

\Helenespeaks\direct{Ironique} De alé te graté! Can mimo lèi pense lleu.

\Cesarspeaks A la plase de diye de counte foule, a queun poueun t'i avouì le trifolle?

\Helenespeaks  Mi mé n'i comenchà a plemì le premiye\ldots 

\Cesarspeaks Senque t'i eun tren de fé? Ti eun tren de t'apresté pe la fèira de sent'Or?

\Helenespeaks N'i comenchà mi si pa véo nen van!

\Cesarspeaks\direct{Dispéoù} Fa le fé perboliye! Fa maque le froté, n'ayò to euspleco-te, n'i finque icri-te to si lo beillette! 

\Helenespeaks Mi t'ayè deu-me té de fé pouèi.

\Cesarspeaks\direct{Comme douàn} Maleur! Trèi trifolle l'a plimo-me! De seu a eun momàn areuvve Hermann lo betchì di Tirol é si l'è mèitchà tédesque! Qui lo sen?! Qui lo arite pamì can comenche a braillì pequé acappe pa preste?!\direct{Ajitoù} Maleur le trifolle!

\StageDir{Cesar tsemin-e a drèite é a goste pe lo palque tanque can lèi veun eugn idoù.}

\Cesarspeaks\direct{A Hélène} T'a eun pappa trop euntellijàn! Te va ba i conjélateur, te lo ivre, a drèite n'a le trèi saque de la Bofrost\ldots te pren si salle é fièn alé salle!

\Helenespeaks Va bièn, mi pappa lo conjélateur iaou l'è?

\Cesarspeaks\direct{Dispéou} Hélène!\direct{Ver Rémy} Va té soplé!

\StageDir{Rémy chor eun gantèn la tita.}
 
\Helenespeaks  Ad\'on mé se n'i fenì pouì alé.

\Cesarspeaks\direct{Ironique} Ouè va maque te fé eun tor eun Veulla, mogà eun Plase Chanoux tanque aoutre a l'Arc, te te meudze eunna bon-a glase\ldots \direct{malechà} mi iaou te pense de alé Hélène?! T'a eunc\'o totte le carote rodze que te fa plimì! Maleur, mi avouì té perdo la pachense! T'ayè le trifolle da froté é le carote rodze da plimì. Doe bague!

\StageDir{Hélène, que l'ie presta a chotre, sbouffe é se avèitse a l'entor pe comprende sen que fé. César dimèn plache douàn lleu le casette di carote rodze.}

\Helenespeaks Senque? To so? Trèi casette?

\Cesarspeaks Ouè trèi casette!

\Helenespeaks To mé soletta?

\Cesarspeaks To té soletta é tsertsa de gavé le \textit{ridotte}.

\StageDir{I galoppe, tsardjà comme eun melette, entre Rémy avouì de grou saque de la Bofrost.}

\Cesarspeaks Oh brao Rémy! Tapèn ba salle trifolle.

\Remyspeaks Fa didzalé-le douàn!

\Cesarspeaks N'a pa lo ten!

\StageDir{Rémy é Cesar vouidzon le saque di trifolle de la Bofrost deun la cachoula.}

\StageDir{Se sen eunna:}

\effet{https://on.soundcloud.com/n5hsi}{Machina que postèdze}{}

\StageDir{Cesar é Rémy son tracachà.}

\Cesarspeaks Saré dza pa Hermann?

\Cesarspeaks N'i pouiye de ouè. 

\StageDir{Rémy tappe ià le saque di trifolle.}

\scene[-- Hermann Betchì]

\StageDir{Entre Hermann avouì eunna VALIGETTA iaou lèi son tcheu le caoutì de llou pe fe betsii. L'è arbeillà avouì le dra tipique di Tirol. Prèdze eun patoué avouì eun for acsàn tirolèis.}

\Hermannspeaks Bondzor a tcheut!

\Cesarspeaks Bondzor Hermann!

\Hermannspeaks Comèn l'è Cesar? Te vèyo eun plèin-a fourma!

\Cesarspeaks Ouè avèitsa: sen tcheut preste!

\Hermannspeaks Tot amoddo pe seuilla?

\Cesarspeaks Tot amoddo!

\Hermannspeaks \direct{I dou mèinoù} Sise dou \textit{poutinèn}\footnote{ Caque cou Hermann se eunvente de paolle pequé sa pa amoddo lo patoué.} qui son-tì?

\Cesarspeaks Son le dou botcha de mé\ldots Rémy é Hélène.

\Hermannspeaks N'ario cheur pa recougni-le. \direct{I dou mèinoù} Pensade de resté lé to rette comme vo fisa dzaloù ou alade me ditsardjì lo Volkswagen? 

\StageDir{Rémy é Hélène chorton i galoppe.}

\Hermannspeaks\direct{Ver Cesar} Donque, n'i sentì tanta Melanie; l'a deu-me que sit an no pouré pa no baillì eunna man pequé l'è maladda.

\Cesarspeaks Malerezemàn ouè.

\Hermannspeaks Pe si motif n'i i eugn idoù: n'i portoù avouì mé eugn amì é sa siaou. Te prézento Peter é Katrin. 

\StageDir{Entron Peter é Katrin. Le dou son arbeillà comme Hermann, avouì le dra tirolèis. Peter entre eun pourtèn eunna grousa casetta avouì dedeun de tseur. Eunc\'o leur dou prèdzon eun micllèn lo patoué avouì lo tirolèis. Peter sa pa tan bièn lo patoué; donque prèdze bièn plan.\\ Eunsemblo a leur, entre eunc\'o Héléne; s'achouatte é comenche a plimé le carote.}

\Peterspeaks\direct{Eun sarèn la man a Cesar} Bondzor!

\Cesarspeaks Bondzor!

\Peterspeaks Mé si Peter Schwartzer é veugno di Tirol. N'i traillà de-z-àn avouì Hermann é si bon a fé betsii.

\Cesarspeaks Spéèn.

\Peterspeaks N'i quetoù la fameuille eun Tirol é so me fé tchica plaoué.

\StageDir{Peter plaoue.}

\Cesarspeaks Mi na Peter!

\StageDir{Hermann gnoue a baillì de creppe si l'itseua de Peter pe lo consolé.}

\Hermannspeaks Soplé Peter, plaoua pa!

\StageDir{Peter se reprèn.}

\Peterspeaks  \ldots é n'i portoù ma siaou avouì mé.

\StageDir{Katrin prèdze avouì eunna vouése pouissante é \textit{autoritaire}.}

\Katrinspeaks\direct{Eun sarèn for la man de Cesar} Plèizì Katrin!

\Cesarspeaks\direct{Éton-où é ipovant\'o} Bondzor!

\Katrinspeaks Mé cougniso \textit{abastansa} bièn lo patoué péqué n'i travaillà dou-z-àn si eun Pila.

\Cesarspeaks  V'ouède cheur fé la \textit{gattista}!

\Katrinspeaks Na pisteur!

\Hermannspeaks\direct{Eun tsandzèn discoù} Ad\'on Cesar, comèn sen betoù seuilla?

\Cesarspeaks Caze amoddo\ldots

\Hermannspeaks\direct{Pa convencù} \ldots le trifolle ou, comme dièn no-z-atre eun Tirol, le \textit{kartoffen} son-tì dza frèide?

\Cesarspeaks Dzalaye!

\Katrinspeaks N'i sentì \textit{kartoffen}?

\Peterspeaks \textit{Kartoffen}!

\sound{https://www.youtube.com/watch?v=7eM6JdE10SE}{Danse tirolèise}

\StageDir{Hermann, Peter é Katrin danchon eunna danse tipique di Tirol. Can fenèi la mezeucca, Cesar l'a le man pe le pèi, dispéoù.}

\Cesarspeaks Mi senque l'è si charabàn?

\Hermannspeaks\direct{Avouì lo flo queur} Squeza-mé Cesar, mi no-z-atre no sen cotemoù can no sentèn seutta paolla\ldots

\Cesarspeaks Queunta?

\Hermannspeaks \textit{Kartoffen}!

\Katrinspeaks \textit{Kartoffen}?

\StageDir{Katrin s'apreste pe la danse, mi l'è to de chouite blocaye pe Peter é Hermann.}

\Hermannspeaks\direct{Eun aritèn Katrin} Na, na!\direct{Ver Cesar} Diavo\ldots no sen cotemoù de fé noutra danse\ldots can mimo, tornèn a no: comèn sen betoù seuilla? Le bouì son dza deun l'éve tsada?

\Cesarspeaks Maleur le bouì! Hermann squeza-mé mi le bouì n'i fran oublia-le.

\Hermannspeaks\direct{Malechà} Mi ad\'on comèn no fièn? Lo \textit{gadinen} saré dza rapoù é pouèn pa catchi-lo tourna dedeun lo bouat, la vatse l'è trèi dzor que l'è pendia é mé n'i pa to si ten a pédre avouì vo! 

\StageDir{Cesar sa pa sen que diye.}

\Hermannspeaks\direct{Ironique} No-z-allèn pamì reterì la vianda, dièn i dzi que son si \og Scherzi a parte\fg ?

\Cesarspeaks \direct{Ajitoù} Acouta\ldots me diplé\ldots renque pe eun per de bouì saré pa la feun di mondo!

\Hermannspeaks Se pou pa! Mèinoù a 50 an! T’isse eunterecha-te de pi l'an passoù, a la plase de maque avèitchì le-z-atre que travaillavon ou alé ià to lo dzor pe fi de commich\'on pe la tanta, mogà ara no sarian betoù mioù!

\Cesarspeaks Mi qui pensè que sit an tanta vegnè maladda?!

\Hermannspeaks\direct{Malechà, eun bouéchèn si la tabla} Maladda ou pa maladda, t'ayè jamì pensou-lèi que eun dzor ou l’atro te faré té prendre eun man la \textit{situachonèn}? 

\Cesarspeaks T'a bièn deu, eun dzor ou l'atro\ldots mé pensao a l'atro.

\StageDir{Entre Rémy.}

\Remyspeaks Lo Volkswagen l'è ditsardjà.

\Hermannspeaks\direct{A Rémy é Hélène, todzor malechà} \'E vo dou, de-z-àn é de-z-àn desì le livro a itedjì é ara v'ouite gnenca bon, diyo pa a chédre la vianda ou tsapotì lo gadeun ou betì de drogue, mi betì si eunna marmitta d’éve tsada! De la mima \textit{forsen} de voutro pappa!

\Hermannspeaks\direct{A Cesar} N'ayoù deu a Katrin é Peter de vin-ì eun Val d'Outa avouì mé, pouèi fiavon eunc\'o eun tor pe la vezeté\ldots mi ara na! Dèyon traillì to lo ten!

\Peterspeaks\direct{A Cesar}  Ouè pensao de fé tchica de vacanse! Vouillavo vère Corméyaou\ldots

\Cesarspeaks Ouè dzenta Corméyaou, fa renque fé tchica attench\'on i bèrio que rebatton\ldots mi piatro dzen.

\Peterspeaks Vouillavo vère Casin\'o\ldots

\Cesarspeaks Pa cheur de lo acapé iver sise dzor \ldots

\Peterspeaks Mi na, ara pouì pa é me fa travaillì pe fota de vo! Mi\ldots

\StageDir{Eun tédesque, Peter dimande eunna baga a Cesar que, naturellamente, comprèn pa. Peter, ad\'on, dimande a Katrin de tradouire pe llou.}

\Katrinspeaks Cesar, sen payà?

\Cesarspeaks\direct{Eun bleuffèn} Ouè, ouè\ldots Hermann vo payerè!

\StageDir{Peter sourì to contèn, se beutte lo faoudè é attaque a trafetchì desì la tabla.}

\Hermannspeaks Ad\'on comenchèn! Cesar\ldots mé n'i dza sèi\ldots beutta foua quetsouza a bèye.

\Cesarspeaks Amoddo! Vi que l'è dza belle nou aoue beutto si eunna cafetchiye!

\Hermannspeaks\direct{Ofenchà} Nou aoue!Varda leunna! No-z-atre lo cafì a nou aoue lo béyèn pa! Sen lévou-no a sinqu'aoue di \textit{matin}, n'en prèi la brétella pe Trento é n'en fi didj\'on bièn devàn que té! Ara beutta maque foua eun tsequet de Vodka ou de Martini blan.

\Peterspeaks \direct{A Hermann} Vodka? Pouèi euntanèn lo Speck que n'en portoù.

\Hermannspeaks \textit{Ià}!

\StageDir{Todzor eun prédzèn tédesque, Peter dimande a Cesar iaou l'è lo ben.}

\Cesarspeaks\direct{A Rémy} Mi mé sit lo comprègno pa, mi senque l'è? Albanèis?

\StageDir{Peter, eun mimèn que dèi pichì, fé la mima dimanda a Hermann, lequel, a jeste é todzor eun tédesque, lèi moutre iaou l'è lo ben. Peter chor.}

\Cesarspeaks Hermann! Mi pequé t'a mando-lo eun \textit{caldaia}?

\StageDir{Peter torne eun dérì. Hermann lèi moutre eugn atra rotta.}

\Cesarspeaks Ah l'a fata de alé i ben!

\Hermannspeaks \textit{Ià}, \textit{pissing}!

\Cesarspeaks  Ah n'i comprèi, \textit{pissing} vou deu pichì!

\scene[-- Te me baille eunna man?]

\StageDir{Areuvve Lucie tsardjaye comme eun melette: bourse de la spèiza, paquet, saquet é eunc\'o eunna bombola di gas.}

\Luciespeaks  Caqueun l’at-ì lo ten é la voya de m’èidjì? 

\Hermannspeaks Lucie!

\StageDir{Rémy galoppe ver sa mamma, lèi pren la bombola é la plache pe alemì lo gas. Eunc\'o Cesar s'aprotse a Lucie pe lèi prende le bourse.}

\Cesarspeaks Que dzen te veure Lucie! T'a to prèi? Lo \textit{nastro}?

\Luciespeaks Ouè l'è to lé.

\StageDir{Cesar teurie foua lo \textit{nastro} é comenche a fichì le cart\'on pe tèra.}

\Luciespeaks\direct{ROMANTICA, ver Hermann} Oh Hermy!

\Hermannspeaks Lucie que dzen te veure! Comèn t'i \textit{dzenten}!

\StageDir{Hermann é Lucie s'eumbrachon.}

\Luciespeaks Comèn l'è? Te t'arite eunc\'o sit an caque dzor eun pi? 

\Hermannspeaks Ouè trèi ou catro dzor eun pi\ldots \direct{malisieu, a vouése basa} mogà eunna souaré a lambo no la spountèn mé é té!

\StageDir{Lucie s'avèitse a l'entor.}

\Luciespeaks Mi seuilla l'è pa eunc\'o preste ren! \direct{Malechaye} L'è eunc\'o to pe l'er!

\Helenespeaks\direct{A Lucie}  Mamma, l'è pa que te pourie baillì té eun vèyo de Vodka a Hermann? Mé si pa iaou l'è.

\Luciespeaks Ouè bondzor Hélène! Féo mé. 

\StageDir{Lucie apreste eun vèyo de \textit{Vodka} pe Hermann. Dimèn, Cesar se euntsambotte avouì lo \textit{nastro}.}

\Remyspeaks  \direct{A Lucie} Mamma, aprì te me baille eunna man, soplé?

\Luciespeaks Ouè areuvvo.

\Cesarspeaks Lucie! N'ario eunc\'o fata mé d'eunna man seuilla avouì le cart\'on.

\Helenespeaks  Ouè Mamma, eunc\'o mé avouì le carote rodze, n'i fata.

\Luciespeaks Mondjeu!\direct{Ironique} Que dzen vo reveure. N’i finque faì tchoué lo téléfonne pequé nen pouò pamì de vo sentì!

\StageDir{Lucie soum\'on la \textit{Vodka} a Hermann.}

\Luciespeaks\direct{A Hermann} Té vèi de pa fé comme l'an pas\'o, que t'ayè tellamente bi que t'a drogoù trèi cou le saouseuse é say\'on eumpecable!

\Hermannspeaks\direct{Fier} Seutta l’a fran ramplachà tanta Melanie!

\Remyspeaks\direct{Dispéoù} Mamma! Te me baille eunna man?

\Luciespeaks Areuvvo!

\Helenespeaks  Mamma! Can t'a fenì veun seuilla avouì seutte carotte rodze!

\Hermannspeaks\direct{Eun desuèn tcheutte} Ouè, Lucie,  mogà can t’a fenì avouì sise trèi inutilo te veun me èidjì a chédre la vianda, a tsacotì é betì le \textit{droguen}? Aprì a modì, eumbouélì é mogà eunc\'o fé sétchì la vianda?

\Luciespeaks\direct{Inervaye} Acoutade! Acoutade-mé tcheutte, mé me fa alé veure tanta Melanie! L'è maladda é l'a fata di medeseun-e\ldots vo v'ouite eunna blita de dzi é donque  dibroillade-v\'o tchica pe inque!

\StageDir{Lucie chor eun mandèn tcheutte a caqué.}

\scene[-- Se comenche!]

\Hermannspeaks\direct{A Cesar} Té t'a fenì avouì si Scotch é sise cart\'on?

\Cesarspeaks Ouè, que traille! Oh \textit{flot}\footnote{ Lo significà de seutta drola espréch\'on l'è décrì deun la séch\'on Queriaouzitoù.}!

\Hermannspeaks Bon, ara caqueun que me baillise eunna man pe ditrafetchì le bouì!

\StageDir{Silanse. Gneun vou se propozé.}

\Hermannspeaks\direct{Eun s'avèitsèn a l'entor} Caqueun?

\Remyspeaks  Hélène! Mé dèyo didzalé le trifolle! 

\Helenespeaks  Mi l'è tro pouè si travaille pe mé.

\Hermannspeaks\direct{Malechà} Va bièn, restade maque \textit{tranquillen}! Sit an le saouseuse, le boudeun é le salàn le beutterèn deun le \textit{vaschette salva freschezzen}!

\Cesarspeaks N’i to comprèi, Hermann t'èidzo mé!

\StageDir{Cesar pren lo sidel di flotte é nen pouze eun per si la tabla. Dimèn Rémy soum\'on a Hermann eugn atro vèyo de Vodka.}

\Remyspeaks\direct{A Hermann} T'i trop inervoù! 

\Hermannspeaks Penso beun!

\Remyspeaks Bèi eunc\'o eunna gotta\ldots é iaou l'è la botèille?

\StageDir{Rémy pren la botèille de Vodka que l'ie deun la crédense.}

\Cesarspeaks\direct{A Rémy} Vouedza-lèi eunc\'o eun vèyo i pouo Hermann!

\Hermannspeaks Ouè pequé oueu saré eunna londze dzornaye.

\StageDir{Rémy reumplèi lo vèyo de Hermann. Hermann lo bèi ba a qui blan.}

\Remyspeaks \direct{Iton-où} Ba de Vodka choueudza!

\Hermannspeaks \textit{Ià}, son de bague de noutra téra.

\StageDir{Rémy li l'ETICHETTA de la botèille.}

\Hermannspeaks Distillach\'on de trifolle ou comme diade vo \textit{kartoffen}.

\Katrinspeaks\direct{Eun braillèn} \textit{Kartoffen}!

\Cesarspeaks Na!

\StageDir{Comme douàn partèi la mezeucca tirolèise. Peter, que l'ie eunc\'o i ben, areuvve a galoppe eun se terièn si le pantal\'on . Katrin, Hermann é Peter atacon la leur danse é teurion dedeun eunc\'o Rémy é Hélène. Le dou frée, pi que danchì, tchappon maque de patèle di momàn que san pa fé eun pa djeusto.}

\StageDir{Can fenèi la mezeucca, tcheutte tournon i leur poste.}

\Hermannspeaks\direct{A Cesar} Squeza-n\'o!

\Cesarspeaks Si sensa paolle.

\Peterspeaks\direct{A Cesar} Vou djeusto terì l’\textit{l'éven}!

\StageDir{Peter chor é entreré to de chouite aprì.}

\Luciespeaks\direct{Dérì le rid\'o, eun braillèn} Na tanta! Tanta Melanie! Na!

\Hermannspeaks Mi senque l'è \textit{capitounen}? Cesar, l'è pa que t'a quetoù la lemie aviaye i corrid\'o?

\Cesarspeaks Crèyo pa. Rémy va veure té, mé si seu avouì seutte flotte.

\Remyspeaks Hélène, va té, mé n'i le trifolle.

\Helenespeaks Na, avouì totte le dzi que n'a séilla, fran mé?

\Cesarspeaks\direct{Stouffie} Maleur! Vo dou v'ouèide praou stoufia-me oueu! Vou mé veure!

\StageDir{Cesar chor.}

\Hermannspeaks\direct{A Peter} Senque saré capitoù?

\Peterspeaks Si pa.

\Hermannspeaks Mistère\ldots \direct{eun se térièn si le mandze} alé! Fa pa pédre de ten. Ara que t'a pichà, te pense de me baillì eunna man?

\Peterspeaks \textit{Ià}!

\Hermannspeaks Ad\'on aprestèn le \textit{droguen} é beten-lé deun la më.

\StageDir{Hermann é Peter atacon a drogué la vianda eun tsantèn eunna tsans\'on tirolèiza.}

\StageDir{Entre Cesar.}

\Cesarspeaks\direct{Sèrieu} Hélène, Rémy. L'è mioù que vegnisa eunc\'o vo avouì mé.

\Remyspeaks Ouè va bièn.

\StageDir{Hélène é Rémy chorton avouì lo pappa. Hermann é Peter recomenchon a tsanté é a traillì. Katrin l'è todzor eun tren de plimì le carote.}

\scene[-- Tan pe diye]

\StageDir{\'Epouvant\'o di-z-eurlo de Lucie, entron Battista é Germene.}

\Germenespeaks Ouè mi senque l’è-tì capitoù inque? De braillo semblablo. 

\Hermannspeaks L'ie Lucie que criave \textit{tanten}, \textit{tanten}, \textit{tanten}! Ara le trèi inutilo son aloù vère sen que l'è capitoù.

\Battistaspeaks  \ldots é mé que n’ayoù la fèi que fusse itoù lo gadeun a querì pequé l’ie panco mor!

\StageDir{Battista ri to solet pe sa battia.}

\Germenespeaks N’are-tì praou a riye? Son pa de bague di-z-ommo, l'è mioù que aliso mé si veure sen que l’è capitoù. Té, Battista, fé attench\'on! Se te torne i mitcho plen comme eunna botalla comme l'an pas\'o, te drime foua avouì lo tseun! 

\StageDir{Germene chor.}

\Battistaspeaks  Sarèn pa magà de bague di-z-ommo mi pe mé si cou tanta Melanie l’a belle fenì de pequì de betsii! 

\StageDir{Battista ri tourna to solet pe sa battia. Peter fé seumblàn de riye.}

\Battistaspeaks  \direct{A Hermann, queriaou}Pèi té te va d’eun coutì a l’atro pe fie la betsii pe le-z-eun é le-z-atre?

\Hermannspeaks \direct{Inervoù, avouì lo caoutì eun man} Ouè perqué se pou pa?

\Battistaspeaks\direct{Eun se squezèn}  Na, na, diavo maque pèi, tan pe diye.

\Hermannspeaks Ah bon!

\Battistaspeaks  \direct{Todzor queriaou} \ldots é véo de dzor te beutte-tì pe fie la betsii d’eun gadeun de 150 quilo é eunna vatse de 25 meuria de cartì?

\Hermannspeaks Depèn de eunna matse de bague. Pe fé eugn izeumplo: véyo de dzi me baillon eunna man é \direct{eun avèitsèn Battista} é véyo de dzi me queutton pa travaillì! Seuilla n'aré pe trèi ou catro dzor.

\Battistaspeaks  Catro dzor?

\Hermannspeaks\direct{Eun sarèn lo caoutì} Ouè, perqué se pou pa?

\Battistaspeaks\direct{Eun se squezèn}  Na, na, diavo maque pèi, tan pe diye.

\Hermannspeaks Ah, bon!

\Battistaspeaks\direct{Todzor a Hermann}\ldots é té t’avèitse-tì la leunna pe fé betsii?

\Hermannspeaks Bièn cheur que l’avèitso. Leunna diya, fret defoua é tsa dedeun iaou se traille la vianda. 

\Battistaspeaks Ah, leunna diya é pa tendro de leunna?

\Hermannspeaks \direct{Comme douàn} Ouè, leunna diya, pa tendro de leunna! Perqué se pou pa?

\Battistaspeaks\direct{Comme douàn}  Na, na, diavo maque pèi, tan pe diye.

\Hermannspeaks Ah, bon!

\StageDir{Battista se trame ver Katrin.}

\Battistaspeaks\direct{Eun sarèn la man a Katrin} Plèizì, Battista.

\Katrinspeaks\direct{Avouì forse} Mé si Katrin!

\Battistaspeaks  V'ouite la fenna de Hermann?

\Katrinspeaks  Na!

\Battistaspeaks  Bièn! Pequé mé cougniso de salle bague\ldots \direct{eun moutrèn Hermann} sitte l’è eun petan-et!

\Peterspeaks\direct{Eun rièn} \textit{Ià, peutaetten}!

\scene[-- R.I.P. Tanta]

\StageDir{Entre Cesar, chouivì pe Rémy é Hélène.}

\Cesarspeaks\direct{Triste, vouése basa} Squezade\ldots Tanta\ldots Tanta l’a quetou-no.

\Hermannspeaks Mi \textit{nein}!

\StageDir{Hermann fé lo signe de la creu é eumbrache Cesar. Peter, Katrin é Battista se beutton eun feulla é, eun aprì l'atro, prézenton leur condoillanse a totta la fameuille. Katrin sare le man é bèije le dzi de fas\'on bièn grouchiya.}

\Battistaspeaks\direct{Eun rontèn lo silanse}  L’ayè dza 80 an! A sit éyadzo t’a dza eun pià dedeun la quése.

\StageDir{Battista ri to solet pe sa battia. Peter fé seumblàn de riye.}

\Hermannspeaks\direct{A Battista} Squeza-mé, mi de queun an t'i té?

\Battistaspeaks Mé si di 32! Classe de Cazemì, Giacco, Filoméne\ldots

\Hermannspeaks Donque vio de-z-àn t’a?

\Battistaspeaks  82 a juliet, pequé se pou pa?

\Hermannspeaks Na, diavo maque pèi, tan pe diye.

\Battistaspeaks Ah bon.

\Cesarspeaks\direct{A Hermann, bièn ajit\'o} Squeza se te direndzo, mi comme te vèi, pe sen que l'è capit\'o\ldots no n'en prédjà eun fameuille é n'arian pens\'o de plantì lé de fé betsii pe sit an.

\StageDir{Hermann fé tsire lo caoutì é avouì la man boueuche for desì la tabla.} 

\Hermannspeaks \textit{Nein}, \textit{nein}, \textit{nein}, \textit{nein}, \textit{nein}!

\Cesarspeaks\direct{\'Epouvant\'o} \textit{Nein}?

\Hermannspeaks\direct{Tourna eun bouéchèn desì la tabla} \textit{Nein}! Pouèn pa quetì la \textit{betsinen}! N'i \textit{deutten} que lo ten l'è pi que adatto, la vatse l'è dza pendia, lo gadeun l'è rapoù é no-z-atre n'en pa to si ten de restì seuilla a fé ren. Eun pi, se tanta Melanie fise eunc\'o eun viya, sariye contenta de veure que totta sa fameuille porte eun devàn sa betsii! \direct{Eun levèn la vouése} A-te comprèi?

\Cesarspeaks\direct{Fase} Djeusto! L'è sen que diavo douàn i meun dou gars\'on! Mi l'an pa voulì m'acouté!

\StageDir{Rémy é Hélène avèitson mal Cesar, lequel, sensa se fé sentì pe Hermann, se squize. Cesar pourte le dou botcha de llou louèn de Hermann é avouì leur tsertse de rezoudre la situach\'on.}

\Cesarspeaks\direct{I seun botcha} Ad\'on v'ouèide sentì Hermann\ldots fièn eunna baga, tsertsèn de no dibroillì avouì la betsii i mentèn di pià. Donque, Hélène\ldots té te fé eun sate tanque ba di prée, te pense eunc\'o i fleur, te sen la tsantii é le pompì. Rémy\ldots té te fé eun sate ba eun Veulla é te tsertse eunna quése.

\StageDir{A Rémy veun eugn idoù é comenche a tsertchì quetsouza pe lo mitcho.}

\Helenespeaks Va bièn, mi pappa\ldots pe avijì le dzi di veladzo comèn fièn?

\Cesarspeaks   Sen lèi pense praou Germene \direct{avouì la man imite lo bec d'eun \textit{canar}} can l'a fenì si dameun avouì tanta!

\Battistaspeaks\direct{A Hélène} Va mai tranquila! Lèi pense lleu!

\StageDir{Hélène chor.}

\scene[-- Quése.com]

\StageDir{Rémy teurie foua di téèn eugn \textit{iPad}.}

\Remyspeaks  Pappa! Avèitsa\ldots

\StageDir{Rémy moutre a Cesar sen que l'a acapoù desì l'\textit{iPad}.}

\Remyspeaks \ldots n’a pa fata de alé ba 
eun Veulla pe comandé la quése.

\Cesarspeaks Ah na?

\Remyspeaks I dzor de oueu, se fé to si Internet! 

\Battistaspeaks\direct{Savèn}  Mé son de-z-àn que eumplèyo Internet!

\Cesarspeaks\direct{\'Eton-où} Té?

\Cesarspeaks Djèique! Se no Prézidàn sen pa eumpléì Internet, l’è mioù que plantisan lé de fé le Prézidàn!

\Hermannspeaks Mi Prézidàn de senque té?

\Battistaspeaks  Prézidàn di groupe di-z-alpeun!

\StageDir{Avouì eun jeste Hermann mande a caqué Battista.}

\Cesarspeaks Alé Rémy, fé-no vère, n'en pa to lo dzor! 

\StageDir{Rémy pouze l'\textit{iPad} desì la tabla é tcheu le-z-atre se plachon dérì llou pe vère sen que fé. Battista alondze chouèn le man pe totchì l'\textit{iPad}.}

\Remyspeaks\direct{Eun icrièn si l'\textit{iPad}} Donque\ldots Google\ldots quése di mor\ldots

\Cesarspeaks\direct{Surprì} Ah, fran pai?

\Remyspeaks Ouè. \direct{Eun baillèn eunna patèla a Battista é Peter} Soplé totsa pa! Totsa pa!

\Hermannspeaks Mi senque l'è salla dzenta fenna patania?

\Remyspeaks Na sen l'è de poublisit\'o.

\Hermannspeaks Mi l'è fran dzenta, semblè eunna di Tirol!

\Remyspeaks Soplé, avèitsèn le quése!

\Cesarspeaks Seutta l'è pa mal, mi se comprèn pa amoddo de sen que l'è féta\ldots é aprì l'è pa marc\'o lo pri, vourio pa que coutisa trop.

\Battistaspeaks\direct{Malechà}  T’aré pa de cou voya de beuté tanta dedeun catro lan de beublo?

\Hermannspeaks Si cou l’a rèiz\'on Battista.

\Battistaspeaks Ouè, avouì to sen que lèi queutte!

\StageDir{Battista ri tourna to solet pe sa battia. Peter fé todzor seumblàn de riye.}

\Cesarspeaks Ad\'on avèitsén-nèn eugn atra eunna mia pi dzenta\ldots

\StageDir{Rémy conteneuvve a moutré le diffiente quése, todzor eunfastedjà pe Battista que molle pa de totchì l'\textit{écran}.}

\Hermannspeaks Pren seutta! Avèitsa queunta dzenta couleur que l'at.

\Battistaspeaks \ldots é seutta que l’at eunc\'o finque lo tsapì di-z-alpeun.

\Cesarspeaks\direct{Eun perdèn la pachense} Maleur, soplé! Me fiade vin-ì matte! Sarè-tì eumpourtanta la couleur ou lo tsapì di-z-alpeun ara? Rémy veun avouì mé eun momàn.

\StageDir{Rémy é Cesar se tramon pe an coueugne.}

\Cesarspeaks\direct{Eun moutrèn eunna quése desì l'\textit{iPad}} Fièn que prendre seutta, que l'è gnenca tan tchiya. Pren salla lé.

\Remyspeaks  L'è eunc\'o eun bon pri\ldots l'è tchica queurta\ldots mi pe tanta va cheur amoddo.

\Battistaspeaks  La pléyèn se reste pa.

\StageDir{Battista tourne riye to solet. Peter lo desue.}

\scene[-- \textit{Sainte} betsii]

\StageDir{Entre lo préye.}

\Preyespeaks L’è-tì permì? 

\Cesarspeaks  Bondzor Monseur! Mi diade-mé, pequé v'ouite entr\'o pe la porta de servicho? Seu l'è to eun pastise, sen eun tren de fé betsii. Pouavade pa entré pe lo port\'on prensipal?

\Preyespeaks Tracachade-v\'o pa. Totte le pourte porton i paradì!

\Cesarspeaks\direct{Superstisieu} Oh maleur! Can mimo, vegnade\ldots

\Preyespeaks\direct{Eun gneflèn} Mi squezade, v'ouèide deu-me que v'ouite eun tren de fi betsii\ldots mi sento pa de flou\ldots

\Hermannspeaks\direct{Malechà} Pe forse sentade pa de flou! Avouì de dzi pai inutilo é eunna morta seu dedeun, penso beun que sentisade pa de flou.

\Preyespeaks Damadzo, n’ario tan l'amoù gouté salla pata. Sade\ldots son de-z-àn que fio pamì de bestii.

\Battistaspeaks  Ouè, son de-z-àn que sitte fé pamì betsii, mi nen peuque pi de no! 

\StageDir{Battista tourne riye to solet, avouì Peter que lo desue.}

\Preyespeaks\direct{Ofenchà} Todzor seumpateuco Battista.

\Cesarspeaks  Monseur vegnade, n’aré pi eugn atro momàn pe fé la counta é gouté le saouseuse. Vo porto si eun tchi tanta\ldots

\StageDir{Cesar s'aprotse ver la pourta, mi lo préye s'arite protso de la cachoula di trifolle.  Rémy lèi nen fé gouté eunna.}

\Preyespeaks Na son eunc\'o criye!

\Cesarspeaks Monseur soplé! Vegnade! Ou mioù\ldots Battista! Porta si té Monseur si dameun, iaou n'a tanta.

\Battistaspeaks Va bièn, vegnade Monseur!

\Cesarspeaks Oh! Si arevoù a nen fé foua dou deun creppe solette!

\StageDir{Battista é Monseur chorton.}

\ridocliou

\scene[-- La më? ]

\StageDir{\Fv{Son pasoù dou dzor. Totta la fameuille l'è eun tren de fenì la betsii. Fa eunc\'o eumbouélé é gropé la dériye tsèin-a de boudeun. Oueu l'è lo dzor de la sepolteua}}

\ridoiver

\StageDir{Eun \textit{scène} n'at Hermann, Peter, Cesar é Rémy. Dérì se vèi la toupie de saouseuse é boudeun pendiye a dou cavalet. Cesar é Rémy son eun tren de eumbouélé lo dérì bouì. Hermann poulite la më.}

\Cesarspeaks Queunt’aoua l’è-tì?

\Hermannspeaks Son dza belle nou aoue.

\Cesarspeaks\direct{Ajitoù} Nou aoue? Mi ad\'on no fa no boudjì! Oueu n'at eunc\'o la sepolteua é fa to dibarachì.

\StageDir{Peter ataque a seblé eunna tsans\'on amuzante pe pendre le saouseuse. Can areuvve protso di meur tsertse de la pendre eun saoutèn. Malerezamente l'è tro base pe lèi arrevé. Rémy, que l'a mioulo-lo, lèi baille eunna man: lo pren é lo levve si de fas\'on que pouise pendre la tsèin-a.}

\effet{https://soundcloud.com/user-234168361/telefono}{Téléfonne}

\Remyspeaks Pappa avèitsa té, mogà l'è lo teun.

\StageDir{Cesar tsertse deun le secotse. L'è pa lo téléfonne de llou que soun-e. Rémy, comme douàn, levve si Peter pe pendre le saouseuse.}

\effet{https://soundcloud.com/user-234168361/telefono}{Téléfonne}[false]

\Cesarspeaks\direct{Malechà} Ad\'on si téléfonne!

\Remyspeaks  Squezade l'è lo meun!

\StageDir{Rémy rep\'on i téléfonne.}

\Remyspeaks Ouè\ldots ouè l'è seuilla\ldots \direct{a Cesar} pappa, l'è pe té.

\Cesarspeaks  N'i pa lo ten Rémy! Si eun tren de gropé.

\Remyspeaks\direct{I téléfonne} Na squezade mi l'è djeusto chortì\ldots ah l'è eumpourtàn?

\StageDir{Silanse. Rémy sa pa senque diye.}

\Remyspeaks Que chanse! L'è tornoù ara. Vo lo paso.

\StageDir{Cesar pren lo téléfonne.}

\Cesarspeaks  Ouè\ldots ouè si mé Cesar\ldots \direct{triste} malerezamente tanta Melanie l'a quetou-no. Ouè djeusto, oueu n'a la sepolteua, a trèi-z-aoue. Ah la quése\ldots ouè n'en prèi-la si \textit{internet}! Diade-me maque\ldots la quése\ldots senque? La fèi n'i pa comprèi amoddo\ldots \direct{eun lévèn la vouèse} senque?! Maleur!

\StageDir{Cesar tappe lo téléfonne ver lo pebleuque.}

\Cesarspeaks\direct{Rodzo comme eun pèivr\'on}  Rémy! Te sa qui l'ie?

\Remyspeaks Sise de la quése.

\Cesarspeaks\direct{Todzor malechà}  Ouè fran leur! Te sa senque l'an deu-me?

\Remyspeaks L'an bèichà lo pri?

\Cesarspeaks Ouè, l'an bèicha-lo totte. No porton pamì la quése! N'at i eun problème avouì \textit{internet}!

\StageDir{Tcheutte se bloccon, paralizoù. Rémy l'a le man pe le pèi.}

\Cesarspeaks Ara senque fièn que oueu n'en la sepolteua? Te me lo di té?

\Hermannspeaks \textit{Nein}! \textit{Nein quésen}!

\Cesarspeaks\direct{Dispéoù} Eh \textit{nein}\ldots na!

\Hermannspeaks Ara comèn fièn?

\Cesarspeaks Si pa\ldots

\StageDir{ Tcheutte san pa senque fé. Hermann tsemie a l'entor de la tabla.  To d'eun creppe s'arite douàn la më. L'avèitse.}

\Hermannspeaks Cesar\ldots n'en pa de quése\ldots

\StageDir{Cesar rep\'on pa. Hermann, protso de la më, avouì la man pren le mezeue de la londjaou.}

\Hermannspeaks\direct{Comme douàn} Pa de quése\ldots Cesar!

\StageDir{Cesar avèitse Hermann é gnouye a douté.}

\Cesarspeaks\direct{A Hermann} Spéo que t'ise pa eun tren de pensé sen que penso\ldots

\Hermannspeaks \direct{Eun lévèn la vouése} Té t'a eugn idoù meillaouza?

\Cesarspeaks Ara comme ara na.

\Hermannspeaks \ldots donque?

\StageDir{Peter é Rémy son derì la tabla chocoù.}

\Cesarspeaks  Eugn éffé\ldots

\StageDir{Hermann é Cesar s'avèitson. Eunsemblo levvon la më; se la tsardzon si le-z-ipale é chorton.}

\ridocliou

\DeriLeRido

\RoleNoms{Avanspettaclle}{Jo\"elle Bollon, Laurent Chuc}
\RoleNoms{\`Eidzo réjì}{Flavio Albaney}
\RoleNoms{Tramamoublo}{Simone Roveyaz}

\end{drama}
