\chapter*{Préfach\'on} 
L’aruye po tcheu le dzor, mimo se vo trailledde i Guetsè leungueusteucco, d’eunvian-é eun voyadzo  pai lon é stimulàn a traé d’eun patouè. Patouè d’an quemeun-a mi, euncó de pi, patouè d’an comunitó, euntenduya comme esprèchòn vivanta de eungn eunsemblo de dzi que se recougnison a l’euntor d’eun projè, d’an lenva é d’eun téritouéo. Vouè, péquè le Digourdì son po mocque de brao-z-atteur, son avàn to de dzoun-io Tsarvensolèn, que l’an lo plèizì de pourté su lo palque leur vécù é leur éritadzo, atò l’entouziasme, la frècheur é la voya d’amuzé eun s’amuzèn que l’apartchàn a leur éyadzo.
 
Bièn cheur tsicque jénérachòn eunterprette a sa modda l’éritadzo quemeun. É l’è fran a traé de heutta eunterprétachòn todzor nouila que lo pasó se fa prézèn - attuel é vivàn - é pouze le fondemèn pe l’avin-ì.  Pouchà d’an curiozitó é de eungn’estraordinéa voya de se beté eun djouà, le Digourdì l’an reléó tcheu le défì que lèi son ihó propouzó, eun parcourèn d’an magnî trasversalla de nombreu janre arteusteucco : de la réalizachòn de queurmétradzo eun patouè comme La reconstitution é Là-bas au front, i spettacle final de la 63a édichòn di Concours Cerlogne, eun pasèn pe l’euvra téatralla L’homme au cœur valdôtain dédiéye a la memouî d’Émile Chanoux a l’occajòn di 50o aniverséo de sa mor. Mi étò la produchòn muzicalla, fétte de reprèize de boucòn fameu é de compozichòn orijinelle, é la réalizachòn de nombreu clip vidéó. Le Digourdì prèdzon eun langadzo moderno, frique é ironeucco é lo partadzon su le preunsipal rézó sosial di dzor de vouì : Instagramm, Facebook, YouTube, TikTok, mi étò su de plattefourme muzicalle comme Spotify é Apple Music. 

La Valoda d’Ouha l’areu po pousù espéré de-z-ambasadeur mèilloi di patouè ! 
Pe levré, dz’ouì souligné eungn otro particulié que fa oneur i Digourdì é que témouagne de la maturitó d’an compagnì ioi que l’éyadzo mouayèn di-z-atteur pase po le 25 an : le Digourdì s’eungadzon étò dedeun la formachòn di patouazàn de demàn. Dze penso a l’organizachòn d’ateillì de téatro pe le mèinoù é a la consecanta créachòn, a l’euntèrieur de la compagnì, di grouppe di Pégno Digourdì de Tsarvensoù que l’a débutó su lo palque di Splendor l’an 2023.
Voualà donque eun dzen témouagnadzo de la vitalitó de nouha queulteua é eun motif d’espouer pe son avin-ì.

La publicachòn de hi livro, l’è eungn otro boucòn que se djoueun a an produchòn dza remarcobla. La révijòn di teste l’a demandó an confrontachòn continuella avouì Jordy, é dze crèyo que l’espérianse siye ihéye eunrechissanta pe tcheu dou. Pe mè avàn to que, a traé de l’étsandzo, n’i pousù vivre di dedeun la jénèze di pyihe, eun m’obledzèn a me confronté avouì eun langadzo que évoluye dézò l’eumpulchòn di tsandzemèn que tsicque jénérachòn que lo prèdze pourte avouì sè ; mi, dze crèyo, eunrechissanta étò pe l’oteur, que dz’i vu crèihe di poueun de vuya de l’atenchòn a la lenva é de la métrize de la grafiya.
Eun parcourèn le dji pyihe, lo lèiteur manquerè po d’aprésié l’évoluchòn arteusteucca de la compagnì. Di premî compozichòn que s’ensérèison dedeun la tradichòn populéa pi classeucca canque i teste pi métatéatral de la dérî produchòn.
Eungn ouvradzo complè, an sorta d’antolojì multimédialla di premî dji-z-àn d’an compagnì nèisuya caze pe riye é que avouì lo ten s’et eumpouzéye comme eunna di réalitó pi dinameucque, novatrise é complette di panoramà téatral valdotèn. 

Eun reprègnèn le mo de l’oteur : eun jeste d’amour.
\newline
\newline
\hspace*{\fill} \textit{Daniel Fusinaz}
