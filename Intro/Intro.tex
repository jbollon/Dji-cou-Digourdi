\chapter*{Introduction} 

%IDEA: APPENDICE CON STATISTISCHE: parole totali, nome più frequente, verbo più frequente ecc.... parola più lunga...; totale e per pièce

% DA QUALCHE PARTE SCRIVERE CHE per i suoni e le musiche c'è un link su soundcloud per trovare ciò che abbiamo creato o salvato. Le canzoni o le sigle non ci sono se non sono state create da noi. Si possono trovare su yt.

% DA QUALCHE PARTE SCRIVERE:  non sono state inserite le indicazione di scena in merito agli ingressi/uscite da dx/sx, se non strettamente necessario... MOtivo: per non appesantire la lettura del racconto con note tecniche di regia.

\paragraph*{Pourquoi \textit{Dji Cou Digourdì}}
\textit{Dji Cou Digourdì} est né de la nécessité de rassembler toutes les pièces de théâtre interprétées par la compagnie Le Digourdì de Tsarvensoù à l'occasion du Printemps Thé\^atral. Presque toujours, les dernières versions des textes sont modifiées à la main lors des dernières répétitions, période durant laquelle, pour une étrange raison, naissent les répliques et les idées les plus irrévérencieuses et efficaces. Par conséquent, au fil des ans, les textes théâtraux, archivés quelque part dans les ordinateurs des acteurs, sont restés figés sur une ancienne version du scénario, peu fidèle à la représentation jouée en direct sur scène.

De plus, les pièces des Digourdì ont souvent intégré au scénario des contenus multimédias, tels que des audios, des bruitages, des vidéos et des images. Ainsi, outre le désir de témoigner fidèlement des pièces interprétées sur la scène du Printemps Thé\^atral, \textit{Dji Cou Digourdì} est également un archivage des contenus multimédias produits par la compagnie théâtrale \textit{tsarvensolentse}.

Bref, \textit{Dji Cou Digourdì} se présente aux lecteurs comme un témoignage et un catalogue de l'art théâtral des Digourdì.

\paragraph*{Le contenu}
\textit{Dji Cou Digourdì} rassemble les dix premières pièces interprétées par la compagnie Le Digourdì de Tsarvensoù à l'occasion du Printemps Thé\^atral de 2009 à 2019\footnote{ Au total, cela fait 11 ans ; cependant, en 2017, Les Digourdì n'ont pas participé au Printemps Thé\^atral. La raison de cette absence est expliquée dans le chapitre dédié à l'année 2017.}. Chaque pièce correspond à un chapitre et s'articule de la manière suivante :
\begin{itemize}
\item[$\bullet$] Couverture de la pièce avec titre, auteurs, lieu et date de la représentation ;
\item[$\bullet$] Photo avec la liste des acteurs et un QR Code pour visionner la vidéo intégrale de la pièce ;
\item[$\bullet$] Entretien avec un acteur comprenant un souvenir personnel et quelques curiosités ou anecdotes liées à la pièce ;
\item[$\bullet$] Description de la scénographie et des principaux accessoires de scène ;
\item[$\bullet$] Liste et description des personnages (par ordre d'apparition) avec l'indication de l'acteur qui l'interprète ;
\item[$\bullet$] Éventuel avant-spectacle ;
\item[$\bullet$] Texte théâtral ;
\item[$\bullet$] Liste des collaborateurs qui ont travaillé en coulisses.
\end{itemize}

\paragraph*{L'habillage}
C'est avec plaisir que je commente également l'habillage qui a été choisi pour agrémenter le contenu de \textit{Dji Cou Digourdì}. J'ai personnellement veillé à ne pas me borner à la rédaction d'un simple volume constitué d'une série de pièces de théâtre classées chronologiquement. Étant donné que Les Digourdì, depuis leur création, ont toujours cherché à proposer un théâtre frais et jeune, j'ai décidé que le recueil devait présenter divers éléments d'originalité, afin de rendre la lecture agréable, colorée et multimédia.

Pour rendre la lecture agréable, j'ai choisi de rédiger \textit{Dji Cou Digourdì} avec \LaTeX, un logiciel gratuit de composition typographique, conçu pour produire des documents d'une qualité typographique élevée. La grande flexibilité de \LaTeX m'a permis de transformer la mise en page graphique d'un script de théâtre classique, propice à un acteur qui doit lire aisément une réplique, en une mise en page fonctionnelle pour un lecteur intéressé par l'intrigue du scénario. En effet, la beauté d'une composition typographique, outre le texte, réside dans le fait que la disposition du matériel à lire ou à consulter n'attire pas l'attention sur elle-même, mais conserve cette sobriété qui permet au lecteur de recevoir le message sans distractions inutiles et sans buter sur la lecture à cause d'espacements irréguliers ou de changements constants de style de caractères\footnote{ Fazio, Luciano. \textit{Introduzione all'arte della composizione tipografica con LaTeX}. Università degli Studi di Messina, n.d., \url{https://mat521.unime.it/~fazio/tesi/GuidaGuIT.pdf}}. Par conséquent, je suis certain que le lecteur appréciera la composition typographique agréable de \textit{Dji Cou Digourdì}.

Pour rendre le texte plus coloré, je me suis inspiré du livre "Computer Science Distilled: Learn the Art of Solving Computational Problems"\footnote{ Waisman, Wladston Ferreira Filho. \textit{Computer Science Distilled: Learn the Art of Solving Computational Problems}. Code Energy LLC, 2017.}, dans lequel les auteurs utilisent des émoticônes pour égayer la lecture de thèmes liés à l'informatique. J'ai exploité cette idée non seulement pour embellir le texte et lui donner une touche d'originalité, mais aussi pour rendre plus didactique la compréhension de certaines expressions ou mots du patois. Le lecteur trouvera donc souvent une émoticône représentant un objet, une émotion ou un animal juste après le mot correspondant : \textit{martelette}\martello, \textit{innamoroù}\inamourou, \textit{gadeun}\gadeun.

Enfin, compte tenu de la fonction d'archive que possède \textit{Dji Cou Digourdì}, j'ai souhaité rendre facilement accessible au lecteur tout le matériel multimédia produit par Les Digourdì au sein de leurs pièces. Grâce à des QR Codes, il est en effet possible d'accéder aux plateformes YouTube, Facebook, Instagram et SoundCloud pour apprécier les vidéos, les photos, les bruitages et les musiques\footnote{ Les bruitages et les musiques sont accessibles via la version en ligne du présent livre. Version archivée au sein du dépôt GitHub (voir \hyperref[vers_num]{Annexe - Version numérique}).} réalisés par Les Digourdì.

\paragraph*{Méthode}
Maintenant que nous avons clarifié la raison d'être de \textit{Dji Cou Digourdì}, son contenu et son habillage, c'est le temps de décrire brièvement sa réalisation.

La première phase a été dédiée à récupérer les copies papier et/ou numériques des textes théâtraux, en privilégiant les versions les plus à jour. Par la suite, j'ai recherché, archivé puis visionné, seconde par seconde, toutes les vidéos des représentations théâtrales des pièces de 2009 à 2019. Ce fut sans aucun doute la phase qui a demandé le plus de temps. En regardant et en écoutant chaque réplique, j'ai mis à jour tous les scripts, y compris avec les improvisations qui, en tant que telles, n'auraient jamais pu être transcrites dans le script original. De plus, j'ai structuré toutes les pièces en scènes, en leur donnant un titre éloquent.

Une fois terminée la transcription de toutes les pièces, la phase de révision linguistique a commencé. En collaboration avec le BREL\footnote{ Bureau Régional Ethnologie et Linguistique.}, l'intégralité du texte de \textit{Dji Cou Digourdì} a été corrigée sur le plan orthographique et lexical.

Parallèlement à ces deux phases principales, j'ai collecté tout le matériel multimédia produit par Le Digourdì\footnote{ Malheureusement, une seule vidéo n'a pas été archivée car elle a été égarée on ne sait où. La vidéo représentait parodiquement la publicité pour l'eau de Santa Colomba « qui, si tu la bois, est une bombe ! » (voir pièce « Eun drolo de distributeur »).} et je l'ai rendu disponible, si ce n'était pas déjà le cas, sur YouTube \href{https://www.youtube.com/@the_digourdi}{\yt}, pour les vidéos, et sur SoundCloud \href{https://soundcloud.com/user-234168361/sets}{\pppp} pour les bruitages et les chansons.

De plus, pour chaque pièce, j'ai mené un entretien avec un ou plusieurs acteurs afin de recueillir ces souvenirs, ces coulisses et ces émotions qui rendent l'interprétation d'une pièce unique et irremplaçable. Les entretiens ont été menés de la manière la plus spontanée possible, individuellement ou en groupe, autour d'un apéritif qui a aidé l'acteur à revivre toute son expérience personnelle avant, pendant ou après la pièce.

Je conclurai par quelques notes méthodologiques concernant la scénographie, les costumes, les avant-spectacles et la citation des collaborateurs.
J'ai rédigé la description de la scénographie dans le seul but d'énumérer sommairement les éléments les plus marquants du contexte dans lequel le scénario se développe, avec une énumération approximative de certains accessoires de scène utilisés par les personnages. Cette liste ne se veut absolument pas une reproduction fidèle de la fiche technique détaillant tous les accessoires et les scénographies.

En ce qui concerne les costumes, j'ai choisi de ne pas les décrire dans une section dédiée. En cas de tenues particulières, celles-ci sont indiquées en note ou dans la description des personnages.
Enfin, j'ai transcrit les avant-spectacles lorsque le script ou la vidéo était disponible\footnote{ En 2014, Jo\"elle Bollon et Laurent Chuc ont chauffé le public avec un avant-spectacle dont, malheureusement, ni le texte ni aucune preuve vidéo n'ont été retrouvés.}. J'ai également cherché à citer, à la fin de chaque pièce, les collaborateurs qui ont travaillé en coulisses. Cependant, la partie finale des remerciements a parfois été coupée de la vidéo. Par conséquent, à travers des photos et des entretiens, j'ai tenté d'identifier les personnes qui ont collaboré tout au long de la production théâtrale, mais je ne peux certainement pas garantir de les avoir toutes citées. Si quelqu'un a été omis, je jure que ce n'est rien de personnel !

\paragraph*{Notes stylistiques}
À mesure que le travail progressait selon la méthodologie décrite dans le paragraphe précédent, j'ai dû faire des choix stylistiques.

En premier lieu, le patois étant une langue minoritaire, plusieurs mots n'existent pas ou se sont perdus avec le temps. Pour combler le vide lexical du patois, il est d'usage courant de s'appuyer sur la langue française. Cependant, dans certains cas, j'ai suivi le conseil de Diego Lucianaz\footnote{ Mes plus sincères remerciements lui sont adressés pour ses nombreuses consultations concernant la recherche de mots ou d'expressions en patois.}, fervent défenseur de Le Digourdì pour le rôle que la compagnie joue dans la diffusion du patois auprès des jeunes. Il m'a suggéré de proposer des néologismes plutôt que d'utiliser le français dans le cas où un mot ne pourrait être traduit. À titre d'exemple, le terme « aspirapolvere » peut être traduit du français par « aspirateur » ; cependant, une solution originale et très théâtrale a été d'inventer le néologisme « peuccapoussa », qui, bien sûr, est légitimé à vivre au sein d'une pièce de théâtre, mais qui, qui sait, pourrait aussi devenir d'usage courant dans les foyers de nombreuses familles. Un deuxième exemple : pour « sedia a sdraio », traduisible par « chaise longue », j'ai proposé le néologisme « caèya di solèi ».
Cependant, chaque fois que je n'ai pas pu ou voulu proposer un néologisme, j'ai emprunté le terme français correspondant en l'écrivant en italique. En général, tous les mots provenant d'une autre langue (italien, français, anglais) ont été écrits en italique pour signaler au lecteur la présence d'un terme écrit avec une graphie différente (\textit{proscenium}, \textit{scène}, \textit{scénographie}).

Les noms propres, quelle que soit leur graphie originale, n'ont pas été écrits en italique : Ferrari, Champagne, Facebook, Orage, Pierre, Francesca... Dans ces cas, la première lettre majuscule identifie un nom propre qui sera lu dans sa graphie originelle, si elle est connue du lecteur.

La ponctuation, l'emploi des majuscules, les critères pour écrire les nombres en toutes lettres et les autres conventions linguistiques suivent, en principe, les règles de la langue française.

Un autre aspect très important pour lequel j'ai dû faire un choix concerne les variantes phonétiques du francoprovençal. Il est connu, du moins parmi ses locuteurs, que le patois varie selon l'endroit où il est parlé, le rendant très hétérogène. Pour des raisons évidentes, la variante utilisée dans \textit{Dji Cou Digourdì} est le patois de la Commune de Charvensod. Cependant, tout en restant dans cette limite linguistique, on y trouve de nombreuses variantes phonétiques : éve/ive, me plé/me pli, féyo/fio, aprestoù/aprestó (en général tous les participes passés en où/ó). Au milieu de cette très vaste richesse phonétique, j'ai cherché un compromis entre cohérence et valorisation des variantes qui, à mon modeste avis personnel, représentent une facette précieuse de l'identité de ses locuteurs. Par conséquent, au sein d'une même pièce de théâtre, le lecteur trouvera le même mot écrit de manière différente ; en même temps, j'ai cherché à éviter que deux variantes ne coexistent dans une même phrase (èitsa/èita, counte/conte).

Par principe, les variantes phonétiques héritées de la langue italienne n'ont pas été transcrites, afin de valoriser, dans la mesure du possible, l'origine française du terme.

Enfin, et ce n'est pas le moins important, la graphie à utiliser pour écrire le patois de Charvensod. J'ai toujours écrit le patois de manière approximative, en m'inspirant de quelques règles phonétiques de la langue française et en demandant conseil à des collègues plus expérimentés. Cependant, pour pouvoir publier \textit{Dji Cou Digourdì}, j'ai dû chercher quelqu'un qui était capable et avait le temps de corriger environ 500 pages de scripts. La seule institution en Vallée d'Aoste que j'ai réussi à identifier pour bénéficier d'un service gratuit de révision linguistique a été le Guichet linguistique francoprovençal de la Vallée d'Aoste, lequel adopte la graphie définie par le BREL\footnote{ Bureau Régional Ethnologie et Linguistique.}. Par conséquent, l'intégralité du texte de \textit{Dji Cou Digourdì}, ayant été révisée par le Guichet linguistique, suit les règles de graphie définies par le BREL. La logique qui imprègne cette graphie consiste à écrire chaque mot exactement comme le locuteur le prononce ; chaque lettre est propédeutique à la reconstruction du son de chaque mot.

Je n'entrerai pas dans le détail des avantages et des inconvénients, des qualités et des défauts de cette graphie plutôt qu'une autre, car ce n'est certainement pas le lieu pour disserter d'un sujet qui exigerait des connaissances et des compétences bien au-delà des miennes. L'objectif de ce bref paragraphe est de justifier le choix stylistique d'adopter cette graphie spécifique. Un choix qui, comme il a été précédemment déclaré, repose sur un aspect de pure praticité, à savoir la possibilité de publier un texte de 500 pages entièrement révisé.

\paragraph*{Un don}
Que représente pour moi \textit{Dji Cou Digourdì} ? Un don pour Le Digourdì, un acte d'amour, car cela m'a demandé une quantité de temps énorme, et le temps donné est amour.
\begin{center}
Il ne me reste plus qu'à vous souhaiter une\ldots\\\vfill bonne lecture !\\\vfill\#todzorpidigourdì
\end{center}


