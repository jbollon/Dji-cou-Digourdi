\chapter*{Euntroduch\'on} 

%IDEA: APPENDICE CON STATISTISCHE: parole totali, nome più frequente, verbo più frequente ecc.... parola più lunga...; totale e per pièce

% DA QUALCHE PARTE SCRIVERE CHE per i suoni e le musiche c'è un link su soundcloud per trovare ciò che abbiamo creato o salvato. Le canzoni o le sigle non ci sono se non sono state create da noi. Si possono trovare su yt.

% DA QUALCHE PARTE SCRIVERE:  non sono state inserite le indicazione di scena in merito agli ingressi/uscite da dx/sx, se non strettamente necessario... MOtivo: per non appesantire la lettura del racconto con note tecniche di regia.

%%%%%%%%%%%%%%%%%%%%%
% STRUTTURA DELL'INTRODUZIONE
%%%%%%%%%%%%%%%%%%%%%%

% PERCHÉ 10 cou digourdì
%\paragraph*{Pequé Dji Cou Digourdì}
%Dji Cou Digourdì nèi de la nésessitoù de recupéré tcheu le teste téatral di pièse eunterprétaye pe la compagnì Le Digourdì de Tsarvensoù eugn occaj\'on di Printemps Théatral. Bièn chouèn, le derie verch\'on di teste son modifiaye a man i dérì momàn, pe le dérie proue iaou, per qualche strano motivo, nèisson le battiye é le-z-ach\'on pi comique. Pertanto, deun le-z-àn le teste téatral prézàn pe le-z-archive di-z-ordinateur sayòn maque de verch\'on viille, bièn louèn de sen que lo pebleuque l'a pouì veure.

\paragraph*{Perché Dji Cou Digourdì}
Dji Cou Digourdì nasce dalla necessità di raccogliere tutti i testi teatrali interpretati dalla compagnia teatrale Le Digourdì de Tsarvensoù in occasione del Printemps Thé\^atral. Quasi sempre, le ultime versioni dei testi erano modificate a penna durante le ultime prove, durante le quali, per qualche strana ragione, nascevano le battue e le idee più irreverenti ed efficaci. Pertanto, nel corso degli anni, i testi teatrali, archiviati da qualche parte nei computer degli attori, corrispondevano solamente ad una vecchia versione della sceneggiatura, poco fedele alla rappresentazione interpretata dal vivo sul palco.

Inoltre, le pièse dei Digourdì hanno sovente integrato nella sceneggiatura contenuti multimediali, quali audio, suoni, video e immagini. Pertanto, oltre a voler testimoniare in modo fedele le pièse interpretate sul palco del Printemps Thé\^atral, Dji cou digourdì vuole essere anche un archivio dei contenuti multimediali prodotti dalla compagnia teatrale tsarvensolentse.

In sintesi, Dji cou digourdì si presenta a voi lettori come testimonianza e catalogo dell'arte teatrale dei Digourdì.

\paragraph*{Il contenuto}
Dji cou Digourdì raccoglie le prime dieci pièse interpretate dai Digourdì in occasione del Printemps Thé\^atral dal 2009 al 2019\footnote{ In totale sono 11 anni; tuttavia, nel 2017 Le Digourdì non hanno partecipato al Printemps Thé\^atral. Il motivo dell'assenza è raccontato nel capitolo dedicato all'anno 2017.}. Ogni pièse corrisponde ad un capitolo strutturato nel seguente modo:
\begin{itemize}
\item[$\bullet$] Copertina della pièse con titolo, autori, luogo e data di rappresentazione;
\item[$\bullet$] Foto con elenco degli attori e QR Code per collegarsi al video integrale della pièse;
\item[$\bullet$] Intervista ad un attore con un suo personale ricordo e qualche curiosità o aneddoto legato alla pièse;
\item[$\bullet$] Descrizione della scenografia e dei principali oggetti di scena;
\item[$\bullet$] Elenco e descrizione dei personaggi (per ordine di ingresso in scena) con indicazione dell'attore che lo interpreta;
\item[$\bullet$] Eventuale Avanspettaclle;
\item[$\bullet$] Testo teatrale;
\item[$\bullet$] Elenco dei collaboratori che hanno lavorato dietro alle quinte.
\end{itemize}

%struttura contenuto: 11 anni-1 (2017); sul perché no 2017 si rimanda al capitolo a esso dedicato; souvenir de l'atteur; link video + scenografia + cast (+ordine per apparizione).. poi testo e deri le rido' + nota sui costumi

% grafia (breve cenno su scelta BREL e poi approfondimento su dibattito)  e struttura del contenuto
\paragraph*{Il vestito}
A mo' di stilista ho il piacere di commentare anche il vestito che è stato scelto per impreziosire il contenuto di Dji cou Digourdì. Qui vive il mio personale impegno a non volermi limitare a scrivere un mero volume costituito da  una serie di pièse teatrali ordinate cronologicamente. Siccome i Digourdì, sin dalla loro nascita, hanno sempre cercato di proporre un teatro fresco e giovanile, ho deciso che la raccolta dovesse possedere vari elementi di originalità, in grado di rendere la lettura piacevole, colorata e multimediale.

Per rendere piacevole la lettura, ho scelto di scrivere Dji cou Digourdì con \LaTeX, un software gratuito di composizione tipografica, progettato per produrre documenti con un'elevata qualità tipografica. La grande flessibilità di \LaTeX mi ha permesso di trasformare il layout grafico di un classico copione teatrale, propedeutico per un attore che deve leggere agevolmente una battuta, in un'impaginazione funzionale ad un lettore interessato alla trama della sceneggiatura. Infatti, la bellezza di una composizione tipografica, a parte il testo, sta nel fatto che la disposizione del materiale da leggere o da consultare non richiama su di sé l’attenzione, ma mantiene quella sobrietà che consente al lettore di recepire il messaggio senza distrazioni inutili e senza zoppicare nel leggere a causa di spaziature irregolari o continui cambiamenti di stile dei caratteri\footnote{ Fazio, Luciano. \textit{Introduzione all'arte della composizione tipografica con LaTeX}. Università degli Studi di Messina, n.d., \url{https://mat521.unime.it/~fazio/tesi/GuidaGuIT.pdf}}. Pertanto, sono certo che il lettore apprezzerà la piacevole composizione tipografica di Dji cou Digourdì.
%La scelta dei font, la loro grandezza, il loro stile, la loro nerezza, la scelta della distanza fra le righe, i margini, gli spazi bianchi lasciati attorno agli oggetti non testuali, sono tutte cose di competenza del “disegnatore editoriale”, non di dilettanti come siamo tutti noi. L A TEX, nei limiti di quello che può fare un oggetto “inanimato” come un programma di elaborazione di dati, sostituisce sia il disegnatore editoriale sia il tipografo.

Per rendere invece il testo più colorato, ho tratto spunto dal libro "Computer Science Distilled: Learn the Art of Solving Computational Problems"\footnote{ Waisman, Wladston Ferreira Filho. \textit{Computer Science Distilled: Learn the Art of Solving Computational Problems}. Code Energy LLC, 2017.}, in cui gli autori, utilizzano degli emoticons per rallegrare la lettura di temi legati alla Computer Science. Ho sfruttato questa idea non solo per abbellire il testo e dargli un tocco di originalità, ma anche  per rendere più didascalica la comprensione di alcune espressioni o parole del patois. Il lettore, quindi, troverà sovente un emoticon raffigurante un oggetto, un'emozione o un animale subito dopo la rispettiva parola: \textit{martelette}\martello, \textit{innamoroù}\inamourou , \textit{gadeun} \gadeun .

Infine, premessa la funzione di archivio che Dji cou Digourdì possiede, ho voluto rendere di facile accesso al lettore tutto il materiale multimedale prodotto dai Digourdì all'interno delle proprie pièse. Attraverso dei QR Code, infatti, è possibile accedere alle piattaforme YouTube, Facebook, Instagram e SoundCloud per apprezzare i video, le foto, i suoni e le musiche realizzate dai Digourdì.

%SPECIFICARE CHE PER I VIDEO CI SONO QRCODE NEL TESTO, MENTRE PER I SUONI E GLI EFFETTI SONORI recuperabili con la versione online del libro, in cui cliccando sul titolo del suono/musica si viene reindirizzati al brano

\paragraph*{Metodo} Chiarito perché è nato Dji cou Digourdì, qual è il suo contenuto e come è stato vestito, ritengo opportuno descrivere brevemente come è stato realizzato.

La prima fase ha visto il reperimento  delle copie cartacee e/o digitali dei testi teatrali, quelli maggiormente aggiornati. Successivamente, ho cercato, archiviato e poi visionato, secondo dopo secondo, tutti i video della rappresentazione teatrale delle pièce dal 2009 al 2019. Questa è stata senza dubbio la fase che ha richiesto più tempo. Guardando e ascoltando ogni singola battuta, ho aggiornato tutti i copioni anche con quelle improvvisazioni che, in quanto tali, non sarebbero mai potute essere trascritte nel copione originale. Inoltre, di mio pugno, con un titolo suggestivo ho strutturato in scene tutte le sceneggiature.

Terminata la trascrizione di tutte le pièse, è iniziata la fase di revisione linguistica. In collaborazione con il BREL\footnote{ Bureau Régional Ethnologie et Linguistique.}, l'intero contenuto di Dji cou Digourdì è stato corretto sotto il profilo ortografico e lessicale.

Parallelamente a queste due macro fasi, ho raccolto tutto il materiale multimediale prodotto dai Digourdì\footnote{ Purtroppo, solamente un video non è stato archiviato in quanto smarrito chissà dove. Il video rappresentava parodicamene la pubblicità dell'acqua di Santa Colomba ``che se la bevi è una bomba!" (vedi pièse 2012).} e l'ho reso disponibile, qualora non lo fosse già, su YouTube \href{https://www.youtube.com/@the_digourdi}{\yt}, per i video, e su SoundCloud \href{https://soundcloud.com/user-234168361/sets}{\pppp}  per i suoni e le canzoni.
%COME È AVVENUTA L'INTERVISTA AGLI ATTORI
Inoltre, ho intervistato, per ogni pièse, uno o più attori per catturare quei ricordi, quei retroscena e quelle emozioni che rendono unici e irripetibili l'interpretazione di una pièse. Le interviste sono state condotte nel modo più spontaneo possibile, singolarmente o in gruppo, con un aperitivo che aiutasse a far rivivere all'attore tutta la propria personale esperienza prima, durante o dopo la pièse.

Concludo, con alcune note metodologiche in merito alla scenografia, ai costum, agli Avanspettaclle e alla citazione dei collaboratori. 
%NOTE PER SCENOGRAFIA
Ho scritto la descrizione della scenografia con il mero scopo di elencare sommariamente le parti più salienti del contesto in cui la sceneggiatura si sviluppa, con un'elencazione approssimata di alcuni oggetti di scena utilizzati dai personaggi. Non vuole essere assolutamente una riproduzione fedele della scheda tecnica nella quale dettagliare tutti gli oggetti di scena e le scenografie. 
% NOTE PER COSTUMI - ? non so se dire che non li ho descritti o semplicemente non dire nulla
Per quanto riguarda i costumi, ho scelto di non descriverli in una sezione ad essa dedicata. In caso di abbigliamenti particolari, essi sono indicati in nota o nella descrizione dei personaggi.
% NOTE PER AVANSPETTACLE E COLLABORATORI: citati laddove è stato possibile capire/ricordarsi chi c'era; avanspettaclle messo se c'era copione o video (in alcuni anni forse era stato improvvisato, e se non registrato, non è stato possibile inserirlo)
Infine, ho trascritto gli Avanspettaclle laddove era disponibile il copione o il video. Ho cercato altresì di citare a fine pièse i collaboratori che hanno lavorato dietro alle quinte. Qualche volta, però, la parte finale dei ringraziamenti è stata tagliata dal video. Pertanto, tramite foto e interviste ho tentato di individuare le persone che hanno aiutato durante la produzione e interpretazione della pièce, ma sicuramente non posso garantire di averle citate tutte. Se qualcuno è stato omesso, giuro, non è nulla di personale!

\paragraph*{Note stilistiche}
Man mano che il lavoro procedeva con la metodologia descritta nel paragrafo precedente, ho dovuto effettuare delle scelte stilistiche.
% uso dei neologismi - Diego motivaizone + esempi
In primo luogo, essendo il patois una lingua minoritaria, diversi vocaboli non esistono o  sono andati perduti nel tempo. Per colmare il vuoto lessicale del patois, è uso comune appoggiarsi alla lingua francese. Tuttavia, in alcuni casi ho accolto il consiglio di Diego Lucianaz\footnote{ A lui vanno i miei più sinceri ringraziamenti per le numerose consulenze circa la ricerca di parole o di espressioni in patois.}, forte sostenitore dei Digourdì per il ruolo che la compagnia ha nel diffondere il patois tra i giovani, il quale mi ha suggerito di creare dei neologismi piuttosto che usare il francese nel caso in cui una parola non si riesca a tradurre. Per fare un esempio, il termine aspirapolvere può essere tradotto dal francese con ``aspirateur"; una soluzione originale e molto teatrale, invece, è stata quella di ideare il neologismo ``peuccapoussa", che, sicuramente, è legittimato a vivere all'interno di una pièce teatrale, ma poi, chissà, potrebbe anche diventare di uso comune all'interno delle case di molte famiglie. Un ultimo esempio: per ``sedia a sdraio", traducibile con "chaise longue" ho proposto il neologismo ``caèya di solèi" [AGGIUNGERE trebeillet e ringraziare Daniel!].
% note sull'uso del corsivo (vedi note su drive): prestiti da altre lingue tra cui ita, inglese e francese soprattutto per termini teatrali (proscenio, scenografica, scena)
Tuttavia, tutte le volte in cui non ho potuto o voluto proporre un neologismo, ho preso in prestito il rispettivo termine in francese scrivendolo in corsivo. In generale, tutte le parole provenienti da un'altra lingua (italiano, francese, inglese) sono state scritte in corsivo per segnalare al lettore la presenza di un termine scritto con una grafia diversa (\textit{proscenium}, \textit{scène}, \textit{scénographie}).

Non sono stati scritti in italico, invece, i nomi propri, quale che fosse la loro grafia originale: Ferrari, Champagne, Facebook, Orage, Pierre, Francesca\ldots in questi casi la prima lettera maiuscola identifica un nome proprio che sarà letto nella sua grafia originale, se nota al lettore.

La punteggiatura utilizzata segue (SFUMARE DICENDE che segue approssimativamente) le regole della lingua francese, analogamente alle convenzioni per scrivere i numeri in lettere, fatta eccezione ....QUIIIIIIIIIIIIIIIIII % per i numeri delle scenografie no, numeri interi)

Un altro importantissimo aspetto per il quale ho dovuto effettuare una scelta riguarda le varianti fonetiche del francoprovenzale. È noto, quantomeno tra i suoi locutori, che il patois varia a seconda di dove viene parlato rendendosi molto eterogeneo. Per ovvie ragioni, la variante utilizzata in Dji cou Digourdì è il patois del Comune di Charvensod. Tuttavia, pur rimanendo in questo confine linguistico, al suo interno troviamo parecchie varianti fonetiche: éve/ive, me plé/me pli, féyo/fio, aprestoù/aprestó (in generale Participes passés en où / ó). Nel bel mezzo di questa vastissima ricchezza fonetica, ho cercato un compromesso tra coerenza e valorizzazione delle varianti, che, secondo un mio modesto parere personale, rappresentano una sfacettatura preziosa dell'identità dei suoi locutori. Pertanto, all'interno di una medesima pièce teatrale il lettore troverà la stessa parola scritta in modo diverso; allo stesso tempo, ho cercato di evitare che in una stessa frase ci fossero due varianti (èitsa/èita, counte/conte).

Come linea di principio, non sono state trascritte quelle varianti fonetiche ereditate dalla lingua italiana, al fine di valorizzare, nei limiti del possibile, l'origine francese del termine. %Tra i vari esempi che si possono citare, quello più comune è "mé pousso" (je peux), sostituito nei vari testi teatrali con "mé pouo/pouì". Allo stesso modo, laddove possibile, per le forme di possesso quali "lo meun tseun, la tin-a coutse, le noutre amì" ho eliminato l'articolo determinativo, peculiare dell'italiano, (mon tseun, ma coutse, noutre amì) oppure ho riformulato il possesso con la forma "lo tseun de mé, la coutse de té, le-z-amì de no".

% NON SPECIFICO note sui numeri

% (last but not least) grafia

Infine, ma non per importanza, la grafia con la quale scrivere il patois di Charvensod. Ho sempre scritto il patois in modo improvvisato, ispirandomi a qualche regola fonetica della lingua francese e chiedendo consigli a qualche amico che lo scrive da più tempo. Tuttavia, per poter pubblicare Dji cou Digourdì ho dovuto cercare qualcuno che fosse in grado e avesse il tempo di correggere circa 500 pagine di copioni. L'unico ente in Valle d'Aosta che sono riuscito a trovare per usufruire di un servizio gratuito di revisione linguistica è stato il Guichet linguistique francoprovençal de la Vallée d'Aoste, il quale adotta la grafia definata dal BREL\footnote{ Bureau Régional Ethnologie et Linguistique.}. Pertanto, tutto il testo de ``Dji cou Digourdì", essendo stato revisionato dal Guichet linguistique, segue le regole di grafia definite dal BREL. La logica che permea questa grafia consiste nel scrivere qualsiasi parola esattamente come il locutore la pronuncia; ogni lettera è propedeutica alla ricostruzione del suono di ogni vocabolo.

Non entro nel merito dei vantaggi e degli svantaggi, dei pregi e dei difetti di questa grafia piuttosto che un'altra, in quanto non è sicuramente questa la sede per disquisire di un tema che richiederebbe conoscenze e competenze che vanno ben al di là di quelle del sottoscritto. L'obiettivo di questo breve paragrafo era di giustificare la scelta stilistica di adottare questa specifica grafia. Scelta che, come pocanzi dichiarato, si fonda su un aspetto di mera praticitià, vale a dire poter pubblicare un testo di 500 pagine completamente revisionato.  


% BUCHI di contenuto: avanspettaclle e video (ester), ringraziamenti finali, + foto per il processo di forum (sulla serata caliente di Tobie?) + Joelle e Laurent nel 2014 hanno fatto avanspettaclle.. ma non abbiamo né video né copione.
--------------------------------------
%Elenco quindi di seguito le peculiarità di questa raccolta:

%- Gli emoticons. Dèi le z an 2000, è ormai prassi comunicare con messaggi di testo arricchiti con i c.d. emoticons, ossia piccolissime "immagini" colorate da inserire nel corpo del testo per esprimere o rafforzare un'emozione o un contenuto particolare del messaggio. Ad esempio, "I love you $<$3", "No vièn pi tard, poudzo! -pollice", "complemen -applausi". Prendendo spunto da "citare computer science" (In nota: ringrazio l'autore per essere stato fonte di ispirazione, anche se credo non leggerà mai questa raccolta), ho deciso di inserire all'interno dei testi teatrali dell'emoticons per due motivi: 1)[un po' da boomer questo punto: magari basta dire per renderlo più colorato]per rendere il testo più giovanile, colorato e più vicino alla forma con cui i ragazzi e le ragazze comunicano; 2) per facilitare la comprensione di alcune parole a chi si sta avvicinando alla lettura del patois. In alcuni casi troverete un oggetto e subito dopo la sua illustrazione in forma di emoticon, i.e. "gadeun" -maiale.

%- Rimanendo sempre sul piano della comunicazione in forma giovanile, ho voluto rendere multimediale la lettura delle pièce dando la possibilità al lettore di ascoltare o visualizzare musiche e video trasmessi durante la pièce. Le Digourdì sono sempre stati multimediali anche nelle loro opere teatrali. Pertanto, anche la raccolta delle relative pièce non poteva essere da meno. Sfogliando il testo troverete dei QRcode che fungono da collegamento internet per visualizzare su YouTube [add logo emoticon] o su Facebook [add logo emoticon] i video realizzati e per ascoltare su SoundCloud [add logo emoticon] le canzoni/musiche utilizzate [audio solo un logo con link a soundcloud - per musiche magari basta solo la cit]. Se state leggendo il cartaceo dovrete scaricare sul vostro cellulare un'apposita app per la codifica del QRcode [add logo emoticon] , altrimenti se state leggendo la versione digitale sarà sufficiente cliccare sul QRcode.

%- In seguito, il lettore prima di immergersi nella trama della pièce potrà scoprire, attraverso un'intervista di uno degli attori, qualche aneddoto simpatico relativo alla pièce: emozioni, errori, retroscena ecc...

%- Infine, tutto è il testo è stato sottoposto a revisione linguistica. Questione della scelta della grafia BREL

\paragraph*{Un dono} cosa rappresenta per me dji cou digourì: un dono per i Digourdì, un atto d'amore, perché mi ha richiesto una quantità di tempo enorme, e il tempo donato è amore. 

%... non mi resta che augurarvi una buona lettura! \# todzorpidigourdì

%Forse quanto segue va nella prefazione. Solo sinossi nel sommario.

%Sintesi della raccolta: troviamo 10 pièse interpretate dai digourdi dal 2009 al 2019. Si tratta di 11 anni, ma nel 2016/17? non si è recitato -> cortometraggio. Le prime xx al giacosa e le altre allo splendor. Prima di ogni pièse ho inserito la lista dei personaggi con una brevissima/succinta caratterizzazione e il nome di chi lo ha interpretato. + Breve descrizione della scenografia +  qualche aneddoto esclusivo sulla pièce raccontato da uno degli attori.

\chapter*{\textit{Sommaire}}

\begin{center}
\begin{itemize}
\item[$2009$] \textbf{L'opetaille moderno} - La modernitoù l'arè-tì pourt\'o de bénéfise pe le-z-opétaillo? Mi sourtoù: éziston-tì le maladdo reutso é le maladdo pouo? Lo séjour de Casimir é Geromine deun l'opetaillo de Veulla no-z-èidzerè a repoundre a seutte dimande.\newline
\item[$2010$] \textbf{Forum valdotèn} - Deun lo \textit{studio} Mediaset de Forum, Rita de l'Eillize no pourte deun eun vrèi tribunal, avouì lo renomoù dzeudzo Senliquer. Dou case: Tobie countre la fenna Bertina é Tchièn Frottapanse countre le carabegnì Tic é Tac.\newline

\item[$2011$] \textbf{La vatse de l'universitoù} - Le vatse rendon pamì comme eun cou; eunc\'o Marietto é lo bitchoulì Sahlam san pamì iaou bouéchì la tita. Mi lo laouréoù Simon, avouì le-z-amì vétérinéo, sa comme modernizé lo baou de pappa Tchièn. \newline

\item[$2012$] \textbf{Eun drolo de distributeur} - Pe la popolach\'on la quemeua de Tsarvensoù l'a atsetoù  eun distributeur d'éve. Dèi l'inaougurach\'on, eunna matse de dzi, avouì de caratéo bièn diffièn, s'euncrouijon douàn seutta bouteucca de l'éve, iaou pe lo Vallet de Veulla l'è pa lèin-o fé reuspété lo reillemèn.
\newline

\item[$2013$] \textbf{Matte sen tcheutte\ldots Matte} - Pe pasì lo ten, dou vétchot avèitson caque trasmech\'on a la télévij\'on iaou semble que sise poussiblo accapé totte le repounse i problème de la viya: comèn s'arbeillì, troué l'ommo parfé, terì si amoddo [COMÈN SE POURIYE DEU "EDUCARE"] le botcha é accapé eun traille.\newline

\item[$2014$] \textbf{Tanta betsii} - Eunna équipe de Tirolèis betchì èidze eunna fameuille euntredeutta a fé betsii. La mor de tanta Melanie, que gneun s'atégnè, reusque de fé saouté totte! Grase a l'équipe tirolèise, totte le saouseuse son eumbouélaye; mi can mimo la më semble eunc\'o plèin-a de tseue\ldots\newline

\item[$2015$] \textbf{Disco Flama} - Passon le-z-àn mi la counta l'è todzor la mima: Pollein é Tsarvensoù alerèn jamì d'acor. Trèi jénérach\'on de Pollentch\'on é de Tsarvensolèn s'accapon todzor i Disco Flama pe fé fita, danchì e pe dimoutrì qui l'a la tita pi diya. Mi l'amour saré pi for de l'orgeuill que divije le doe Quemeue? \newline

\item[$2016$] \textbf{N'en pa lo ten} - Lo \textit{laxisme} di-z-eumpléyà pebleucco countre le prateuque di penchon-où, le tsachaou countre le-z-ambientaliste; eunc\'o finque de végàn é de maladdo di \textit{social}. Tcheut eunsemblo pe eunna bataille idéolojique deun eugn uficho pebleucco iaou gneun l'a lo ten pe traillì, mi tcheutte l'an lo ten pe pédre de ten.\newline

\item[$2017$] \textbf{LA RECONSTITUTION | La vèille di gran dz\^{o}} - Aprì la secounda Guéra Mondiale, lo 27 janvieur 1946, 424 Tsarvensolèn é Tsarvensolèntse dimandon a la Préfetteua de Veulla la reconstituch\'on de la Quemeua de Tsarvensoù. Lo 30 avril 1946 la Quemeua l'è reconstituite é lo 30 joueun 1946 Aimé Borbey, Emerico Comé, Justin Donzel, Louis Lucianaz é César Savioz son euntsardjà de organizé le premie-z-éléch\'on de Tsarvensoù reconstituite: lo gran dzô. Le-z-émoch\'on viquiye lo lon de la vèille de si gran dzô pouèn maque le-z-imajin-ì.

\item[$2018$] \textbf{Todzo pi digourdì} - Pagàn Paul vo conte la counta di straordinéo sucsé d'eunna di pi fameuze Compagnì téatrale cougniye pe to lo moundo: Le Digourdì de Tsarvensoù. Mi pe sèi la vrèya istouére di Digourdì, de sen que conte Pagàn fodriye coppé la mèitchà, vardé eun car é tsapoté eunc\'o quetsouza!\newline

\item[$2019$] \textbf{Sèidepou(vo)èr} - La sèi de pouer l'a fé tsire lo Gouvernemàn Jorrioz. Le Digourdì l'an pamì eun Prézidàn. Le-z-éléch\'on antisipaye pe voté lo nouo Prézidàn di Digourdì divij\'on totta la parotse de Tsarvensoù eun doe fach\'on: le rouza, lista numér\'o 1, pe voté Marlène Jorrioz é
le dzano, lista numér\'o 2, pe voté Jordy Bollon. \textit{Rendez-vous} lo 9 marse i téatro Splendor de Veulla pe cougnitre eun directe lo \textit{vainqueur} di-z-éléch\'on.

\end{itemize}
\end{center}
