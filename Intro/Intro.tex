\chapter*{Introduction} 
\markboth{\MakeUppercase{Introduction}}{\MakeUppercase{Introduction}}

%IDEA: APPENDICE CON STATISTISCHE: parole totali, nome più frequente, verbo più frequente ecc.... parola più lunga...; totale e per pièce

% DA QUALCHE PARTE SCRIVERE CHE per i suoni e le musiche c'è un link su soundcloud per trovare ciò che abbiamo creato o salvato. Le canzoni o le sigle non ci sono se non sono state create da noi. Si possono trovare su yt.

% DA QUALCHE PARTE SCRIVERE:  non sono state inserite le indicazione di scena in merito agli ingressi/uscite da dx/sx, se non strettamente necessario... MOtivo: per non appesantire la lettura del racconto con note tecniche di regia.

\paragraph*{Pourquoi \textit{Dji Cou Digourdì}}
Le recueil \textit{Dji Cou Digourdì} est né de la nécessité de rassembler toutes les pièces de théâtre interprétées par la compagnie Le Digourdì de Tsarvensoù à l'occasion de ses participation au Printemps Thé\^atral. Presque toujours, les versions finales des textes sont retouchées à la main lors des dernières répétitions, ce moment mystérieux où surgissent, pour une étrange raison, les répliques et les idées les plus drôles, pétillantes et piquantes, et donc les plus efficaces pour l’essor du spectacle. Pour cette raison, les textes théâtraux, archivés quelque part dans les ordinateurs des acteurs, sont restés figés sur une ancienne version du scénario, peu fidèle à la représentation réellement jouée sur scène.

Par ailleurs, les pièces des Digourdì ont souvent intégré au scénario des contenus multimédias: enregistrements sonores, bruitages, vidéos ou images. Ainsi, au-delà de la volonté de témoigner fidèlement des pièces interprétées sur la scène du Printemps Thé\^atral, \textit{Dji Cou Digourdì} constitue également un archivage des contenus multimédias créés par la compagnie théâtrale \textit{tsarvensolentse}.

Bref, \textit{Dji Cou Digourdì} s'offre aux lecteurs comme un témoignage vivant et un inventaire précieux de l'art théâtral cultivé par les Digourdì.

\paragraph*{Le contenu}
\textit{Dji Cou Digourdì} rassemble les dix premières pièces interprétées par la compagnie Le Digourdì de Tsarvensoù à l'occasion de ses participation au Printemps Thé\^atral entre le 2009 et le 2019\footnote{ Au total, cela fait 11 ans ; cependant, en 2017, Les Digourdì n'ont pas participé au Printemps Thé\^atral. La raison de cette absence est expliquée dans le chapitre dédié à l'année 2017.}. Chaque pièce constitue un chapitre et s'articule de la manière suivante :
\begin{itemize}
\item[$\bullet$] Couverture de la pièce avec titre, auteurs, lieu et date de la représentation ;
\item[$\bullet$] Photographie de groupe accompagnée de la liste des acteurs et d'un QR Code pour visionner la vidéo intégrale de la pièce ;
\item[$\bullet$] Entretien avec un acteur, incluant un souvenir personnel et quel\-ques curiosités ou anecdotes liées à la pièce ;
\item[$\bullet$] Description de la scénographie et des principaux éléments de décor utilisés sur scène ;
\item[$\bullet$] Présentation des personnages, dans l’ordre d’apparition, avec le nom de l’acteur qui les incarne ;
\item[$\bullet$] Texte de l’avant-spectacle, le cas échéant ;
\item[$\bullet$] Texte intégral de la pièce ;
\item[$\bullet$] Liste des collaborateurs ayant travaillé en coulisses.
\end{itemize}

\paragraph*{L'habillage}
C'est avec plaisir que je commente également le soin apporté à l'habillage du contenu de \textit{Dji Cou Digourdì}. J'ai personnellement veillé à ne pas me limiter à la simple rédaction d'un volume réunissant une série de pièces de théâtre classées par ordre chronologique. Étant donné que les Digourdì, depuis leur création, ont toujours cherché à proposer un théâtre novateur, dynamique et créatif, j'ai voulu que ce recueil  reflète cette philosophie à travers une présentation originale, agréable à lire, visuellement attrayante et enrichie d’éléments multimédias.

Pour cela, j'ai choisi de concevoir \textit{Dji Cou Digourdì} avec \LaTeX, un logiciel gratuit de composition typographique, reconnu pour la qualité exceptionnelle des documents qu’il permet de produire. La grande flexibilité de \LaTeX\ m’a permis d’adapter la mise en page classique d’un script théâtral - pensée à l’origine pour faciliter la lecture par un acteur - en une en une version plus fluide et lisible pour un lecteur intéressé par l’intrigue du scénario. En effet, la beauté d’une composition typographique ne réside pas uniquement dans le texte lui-même, mais aussi dans la manière dont ce texte est disposé : une mise en page réussie sait rester discrète, sobre et cohérente, sans attirer inutilement l’attention ni nuire à la lecture par des espacements incohérents ou des variations injustifiées de style ou de caractère\footnote{ Fazio, Luciano. \textit{Introduzione all'arte della composizione tipografica con LaTeX}. Università degli Studi di Messina, n.d., \url{https://mat521.unime.it/~fazio/tesi/GuidaGuIT.pdf}}. Par conséquent, je suis certain que le lecteur saura apprécier le soin de la composition typographique de \textit{Dji Cou Digourdì} autant que la richesse de son contenu.

Pour apporter une touche de couleur au texte, je me suis inspiré du livre \og Computer Science Distilled: Learn the Art of Solving Computational Problems\fg\footnote{ Waisman, Wladston Ferreira Filho. \textit{Computer Science Distilled: Learn the Art of Solving Computational Problems}. Code Energy LLC, 2017.}, dans lequel les auteurs utilisent des émoticônes pour égayer la lecture de sujets liés à l'informatique. J'ai repris cette idée non seulement pour enrichir visuellement le texte et lui donner une touche d'originalité, mais aussi dans un but didactique: faciliter la compréhension de certains termes ou expressions en patois. Ainsi, le lecteur rencontrera régulièrement des émoticônes représentant un objet, une émotion ou un animal immédiatement après le mot correspondant : \textit{martelette}\martello, \textit{innamoroù}\inamourou, \textit{gadeun}\gadeun.

Enfin, compte tenu de la fonction d'archive de ce recueil, j'ai souhaité rendre facilement accessible au lecteur tout le matériel multimédia produit par les Digourdì au fil des pièces. Grâce à des QR Codes intégrés dans le texte, le lecteur peut accéder directement aux plateformes YouTube, Facebook, Instagram ou SoundCloud pour découvrir les vidéos, les photos, les bruitages et les musiques\footnote{ Les bruitages et les musiques sont accessibles via la version en ligne du présent recueil. Version archivée au sein du dépôt GitHub (voir \hyperref[vers_num]{Annexe - Version numérique}).} réalisés par les Digourdì.

\paragraph*{Méthode}
Maintenant que la raison d'être de \textit{Dji Cou Digourdì}, son contenu et son habillage ont été clarifiés, il est temps d’en présenter brièvement la réalisation.

La première phase a consisté à rassembler les textes des pièces, sous format papier et/ou numérique, en privilégiant les versions les plus à jour. Ensuite, j'ai entrepris un long travail d’archivage et de visionnage minutieux de toutes les captations vidéo des représentations théâtrales de 2009 à 2019. Ce fut sans aucun doute la phase qui a demandé le plus de temps. En regardant et en écoutant chaque réplique, j'ai mis à jour tous les scripts, en y intégrant également les improvisations - par nature absentes des textes d’origine. De plus, j'ai structuré toutes les pièces en séquence de scènes, chacune dotée d’un titre éloquent.

Une fois terminée la transcription intégrale de toutes les pièces, une phase de révision linguistique a été menée en collaboration avec le BREL \footnote{ Bureau Régional Ethnologie et Linguistique.}. Tout le contenu de \textit{Dji Cou Digourdì} a été corrigé, tant sur le plan orthographique que lexical.

En parallèle de ces deux grandes étapes, j'ai aussi collecté l'ensemble du matériel multimédia produit par Le Digourdì\footnote{ Malheureusement, une seule vidéo n'a pas été archivée car elle a été égarée on ne sait où. La vidéo représentait parodiquement la publicité pour l'eau de Santa Colomba « qui, si tu la bois, est une bombe ! » (voir pièce « Eun drolo de distributeur »).} et je l'ai rendu disponible, lorsque ce n’était pas encore le cas, sur YouTube \href{https://www.youtube.com/@the_digourdi}{\yt} pour les vidéos, et sur SoundCloud \href{https://soundcloud.com/user-234168361/sets}{\pppp} pour les bruitages et les chansons.

De plus, pour chaque pièce, j'ai mené un entretien avec un ou plusieurs acteurs afin de recueillir ces souvenirs, anecdotes et émotions qui rendent chaque interprétation unique et irremplaçable. Ces échanges se sont voulus aussi spontané que possible, menés individuellement ou en petit groupe, autour d'un apéritif convivial qui a aidé l'acteur à partager toute son expérience personnelle avant, pendant ou après les représentations.

Je terminerai par quelques notes méthodologiques concernant la scénographie, les costumes, les avant-spectacles et la mention des collaborateurs.
La description de la scénographie vise uniquement à relever les éléments les plus marquants du cadre dans lequel se déploie l’action. Elle s’accompagne d’une liste approximative de certains accessoires de scène utilisés par les personnages. Cette énumération ne se veut absolument pas une reproduction fidèle de la fiche technique détaillée.

Quant aux costumes, j'ai choisi de ne pas leur consacrer une section dédiée. En cas de tenues particulières, celles-ci sont indiquées en note ou dans la description des personnages, lorsque cela s’avérait pertinent.
J'ai également transcrit les avant-spectacles lorsque le script ou l'enregistrement vidéo était disponible\footnote{ En 2014, Jo\"elle Bollon et Laurent Chuc ont chauffé le public avec un avant-spectacle dont, malheureusement, ni le texte ni aucune preuve vidéo n'ont été retrouvés.}. Enfin, j’ai tenu à mentionner, à la fin de chaque pièce, les collaborateurs ayant travaillé en coulisses. Cependant, certains enregistrements de pièce ne comportaient pas les remerciements finaux, j’ai dû m’appuyer sur des photos et des témoignages pour tenter d’identifier les membres de l’équipe. Il est possible que certaines personnes n’aient pas été citées, et si tel est le cas, je l’assure : ce n’est nullement intentionnel !

\paragraph*{Notes stylistiques}
À mesure que le travail avançait, selon la méthodologie exposée dans le paragraphe précédent, j'ai dû faire des choix stylistiques.

Tout d’abord, le patois étant une langue minoritaire, plusieurs mots n'existent pas ou se sont perdus avec le temps. Pour combler ce vide lexical, il est d'usage courant de recourir à des emprunts à la langue française. Toutefois, dans certains cas, j'ai suivi les conseils de Diego Lucianaz\footnote{ Mes plus sincères remerciements lui sont adressés pour ses nombreuses consultations concernant la recherche de mots ou d'expressions en patois.}, ardent défenseur de Le Digourdì pour le rôle que la compagnie joue dans la diffusion du patois auprès des jeunes générations. Il m'a suggéré de privilégier, lorsque cela était possible, la création de néologismes plutôt que le simple emprunt au français. À titre d'exemple, le terme « aspirapolvere » peut être traduit du français par « aspirateur » ; cependant, une solution originale et très théâtrale a été d'inventer le néologisme « peuccapoussa », qui, bien sûr, est légitimé à vivre au sein d'une pièce de théâtre, mais qui, qui sait, pourrait aussi devenir d'usage courant dans les foyers de nombreuses familles. Un deuxième exemple : pour « sedia a sdraio », traduisible par « chaise longue », j'ai proposé le néologisme « caèya di solèi ».
Cependant, chaque fois que je n'ai pas pu ou voulu proposer un néologisme, j'ai emprunté le terme français correspondant en l'écrivant en italique. En général, tous les mots provenant d'une autre langue (italien, français, anglais) ont été écrits en italique: \textit{proscenium}, \textit{scène}, \textit{scénographie}.

Les noms propres, en revanche, ont conservé leur graphie d'origine sans italique : Ferrari, Champagne, Facebook, Orage, Pierre, Francesca... Dans ces cas, la première lettre majuscule suffit à signaler au lecteur qu’il s’agit d’un nom propre à lire tel quel, dans sa forme d’origine.

La ponctuation, l'emploi des majuscules, la graphie des nombres en toutes lettres ainsi que les autres conventions linguistiques suivent, en principe, les règles de la langue française.

Un autre aspect très important ayant nécessité une prise de position concerne les nombreuses variantes phonétiques du francoprovençal. Il est connu, du moins parmi ses locuteurs, que le patois varie fortement d’une commune à l’autre, ce qui en fait une langue profondément hétérogène. Pour des raisons évidentes, la variante utilisée dans \textit{Dji Cou Digourdì} est le patois de la Commune de Charvensod. Cependant, même au sein de cette seule aire linguistique, on y trouve de nombreuses variantes phonétiques : éve/ive, me plé/me pli, féyo/fio, aprestoù/aprestó (en général tous les participes passés en où/ó). Face à cette très vaste richesse phonétique, j'ai cherché un compromis entre cohérence et valorisation des variantes qui, à mon modeste avis personnel, représentent une facette précieuse de l'identité de ses locuteurs et méritent d’être conservées. Par conséquent, dans une même pièce de théâtre, le lecteur trouvera le même mot écrit de manière différente. J’ai toutefois veillé à ne pas faire coexister deux variantes en même temps dans une même phrase (èitsa/èita, counte/conte).

Par principe, les variantes phonétiques héritées de la langue italienne n'ont pas été transcrites, afin de valoriser, dans la mesure du possible, l'origine française du terme.

Enfin, et ce n'est pas le moins important, la graphie à utiliser pour écrire le patois de Charvensod. J'ai toujours écrit le patois de manière approximative, en m'inspirant de quelques règles phonétiques de la langue française et en demandant conseil à des collègues plus expérimentés. Cependant, pour pouvoir publier \textit{Dji Cou Digourdì}, j'ai dû chercher quelqu'un qui était capable et avait le temps de corriger environ 450 pages de scripts. La seule institution en Vallée d'Aoste que j'ai réussi à identifier pour bénéficier d'un service gratuit de révision linguistique a été le Guichet linguistique francoprovençal de la Vallée d'Aoste, lequel adopte la graphie définie par le BREL\footnote{ Bureau Régional Ethnologie et Linguistique.}. Par conséquent, l'intégralité du texte de \textit{Dji Cou Digourdì}, ayant été révisée par le Guichet linguistique, suit les règles de graphie définies par le BREL. La logique qui imprègne cette graphie consiste à écrire chaque mot exactement comme le locuteur le prononce ; chaque lettre est propédeutique à la reconstruction du son de chaque mot.

Je n'entrerai pas dans le détail des avantages et des inconvénients, des qualités et des défauts de cette graphie plutôt qu'une autre, car ce n'est certainement pas le lieu pour disserter d'un sujet qui exigerait des connaissances et des compétences bien au-delà des miennes. L'objectif de ce bref paragraphe est de justifier le choix stylistique d'adopter cette graphie spécifique. Un choix qui, comme il a été précédemment déclaré, repose sur un aspect de pure praticité, à savoir la possibilité de publier un texte de 500 pages entièrement révisé.

\paragraph*{Un don}
Que représente pour moi \textit{Dji Cou Digourdì} ? Un don pour Le Digourdì, un acte d'amour, car cela m'a demandé une quantité de temps énorme, et le temps donné est amour.
\newpage
\begin{center}
Il ne me reste plus qu'à vous souhaiter une\ldots\\\vfill bonne lecture !\\\vfill\#todzorpidigourdì
\end{center}


