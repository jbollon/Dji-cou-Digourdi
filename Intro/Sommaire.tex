\chapter*{\textit{Sommaire}}

\begin{itemize}
\item[$2009$] \textbf{L'opetaille moderno} - La modernitoù l'arè-tì pourt\'o de bénéfise pe le-z-opétaille? Mi sourtoù: éziston-tì le maladdo reutso é le maladdo pouo? Lo séjour de Casimir é Geromine deun l'opetaille de Veulla no-z-èidzerè a repoundre a seutte dimande.\newline
\item[$2010$] \textbf{Forum valdotèn} - Deun lo \textit{studio} Mediaset de Forum, Rita de l'Eillize no pourte deun eun vrèi tribunal, avouì lo renomoù dzeudzo Senliquer. Dou case: Tobie countre la fenna Bertina é Tchièn Frottapanse countre le carabegnì Tic é Tac.\newline

\item[$2011$] \textbf{La vatse de l'universitoù} - Le vatse rendon pamì com\-me eun cou; eunc\'o Marietto é lo bitchoulì Sahlam san pamì iaou bouéchì la tita. Mi lo laouréoù Simon, avouì le-z-amì vétérinéo, sa comme modernizé lo baou de pappa Tchièn. \newline

\item[$2012$] \textbf{Eun drolo de distributeur} - La Quemeua de Tsarvensoù l'a atsetoù eun distributeur d'éve pe la populach\'on. Dèi l'inaougurach\'on, eunna matse de dzi, tsaqueun avouì son caratéo, s'euncrouijon douàn seutta bouteucca de l'éve, iaou pe lo vallet de Veulla l'è pa lèin-o fé reuspété lo reillemèn.
\newline

\item[$2013$] \textbf{Matte\ldots sen tcheutte matte} - Pe pasì lo ten, dou vé\-tchot avèitson caque trasmech\'on a la télévij\'on iaou sem\-ble que sise poussiblo accapé totte le repounse i problème de la viya: comèn s'arbeillì, troué l'ommo parfé, terì si amoddo le bo\-tcha é accapé eun traille.\newline

\item[$2014$] \textbf{Tanta betsii} - Eunna équipe de Tirolèis betchì èidze eunna fameuille euntredeutta a fé betsii. La mor de tanta Melanie, que gneun s'atégnè, reusque de fé saouté totte! Grase a l'équipe tirolèise, totte le saouseuse son eumbouélaye; mi can mimo la mii semble eunc\'o plèin-a de tseue\ldots\newline

\item[$2015$] \textbf{Disco Flama} - Passon le-z-àn mi la counta l'è todzor la mima: Pollein é Tsarvensoù alerèn jamì d'acor. Trèi jénérach\'on de Pollentch\'on é de Tsarvensolèn s'accapon todzor i Disco Flama pe fé fita, danchì é pe dimoutrì qui l'a la tita pi diya. Mi l'amour saré pi for de l'orgeuill que divije le doe Quemeue? \newline

\item[$2016$] \textbf{N'en pa lo ten} - La flemma di-z-eumpléyà pebleucco countre le prateuque di penchon-où, le tsachaou countre le-z-ambientaliste; eunc\'o finque de végàn é de maladdo di \textit{social}. Tcheut eunsemblo pe eunna bataille idéolojique deun le bur\'o pebleucco iaou gneun l'a lo ten pe traillì, mi tcheutte l'an lo ten pe pédre de ten.\newline

\item[$2017$] \textbf{LA RECONSTITUTION | La vèille di gran dz\^{o}} - Aprì la Secounda Guéra mondiale, lo 27 janvieur 1946, 424 Tsarvensolèn é Tsarvensolentse dimandon a la Préfetteua de Veulla la reconstituch\'on de la Quemeua de Tsarvensoù. Lo 30 avrì 1946, la Quemeua l'è reconstituite é lo 30 joueun 1946 Aimé Borbey, Emerico Comé, Justin Donzel, Louis Lucianaz é César Savioz prègnon l'euntsardzo de organizé le premie-z-éléch\'on de Tsarvensoù reconstituite: lo gran dzô. Le-z-émoch\'on viquiye lo lon de la vèille de si gran dzô pouèn maque le-z-imajin-ì.\newline

\item[$2018$] \textbf{Todzo pi digourdì} - Pagàn Paul vo conte la counta di straordinéo sucsé d'eunna di pi fameuze Compagnì téatrale cougniye pe to lo moundo: Le Digourdì de Tsarvensoù. Mi pe savèi la vrèya istouére di Digourdì, de sen que conte Pagàn fodriye coppé la mèitchà, vardé eun car é tsapoté eunc\'o quetsouza!\newline

\item[$2019$] \textbf{Sèidepou(vo)èr} - La sèi de pouer l'a fé tsire lo Gouvernemàn Jorrioz. Le Digourdì l'an pamì eun Prézidàn. Le-z-éléch\'on antisipaye pe voté lo nouo Prézidàn di Digourdì divij\'on totta la parotse de Tsarvensoù eun doe fach\'on: le rouze, lista numér\'o 1, pe voté Marlène Jorrioz é le dzano, lista numér\'o 2, pe voté Jordy Bollon. \\\textit{Rendez-vous} lo 9 marse i téatro Splendor de Veulla pe cougnitre eun directe lo \textit{vainqueur} di-z-éléch\'on.

\end{itemize}