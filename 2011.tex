\title{LA VATSE DE L’UNIVERSITO\`U}
\author{Pièse icrita pe Le Digourdì}
\date{Téatro Giacosa de Veulla, 6 mi 2011}

\maketitle

\fotocopertina{Foto/2011/gruppo.jpg}{Jo\"{e}l Albaney, Simone Roveyaz, Jasmine Comé, Francesca Lucianaz, Marco Ducly, Ilaria Linty}{Paolo Cima Sander, Laurent Chuc, Giada Grivon, Jo\"{e}lle Bollon, Ester Bollon, Pierre Savioz}{2011}

\LinkPiese{La vatse de l'universitoù}{https://www.youtube.com/watch?v=GhEbo6vx-xY&list=PLBofM-NS_eLJUln45l7VH457fGak_Bk5O&index=11}{.75}

\souvenir{L'è itaye la premiye pièse iaou n'en voulì clloure la counta avouì eun messadzo émouvàn. L'è it\'o bièn drolo clloure de seutta fas\'on\ldots lo pebleuque se lo attégnè pa é eunc\'o no sen senti-no eunna mia foua plase, sourtoù aprì èi fé eunna londze counta que l'ie complètemàn comique.
\\ Can mimo n'ayoon fé eun grou traille, avouì le costume, la réjistrach\'on eun Cittadella de Waka Watse é la \textit{chorégraphie} que n'en danchà si lo palque. Eun pi, se me troumpo pa, l'è it\'o lo premì cou que n'en realiz\'o no mimo eunna matse de \textit{scénographie}.
\\ Mi se dèyo fran terì foua lo pi dzen souvenir que n'i, me fa diye que son le botcha de la parotse que, le dzor aprì ou finque aprì de mèis, repettaon noutre battiye, salle que lèi fiaon pi riye; finque té (Jordy Bollon, N.d.R.) si dzo que sayò pasoù te veure t'ayè deu-me, si la louye, \og Areuvvo, prègno l'achenseur\fg, eunna di battiye di pouo Tchièn Frottapanse!\\L'ie cheur plèizèn veure la salle di Giacosa plèin-a de dzi (é le premi-z-àn no no lo attégnaon cheur pa), mi la soddisfach\'on pi grousa l'ie comprendre que t'ayè quetoù eun vrèi sourì, de grouse riaye, dedeun le dzi. 
%finale con morale, abbiamo voluto fare questo finale.. ma non ci ha convinto
%grande lavoro sui costumi e registrazione in cittadella; joelle coreografia
%come scenografie è stata la prima volta che abbiamo fatto un sacco di cose noi!
%Bello il fatto di avere un riscontro, anche da parte dei bambini, in giro per ilpaese che ti citavano le battute...
%Anche Jordy disse "prégno l'achenseur"!
%e anche il successo di pubblico!
}{Pierre Savioz}
\queriaouzitou{
\begin{itemize}
\item[$\bullet$] Waka Watse l'è itaye la premiye \textit{cover}\footnote{ Deun lo 2009, l'ie itaye djeusto réjistraye la parodì d'eun \textit{refrain} publisitère.} di Digourdì. Avouì seutta tsans\'on le Digourdì l'an eunc\'o réalizoù, pe lo premì cou de leur istouéré, eunna \textit{chorégraphie} que l'an danchà i mentèn de seutta pièse.

\item[$\bullet$] La \textit{chorégraphie} de Waka Watse l'è itaye repropouzaye i mariadzo de Pepe e Paoletta.

\item[$\bullet$] Pe la \textit{vidéo} Catalogue Boeuf n'at eun bitchoulì que pe eun per de seconde arrevè pa a comprendre comèn eun bou piémontèis pouè prédjì devàn a eunna machin-a di reprèize!
\end{itemize}
}

%\subsection*{Queriaouzitoù}

\Scenographie
\textbf{A gotse, pe la mèiz\'on di lasì}:
\begin{itemize}
\item[$\bullet$] 1 tsaoudée;
\item[$\bullet$] 1 tabla \tabla\ avouì desì an matse de papì é documàn;
\item[$\bullet$] 2 caèye de bouque \sediaLegno\ a l'entor de la tabla;
\item[$\bullet$] 1 scaffal avouì eunna bouite de caille pe fé lo fromadzo, de platte, de tasse, eunna botèille de rodzo é de vèyo;
\item[$\bullet$] 1 modeun de la caillà;
\item[$\bullet$] 1 sidel di lasì;
\item[$\bullet$] 1 devella piastrellaye avouì eunna fin-itra, de fas\'on que se pouisse vère qui pase dérì.
\end{itemize}
\vspace*{0.25cm}

\textbf{I mentèn}:

\begin{itemize}
\item[$\bullet$] 1 chouta de bouque;
\item[$\bullet$] 1 batchì;
\item[$\bullet$] 1 radi\'o \radio .
\end{itemize}
\vspace*{0.25cm}

\textbf{A drèite, pe lo parque di vatse}:

\begin{itemize}
\item[$\bullet$] De bale de fen;
\item[$\bullet$] 1 arieuza\footnote{ Lamèn la querì eunc\'o trepeullalasì.};
\item[$\bullet$] 2 rètse pe apeillì le vatse;
\item[$\bullet$] 1 icaoa \scopa ;
\item[$\bullet$] 2 saque de lets\'on ou de \textit{mangime}.
\end{itemize}

\setlength{\lengthchar}{3cm}

\Character[SAHLAM]{SAHLAM}{Sahlamm}{Maroqueun aourì de Tchièn, todzor avouì la baquetta eun man é le gambal, \name{Paolo Cima Sander}}

\Character[PICCINA]{PICCINA}{Piccina}{Rèina di lasì, blantse avouì de tatse maròn, calma é avouì bièn d'ichèn, \nameF{Jo\"{e}lle Bollon}}

\Character[COBRA]{COBRA}{Cobra}{Rèina di baou, nèye, aggressive é todzor presta a féye de djestre, \nameF{Ester Bollon}}

\Character[TSARDOÙN]{TSARDOÙN}{Tzardoun}{Rèina viille, tsatagnaye, bièn saventa, \nameF{Francesca Lucianaz}}

\Character[TCHIÈN]{TCHIÈN}{Cien}{Propriétéro di vatse, ommo de Iolanda é pappa de Simon, \name{Pierre Savioz}}

\Character[IOLANDA]{IOLANDA}{Iolanda}{Fenna de Tchièn, mamma de Simon, comandanta é déterminaye, \nameF{Jasmine Comé}}

\Character[SIMON]{SIMON}{Simon}{Gars\'on laoureoù de Tchièn é Iolanda, l'a pa tan voya de se pouertchì le man, \name{Laurent Chuc}}

\Character[MARIETTO]{MARIETTO}{Marietto}{Pantal\'on tanque a l'amboueuille\footnote{ Lamèn lo querì eunc\'o bot\'on di ventro.}, bretelle de pappagran é tsapì rodzo eun tita. L'è l'amì de Tchièn, lo vezeun di mitcho todzor contèn é pozitif, eunna mia tro queriaou, \name{Jo\"{e}l Albaney}}

\Character[CHEF\\ MOTTINHO]{MOTTINHO}{Chef}{Lo cape de l'équipe spésialla di vétérinéro, \nameF{Ilaria Linty}}

\Character[VET.\\ GILLES]{ GILLES}{Maracana}{Vétérinéro d'orijine amériquène, bièn sportif, spésializoù eun sosializach\'on é mouvemàn de la vatse, \nameF{Giada Grivon}}

\Character[VET. SCHULZ]{ SCHULZ}{Schulz}{Vétérinéro d'orijine tédesque, spésializoù deun la psicolojì de la vatse, \name{Marco Ducly}}

\Character[VET.\\ TRINCHERO]{ TRINCHERO}{Trinchero}{Vétérinéro jamaiquèn spésializoù deun l’alimentach\'on de la vatse, \name{Simone Roveyaz}}

\Character[]{TCHEUTTE}{Tutti}{\hspace{1cm}}

\DramPer

\act[Acte I]

\ridoiver

\scene[-- Cobra, Piccina é Tsardoùn]

\StageDir{La \textit{scène} l’è divijaye eun trèi partiye: d’eun coutì, a drèite, lo parque di vatse eun stabulach\'on libra; i mentèn eunna pégna chouta avouì le bague que se eumplèyon de couteumma i baou; a gotse eunna devella que pourte a la mèiz\'on di lasì. Dedeun la quezeun-a n’atte eunna tabla plèin-a de papì é documàn, avouì dérì eunna tsaoudée iaou se feré lo lasì. \\ Entre deun lo parque lo bitchoulì, aprì le vatse: Cobra, Piccina é Tsardoùn. Se vèi maque lo maroqueun que avèitse foua \textit{scène} é prèdze avouì le vatse \mucca \mucca \mucca . Se sen lo}

\effet{https://soundcloud.com/user-234168361/muggito}{Borello}

\StageDir{di vatse\footnote{ Lo son de la vatse (\textit{muggito}).}.}

\begin{drama}

\Sahlammspeaks Veun seu veun! Veun seu veun! Alé, sah, ommo, que fa arrevé i parque. Ommo Dik! Va ba dérì! Maladetto d'eun tseun, m'acoutte pa\ldots 
Sa, sa, Tsardoùn, alé! Boudza-té-vèi!

\StageDir{Entre totta flappa la premie vatse\footnote{ Pe rendre l'eunterprétach\'on bièn pi téatralla, totte le vatse se boudzon si do pià/patte \riye .}. }

\Sahlammspeaks Mondjemé se fisan totte lente comme té n’ario cheur jamì lo flou queurt. Cobra, Cobra!

\StageDir{Entre la secounda vatse é eunc\'o lleu tsemie flappa avouì la tita que nante. Eunsemblo a Tsardoùn se proméon douàn lo palque é de tenzentèn se frotton eunna countre l'atra ou desì de bale de fen.}

\Sahlammspeaks Veun Piccina, eunc\'o té\ldots

\StageDir{Entre la trèijima é dérì vatse. Comme le-z-atre l'a la fiacca de tsemiì.}

\Sahlammspeaks \ldots saye lleu.

\StageDir{Sahlam entre deun la mèiz\'on di lasì pe prendre lo nésesséo pe pouèi aryì. Le vatse son solette é pouon prédjì tranquillamente.}

\Piccinaspeaks Sit maroqueun no queutte pa tranquille eun momàn!

\Cobraspeaks  Vouè t’a rèiz\'on. L’è todzor d’entor é no queutte jamì pequì eun pése.

\Tzardounspeaks \'E aprì avouì si tseun \cane\ que l’è todzor dérì a no moudre le zaret. L’è fran gramo. A mé cheurtoù que n’i le pià tan sensiblo é delecatte.

\Cobraspeaks Mi iaou gramo? Mé eun dzor avouì la mia corna drèita n’i levou-lo desì mé é n'i caya-lo bièn louèn. \direct{Bièn fièira} Eugn éffé a mé reste pa tan protso é fé bièn attench\'on.

\Piccinaspeaks \'E aprì comenche a itre vioù é chor. Vèyade beun que l’é pi sen que fi de sin-a tita que sen que lèi diyon de fére.

\Tzardounspeaks Dé, fé attench\'on a sen que te di di viille eh! Mé si la pi viille di baou mi salla que sa todzor de pi de vo totte. Rappèla-té que, sensa de mé, vo ariya jamì aprèi eunna matse de bague. Té, Cobra, si pa se te rappèle euncoa: l’an que t’i nèissia, mé é mamma de té allavon bièn d’accor é\ldots

\Cobraspeaks Voualà que torne reprende le seun discour stoufiàn pa pocca.

\Piccinaspeaks Vouè, sen dza totte de salla conta li\ldots

\Tzardounspeaks Reuspé vo doe todzor mouèn?!

\Cobraspeaks Can mimo, v'ouèi vi Sahlam? L’è dza torna aoura de no aryì.

\Piccinaspeaks Fran té te prèdze. Avouì la squeuza di bataille di rèine te queutton pa crèitre le teteun é te teurion cheur pa de lasì\latte. La fontin-a la féyon avouì mé! \direct{En moutrèn son teteun} Avèitsa mé que mezeua: la satchima!

\Tzardounspeaks Finque ézajeroù! Pensa que Tchièn, lo noutro propriétéro, l’a de foto de mé da dzouvin-a que\ldots le n’ayoù pa magà grouse comme le tin-e, mi bièn pi caraye é diye comme eun bèrio! Atro qué!

\StageDir{Piccina é Cobra s'avèitson.}

\Piccinaspeaks  \direct{Eun se moquèn de Tsardoùn} Ara te te pouriye fé eun \textit{lifting}! N’i sentì que refàn eunc\'o le sen pe le vatse ara.
 
\StageDir{Cobra é Piccina se beutton a riye.}

\Tzardounspeaks Ad\'on vo dovve vo pouria dza fére demanda: eunna  pe beutté oumouente eunna premie é l’atra pe beutté su de rebeuste bretelle!

\Piccinaspeaks Te la sa todzor pi londze té! 

\Tzardounspeaks Vouè, l’é fran pouèi. Can mimo seutte machine pe no-z-aryì son pa pouì tan delecatte.

\Piccinaspeaks Pe ren. Me fan eun mou di matte! Tcheu le cou que me teurie eunna peuppa vèyo le-z-itèile \stella \dolore \stella \dolore \stella .

\Tzardounspeaks Mi eun dzor ou l’atro salla \textit{machinetta} li no teurie si eunc\'o le teteun. Te vèi pi!

\Piccinaspeaks Spérèn fran de na.

\Tzardounspeaks Can mimo mé me rappèlo eun cou can Tchièn me ariave a man! Sen vouè que l'ie dzen. Cheurtoù can n’ayoù de maleusse avouì caqueun ou la dzournoù l'ie pa tan allaye bièn vèyao pa l’aoua de itre accaéchaye é baillì lo meun bon lasì. Sario restaye pouza é pouza a itre patchocaye pe llu. Cheurtoù can baillave campa i sin-e dzente tsans\'on\ldots a mé me s’eunplissave lo queur d’amour. Belle se Tchièn l’a jamì tsantoù a Sent-Or, \direct{avèitse l’atra} eh\ldots t’a comprèi\ldots sayè me fie relassì.

\Piccinaspeaks Comèn n’ario lamoù passì eunc\'o mé sise dzen momàn. Mé si nèisia caze eunsemblo a si trepeullalasì électric. L’è dèi can si modze que Sahlam eumplèye si machèn avouì mé.

\Tzardounspeaks T’a rèiz\'on é cheurtoù noutro Tchièn n’en pamì tan vi-lo seuilla i baou. Sembleavì que, dèi can l'è arrevoù sit aourì, llu lise queuttoù totte é lèi euntéressise pamì ren de no.

\Cobraspeaks De mé se euntéresse euncoa. S’i jamì itaye fourtin-aye ià pe le concour. Mé si caze chia que saroù Rèina Réjonalla 2011 é n’aret Gueuste que di a Touri de me beuttì si dzen bosquet que tan sondzo!

\Tzardounspeaks Te lame sondjì eh?!  Mi acoutà Cobra\ldots mi l’è vrèi que t’a de problème a restì prègne? Perqué n’atte de vouése que vouyadzon pe seuilla\ldots

\StageDir{Cobra reste sourprèya de sen que l'a deu Tsardoùn, mi comprèn to de chouite qui l'a fé l'espi\'on: se vionde ver Piccina é la eumbranque pe eunna corna.}

\Cobraspeaks Vouè sarè seutta sé que l’arè de-te-l\'o.

\Piccinaspeaks Na, na, si pa mé!\imbarazzatoo\ Mi l’a pa de eumportanse, di maque\ldots

\Cobraspeaks Ouè l’è vrèi. Mé, dèi can n’i prouvoù Mol\'on, lo bou de la grandze de Marteun, n’i todzor pensoù a llu é djeusto a llu. S’i eunnamoraye fran!

\Piccinaspeaks Oh poua matasse\ldots

\Cobraspeaks  L’ayè doe grouse-z-ipale, eugn itsin-a é eun cotz\'on que fiave pouiye de vère! \'E aprì de grouse corne!

\Piccinaspeaks \ldots é ara?

\Cobraspeaks \ldots é ara l’è dou-z-àn que lo vèyo pamì.

\Tzardounspeaks Cobra, te lo vèi pamì, resta tranquila. Seuilla te fa pa te eunnamorì \inamourou\ d’eun bou! Tcheu le-z-àn se tsandze. Eun nouo é dzouvin-o tcheu le cou.

\Piccinaspeaks Eunc\'o mé sayao eunnamoraye\ldots de Topo!

\Tzardounspeaks De qui?

\Piccinaspeaks Topo.

\Tzardounspeaks Qui?

\Piccinaspeaks \direct{Eun braillèn} Topo! N’ayoù cogni-lo a Gressan a l’Anaborava, mi can l’an deu-me que l’ayè 3000 mèinoù\ldots

\Tzardounspeaks 3000 mèinoù\ldots

\Cobraspeaks 3000?

\Piccinaspeaks Ouè\ldots 3000 paillette!

\Tzardounspeaks Ah l'è diffièn!

\Piccinaspeaks \ldots é beun n’i to de chouite oublia-lo!

\Cobraspeaks \direct{A Tsardoùn} Pensa 3000 ratte seu dedeun! Can mimo m’euntéresse pa de Topo! Mé vouillo lo meun Mol\'on!

\scene[-- Tchièn l’éléveur]

\StageDir{Torne Sahlam. Pourte Tsardoùn é Cobra foua di parque. Alleume la radi\'o avouì de mezeucca arabe é eun tsantèn tsertse d'atatchì l'arieuza a Piccina, que, can lo vèi arrevé, s'ajite.}

\sound{https://www.youtube.com/watch?v=66Essn5jIUA}{Cheb Hasni - A la besse}

\StageDir{Can l’a fenì de beuttì su totte, icaoue an mia pe tèra, mi aprì do seconde se beutte protso a eunna balla de fen é s’eundrumme. Aprì pocca entre deun la mèiz\'on di lasì Tchièn l’éléveur. L’è restoù to lo mateun eun Veulla a fere de commech\'on é l’è bi-se dza caque apéritif, donque l’è eunna mia gramo. Ren lèi va bièn.\\ Se gave lo palt\'o é lo tsapì é va ver lo parque pe vère sen que l'è seutta mezeucca que sen. Can areuvve deun lo parque vèi Sahlam eundrumì.}

\Cienspeaks \direct{Eun tchouéyèn la radi\'o} Deh s’i arrevoù!

\StageDir{Sahlam se levve to de chouite eun pià, eun se fr\'otèn le joueu.}

\Sahlammspeaks Vouè mesieu, bondzor Tchièn!

\Cienspeaks Vouè bondzor eunc\'o té. Vou-te cafilasì é brioche vi que t’i djeusto levoù?

\Sahlammspeaks Na na mersì \direct{mortifià} l'iyo eun tren de tsertchì eun ratte \topo\ que l’é dza caque dzor que viounde pe seuilla.

\Cienspeaks \ldots é Iolanda?

\Sahlammspeaks Qui?

\Cienspeaks Sahlam, la fenna!

\Sahlammspeaks Ah Iole! Ouè l’è allaye a messa. L’a deu-me de te diye de beuttì su l'ive pe la pasta.

\Cienspeaks Mi lèi penso gnenca! \direct{Bièn arogàn} L'è pa eun travaille de mé sen. L’a djeusto de arrevì tchica devàn lleu, pitoù de tchacolé foua de messa avouì le sin-e amie; ou diye i prie de copé an mia lo serm\'on. Fé todzor de counte si prée\ldots

\StageDir{Entron Iolanda é Simon.}

\Iolandaspeaks \direct{Eun braillèn} Tchièn! Iaou t'i?

\StageDir{Tchièn entre deun la lèitì. Dimèn Sahlam conteneuvve a poulitì deun lo parque, mi l'a todzor eun bouigno\orecchio\ que acoute ver la lèitì.}

\Cienspeaks \direct{Avouì eun ton bièn pi fèiblo, mouèn arrogàn de douàn} Oh, Iolanda! Bièn alaye Messa?

\Iolandaspeaks Ouè, t’a beto-nen de ten pe fé totte le commech\'on de té!

\Cienspeaks Eh ouè\ldots euh, lo serm\'on l'ie-tì dzen?

\Iolandaspeaks  Ouè, prochèn cou te pou viì eunc\'o té!

\Cienspeaks Na, na! Pren-te varda, n’ayè eunna quiya a l’AREV!

\Iolandaspeaks Eunmajin-o!

\Simonspeaks Pappa! Di pa de counte foule, te cougnissèn! Te passe de-z-aoure é de-z-aoure avouì Calliste é Jérôme a bèye de vèyo tanque a noua i Bar de l’Arena.

\Cienspeaks Mi deh Simon! Fé attench\'on eunc\'o té! Oueu l’è pa dzournoù!

\Iolandaspeaks T’ise bi dou blan \vino\ eun mouèn t’arie tchica pi dzen carattéro.

\Sahlammspeaks \direct{Eun rièn}  Eh, eh, eh, grama seutta eh?

\Cienspeaks Sahlam souplì eh! Va fenì sen que t’ie eun tren de fére.

\Sahlammspeaks Vouè, vouè, to de chouite patr\'on.

\StageDir{Sahlam tourne a sa plase. Ara l'è eun tren de brochì Piccina.}

\Cienspeaks \direct{Eun s'achouatèn a tabla} Bon ara medjì caque tsouza perqué comencho a avèi tchica d’appetì.

\Sahlammspeaks \direct{Eun rièn} Eh, eh, eh! Te fa ichouiì sen que t’a bi seu mateun. 

\StageDir{Tchièn bièn eunfastedjà, se levve é va aoutre deun lo parque iaou n'atte Sahlam.}

\Cienspeaks \direct{Eun desuèn Sahlam} Eh, eh, eh! Ano! Sah, veun avouì mé\ldots

\StageDir{Tchièn pren Sahlam é lo pourte ver la drèite di palque, caze foua.}

\Cienspeaks \direct{Eun moutrèn foua \textit{scène}} Te vèi ba lé i fon de la toula? Va ba me tramé la souye! Ara! Viyto! 

\Sahlammspeaks Vouè, vouè, galoppo\galoppe.

\StageDir{Sahlam chor foua é queutte Piccina avouì lo trepeullalasì apeillà.\\ Tchièn salie Piccina avouì eunna caresse é tourne dedeun la mèiz\'on di lasì. Dimèn Iolanda l'è chortia.}

\Simonspeaks Vouè mi pappa, avèitsa que confuj\'on que n’a seuilla. Comèn te fi a comprendre caque tsouza?

\Cienspeaks \direct{Eun s'achouatèn} T’a rèiz\'on mi mé dedeun si cazeun si iaou queutto le bague. Resta tranquilo que perdo ren.

\Simonspeaks N’isso totte mé eun man seutta baracca!

\Cienspeaks Pe ara si eunc\'o mé que désido le bague é \direct{eun moutrèn la tabla} teugno bièn lo meun oufficho.

\Simonspeaks Pappa te fa comprendre que mé n'i itedjà! Mé que si laouréoù, si de bague que té t'emajin-e gneunca. T’a jamì sentì prédzì de \textit{dematerializzazione}? Tcheu le documàn allérian deun lo \textit{computer} é d’eunna secounda te trouveriye totte sen que t’a fata.

\Cienspeaks Acouta a mé va eunc\'o to bièn pouèi, n’i pa fata de novitoù.

\Simonspeaks Pappa l’éléveur de oueu l’é pamì si de 20 an fé! T'ioù lo comprendre?

\Cienspeaks Péqué?

\Simonspeaks Comèn pequé? Aya l'è to diffièn!

\Cienspeaks Argh acoutta Simon, n'i pa fata de nouvot\'o! Pappa l’é contèn pouèi\ldots é seunséramente se t’isse pa fenì de itedjì a 40 an t’arie pouì me éidjì tchica devàn pe seuilla. N’ario cheur pa avì fata de prende Sahlam!

\Simonspeaks Mi pappa! Mé n'i itedjà! Mé si laouréoù! Mé pourio jamì fére lo travaille de Sahlam. Mé n'i itedjà! Te vou lo comprende?

\scene[-- Marietto]

\StageDir{Marietto areuve a la fenitra \finestra\ é vèi le dou chattoù a medjì.}

\Mariettospeaks Oué Tchièn!

\Cienspeaks Oh Marietto! \direct{A Simon} L’a sentì lo flou di medjì.

\Mariettospeaks Veugno dedeun Tchièn?

\Cienspeaks Vouè, veun, fé lo tor\ldots

\StageDir{Marietto fé lo tor é entre.}

\Mariettospeaks Eh le-z-amì! Ad\'on comèn l’è?

\Cienspeaks Amoddo. Te me semble eun prie que cougnisavo; que arevave todzor fran a l’aoura de maenda pe baillì la bénédich\'on i mitcho. Comme eunna moutra souisse!

\Mariettospeaks Ah, é té Simon? Te martse l’icoula?

\Simonspeaks Icoula?! Marietto, l’é pa fran eunna icoula\ldots

\Mariettospeaks Vouè é comèn te vou la criì?

\Simonspeaks Te sa Marietto totte le pi grouse veulle di mondo l’an eun eumpourtàn \textit{Centre d'étude}. L’è fameuza l’Universitoù de Toronto, Boston, ll'è la Sapienza a Romma, la Bocconi a Milan é\ldots la Testafochi eun Veulla!

\Mariettospeaks Mi pensa té! Ah vouè n’i vi lo téléjournal. Pensa Tchièn que can n’i sentì predjì de \textit{campus} eun Veulla n’i pensoù que l’isan fé eun grou courtì pe tcheu le Veullatchì. Can mimo t’a belle fenì le-z-iteudzo?

\Simonspeaks Vouè caze. Me mancon djeusto trèi ou catro izamèn.

\Mariettospeaks L'ie beun l’aoura! Mé é Tchièn l'ian stouffie de travaillì to lo dzor eun campagne. Pe fourtin-a si cou t’i eunc\'o té que te no baille eunna man.

\Simonspeaks Mi pensa-lèi gnenca! Mé n'i itedjà! Mé si laouréoù! Mé n’i dza eunna idì de sen que fiye pe seutta fameuille. Mé vouillo jestì!

\Cienspeaks Jestì?!

\Mariettospeaks Deh Tchièn t’a fé eun botcha bièn feun! Va savèi vio de sou te fé pouì.

\Cienspeaks Souplì va!

\Mariettospeaks Deh m’achatto! \direct{Eun s'achouatèn} Mi te meudze totte a sèque?

\Cienspeaks Avouì té l’è eumpoussiblo. \direct{A Simon} Deh laouré\'o! Va quiì lo veun!

\Mariettospeaks Ouè va quiì lo veun a pappa!

\Simonspeaks \direct{Démoralizoù} Ouè pappa! De rodzo?

\Mariettospeaks Ouè va bièn!

\StageDir{Simon soum\'on dou vèyo de rodzo\bicchiererosso \bicchiererosso\ a Marietto é i pappa.}

\Simonspeaks A pappa betèn mèitchà vèyo\ldots

\Cienspeaks Ouè, ouè pa de pi!

\Mariettospeaks \ldots é a Marietto beutta mai plen!

\Simonspeaks \ldots é queutto sé la botèille\wine\ pe Marietto!

\Cienspeaks \ldots é Simon aprì beutta tchica eugn odre seutta tabla plèin-a de papì!

\Simonspeaks Ouè pappa!

\StageDir{Simon, dimèn que Marietto é Tchièn fan la counta, beutte eugn odre le papì.}

\Mariettospeaks \direct{A Tchièn} T’a de novitoù desì Cobra? L’a tin-ì lo bou de Ginetto?

\Cienspeaks Mi na, ren! Nen prédzavo fran si mateun avouì Jérôme é eunc\'o llu di que n’i prouvoù de totte é ren a fére.

\Mariettospeaks Tchica te le eumplèi de \textit{mangime}. Trop, can l’è trop l’è trop!

\Cienspeaks Si pa te diye comèn. Si djeusto que l’è dou-z-àn que me queutte sensa vi. L’è pa poussiblo. Se fisse pa tan grama n’ario cheur pa vardou-la.

\Mariettospeaks Grama?! Dedeun lo teun baou? T’a maque de croué vatchine!

\Cienspeaks \direct{Caze avouì le larme i joueu \plaoue\ é eun se levèn} Queutta-mé va pe plèizì. Te sa bièn comèn terì si lo moral di-z-amì té.

\Mariettospeaks Squersavo, squeza-mé! Veun sé!

\StageDir{Tchièn se nen va ver lo parque.}

\scene[-- N'a manca de lasì!]

\StageDir{Tchièn deun lo parque vèi arevé Sahlam avouì eun sidel de lasì demì vouido.}

\Cienspeaks Dé mi l’è to li lo lasì?

\Sahlammspeaks Vouè. N’a pa de gran bague.

\Cienspeaks Avèitsa seuilla Marietto. Gnenca de lasì n’i, pamì ren. Va totte todzor pi mal! Le vatse rendon fran pamì comme eun cou.

\StageDir{Marietto va vère dedeun lo sidel.}

\Mariettospeaks Mi fran ren\ldots é mé que si eunc\'o viì pe te dimandì tchica de lasì. Ara ouzo pamì. Se te lèi beutte dou biscouì \biscotto\ lo mateun t’a belle feni-lo!

\StageDir{Tchièn gante la tita. Sahlam va vouidjì lo lasì deun la tsaoudée.}

\Mariettospeaks Sah alèn no chouaté!

\Cienspeaks Ouè fé maque comme se te fise i mitcho de té!

\Mariettospeaks Ouè, ouè veun maque!

\StageDir{Tchièn é Marietto s'achouatton tourna a tabla.}

\Mariettospeaks Bon a par le squerse Tchièn t’a beun rèiz\'on. \'E aprì seutte vatse san pamì sen que peuccon: d’iveur peuccon, can va bièn, sinque caletoù de fen, si de plan, si di mayèn, lo pièmontèis, lo fransé é lo sanfouèn di-z-Ile!

\StageDir{Dimèn Sahlam pourte foua Piccina.}

\Cienspeaks \direct{Démoralizoù} Eh, lo si praou!

\Mariettospeaks \'E le \textit{mangimi}? Sen-n\'o senque lèi beutton dedeun?

\Cienspeaks Cheur que na!

\Mariettospeaks Te vèi beun que tcheu le momàn nen accapon: caque fabreuque que trafeutson le \textit{mangimi} avouì le pouse de mabro ou de faène drole.

\Cienspeaks \direct{Acabloù} Eh\ldots 

\Simonspeaks \direct{Fier de llu} Pappa!  T’isse deu-me devàn de tcheu sise problème, mé pouavo fé quetsouza. Mé que n'i itedjà, n'ayò la soluch\'on!

\Cienspeaks Acouta, mi pe plèizì, mi queutta-mé pédre, té é tia icoula!

\Mariettospeaks Mi ouè queutta ité pappa!

\Simonspeaks Pappa, crèi-mé, mé si sen que fé di noutre vatse! 

\Cienspeaks \ldots é senque te vourie fé?

\Mariettospeaks \ldots é senque te vourie fiye?

\Simonspeaks Ad\'on mé, mé\ldots mé que n'i fé de-z-iteudzo tchica économique pouì pa fére tan pe seutta situach\'on. Mi n’i de-z-amì djeusto laouréoù que l’an itedjà sen que servèi a no.

\Cienspeaks \ldots é pe senque?

\Mariettospeaks \ldots é pe senque?

\Simonspeaks Se te me baille papì blan pe le vatse, te spleucco totte de seu a eunna pouza.

\Cienspeaks Papì blan? Mi t'i ià de tita!

\Mariettospeaks Mi Tchièn n’en prouvoù eunna blita de bague pe noutre vatse! 

\Cienspeaks Noutre? \doute

\Mariettospeaks  Ouè le min-e, le tin-e\ldots ouè de no! Prouvèn sen que te di lo gars\'on!

\Simonspeaks Ouè pappa, dai!

\Mariettospeaks \direct{Eun baillèn eun cré creppe si l'ipala de Tchièn} Dai!

\Cienspeaks Te di?

\Mariettospeaks \direct{Tourna eun baillèn eun cré creppe} Dai!

\Cienspeaks \'E beun\ldots Vouè\ldots prouvèn eunc\'o so.

\Simonspeaks Ad\'on pappa vou tchertchì sise vétérinéro que n’i deu-te é no véyèn de seu a pocca. Salì! Salì Marietto!

\StageDir{Simon chor.}

\Mariettospeaks Te vèi que me fayè itedjì eunc\'o mé.

\Cienspeaks Mi ouè té! L’ian d’atre ten can l'ian dzouvin-o no, Marietto! Aprì Marietto fa diye que qui pou savèi di vatse de pi que no dou? De botcha vétérinéro? Ah, ah, ah!

\Mariettospeaks Ah, ah, ah! \'E ad\'on\ldots santé! 

\StageDir{Marietto, douàn que levé lo vèyo, lo eumplèi tourna de rodzo.}

\Cienspeaks L'è bon si rodzo eh! Santé!

\StageDir{Le dou bèyon eun creppe.}

\Cienspeaks Te sa senque pourian fé ara? Alèn ba i parque, te fio vére Cobra: mondjeu que dzenta que n'i aprésto-la! L'è presta pe lo concoù! Veun veure!

\StageDir{Le dou van ver lo parque. Douàn de chotre euncrouijon Sahlam que entre deun la lèitì.}

\Sahlammspeaks \direct{Eun se plégnèn} Todzor a fé la conta sise dou li! Mimi é Coco! Travaillì todzor le mimo!

\StageDir{Sahlam avèitse la tabla \tabla.}

\Sahlammspeaks Ah lasì pocca, touteun de veun!

\StageDir{Sahlam pren eun di dou vèyo é agoutte.}

\Sahlammspeaks Bon deh!

\StageDir{Aprì èi bi eunna goloù, Sahlam vouidje lo sidel demì plen de lasì deun la tsaoudée é beutte an mia de caille.}

\Sahlammspeaks Ad\'on comme di Tchièn, eun tchitchareun de caille eun poussa pe 300 litre de lasì\ldots \direct{Avèitse douteu le pocca litre de lasì} prouèn can mimo a beuté eun tchitchareun é véyèn comèn vat!

\StageDir{Pe queriaouzitoù Sahlam nieufle la bouite di caille. La réach\'on que chouivèi l'è la souivante: Sahlam iternèi é to lo caille fenèi dedeun la tsaoudée.}

\Sahlammspeaks Oh mondjeu senque n'i combinoù! To lo caille dedeun 20 litre! Fa to de chouite viondì!

\StageDir{Pren lo modeun é viounde lo lasì, mi aprì do seconde lo lasì veun deur comme de bèrio \sasso\ é Sahlam areuve pamì a terì foua lo modeun.}

\Sahlammspeaks Oh a forse l'a caillà!

\StageDir{Avouì totte sé forse areuvve a teri-lo foua! Lo beutte eun caro é s'avèitse a l'entor.}

\Sahlammspeaks Bon alèn ià de sé!

\StageDir{Sahlam bèi eunc\'o eunna gotta de rodzo é chor ver lo parque iaou euncrouije Marietto é Tchièn.}

\Cienspeaks \direct{A Sahlam} Poste?

\Sahlammspeaks Ouè bièn aloù, pa fé de mou!

\StageDir{Sahlam chor.}

\Cienspeaks \direct{A Marietto} Sit an le fontin-e son quetsouza\ldots

\Marietto Ah nen si quetsouza eunc\'o mé!

\StageDir{Le dou entron deun la mèiz\'on di lasì, douàn Marietto é aprì Tchièn.}

\Cienspeaks Entra maque Marietto!

\Mariettospeaks Ouè tante si di mitcho!

\StageDir{Le dou s'achouatton a tabla.}

\scene[-- Chef d’équipe Mottinho]

\StageDir{Arreve Iolanda.}

\Iolandaspeaks Oh Marietto drolo te vère pe seuilla l’aoura de maenda.

\Cienspeaks Ouè drolo!

\Mariettospeaks Ouéla Iole! Vatte?

\Iolandaspeaks Vouè totte amoddo. Te pense de t’arritì eunc\'o oueu a maenda ?

\StageDir{Marietto l’è preste a diye de vouè mi Tchièn lo antisipe.}

\Cienspeaks Na, oueu na Iolanda. Oueu n’en de grouse novit\'o é eunc\'o bièn eumportante! Dèyon arevé avouì Simon, n'en pa lo ten de medjì!

\Iolandaspeaks Comme vouillade. Vi que gneun l'a beuttoù si le pate, n’i terià foua di conjélateur  de bon cannell\'on é son djeusto chortì di for.

\Mariettospeaks Mondje le cannell\'on! Avouì lo ragoù?

\Iolandaspeaks Ouè!

\Mariettospeaks Avouì la béchamelle?

\Iolandaspeaks Ouè!

\Mariettospeaks \direct{A Tchièn, eun se levèn} Ad\'on mé vou avouì\ldots

\Cienspeaks \direct{Eun tegnèn Marietto pe eunna mandze} Na, na, Iolanda tracacha-te pa pe le cannell\'on! Le medzèn a sin-a, tante Marietto l'è pi ià!

\Iolandaspeaks Ah comme vouillade, le meudzo pi tcheu mé!

\StageDir{Iolanda chor é Marietto, bièn eunfastedjà \arrabbiato, se gave lo tsapì, lo boueuche countre la tabla é s'achouatte démoralizoù. Eun mimo ten, di coutì di parque, entre Simon avouì lo fameu Chef d’équipe Mottinho.}

\Simonspeaks \direct{Eun braillèn} Pappa, pappa!

\Cienspeaks Mondjeu, mondjeu l'è Simon, l'è arrevoù avouì le vétérinéro! Veun Marietto, vito!

\Mariettospeaks Ouf, oueu sen totte a galoppe!

\StageDir{Tchièn é Marietto, a galoppe, van ver lo parque.}

\Simonspeaks \direct{Eun braillèn} Pappa, sitte l’é lo vétérinéro Mottinho.

\Chefspeaks \direct{Eun baillièn la man a tcheut} Bondzor a tcheut. 

\Simonspeaks \direct{Emochon-où} Bondzor!

\Mariettospeaks Marietto!

\Chefspeaks Mé si lo Chef d’équipe Mottinho. N'i prèi satte \textit{lauree} é s’i eun di pi capablo é aprestoù vétérinéro de la Val d'Outa. S’i seuilla a voutra dispozich\'on pe rezoudre tcheutte voutro problème avouì le bitche pe djeusto 128 \textit{euro} la meneutta.

\Mariettospeaks \direct{A Tchièn} Eunna fontin-a la meneutta. Que onéto!

\StageDir{Dimèn areuvve eunc\'o Sahlam que icaoue l'entrada de mèiz\'on.}

\Cienspeaks \direct{Eun prégnèn Simon a coutì de llou} Acoutta Simon, \direct{tracachà} me semble tchica tcheur!

\Simonspeaks Mi pappa! Te dèi pa avèitchì l'aspé économique. Te sa beun comèn van seutte bague: lèi queuttèn pouì djeusto eun papì que l’è restoù avouì no é la Réj\'on no paye pouì caze totte.

\Cienspeaks Ah bon se l’è pouèi. Bon, ad\'on\ldots \direct{eun s'avèitsèn a l'entor} Sahlam!

\StageDir{Sahlam, que l'ie eun tren d'acouté, se beutte si l'\og attenti\fg.}

\Sahlammspeaks Vouè!

 \Cienspeaks Mi que te fé li dret?
 
 \Sahlammspeaks Sayoù eun tren d'icaoué!

 \Cienspeaks Mi te vèi pa que l'è arevó lo vétérinéro?!\\ \direct{Malechà \malecha} Pourta eun sé le bitche! Fé-me pa fé seutte fegueue!
 
 \StageDir{Marietto, que l'è todzor i mentèn di pià, fé eunc\'o llou semblàn de reprodjì Sahlam.}
 
 \StageDir{Sahlam chor foua eun galopèn.}

\Chefspeaks Bon tornèn a no? Mesieu Tchièn, v'ouite vo eh? Bièn voutro gars\'on l’a spleuccou-me tchica la situach\'on é ad\'on n’i pourtoù avouì mé le tecnisièn que me servisavon de pi. Pouì le fé euntrì?

\Cienspeaks Ouè, ouè!

\Chefspeaks Ad\'on vo le prézento: vétérinéro Gilles vegnade!

\StageDir{Entre eun galopèn lo vétérinéro Gilles. Ite jamì rito é saoutaille to lo ten. L'a la blouza di medeseun, mi déz\'o l'è arbeillà sportivamente.}

\Maracanaspeaks \direct{Eun sarèn la man a Tchièn} \textit{Hi}\footnote{ Liye eun angllé. Si vétérinéro salieré todzor eun angllé.} Tchièn!

\Cienspeaks V'ouèide prisa?

\Maracanaspeaks Na, na! \direct{Eun salièn Marietto} \textit{Hi}!

\Mariettospeaks \textit{Hi}!

\StageDir{Gilles se plache protso de Mottinho.}

\Chefspeaks Lo sec\'on l'è véterinéro Schulz!

\StageDir{Lo vétérinéro Schulz entre.}

\Schulzspeaks \direct{Ver lo pebleucco, eun bouéchèn for eun pià\piede\ pe téra é avouì acsàn tédesque} Ai!

\StageDir{Lo vétérinéro Schulz salie Tchièn é Marietto eun sarèn la man.}

\Schulzspeaks \direct{A Tchièn} Ai!

\Schulzspeaks \direct{A Marietto} Ai!

\Mariettospeaks \direct{Eun braillèn di mou pe la man tro saraye de Schulz} Ahi!

\StageDir{Schulz se plache protso a Gilles.}

\Chefspeaks \ldots é lo trèijimo l'è vétérinero Trinchero!

\StageDir{Entre lo vétérinéro Trinchero avouì le-z-\textit{infradito}, pantalon\-tcheun tanque le dzegnaou, pèi \textit{rasta} é bounet jamaiquèn.}

\Trincherospeaks \direct{A Tchièn} Salù!

\StageDir{I contréo, lo vétérinéro Trinchero salie Marietto avouì eun djouà di man juvenile: se baillon lo sinque, deun coutì é de l'atro, é eun creppe de panse eun countre l'atro. Aprì èi salià, lo vétérinéro Trinchero se plache protso le-z-atre collègue.}

\Chefspeaks Voualà, ara vo prédzon de leur spésializach\'on. Vo dou \direct{a Gilles é Trinchero} alade maque vo presté dimèn que comenchèn avouì Schulz.

\StageDir{Lo vétérinéro Trinchero é Gilles chorton.}

\Chefspeaks Simon alèn veure sise papì \documenti\ que t'ayè deu-me?

\Simonspeaks Ouè alèn veure!

\Chefspeaks Bièn. \direct{A Tchièn} Fiade entré le vatse!

\StageDir{Simon é Mottinho entron deun la mèiz\'on, eun quetèn solette Tchièn é Marietto avouì lo vétérinéro Schulz. Dimèn, to todzèn, Sahlam pourte le vatse eun \textit{scène}.}

\scene[-- Vétérinero Hermann Schulz]


\Schulzspeaks \direct{A Tchièn é Marietto} Salù! Ad\'on mé si lo docteur vétérinéro Hermann Schulz é si laouréoù eun psicolojì de la vatse é stress di baou. I dzor de oueu eun argumàn todzor pi eumportàn pe la bon-a calétoù de la rase.

\Cienspeaks An\ldots

\Schulzspeaks Pouì comenchì la min-a térapì?

\Cienspeaks Vouè, vouè, fiyade maque.

\Schulzspeaks Ad\'on n’i djeusto fata de tcheu vo pe beutti-me voutre vatse totte a l'entor de mé. Comme eun serclle.

\Mariettospeaks \direct{Sensa se fé sentì da Schulz} Me semble eunna bagga tchica drola a mé.

\Cienspeaks Marietto! Lèicha travaillì lo vétérinéro! Pitoù va baillì eunna man a si Sahlam lé!

\StageDir{Marietto èidze Sahlam a plachì le vatse douàn Schulz.}

\Schulzspeaks Ad\'on comenchèn avouì Cobra. Di-mé viyo de-z-àn t’a?

\Cobraspeaks Meuh!

\Schulzspeaks Bièn, ouette!

\StageDir{Tchièn réajèi ibaì \ouaou!}

\Cienspeaks Ouè mi lèi crèyo pa!

\Schulzspeaks \'E di-mé perqué t’a de problème de t’accoblì avouì de bou?

\StageDir{Cobra di quetsouza i bouigno de Schulz. Eunc\'o Tsardoùn é Piccina diyon quetsouza eun secré a Schulz.}

\Schulzspeaks\direct{A Tchièn} Squezade pouide tornì tcheut d’eun la mè\-i\-z\'on di lasì perqué la vatse l'a de problème a predzì avouì totte seutte dzi.

\Cienspeaks Le vatse l'an de tracasse a prédjì?

\Schulzspeaks \direct{Todzor avouì acsàn tédesque} Ouè! Le vatse l'an de tracasse a prédjì!

\Cienspeaks \direct{Mortifià} Oh, squezade! Marietto, Sahlam! Quetèn lo vétérinéro prédjì avouì le vatse.

\Mariettospeaks Mi mé vouillavo djeusto sentì sen que l'ayèt a lèi diye.

\Cienspeaks Sah, Marietto! Veun que te voueudzo eun vèyo!

\Mariettospeaks Ah! Areuvvo!

\StageDir{Se nen van tcheutte deun la mèiz\'on, sof Sahlam que chor foua ver le pro a drèite, é queutton le vatse avouì lo vétérinéro que l’èi prèdze. Partèi:}

\sound{https://www.youtube.com/watch?v=OkLgTsbCxik}{Il Ritorno Del Ringo - Nazca}

\StageDir{Lo vétérinéro Schulz acoutte totte le vatse, lequelle, sensa prédjì, maque avouì de jeste, se plègnon de sen que va pa. Aprì èi acoutto-le totte é trèi, le ipnotize avouì de movemèn de la man que le trèi vatse chouivèison euntsantaye. A la feun de la mezeucca, Schulz fenèi sa térapì é le trèi vatse son deun étà de transe: son rite é baoudzon maque la tita a drèite é a gotse. \\ Schulz torne criyì tcheut deun la mèiz\'on di lasì, iaou Simon, Mottinho, Tchièn é Marietto sayòn eun tren d'avèitchì de papì.}

\Schulzspeaks Vegnade maque.

\Cienspeaks Ouèi fé?

\Schulzspeaks Ouè!

\StageDir{Marietto é Tchièn galoppon ver lo parque.}

\Cienspeaks \direct{A Schulz} Ouè diade-mé queun l’è lo problème.

\Schulzspeaks De problème n’a tan. Le vatse l’an fran perdì lo leur caratéro é se senton sfruttaye sensa pamì eunna rèiz\'on de viya é, cheurtoù, vouillon chédre leur lo leur bou.

\Cienspeaks Mi\ldots comèn fiyo?

\Mariettospeaks\direct{Todzor i mentèn di pià} Eh, comèn?

\Schulzspeaks Tracachade-v\'o pa! n’i totte. Ad\'on \direct{avèitse le vatse}  Cobra, Piccina é Tsardoùn!

\StageDir{Le vatse son eunc\'o deun eun étà de transe, mi Schulz, avouì eun claquemèn \claque\ di dèi, le fé tournì deun lo prézèn.}

\Schulzspeaks\ldots ara vo fio vère le bou que n’en a dispozich\'on é vo vo lo cherdéré. Tot comprèi?

\StageDir{Cobra, Piccina é Tsardoùn repondoun avouì eun \og meuh\fg!}

\Schulzspeaks Bièn, ad\'on Vidéo!

\StageDir{Partèi eunna vidéo iaou trèi bou diféèn se prézenton: lo bou chouisse, bièn fier, dzen, prestàn é rebeusto; lo bou piémontèis, drolo mi seumpateucco; lo bou valdotèn, croué bagga é finque tchica euntredeutte.}

\StartVideo{https://www.youtube.com/watch?v=WFADoepxN1c}{Catalogue Boeuf}

\Schulzspeaks Ad\'on, mé firiyo chédre pe premie a la rèina di baou. Cobra totse a té! Queun bou t'a cherdì??

\Cobraspeaks\direct{Eun moutrèn Boston} Meuh!

\Schulzspeaks An t’a cherdì Boston, lo bou chuisse. Piccina ara totse a té, vi que t’i la rèina di lasì seuilla. Queun bou t'a cherdì??

\Piccinaspeaks\direct{Eun moutrèn Birillo} Meuh!

\Schulzspeaks Bièn! Birillo si piemontèis. Voualà que a Tsardoùn totse Tanteun, lo bou valdotèn.

\StageDir{Tsardoùn l'è bièn ba de moral é plaoue.}

\Schulzspeaks\direct{Eun consolèn Tsardoùn} Tracacha-té pa! Te vèi pi que l'aré bièn de caletoù catchaye! \direct{A Tchièn} Bon mé sé n'ario fenet, n'i djeusto fata de eunna baga: que vo a feun senà m'atsetade pe voutre vatse eun ordinateur portablo; pouai l'an pi comoddo se cougnitre avouì le bou cherdì\ldots pe mézo de la \textit{chat}!

\Cienspeaks De tsat?

\Schulzspeaks La \textit{chat}!

\Mariettospeaks\direct{A Tchièn, eun fièn seumblèn de savèi} Ouè mi deh, la tchat!

\Cienspeaks\direct{Dizorientoù} \ldots d'acor comme vouillade!

\Schulzspeaks Tracachade-vo pa! Véyade aprì que rezultà aprì to so! Ad\'on mé marco le-z-aoure beutto trèi, fiyèn sinque va, tante no paye la Réjón. Va bièn igale?

\Cienspeaks Vouè, vouè, va bièn, fiyade vo que vo sade.

\StageDir{Schulz sare la man a tcheutte é se nen va. Dimèn Simon é Mottinho chorton de la mèiz\'on di lasì.}


\Simonspeaks Pappa t’a vi? So vouè que l’è lo tsandzemèn que te prédzavo. Te te ren contcho iaou pouèn arrevì no que n’en itedjà?

\Cienspeaks Vouè, vouè. S’i fran vioù se vèi.

\Mariettospeaks Ah ouè, t'i fran vin-ì vioù!

\scene[-- Vétérinéro Gilles]

\Chefspeaks Vouillèn veure lo sec\'on vétérinéro? Lo ferio entré é véyèn sen que no di. Vétérinéro Gilles, vegnade!

\StageDir{Entre eun galopèn lo vétérinéro Gilles. Marietto saoutaille comme Gilles.}

\Maracanaspeaks Bondzor a tcheutte. Mé si lo docteur vétérinéro Gilles, frique frique d’étude. Si djeusto laouréoù eun sosializach\'on é mouvemèn pe la vatse.

\Cienspeaks Pe la vatse?

\Maracanaspeaks Ouè pe la vatse!

\Mariettospeaks Pe la vatse!

\Maracanaspeaks Ouè ad\'on, seutte vatse le vèyo tan tan rèide. Leur l’an fata de caro\ldots

\Mariettospeaks Caro!

\Maracanaspeaks \ldots é de caqueun que le idjisse a fé chotre la voya de vivre!

\Cienspeaks Voya de vivre?

\Mariettospeaks\direct{Todzor eun saoutèn comme Gilles é bièn euntouziaste} Mi ouè, la voya de vivre!

\Maracanaspeaks  Ad\'on mé n’i dza aprestoù dedeun lo meun oufficho eunna mezeucca avouì eun course que pou le-z-èidjì a itre pi sportive!

\Mariettospeaks Sportive!

\Cienspeaks Sportive?
  
\Maracanaspeaks Ouè, se v'ouite d’acor mé comencherio la térapì!

\Cienspeaks Vouè fiade maque.

\Maracanaspeaks Ok n'i djeusto fata de tchica de caro!

\Mariettospeaks Va bièn tramèn la tabla!

\StageDir{Gilles pouze l'abrosaque é teurie foua trèi jupe awouayenne pe le vatse, que, totte contente, se le beutton.}

\Mariettospeaks \ldots é ad\'on, \textit{one, two and one, two, three}, mezeucca!

\StageDir{Partèi la tsans\'on réalizaye pe Le Digourdì:}

\sound{https://youtu.be/HirTuF1oA6g}{Waka Watse}

\StageDir{Gilles, i mentèn di palque, danche é fé vère le premì pa. Le trèi vatse comenchon a aprende le pa é se queutton alì. Tcheut le-z-atteur (eunc\'o Iolanda, Sahlam, Trinchero é Schulz) se fion prende é comenchon a danchì, mouèn que Tchièn que l’è lo seul que l’è pa tan cheur di bon rézultà.}

\StageDir{Fenia la tsans\'on tcheutte tournon i leur poste, iaou sayòn douàn.}

\Maracanaspeaks Ok! Parfé!

\Mariettospeaks So te pren eh! Que chouéyaye!

\Cienspeaks Vouè, mi squezade vétérinéro mi l’è garantì si prodouì ?

\Maracanaspeaks Mi ouè l’è sertifià. Mé l’è dza lo catrimo baou que viondo seutta senà avouì ma térapì é son tcheutte reustoù contèn. 

\Cienspeaks Tcheu contèn?

\Maracanaspeaks Ouè! Pe la fatteura beutto caque-z-aoure eun pi eh, tante\ldots

\Mariettospeaks \ldots tante paye la Réjón!

\Cienspeaks Ouè, ouè, tante l'è la modda de la dzournoù!

\Mariettospeaks Ad\'on ok, salù a tcheutte, salù!

\StageDir{Gilles chor foua eun galopèn.}

\scene[-- Vétérinéro Trinchero]

\Chefspeaks Bièn, mé ferio entré lo dérì vétérinéro, pouèi fenissèn la térapì: vétérinéro Trinchero, vegnade!

\StageDir{Lo vétérinéro Trinchero areuvve pa.}

\Chefspeaks Vétérinéro Trinchero, vegnade!

\StageDir{Lo vétérinéro Trinchero areuvve pa.}

\Tuttispeaks Trinchero!

\StageDir{Entre lo vétérinéro Trinchero avouì an flemma que la mèitchà baste.}

\Trincherospeaks\direct{Bièn relassoù} Ouè salì, mersì!

\Mariettospeaks Mi Trinchi!

\Trincherospeaks\direct{I vatse} Mé si lo Medeseun vétérinéro Trinchero é\ldots

\Cienspeaks Squezade! Si seu!

\StageDir{Trinchero se vionde ver Tchièn.}

\Trincherospeaks Ah ouè.  Salì, mé si lo Medeseun vétérinéro Trinchero é si lo pi vioù vétérinéro euntre no perqué n’i fallì voyadzì de-z-àn é de-z-àn pe trouvì la soluch\'on a seutte vatse que l’an todzor de problème. Diyèn que si restoù cattro, seunqu'an an eun Hollande iaou n’i accapoù l’erba \cannabis\ pi boa pe voutre bitche. Dièn que miclaye avouì d’agroù é de dzén-épì de voutre montagne baille eun prodouì euncomparablo. To so pou prende lo poste di souye que ba\-illa\-de tcheu le dzor a voutre vatse sensa problème.

\Cienspeaks Mondje!

\Trincherospeaks N’i djeusto fata de savèi se v'ouèi fata ara ou pouade attendre caque dzor perqué va comme lo pan.

\Cienspeaks Vi que sen eun tren fièn totte ara!

\Trincherospeaks Bièn. Ad\'on fièn l’odre desù Internet é d’eun momàn dérriye arrevì totte.

\StageDir{Trinchero teurie foua lo portable é fé l'odre.}

\Cienspeaks \direct{A Marietto} Internet? Te vèi i dzor de oueu que bague? Mé que l'io eunc\'o acotemoù a alì prende eun tchi Paolina a Gressan le bague que t’ayè fata. 

\Mariettospeaks Ah se te pren eunc\'o té Internet lo Pezzoli va clloure.

\StageDir{Sonne la pourta.}

\Cienspeaks Qui l'è que soun-e a mèiz\'on, Simon te va tè?

\Mariettospeaks Ouè va vère!

\Simonspeaks Ouè pappa. 

\Cienspeaks Atégnao ren mé.

\StageDir{Simon entre avouì eun paquet.}

\Simonspeaks Pappa l’iye lo postiill\'on que l’a portoù si paquet.

\Cienspeaks Ah la fèi n'i comprèi. Seràn torna salle pomate de la \og Djeust\fg que pren to di lon la fenna! Se son sen le féo tchapé eun vaoulo!

\Simonspeaks Mi na pappa! No n'en itedjà! Seutta l'è la tecnolojì: son le-z-erbe que t’a djeusto prèi avouì Mesieu Trinchero.

\Cienspeaks Mi lèi crèyo pa!

\StageDir{Tchièn baille lo paquet a Trinchero.}

\Trincherospeaks Eh ouè son dza le-z-erbe de mé. Ah, sentade que parfeun!

\StageDir{Anneuflon tcheut.}

\Mariettospeaks Mondjemé n’i dza mou a la téta Tchièn!

\StageDir{Trinchero baille l'erba i vatse. Son de grou can\'on \spinello\ i\-pes\-se que Trinchero aleumme avouì lo briquette \accendino. Eunc\'o Sahlam nen pren eun.}

\Trincherospeaks Voualà eunna pe eun. Fa le-z-alllemì avouì lo briquet é lo djouà son fé. Aprì eun dzen momàn de mouvemèn, so l’é sen que vo idze a vère to roze. To so se criye térapì \textit{Rave}\ldots Na! Rèina \textit{party}. N’i eunc\'o eunna mezeucca pe l’occaj\'on!

\StageDir{Partèi la mezeucca:}

\sound{https://www.youtube.com/watch?v=RIMxmnfDSOs}{Jamming - Bob Marley \& The Wailers}

\StageDir{Totte le vatse comenchon a alé d’eun coutì a l’atro di palque comme se fusson drogaye é bièn relassaye. Eunc\'o Trinchero se relasse. Dérì Sahlam é Marietto prouon a pipé eunc\'o leur.}

\Trincherospeaks Ah, n’i dza comandoù eunna conféch\'on de seutte pe senà pe to l’an, pouèi v'ouite tranquilo pe tchica de ten. 

\Cienspeaks Mondjeu!

\Trincherospeaks Pe la prestach\'on\ldots

\Cienspeaks Ouè, ouè, si dza totte!

\Trincherospeaks Va bièn salù a tcheut!

\StageDir{Trinchero chor.}

\Chefspeaks Voualà, ara que vo vèyo pi tranquilo é  que seutte bitché l’an tchandjà plima, me nen allério eunc\'o mé. Vo queutto maque eun per de  bolleteun pe pouèi paì eun 36 comode rate.

\Cienspeaks Que jantila!

\StageDir{Mottinho teurie foua de papi é le baille a Tchièn.}

\Cienspeaks A vo! Payade maque avouì calme. Salù é mersì!

\Simonspeaks Pappa l’accompagno mé a la porta.

\StageDir{Simon é Mottinho chorton.}

\Cienspeaks Deh Tchièn! Avèitsa: vaoulo!

\StageDir{Marietto se beutte a saouté i mentèn di palque eun bouéchèn le bri comme de-z-ale!}

\Sahlammspeaks Eh patroùn! Avèitsa, si to blan Tchièn! 

\Mariettospeaks \ldots é Sahlam l'è viì to blan! 

\StageDir{Tchièn sa pa senque die é le-z-avèitse éton-où é deperdù \confuso.}

\Mariettospeaks Sahlam alèn ià!

\StageDir{Marietto chor eun vaoulèn avouì Sahlam. Tchièn se beutte le man douàn lo vezadzo, dispéoù é preste a plaoué.}

\StageDir{Teuppe \lemieBa .}

\StageDir{\Fv{Dou mèis aprì}}

\StageDir{Lemie \lemieSi .}

\scene[-- La morale]

\StageDir{Le vatse son tsandjaye. Eun scène n'a doe caèye di solèi avouì le vatse itaoulaye desì é eun ombrell\'on \ombrellone. Eunna vatse l’a la bandjèira de la CGIL.}

\Cienspeaks \direct{Moral a téra, i pebleuque} L’é dza tchica que va eun devàn seutta conta é le bitche de mé son todzor pi drole é te areuvve pamì a lèi fi fiye sen que te vou. Avèitsa Cobra, ara que l’é icrita i sindacat bare fran pamì ren. Pa que devàn l’isse itaye eunna gran rèina mi ara eumpeugne djeusto se te lèi promé pe icrì sinque potchà de \textit{mangime} pe souye; é Piccina é Tsardoùn reston to lo dzor i solèi é dèyo eunc\'o atseti-lei eunna pomatta pe le berleu di solèi. Mi cheurtoù ara que son icrite a la CGIL me baillon lo lasì eun cou pe dzor é djeusto can l’an voya leur\ldots é fenèi pa seuilla. Son doe senà que me queutton pa entrì dedeun lo parque é perqué? Perqué le-z-itreuillo pa tcheu le dzor é leur lèi teugnon tan a itre prope.  Sise drouet l’an rovin-ou-me-lé\ldots \direct{Pause} na mi si cou, praou seutta bagga! \direct{Malechà é déterminoù} Ara fa tsandjì! Si cou\ldots ara Simon me sen! Si cou si botcha va me sentì!

\StageDir{Tchièn va dedeun la mèiz\'on.}

\Cienspeaks Simon, Simon veun seuilla!

\Simonspeaks Ouè pappa arreuvo.

\Cienspeaks Avèitsa sen!\direct{Moutre lo parque} Avèitsa sen que v'ouèide combin-ou-me i vatse!

\Simonspeaks Mi comèn sen que n’en fé? N’en fé-le restì mioù é itre contente avouì tcheut. 

\Cienspeaks Ah reston mioù? Reston troppe amoddo! Si cou la fèi n'i fenì eunc\'o d'avèi eun icouila de lasì pe dzor. 

\Simonspeaks Mi pappa! Te fa pa avèitchì vio te ren la vatse, mi se reste bièn avouì le-z-atre é se l’é contenta de sa viya!

\Cienspeaks Ouè! Mi rapella-té que tanque a oueu l'ie eunc\'o mersì a salla pégna icouila de lasì que t'a pouì itedjì tanque a 40 an! 

\Simonspeaks\direct{Pi tracachà pe son pappa} Mi pappa fa pensì a la nateuva. Fa pensì que leur comme no dèyon itre libre de féye sen que voulon.

\Cienspeaks Praou, praou Simon! n’i fran comprèi que lo mondo l’a tchandjà. Lamao tan lo Simon que l’ayè nou-z-àn é criyave pappa, pappa pourta-mé ba i baou que veugno avouì tè baillì lo fen a la viille Poudre. Poudre\ldots Poudre te la lamave é lleu lamave té. Te chortave de l’icoula é te galoppa-ve eun tsan pe pouèi la vère é l’accaéchì. T'iye bièn dzouvin-io mi bièn pi sensiblo!

\Simonspeaks Mi\ldots

\Cienspeaks Acoutta! N’i desidoù eunna bagga. Té é le teun amì vétérinéro dèi oueu queutta-lé maque louèn de si baou é\ldots a si Sahlam lèi accapèn pi caque d’atre travaille. Si noureun vouì lo vardì  eunsemblo a ta mamma. Vouì lo traillì tcheu le dzor é vouì appréchì le-z-émoch\'on que te fan prouì le bitche tcheu le dzor\ldots que chouèn te baillon bièn pi di dzi\ldots

\ridocliou

\DeriLeRido
\RoleNoms{Chorégraphie}{Jo\"elle Bollon}
\RoleNoms{Costume}{Roberta Charrere, Ornella Grivon}
\RoleNoms{Mezeucca}{Josette Barailler}
\RoleNoms{Tramamoublo}{Jean-Pierre Albaney, Louis Bollon, Jerome Saccani}

\end{drama}