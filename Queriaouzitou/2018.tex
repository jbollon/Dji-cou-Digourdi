\queriaouzitou{
\begin{itemize}
\item[$\bullet$] La grafie di titre de la pièse l'è pa correcte:  \og Todzo\fg{} se devriye icriye \og Todzor\fg{}. Dièn pa lo non di/de la responsablo/a de sit erreur pe pa provoqué de discuch\'on inutile. Can mimo, va la pèin-a de soulignì pequé n’en desidoù de vardé lo titre trompoù. Pe doe rèiz\'on: premì, pe no rappelé de fé pi attenchón avouì la grafiye di teste que no publièn; secón, di momàn que la pièse counte de fasón \textit{autoironique} sen que vou deu itre Digourdì, eun titre avouì eun petchoù erreur l'ie bièn reprézentatif di sujé de la pièse mima.\\

\item[$\bullet$] Deun la scène X - La Nite di-z-Oscar, noutro Paolo Cima Sander repette, deun eunna meneutta é demì, nou cou \og can mimo\fg{} é chouì cou \og eumpourtàn\fg{}!

\refstepcounter{videos}
\begin{figure}[H]
\vspace*{-5pt}
\centering
\begin{subfigure}{.75\textwidth}
\centering
\video\hspace*{0.5mm} \textsc{\small Can mimo vs Eumpourtan}\hspace*{0.5mm} \video\\\vspace*{1mm}
    \qrcode[hyperlink, height=0.5in]{https://www.youtube.com/watch?v=tCJzWhbbfCQ}
\end{subfigure}%
\addcontentsline{vds}{section}{Canmimo vs Eumpourtàn}
\end{figure}

\item[$\bullet$] Deun l'avanspettaclle, le dou prézentateur son réel\-la\-men\-te lo Seunteucco é lo Vise Seunteucco de la Quemeua de Tsarvensoù. 
\end{itemize}
}