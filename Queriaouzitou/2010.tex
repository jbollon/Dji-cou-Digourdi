\queriaouzitou{
\begin{itemize}

\item[$\bullet$] Lo vrèi dzeudzo Santi Licheri, que Jo\"{e}l Albaney imite, l'è mor eun mèis douàn de la prézentach\'on de Forum Valdotèn (4 avrì 2010). Donque, sensa lo volèi, la pièse l'è itaye eunc\'o eun ommaje a la mémoué de Santi Licheri, fameu majistrà é personadzo télévisif.\\

\item[$\bullet$] Orijinellamente, le doe \textit{scène} prensipale sayòn deun l'odre eunvése: douàn la counta de Tchièn Frottapanse é aprì salla di dou-z-ipaou. I dérì momàn son itaye eunvertiye, vu que lo ritme de la premie l'ie pi ate.\\

\item[$\bullet$] La \textit{vidéo} de l'arrestach\'on di pouo Tchièn Frottapanse l'è itaye réalizaye avouì eun seul \textit{plan-séquence}\footnote{ \textit{Dans le plan-séquence, ces plans sont filmés sans interruption, ni montage. Un plan-séquence ne dispose donc pas de coupure et se caractérise par son action continue. Le spectateur peut ainsi suivre une action du début à la fin sans qu'il y ait d'arrêts}. Source: \href{https://www.cinecreatis.net/lexique/le-plan-sequence/}{CinéCréatis}.}. Catro meneutte de \textit{vidéo} que le-z-atteur son arrev\'o a eunterpretì to deun creppe. Mi l'è pa it\'o seumplo, di momàn que Rosina, Gilda é d'atre campagnar avouì l'Ape s'aprotsoon pe dimandé (mogà finque eun braillèn): \og Senque l'è capit\'o? \fg, \og Senque v'ouite eun tren de fé?\fg ou \og Pequé n'a le carabegnì?\fg.
\end{itemize}
}