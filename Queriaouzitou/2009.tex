\queriaouzitou{
\begin{itemize}

%\item[$\bullet$] Paolo Cima Sander l'ie estrèmamente ajitoù, tanque i poueun de secaillì ou pa arrevé a fé de bague seumple. Pe lo èidjì
% - non riuscivo a prendere la pressione (joel mi ha aiutato dicendo "tante n'en totte la dzornoù")
\item[$\bullet$] La \textit{scène} di maladdo \textit{psychiatrique} l'è itaye eumprovisaye deun la dérie proua jénérale sensa que Ester diise ren a gneun! Pouade vo imajiì la réach\'on é le riaye de totta la compagnì.

\item[$\bullet$] Lo Téatre Giacosa de Veulla l'a lo palque eunna mia dicllo ver lo pebleuque. Donque, tcheu le-z-objé avouì de raoue dèyon itre blocoù pe pa colaté. Malereuzamente, d'eunna \textit{scène} avouì bièn de trimadzo, le fren di raoue de la coutse de Geromine son digantsa-se. Bièn aloù que lo noutro Paolo Cima Sander l'è arrevoù  a vardé lo pèise de la coutse que l'ie eun tren de colaté ver lo pebleuque! Queriaou de savèi pe queunta \textit{scène} capite? Alade a lo tsertchì deun la \textit{vidéo} de la pièse\footnote{ QR code a padze \pageref{link}}.

\item[$\bullet$] Pe la pièse l'Opetaille moderno le Digourdì l'an rejistroù leur premiye parodì d'eun \textit{refrain} publisitère: ``Di Congo pe travaillì". La publisit\'o orijinelle l'ie seutta:

\begin{figure}[H]
%\vspace*{-5pt}
\centering
      \begin{subfigure}{.75\textwidth}
  \centering
    \video\hspace*{0.5mm} \textsc{\small Bollywood TVC - Rio Casa Mia}\hspace*{0.5mm} \video\\\vspace*{2mm}
    \qrcode[hyperlink, height=0.5in]{https://www.youtube.com/watch?v=EoVlYvr5JbI}
  \end{subfigure}%
\end{figure}

\item[$\bullet$] Sel\'on lo premì Prézidàn di Digourdì, Jerome Saccani lamè traillì derì le rid\'o pe pouèi veure le dzente feuille de la compagnì eun ganeuss\'on!
\end{itemize}
}

%