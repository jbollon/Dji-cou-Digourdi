\title{L’OPETAILLE MODERNO}
\author{Pièse icrita pe Jo\"{e}l Albaney}
\date{Téatro Giacosa de Veulla, 15 mi 2009}

\maketitle

\fotocopertina{Foto/2009/gruppo.jpg}{Paola Lucianaz, Francesca Lucianaz, Jo\"{e}l Albaney, Jasmine Comé, Paolo Cima Sander, Giada Grivon, Paolo Pession, Pierre Savioz, Ilaria Linty}{Valeria Brunod, Serena Giorgi, Jo\"{e}lle Bollon, Simone Roveyaz, Laurent Chuc, Ester Bollon, Marco Ducly}{2009}\label{link}

\LinkPiese{L'opetaille moderno}{https://www.youtube.com/watch?v=lP22_oK_hws&list=PLBofM-NS_eLJUln45l7VH457fGak_Bk5O&index=10}{.5}

\souvenir{Mon souvenir de l'Opetaille moderno l'è, sensa doute, salla sensach\'on que sayò eun tren de vivre eun sondzo, eugn'émoch\'on euncrouayable. L'ie lo premì cou que noutra compagnì, djeusto nèisiya, pouyè si eun vrèi palque\ldots é lo fé que si palque l'ie fran lo Téatre Giacosa, \textit{symbole} di téatro populéro valdotèn, l'ie eunna \textit{source} pouissante de jouà é responsabilitoù. Dze n'i eun dzen souvenir pe la collaborach\'on que n'en i avouì lo Charaban, eun particulié avouì Vittorio Lupi, Mauro Rossi é Giovanni Neri. L'an fé-no découvrì comme s'apreste a niv\'o professionnel eun spectaclle téatral, sourtoù pe sen que regarde la \textit{scénographie} é le costume. Me rappelo avouì plèizì que Aldo Marrari l'ayè chouivino deun la produch\'on de totta la pièse.

Deun mon \textit{cœur} l'Opetaille moderno l'a cheur eun caro eumpourtàn, pequé l'a dimoutro-me que n'a ren de pi dzen que traillì avouì de dzi é de-z-amì que l'an la mima pach\'on de té.}{Jo\"el Albaney}

%
\queriaouzitou{
\begin{itemize}

\item[$\bullet$] La \textit{scène} di maladdo \textit{psychiatrique} l'è itaye eumprovisaye deun la dériye proua jénérale sensa que Ester diise ren a gneun! Pouade imajiì la réach\'on é le riaye de totta la compagnì.

\item[$\bullet$] Lo Téatre Giacosa de Veulla l'a lo palque eunna mia dicllo ver lo pebleuque. Donque, tcheu le-z-objé avouì de raoue dèyon itre blocoù pe pa colaté. Malereuzamente, d'eunna \textit{scène} avouì bièn de trimadzo, le fren di raoue de la coutse de Geromine son digantsa-se. Bièn aloù que noutro Paolo Cima Sander l'è arrevoù  a vardé lo pèise de la coutse que l'ie eun tren de colaté ver lo pebleuque! Queriaou de savèi pe queunta \textit{scène} capite? Alade la tsertchì deun la \textit{vidéo} de la pièse\footnote{ QR code a padze \pageref{link}}.

\item[$\bullet$] Pe la pièse l'Opetaille moderno le Digourdì l'an rejistroù leur premiye parodì d'eun \textit{refrain} publisitère: \og Di Congo pe travaill\fg. La publisit\'o orijinelle l'è seutta:

\begin{figure}[H]
%\vspace*{-5pt}
\centering
      \begin{subfigure}{.75\textwidth}
  \centering
    \video\hspace*{0.5mm} \textsc{\small Bollywood TVC - Rio Casa Mia}\hspace*{0.5mm} \video\\\vspace*{2mm}
    \qrcode[hyperlink, height=0.5in]{https://www.youtube.com/watch?v=EoVlYvr5JbI}
  \end{subfigure}%
\end{figure}

\item[$\bullet$] Sel\'on lo premì Prézidàn di Digourdì, Jerome Saccani lamè traillì derì le rid\'o pe pouèi veure le dzente feuille de la compagnì totte eun ganeuss\'on!
\end{itemize}
}

\Scenographie
\begin{itemize}
\item[$\bullet$] 2 coutse d'opetaille é 1 coutse a doe plase bièn comodda \lettodoppio ;
\item[$\bullet$] 3 \textit{tables de nuit};
\item[$\bullet$] 2 armouére;
\item[$\bullet$] 2 caèye nèye eun plasteucca ;
\item[$\bullet$] 1 pourtamantì;
\item[$\bullet$] 1 poltronna bièn comodda;
\item[$\bullet$] 1 fondal téatral avouì eunna pourta i mentèn pe laquella le-z-atteur pouon entré é chotre;
\item[$\bullet$] 1 bourset pe le grou sou é 1 pe le sou tri;
\item[$\bullet$] 1 campaneun \campanellino ;
\item[$\bullet$] 3 botèille de \textit{gatorade} ;
\item[$\bullet$] 1 flute ;
\item[$\bullet$] 2 cabaré d’ardzèn: 1 pe pourtì eun \textit{cracker} é l'atro pe la cachoula de la \textit{bourguignonne} ;
\item[$\bullet$] 1 per de motchaou;
\item[$\bullet$] 2 platte de papì, de fortsette é de caoutì \posate ;
\item[$\bullet$] 2 cachoule: eunna grousa pe fé la puré é l'atra pi pégna pe fé bolequé  l'ouillo de la \textit{bourguignonne};
\item[$\bullet$] 1 vèyo é 1 paille.
\end{itemize}

\setlength{\lengthchar}{4cm}

\Character[PIERRE]{PIERRE}{Pierre}{Premì prézentateur, \name{Pierre Savioz}}

\Character[LAURENT]{LAURENT}{Laurent}{Sec\'on prézentateur, \name{Laurent Chuc}}

\Character[GEROMINE]{GEROMINE}{Geromine}{Eunna viille madama\viille\ recovéraye a l'opetaille, \nameF{Jo\"{e}lle Bollon}}

\Character[CASIMIR]{CASIMIR}{Casimir} {Vétchot \viou\ campagnar  recoveroù a l'opetaille, \name{Jo\"{e}l Albaney}}

\Character[ARISTOCRATE]{ARISTOCRATE}{PersEmpourtanta}{Mesieu bièn eumpourtàn, noble é aristocrate, recoveroù a l'opetaille, \name{Marco Ducly}}

\Character[GASTON]{GASTON}{Eunfeurmi}{L’eunfermì personnel de l'Aristocrate, \name{Laurent Chuc}}

\Character[PRIE]{PRIE}{Prie}{Lo prie de l'opetaille \prete , \name{Pierre Savioz}}

\Character[FENNE DI POULISIE]{FENNA DI POULISIE}{Fennepulisie}{Femalle arbeillaye comme le méccanisièn de la \textit{Ferrari}, eunterprétaye pe \textsc{Jasmine Comé}, \textsc{Giada Grivon} é \textsc{Serena Giorgi}. Giada é Jasmine feràn eunc\'o le-z-eunfermie de la \textit{psychiatrie}.}

\Character[STARTER]{STARTER}{Starter}{Cape di fenne di poulisie, \name{Simone Roveyaz}}

\Character[EUNFERMIE]{EUNFEURMIE}{Eunfeurmie}{Eunfermie de l'opetaille avouì eun caratéo d'eun maréchal, \nameF{Francesca Lucianaz}}

\Character[JOSETTE]{JOSETTE}{Felie}{La feuille de Geromine, \nameF{Ilaria Linty}}

\Character[MEDESEUN MITCHO]{MED. MITCHO}{MedMitcho}{Lo \textit{Doctor House} de l'opetaille, \name{Paolo Cima Sander}}

\Character[FERNANDA]{FERNANDA}{Fernanda}{Lo Chef de l’opetaille, \nameF{Ester Bollon} Ester ferè eunc\'o lo maladdo \textit{psychiatrique}.}

\Character[TSAMBR\`I]{TSAMBR\`I}{Tsambri}{Tsambrì personnel de l'Aristocrate, \name{Pierre Savioz}}

\Character[]{FENNA DI POULISIE I}{FennepulisieA}{\hspace*{0cm}}
\Character[]{FENNA DI POULISIE II}{FennepulisieB}{\hspace*{0cm}}
\Character[]{FENNA DI POULISIE III}{FennepulisieC}{\hspace*{0cm}}

\DramPer

\act[\avanSpect\ Avanspettaclle \avanSpect]
\StageDir{\hspace*{2.5em}Se senton de vouése que veugnon de dérì la tèila.}

\begin{drama}

\Pierrespeaks Laurent entra té, mé n'i pouiye.

\Laurentspeaks Pouiye? Feulla!

\Pierrespeaks \direct{Ajitoù\ajitou} Soplé va té!

\Laurentspeaks T'i tan grou é t'a pouiye de cattro personne que son vie no vére?!

\Pierrespeaks Ad\'on fièn na baga? Baillèn dou creppe a la moura!

\StageDir{Le dou fan eunna man a la moura (``tchisse, cattro, totta man, satte\ldots '').}

\Laurentspeaks Ah! Pierre, deusteun! Feulla!

\Pierrespeaks\direct{Eun prégnèn coadzo} Ad\'on vou mé!

\StageDir{Lemie \lemieSi\ desù lo \textit{proscenium}. Pierre chor de dérì la tèila.}

\Pierrespeaks Bonsoir a tcheutte!

\StageDir{Entre eunc\'o Laurent é se plache a drèite de Pierre.}

\Pierrespeaks\direct{A Laurent} Mondjemé véo de dzi oueu lo nite! N'ayoù pa la fèi tan pai!

\Laurentspeaks\direct{Eun avèitsèn lo pebleucco} N'ayoù pa la fèi tan pouèi!

\Pierrespeaks A mé l'ay\'on deu-me que dèi seu desù lo palque te vèyave ren de sen que capitae déz\'o lo palque! \direct{Ajitoù} Mi me semble pa! Se vèi totte!

\Laurentspeaks Eh ouè! Mi se t'avèitse amoddo trèi car de la sala son pa de Tsarvensoù!

\Pierrespeaks T'a rèiz\'on! Can mimo\ldots a Fin-is l'an pi eunc\'o de dzente feuille! \`Eita salle do lé eun premì feulla!

\StageDir{Laurent seuble i feuille.}

\Pierrespeaks Mi va savèi péqué n'a pi de dzi de Fin-is oueu!

\Laurentspeaks Pierre! Avouì tcheu le tracasse que le Tsarvensoulèn l'an i dzor de oueu, fegueua-té se l'an lo ten de vin-ì sé pe avèitchì no!

\Pierrespeaks T'a rèiz\'on! Son tcheu pi greundzo\ldots pe deue\ldots té pensa que eun cou t'ayè praou te réchì a satte é demì pe alé eun Veulla a ouètt'aoue. Ara, avouì seutta dzenta arionda que l'an fé-no ba i Pon-Suà te fa te réchì a chouì é demì! Péqué eun cou que t'i  ba, te fa baillì la présédanse i Gressaèn; aprì pi de 50 an que n'en i no si drouette! \'E comme se bastise pa, can te feullon devàn l'an eunc\'o eun grou sourì a 32 di \sorrisone .

\Laurentspeaks T'a rèiz\'on! Aprì te fa eunc\'o chotre demì aoua devàn di travaille pe alì a l'icoula prende le botcha é alì i mitcho pe fé medjì!

\Pierrespeaks Mi na! Senque te me di? L'è pa lo travaille di-z-ommo sitta! Son le fenne que dèyon pensé i megnadzo!

\Laurentspeaks \direct{Ironique} Eh brao! Eun cou l'ie pouèi! Aya a Tsarvensoù, dèi can la betoù si do formach\'on di femalle di fiolet, leur pi grou tracasse l'è si d'aprestì l'èima, baillì la biacca i fiolet é prenotì le campe réjonal pe la demendze.

\Pierrespeaks Ouè, ouè\ldots sitta cou va fran tott'a bal\'on!

\Laurentspeaks Te pou pa resté avouì seutta vya! T'i stressoù! Aprì fenèi que eun se beutte maladdo!

\Pierrespeaks Queutta pédre! Te la sa la dériye?

\Laurentspeaks Di-mé.

\Pierrespeaks Noutro Senteucco, Ennio Subet, dèi can l'a si qué que lo Ministre Brunetta l'è fé-se eunna loué si lo travaille, l'a voulì eunc\'o llou fé la sin-a.

\Laurentspeaks N'ayò sentì de seutta loué: la \textit{Subettina}. Espleucca vèi comèn fonchoun-e!

\Pierrespeaks Ad\'on, te sa que Brunetta l'a desid\'o d'aoumenté l'oréo de la \textit{mutua}; donque, lo noutro Senteucco la betoù na riilla pe tcheut le sitouayèn! Dèi ara can eun l'è maladdo, l'è maladdo i $100\%$! Te pou gnenca pamì alì ba a la crotta pe prendre eunna boua botèille de veun ou alì de foua pe couillì le pomme \mela .

\Laurentspeaks Ah ara n'i comprèi péqué l'è na senâ que vèyo pa lo meun vezeun di mitcho é pe tèra n'at totte le pomme pouriye.

\Pierrespeaks \ldots é fa fé attench\'on! Péqué lèi  son le vallet de veulla que pasoun, dzor é nite, mitcho pe mitcho pe verifiì. An effè, l'an pamì lo ten de pasì i mentèn de la parotse pe baillì la multa i machine parquédjaye mal!

\Laurentspeaks \ldots é comme se n'isse panco praou, n'a eunc\'o d'atro mé cher-z-amì, péqué eun cou que t'a aprést\'o sin-a, t'a baillà medjì i mèinoù é t'a lav\'o le-z-éze\ldots te l'amerie pa t'itaoulé eun momàn desù lo chofà pe avèitchì lo \textit{Grande Fratello}?

\Pierrespeaks Ouè!

\Laurentspeaks Eh na! Péqué totte le souaré, can si preste m'achouaté si lo chofà lo tseun comenche a djapì é molle pa tanque lo pourto pa féye eun tor deun lo péi!

\Pierrespeaks Mi senque te me di? Itèn pa pe eunna grousa Veulla! T'a praou baillì campa a la tsèin-a é lo tor se lo fé da solette!

\Laurentspeaks A Tsarvensoù?! Mi t'i matte! Fenèi que lèi teurion desì avouì lo fezì!

\Pierrespeaks Me rappelao pamì de seutta counta! \'E t'a sentì que n'at eunc\'o la finanse é le carabignì di bouque que son eun tren de cayì to pe l'er lo péi pe tchertchì seutte-z-arme?

\Laurentspeaks N'ay\'o sentì de seutta counta; é say\'o tellemàn tracachà que n'i falì euntéré deun lo courtì le dou pistolet a éve: le dou \textit{Super Liquidator} di botcha!

\Pierrespeaks Acouta vèi seuilla.

\StageDir{Laurent s'aprotse.}

\Pierrespeaks \direct{Eun moutrèn quetsouza déz\'o la djacca} Te vèi lo pistolet a éve que n'i seuilla?
Me fa todzor lo pourté aprì\ldots l'è pa denonchà.

\Laurentspeaks Oh pouff, que viya! No n'en eunc\'o baillà lo non Digourdì a noutra compagnì!

\Pierrespeaks Ouè que coadzo!

\Laurentspeaks  Spéèn que le tréze que son séilla dérì \direct{moutre la tèila} teugnon pi ate lo non di groupe!

\Pierrespeaks Can mimo douàn can sen chortì di tend\'on n'ay\'on tan pouye di dzi\ldots ara n'en finque prèi tro de coadzo.

\Laurentspeaks N'en eunc\'o prao deu-nen! Spéèn can mimo que la quemeua é tcheutte le Tsarvensolèn contenisa no baillì na man; piatro sit an l'è lo premì é dérì an que fièn lo \textit{Printemps Thé\^{a}tral}.

\Pierrespeaks Ouè! \direct{Eun avèitsèn la moutra \orologio} Ara senque te nen di se prézentisan la pièse?

\Laurentspeaks \direct{Eun avèitsèn lo seun pouse sensa moutra \footnote{ Malerezamente Laurent l'è oublia-se de betì la moutra, mi lo pebleucco l'a bièn riette pe seutta gaffe.}} Ouè l'è l'aoua de comenchì, mi devàn vourio fé na présizach\'on pe tcheu le Tsarvensolèn: tcheu sise manifeste que n'en bet\'o ia pe la parotse, l'è pa pequé gavon l'éve ou l'élétrisit\'o, mi l'è pe lo spétaclle de seutta nite! Tracachade-v\'o pa!

\Pierrespeaks Amoddo! Mogà no sistémèn an miya pe la prézentach\'on?

\StageDir{Pierre é Laurent se cllouzon la djacca é drichon lo papillon \papillon . Eun pi, Laurent se beutte si le poueunte di pià é Pierre se plèye si le dzegnaou de fas\'on que sisan ate tcheut dou igale.}

\Pierrespeaks \textit{Mesdames} é \textit{Messieurs} a vo ``L'opetaille moderno''. Pise icrita pe Jo\"{e}l Albaney.

\Laurentspeaks Boun-a souaré a tcheut!

\StageDir{Pierre é Laurent chorton. Teuppe \lemieBa .}

\act[Acte I]

\ridoiver

\scene[-- L'Aristocrate é le pouo matasse]

\StageDir{Partèi la tsans\'on:}

\sound{https://www.youtube.com/watch?v=cntvEDbagAw}{A Message to You Rudy -
The Specials}

\StageDir{Eun \textit{scène} n’a trèi coutse. Doe coutse, salle iaou son itaouloù Casimir é Geromine, son bièn seumple é protso n'a maque doe caèye é doe \textit{tables de nuit}; i contréo, la trèjima coutse (a gotse) salla de la personna eumpourtanta l’è d’ott\'on, queusseun de sèya, \textit{couvre-lit} avouì le frandze. Acoutì n'at eunna comodda poltronna é eunna \textit{table de nuit}. Pe completì la \textit{scénographie}, i fon di palque n'a dou-z-armouére eun feur, eun pourtamantì é eun grou fondal avouì eunna pourta i mentèn.\\ L'aristocrate l’è catchà déz\'o le queverte é l'è eun tren de drimì. Geromine l’et eun tren de fé lo tsaous\'on é lo vioù l’et eun tren de lie lo journal avouì le lenette ba desì lo na.}

\StageDir{Se sen pamì la tsans\'on.}

\Gerominespeaks \direct{A Casemir} L’è na senâ que si seuilla dedeun é si pa senque atègnon a me fé chotre.

\Casimirspeaks T’atten pi eunc\'o té comme tcheutte!

\Gerominespeaks Si pi praou coudzia. Mi mé sensa lo meun courtì é le mie dzeleunne \gallina\ si perdiya!

\Casimirspeaks \direct{Euntre lli} Eunna pi, eunna mouèn\ldots

\Gerominespeaks N’i pa comprèi\ldots senque t’a deu?

\Casimirspeaks Ren, ren! Diavo maque que te pouave te le porté avouì té! Omouente n’ario magà pouì féye de nouo discour avouì caqueun. Le-z-aoue comenchon itre londze seuilla dedeun\ldots é lo spettacllo l’et todzor lo mimo \direct{avouì lo dèi moutre \deidreite\ Geromine} é ta viya la si perqueue: l’è chouì cou que te me la conte!

\Gerominespeaks Te pense que a mé fiyisse plèizì restì seuilla avouì eun rabadàn comme tè? Surtoù veure, can te tsandzon le fise, eun moustre patanì pi é ouse, blan comme lo lasì é sensa pamì ren de euntéressàn pe le femalle.

\Casimirspeaks Magara pe de femalle comme té! Se le femalle que travaillon seuilla son jantile maque avouì mé n’aret eunna rèizón! \'E té te fa pa itre dzalaouza! \'E aprì resta maque tranquilla, itsauda-té pa\ldots piatro te pren eugn atro \textit{infarto}!

\Gerominespeaks Mé ara resto adì bièn é si pa senque baillerio pe itre eun campagne.

\Casimirspeaks Ouè fa beun diye que s’en eun trèi malado eun tsambra é l’atro \direct{moutre la personna eumpourtanta} l’a panco t’an prédzà.

\Gerominespeaks Ouè ren! S’en gneunca lo non. Se n'i comprèi amodo dèi itre euna grousa tita.

\Casimirspeaks Eh vouè, t’ayè panco comprèi-lo? Mi te avèitse pa lo TG3? Sitte l'é tcheu le momàn eun télévij\'on \tv . Eun dzor pe an bagga é eun dzor pe eugn'atra.

\Gerominespeaks Na. Mé acouto maque la \textit{Voix de la Vallée} desù la radi\'o \radio .

\Casimirspeaks Ad\'on l’é l’aoura que te te modernizisse an miya!

\Gerominespeaks Mé reusto bièn pai: de ouette a onze se si i mitcho \textit{Radio Zeta} avouì Ciccetti é a maenda lo \textit{Gazzettino}, avouì Cesarino!

\StageDir{Se rèche l'aristocrate que l’ie eun tren de drimì. Se derèidì é aprì pren lo campaneun \campanellino\ pe querì lo seun valet.}

\PersEmpourtantaspeaks Gaston! Gastooon!

\StageDir{Entre Gaston.}

\Eunfeurmispeaks Bondzor mesieu! Sade dza réchà?

\PersEmpourtantaspeaks Djaque!  Queun dzor l'é oueu? 

\Eunfeurmispeaks Oueu l'è devendro 15 Mi. L'è onj'aoure é sinque di mateun, lo ten l'è séèn é eun Val d'Outa l'è pa capitoù ren d'eumpourtàn.

\PersEmpourtantaspeaks Amoddo, ad\'on refèicha-mè djeusto le queverte.

\Eunfeurmispeaks To de chouite!

\StageDir{Gaston gnouye refèichì le queverte.}

\PersEmpourtantaspeaks Tro itrèite! 

\Eunfeurmispeaks\direct{Eun refèichèn} Va mioù pouai?

\PersEmpourtantaspeaks Tro lardze!

\StageDir{Gaston moungouye eunc\'o eun momàn.}

\Eunfeurmispeaks Va bièn ara?

\PersEmpourtantaspeaks Ouè va bièn.

\StageDir{Lo vioù, que l'a vi la \textit{scène}, l’a i fon de la coutse le queverte beuttaye mal é lo toppon pa totte. Ad\'on pe pa trebelì proue a se fé èidjì.}

\Casimirspeaks Gaston! Squezade Gaston! 

\StageDir{L’eunfeurmi fé semblàn de ren, ad\'on lo vioù tourne lo criì eun vouaillèn.}

\Casimirspeaks Deh, Squezade! 

\Eunfeurmispeaks Ouè.

\Casimirspeaks Me refèichade la coutse i fon pe plèizì?

\Eunfeurmispeaks Mi squersade pa mesieu, pouì pa vegnì tanque lé! Aprì qui reste seuilla a avèitchì lli?

\Casimirspeaks Que demanda foula que n’i fé. L’é vrèi n’ayoù pa pensou-lèi.

\StageDir{Se viounde di coutì de la viille é sopatte la tita.}

\scene[-- Le sen-z-ouillo]

\StageDir{Entre, de la pourta i fon di palque, eun prie tot arbeillà de neur, avouì la Bible eun man.}

\Priespeaks\direct{Eun fièn la croueu} Bondzor a tcheutte.

\Casimirspeaks Bondzor Monseur.

\StageDir{Lo prée s'aprotse a Gaston, que l'ie eun tren de poulitì le-z-onlle de l'aristocrate.}

\Priespeaks Mesieu\ldots

\Eunfeurmispeaks Ouè.

\Priespeaks \ldots qui l'è lo pi vioù é pi maladdo seuilla?

\Eunfeurmispeaks \direct{Eun moutrèn Casimir}
L'è si lé!

\Priespeaks Mersì.

\StageDir{Lo prie s'aprotse a la coutse de Casimir é dousemàn, avouì bièn de pitié, lèi fé la croueu \croce\ si lo fron.}

\Casimirspeaks\direct{Eun branquèn lo bri di prée} \'E so?! Que fiade? Mi squersade pa!

\Priespeaks \direct{Eun lèi prédzèn comme fuse eun pouo matasse} Mesieu, comme la demando-me voutra fameuille si vin-ì vo saliì pe lo dérì cou.

\Casimirspeaks \direct{Eun se totsèn le-z-attribus} Mi na na! Que fiade? Si pa preste a moueure mé!

\Gerominespeaks Mi senque te te totse? N'a pamì ren lé \riye !

\Casimirspeaks Ah! souplì là!

\Priespeaks \direct{\'Eton-où \ouaou} Comèn?! Mi seutta l’è pa la tsambra numer\'o 17?

\Casimirspeaks Na, na seutta l’é la tsambra nimér\'o\ldots \direct{se veurie ver Geromine} nimér\'o?

\Gerominespeaks \direct{A basa vouése} 23.

\Casimirspeaks Seutta l’è la 33!

\Gerominespeaks \direct{Eun vouaillèn} 23!

\Priespeaks Boundjeu de la France! Squezade-mé tan! Si fran trompou-me! Ad\'on me fa belle partì, piatro me scappe finque lo mor!

\StageDir{Baille la bénédich\'on a la tsambra é can l'è preste a chotre tourne eun derì ver Casimir.}

\Priespeaks Mesieu, si pa se v'ouèide sentì que no fa refé lo tette de l’éillize. Se vouillade pouade fé can mimo eunna queilletta pe l’éillize\ldots

\Casimirspeaks Na mé n'i dza baillà proou a meun ten.

\Priespeaks\direct{Diplèizì} Ah. \direct{Ver Geromine} Vo madama?

\Gerominespeaks Me deplì, mi mé se dérì n'i pa lo boursette. Pi que de bot\'on n'i pa.

\Priespeaks Fé pa ren ad\'on. Fé pa ren. Tracachade vo pa.

\Casimirspeaks Acoutade na baga: \direct{eun moutrèn la personna eumportanta} deumandade a llou que nen n'a d’avanse!

\Priespeaks Ah mersì bièn.

\StageDir{Lo prie va ver l'aristocrate.}

\Priespeaks\direct{A Gaston} Bondzor mesieu. Poui-dz\'o lo réchì?

\Eunfeurmispeaks Eh na, l'è eun tren de drimì. L'è mioù que lo rècho mé.

\Priespeaks Va bièn. Vito que n'i prisa.

\StageDir{Gaston to todzèn tsertse de réchì lo seun mesieu.}

\Eunfeurmispeaks\direct{Eun totsèn la man a l'aristocrate} Mesieu. Mesieu\ldots

\StageDir{L'aristocrate se rèche pa.}

\Priespeaks \ldots squezade. Mi n'i fran prisa. Lèi penso mé.

\Eunfeurmispeaks Va bièn, comme vouyade.

\StageDir{Lo pri beutte eunna man si l'ipala de l'aristocrate.}

\Priespeaks\direct{Eun braillèn} Mesieu!

\StageDir{L'aristocrate se rèche épouvant\'o .}

\PersEmpourtantaspeaks Ah! Eh! Oh, senque?!

\Priespeaks Bondzor.

\PersEmpourtantaspeaks Salì!

\Priespeaks Mesieu l’an deu-me que v'ouèi de voye moustre de fé de bienfezanse pe l’éillize.

\PersEmpourtantaspeaks Oué\ldots \direct{Ver Gaston} va prende lo bourset.

\StageDir{Gaston pren lo bourset di mesieu é teurie foua bièn de sou \cash . Lo mesieu lèi di de na avouì la tita; ad\'on Gaston nen teurie foua eugn atro pi petchoù é lo mesieu tourne lèi diye de na.}

\Eunfeurmispeaks Monseur n'i pa de tri. Vou a vére pe lo bourset di pise.

\StageDir{Gaston tourne eun derì avouì eun per de sou tri eun man.}

\Eunfeurmispeaks N'i acap\'o so!

\StageDir{Gaston baille i prie le dou tri que l'a prèi pe lo bourset.}

\Priespeaks Oh mersì. \direct{Ironique} V'ouite fran itoù jantilo.

\Eunfeurmispeaks Mersì a vo.

\Priespeaks Oueu lo nite fiyo eunna prière pe tcheu vo é tourno vo troué demàn mateun. Repouzade-v\'o. Orvouar a tcheutte.

\StageDir{Lo prie fé la marca de la croueu é chor.}

\scene[-- Ferrari \ferrari\ poulisie]

\StageDir{Entron trèi femalle di poulisie: son arbeillaye avouì eun tone rodzo de la Ferrari. Eunna avouì l'icaoua \scopa , l’atra avouì de prodouì pe poulitì é la dérie avouì eun tchappapoussa.}

\Fennepulisiespeaks Bondzor a tcheutte!

\Casimirspeaks \direct{Eun fièn vère que l’è to contèn} Bondzor! Voualà l'è dza torna aoura di poulisie!

\FennepulisieAspeaks Bièn lévoù?

\Gerominespeaks Mi ouè, mersì, é vo?

\FennepulisieBspeaks To bièn, mersì madama.

\Gerominespeaks V'ouite dza torna preste pe eunna nouva galoppaye?

\FennepulisieBspeaks Eh no totse\ldots

\Casimirspeaks Ouè mi sen comèn l’an aprèi amoddo lo patoué seutte!

\FennepulisieCspeaks Sen itaye coudziye. Diyon que lo maladdo se te l’èi prèdze sin-a lenva se queutte bièn pi alé\ldots

\Casimirspeaks Ouè, ouè fran pèi!

\StageDir{A si poueun, entre lo cape avouì eun cronomètre \cronometro\ é eunna bandjèira a car\'on blan é ner (salla que emplèyon pe le compétich\'on di machine).}

\Starterspeaks Sah, sah, ad\'on v'ouite tcheu preste? Dai eh! Beuttade-v\'o totte eun reugga, eunc\'o té!

\FennepulisieAspeaks Mi vouè, resta tranquilo. Oueu si eun plèin-a forma!

\Starterspeaks No fa betì mouèn de do meneutte, piatro no payon pamì! Perqué lo cou passoù \direct{avèitse eunna fenna di poulisie} l'amia de té l’an spédi-la ià perqué l’a ralent\'o totte le-z-atre!

\FennepulisieBspeaks Ouè, fa beun diye que l’an beutt\'o pi de do meneutte!

\FennepulisieCspeaks Pensa que salla poura matassa perdave de ten a poulitì tcheut le cassette!

\FennepulisieAspeaks Bondàn ézajér\'o! L’ayè panco comprèi comme martse seuilla dedeun. Djeusto la mandì i mitcho!

\Starterspeaks Bon perdèn pa de ten, prest? Trèi, dou, eun\ldots Via!

\StageDir{Fé partì lo cronomètre é sopatte la bandjèira \bandiera\ pe l’er. Le femalle totte a galoppe pouliton lo pi vito poussiblo; saouton d’eun coutì a l’atro; eun mimo ten tsanton é danchon totte eunsemblo, avouì finque le dou vioù, la tsans\'on:}

\sound{https://soundcloud.com/user-234168361/di-congo-pe-travailli}{Di Congo pe travaillì - Le Digourdì}

\StageDir{Can la mezeucca fenèi, se plachon protso i leur cape.}

\Starterspeaks Stop! Brave!

\FennepulisieBspeaks Si cou l’è fran bièn aloù.

\Starterspeaks Ouè, an meneutta é 36 seconde! Eh, can v'ouite vo trèi l’è totte eugn atro travaillì! 

\Fennepulisiespeaks Iuhuh!

\Starterspeaks Ara vo béyade maque eun bon \textit{Gatorade} pe eunna\ldots

\FennepulisieCspeaks Vito que n'en sèi!

\StageDir{Lo cape distribuèi lo bèye eun terièn foua de l'abrosaque trèi botèille de \textit{Gatorade}.}

\Starterspeaks \ldots é lo \textit{champagne} \champagne\ vo atèn pi de foua!

\Fennepulisiespeaks Iuhuh!

\StageDir{Chorton tcheu catro.}

\Gerominespeaks Son tcheu mat seuilla dedeun!

\Casimirspeaks Dériàn recoverì seutte dzi é pa no!

\Gerominespeaks Crèyo beun!

\scene[-- L'aoua di vezeutte]

\StageDir{Entre l’eunfermie.}

\Eunfeurmiespeaks \direct{Eurle avouì acsàn tédesque} L'oréo di vezeutte comenche ara!

\Eunfeurmispeaks Sht! Te vèi pa que lo meun Chef drime!

\Eunfeurmiespeaks \direct{Arroganta} Me euntéresse pa, sitte l'è lo meun répar é fio sen que n’i voya!

\Eunfeurmispeaks Avouì llou \direct{moutre lo seun Chef} séilla véyèn pi tanque can!

\Eunfeurmiespeaks Acouta bièn pégno eunfermì de prouva: mé n'i 27 an de servicho é de seutta plasse l’a jamì tramou-me gneun! A-te comprèi?

\Eunfeurmispeaks Mah, lèi dzouyério pa le mitcho.

\Eunfeurmiespeaks Can mimo torno repété: \direct{eun vouaillèn} L’oréo di vezeutte comenche ara!

\StageDir{L'eunfermie chor de \textit{scène}.}

\Gerominespeaks Sen pamì tan dzoun-euo mi n’en praou de sentì eun cou pe comprendre senque no dion.

\Casimirspeaks Ah se l’è comme totta la senâ vou cheur me lagnì avouì to si mondo que veun me trouvì.

\Gerominespeaks Compregno beun perqué t’a pa gneun. T’i gramo comme eun pet.

\Casimirspeaks \direct{Eun desuèn la vouése de Geromine} T'i gramo comme eun pet!

\PersEmpourtantaspeaks \direct{Ver l’eunfermì} Acoutta, n’i sèi.

\Eunfeurmispeaks Ouè,  vo pourto eun vèyo d'éve \bicchiere ?

\PersEmpourtantaspeaks Vèi té.

\Eunfeurmispeaks Vouillade salla di grousse boule ou salla di seunteucco?

\PersEmpourtantaspeaks Salla di boule va bièn.

\Eunfeurmispeaks Ouè to de chouite, mesieu.

\StageDir{L'eunfermì reumplèi eunna flute d'éve é la soum\'on a l'aristocrate (djeusto eunna pégna gotta). Aprì, reprèn lo vèyo é lo beutte eun caro. Dimèn entre eunna dzenta feuille blounda, Josette, la feuille de Geromine.}

\Feliespeaks Bondzor a tcheutte!

\Casimirspeaks \direct{Eun se terièn si de la coutse pe vère mioù} Bondzor!

\Gerominespeaks Oh! Avèitsa qui se vèi. Ad\'on te sa qué que t’a na mamma?

\Feliespeaks Mamma di pa pouèi. N’i pa i lo ten devàn é avouì tcheu le-z-eungadzemèn que n'i.

\Gerominespeaks Ouè, ouè, bailla-mé eun poteun.

\StageDir{Se baillon eun poteun. Josette s'achouatte si la quèya protso de la coutse.}

\Feliespeaks Can mimo to bièn? Comme se reuste seuilla?

\Gerominespeaks \direct{D’eun ton ironique} Se reste bièn. L’è caze eun oberdze. Si poste seuilla ara l’a maque trèi-z-itèile \stella\ pequé la catrima l’an djeusto gavou-la. \direct{Avèitse Casimir ironiquemàn} Aprì la compagnì l’è fran dzenta! Fiyèn de salle riaye é de salle fite que avèitsa!

\Feliespeaks S’i fran contenta. Ad\'on t’a fran fata de ren ad\'on? Mé n'i pensoù de pourti-te an bouite de biscouì.

\Gerominespeaks Oh mersì Josette. \direct{Avèitse la conféch\'on} Lo pri te pouave lo gavì!

\Feliespeaks An ouè n’i oublià! Squeza-mé!

\Gerominespeaks Pe $0.80$ santime t’aré cheur atseuttou-me de biscouì \biscotto\ bièn bon!

\Feliespeaks Mamma n’ayet l’offre a la LIDL é pouèi n’i fé que spendre an mia de pi pe atsetì doze conféch\'on.

\Casimirspeaks Seutta l’è cheur tin-a feuille!

\Gerominespeaks Ita tchica quèi té! Pa fata d'acouté noutre conte!

\PersEmpourtantaspeaks \direct{Ver l’eunfermì} Gaston, n’i fan: quetsouza a medjì.

\Eunfeurmispeaks Vo baillo eun \textit{cracker}. 

\PersEmpourtantaspeaks Va bièn.

\Eunfeurmispeaks To de chouite, mesieu. Vouillade sise avouì la sa ou sensa sa?

\PersEmpourtantaspeaks Avouì la sa. 

\StageDir{L'eunfermì, avouì eun cabaré d'ardzèn, lèi pourte eun \textit{cracker}; l'aristocrate lèi baille eunna mordia é tourne lo pouzé desì lo cabaré. L'eunfermì beutte tourna totte eun caro.\\ Entre lo medeseun.}

\scene[-- Medeseun Mitcho \casa]

\MedMitchospeaks Bondzor a tcheutte!

\Eunfeurmispeaks  Bondzor mesieu! L'è arreuvoù Doctor \textit{House}!

\Casimirspeaks Qui?

\Eunfeurmispeaks Medeseun Mitcho!

\MedMitchospeaks Ad\'on comme l'è, to bièn pe seuilla?

\Gerominespeaks Todzor bièn mesieu lo medeseun. Mé resto bièn eunc\'o oueu.

\MedMitchospeaks Véyèn pi ara, madama. Ad\'on, teuriade si la mandze é baillade-mé lo bri.

\StageDir{Lo medeseun mezeue la préch\'on.}

\MedMitchospeaks\direct{Eun s’apesissèn de la feuille \inamourou} \ldots é seutta dzenta demouazella de iaou chort?

\Gerominespeaks L’è ma feuille.

\MedMitchospeaks Ah n’i comprèi\ldots

\StageDir{Lo medeseun fenèi de prendre la préch\'on.}

\MedMitchospeaks Voualà lèi s’en caze madama! Dou où trèi dzor é vo mando i mitcho.

\Gerominespeaks Ouè l’è devendro é vo èi dza deu-me seutta baga demarse!

\MedMitchospeaks Me diplì mi la voutra préch\'on monte é bèiche tcheu le momàn; é aprì avouì an feuille pai penso que vo féyo pi restì eunc\'o seuilla doe ou trèi senâ.

\Gerominespeaks Ah! Restade maque tranquillo que vo èi vi-la oueu é la vèyade cheur pamé a me trouvì!

\Casimirspeaks Pa deutte\ldots l’a eunc\'o onze conféch\'on de biscouì a fére foura!

\Gerominespeaks Resta tchica quèi té!

\StageDir{Lo medeseun fé lo tor de la coutse é euncrije le joueu de Josette, laquelle se trame pe lo quetì traillì.}

\MedMitchospeaks\direct{A Geromine} Mé aya l'amerio\ldots teriàde-vo si\ldots

\StageDir{Geromine se beutte achouataye si la coutse avouì eunna mia de difficult\`o.}

\MedMitchospeaks Amoddo, teriade si le bri, si lo cou, cllouzade le joueu, sarade bièn la botse é ara sarade é cllouzade le man pe trèi car d'aoua de ten\ldots 

\StageDir{Geromine fé totte senque lèi di lo medeseun é, dimèn que ivre é cllou le man, lo medeseun nen profite pe fé lo fleungàn avouì Josette.}

\MedMitchospeaks\direct{A Josette, bièn malisieu \malisieu} Ad\'on squezade, mi queun l'è lo voutro non?

\Feliespeaks Si Josette, plèizì!

\MedMitchospeaks Oh Josette! Mé si \textit{House}! Medeseun \textit{House}, mi criàde-me maque Mitcho; é diade-mé, an fenna dzenta comme vo senque fé? Queun travaille fé?

\Feliespeaks Si dirijanta eun Réj\'on.

\Gerominespeaks\direct{Eunfastedjaye \malechaa pe itre itaye abandonaye} Va bièn pouèi medeseun!

\MedMitchospeaks Ouè madama! Trèi car d'aoua n'en deutte.

\Gerominespeaks Mi comencho a itre lagnaye!

\MedMitchospeaks Alé, conteniade.

\StageDir{Geromine conteneuvve a clloure é ivrì le man avouì le joueu clloujì. Lo medeseun Mitcho reprèn a martélé Josette.}

\MedMitchospeaks\direct{A Josette} Ad\'on diavo, an dzenta feuille comme vo l'è mariaye?

\Feliespeaks Na si \textit{single}\ldots

\MedMitchospeaks Ah v'ouite \textit{single}! Me deplì\ldots

\Gerominespeaks A mé na! Ad\'on va bièn pai medeseun?

\MedMitchospeaks Ouè madama! 

\StageDir{Lo medeseun tourne s'occupé de Geromine.}

\MedMitchospeaks Ad\'on madama, teriade foua le pià de la coutse é itade achouataye. Levade si le man é aprì pléyade l'itseun-a é portade le man tanque i fon di pià.

\StageDir{La poua Geromine se plèye si é ba eun per de cou avouì l'èidzo di medeseun que la cllou é l'ivre comme eun livro.}

\MedMitchospeaks Ba é si, si é ba\ldots tourna trèi car d'aoua pai.

\Gerominespeaks Todzor avouì le joueu cllouzì?

\MedMitchospeaks Ouè.

\StageDir{Lo medeseun tourne avouì Josette.}

\Casimirspeaks\direct{Eun desuèn Josette} Si! Ba. Si é ba! Si, si, si!

\StageDir{To d'eun cou, entre eun maladdo psychiatrique que, eun trambélèn, eurle:``Tchouf, tchouf! Tchouf, tchouf!''. Aprì eun per de seconde doe eunfermie entron i galoppe, fan eunna pouenteua i pasiàn ià de tita, l'eumbrancon é lo teurion foua de pèise. Dimèn medeseun \textit{House} tranquilize Josette é Geromine.}

\MedMitchospeaks\direct{Ver Casimir} Ara pasèn a vo. 

\Casimirspeaks Ouè, prest!

\StageDir{Medeseun \textit{House} s'aprotse a Casimir.}

\MedMitchospeaks Lo bri mersì é si la mandze.

\StageDir{Lo medeseun mezeue la préch\'on, mi dimèn avéitse Josette é sensa s’apesèivre countenie a saré la poumpetta de la préch\'on que sare todzor pi lo bri di pouo Casimir.}

\Casimirspeaks\direct{Eun souffràn \dolore} Ahi, argh, ahi!

\MedMitchospeaks Ouè mi que préch\'on ata! 

\Casimirspeaks Que drolo! La feuille que l’è seuilla protso l’è dza eun momàn que l’è arrevaye é penso fran que sise sen!

\MedMitchospeaks Souplé! Son pa de bague pe voutro éyadzo!

\Casimirspeaks Avèitsade que mé tanque ara le pastiille bleuve n’i maque vi-le eun publisitoù!

\MedMitchospeaks Lèi manque eunc\'o finque lo \textit{Viagra}, pai te me stchoppe comme eun pallontcheun! Sa, ara ivrade la botse\ldots

\StageDir{Avouì eunna lemie medeseun Mitcho avèitse dedeun la botse de Casimir, mi teurie vitto eun déri la tita.}

\MedMitchospeaks Ouè mi que bon flo de crotta! Senque vo baillon medjì seuilla dedeun lo mateun? \textit{Bagna cauda} é biscouì? 

\StageDir{Lo medeseun pren coadzo é avèitse tourna deun la botse de Casimir.}

\MedMitchospeaks Diade ouè\ldots

\Casimirspeaks\direct{Avouì la botse iverta} Ouè!

\MedMitchospeaks Diade na\ldots

\Casimirspeaks\direct{Avouì la botse iverta} Na!

\MedMitchospeaks Ouè\ldots Mogà oueu lo nite devàn que alì a drimì queuttade la dentchiye \dentiera\ dedeun la \textit{conegrina}!

\Casimirspeaks Va bièn, comme vouillade vo medeseun mitcho.

\MedMitchospeaks \ldots é demàn mateun lèi baliade eunc\'o eunna dzenta frotaye. Ara teriàde-vo si.

\StageDir{Casimir, avouì an miya de difficult\'o, se beutte achouatoù si la coutse.}

\MedMitchospeaks Teurriade si la flanella\ldots

\Casimirspeaks  Eh fiade vo mé arevo pa!

\MedMitchospeaks\direct{Eun terièn si la ``canotta''} Oh que bon flo de servadzo! Demàn mateun mogà frottade eunc\'o déz\'o le bri! Ara fiade de pégno crep de tosse.

\StageDir{Casemir fé fran de son pezàn é plen de catarre; lo medeseun dimèn que lo vezeutte avèitse to di lon la dzenta Josette, laquelle, de louèn, rep\'on avouì de sourì malisieu.}

\MedMitchospeaks N’i deu pégno!

\StageDir{Lo vioù le fé pi pégno.}

\MedMitchospeaks Lèi sen caze. Sel\'on mé, se vouillade eunna tosse tchica pi sètse vo consèillo de passé de trèi a dou paquet de \textit{trinciato} pe dzor!

\Casimirspeaks Voué, va bièn medeseun, comme vouillade vo.

\MedMitchospeaks \ldots aya le tsambe é aprì n'en fenì. 

\StageDir{Lo medeseun fé lo tor de la coutse é se pourte devàn Casimir; avouì eun martelette \martello\ léi baille eun crep deur desì lo dzegnaou, mi Casimir levve eun bri; lo medeseun lèi baillie eugn atro crep mi, comme douàn, Casimir levve lo bri. Dimèn Josette avouì la squiza de beté eun plase la coutse de la mamma se fé avèitchì lo qui di medeseun, loquel, distré pe Josette, baille eun crep de martelette si la tsamba de Casimir: la réach\'on, si cou, l'è eun caouse i bale i medeseun que fenèi pe tèra. Josette lo èidze a se terì si.}

\Feliespeaks \direct{Eun galoppèn vitto ver lo medeseun} Medeseun, medeseun, v'ouite fé-vo de mou?

\MedMitchospeaks  Na, na, to a poste! Tracachade vo pa. 

\StageDir{Josette tourne protso a Geromine.}

\Casimirspeaks\direct{I medeseun} Eh squezade mé!

\MedMitchospeaks Ouè, ouè\ldots eun jénéral le bague semblon normalle. Ara passerio i noutro pi eumpourtàn cliàn!

\StageDir{Lo medeseun s'aprotse a l’eunfermì.}

\Eunfeurmispeaks Eh drime, me diplì tan!

\MedMitchospeaks Gneun problème. Passo pi mé pi tart. An djeusto eunna bagga: va-tì bièn la coutse que n’en beuttou-lèi protso a la fenitra? Sade, l’amen fére contèn le noutro pi eumportàn cliàn.

\Eunfeurmispeaks Ouè, fran amoddo! Aprì de seuilla vèi bièn lo Mont-\'Emilius que llu l’ame da mat.

\MedMitchospeaks Si fran contèn. Ad\'on tourno pi tar. Tanque!

\Eunfeurmispeaks Ouè vo remersio. Tanque!

\StageDir{Med. Mitcho torne avouì Casimir.}

\MedMitchospeaks\direct{Ver Casimir} Bon ara me nen vou. Mogà tourno pi tar pe eun dzen \textit{clistere}! \direct{Avèitse la dzenta feuille} \ldots é vo\ldots v'ouèide eunna beurta grima, vo vèyo an mia blaye. Pe mé v'ouèide fata de eunna pégna vezeutta!

\Gerominespeaks A mé semble pa: l’è todzor igalla.

\Feliespeaks Mamma mé vou vère. Vourriyo jamì que aprì me sento mal djeusto foura de l’opetaille. 

\Gerominespeaks Josette fé atench\'on!

\StageDir{Josette baille dou pouteun \bacino\ a la mamma é chor avouì lo medeseun.}

\Gerominespeaks \direct{Avèitsèn lo medeseun} Té pren-te varda!

\MedMitchospeaks Ouè. \direct{Eun chortèn avouì Josette} \textit{Au revoir} a tcheutte!

\StageDir{Josette é medeseun Mitcho chorton.}

\scene[-- Lo motchaou]

\StageDir{Lo vioù pren eun motchaou desù sin-a \textit{table de nuit} mi lèi tsi pe tèra. A si poueun pe lo prendre se dèi caze cayì pe tèra é se trèinì pe lo prendre. L’eunfermì, que l'a vi amoddo la \textit{scène}, baoudze pa eun dèi. Lo maladdo pren lo motchaou eun se lagnèn é a fose se souffle lo na.}

\StageDir{L'aristocrate iternèi \starnutire .}

\Eunfeurmispeaks Mondje prégnade pa de frette! Vo prègno to de chouite eun motchaou!

\StageDir{L'eunfermì galoppe pe prende eun dzen motchaou de tissù.}

\PersEmpourtantaspeaks Fiade vito!

\Eunfeurmispeaks Ouè, ara fiade eunna dzenta soufflaye\ldots

\StageDir{L'eunfermì pourte lo motchaou i na de Gaston, lequel souffle détchis.}

\Eunfeurmispeaks Oh\ldots fran eunna dzenta soufflaye! V'ouèide caze eumpli-lo!

\Gerominespeaks Na, mi se pou pa itre fagnàn pouèi?

\Casimirspeaks Ouè, l’atro l’at caze fé-me moueure pe coillì lo motchaou é avèitsa avouì sit ommo senque l’a fé? 

\StageDir{Dimèn l'eunfermì arendze l'atchaou di queseun a Gaston.}

\scene[-- L'aoua de maènda]

\StageDir{Entre l’eunfermie maréchal avouì  eun per de platte, vèyo é fourtsette de papì.}

\Eunfeurmiespeaks L’oréo di vezeutte fenèi ara! \direct{Ver Casimir é Geromine} La maenda l’è caze presta. A vo lo plat é le fortsette de papì.

\StageDir{L'eunfermie tappe le-z-éze si la coutse de Casimir é Geromine.}

\Gerominespeaks Mersì! Todzor de papì?

\Eunfeurmiespeaks Vouillade lavé vo le-z-éze \posate?

\Gerominespeaks La mia l'ie eunna demanda\ldots feyavo pe riye\ldots mondje!

\Casimirspeaks Qui areuvve no pourté a medjì?

\Eunfeurmiespeaks Areuvve ara Fernanda vo servì. \direct{Ver l'eunfermì} \ldots é pe llu \direct{avouì la tita moutre Gaston} areuvve eunc\'o lo menù spésial. \textit{Au revoir}!

\StageDir{L'eunfermie chor.}

\Casimirspeaks\direct{A Geromine} Mé medzì seuilla dedeun lamo ren!

\Gerominespeaks A qui te lo di!

\Casimirspeaks Me semble de itre a eunna de salle fìte di Pro Loco.

\Gerominespeaks Aprì fa totte pouzì desù le queverte pequé te te beurle le tsambe.

\Casimirspeaks Vrèi eunc\'o sen.

\StageDir{Entre Fernanda, lo Chef de l’ipetaille, eunna fenna grousa, avouì eun gran faoudè de la Pro Loco é eun grou sourì que moutre eunna groussa di nèye. Avouì eun bri pourte eunna grousa cachoula é avouì l'atra man teun eunna potse que eumplèye pe fé an miya de ravadzo eun la bouéchèn countre la cachoula.}

\StageDir{Partèi lo jénérique de:}

\sound{https://soundcloud.com/user-234168361/extrait-de-il-pranzo-e-servito}{Il pranzo è servito (Sigla) - Augusto Martelli}

\Fernandaspeaks V'ouite pa eunc\'o mor?

\Casimirspeaks Ouélla Nanda, t’areuvve pountuella eunc\'o oueu!

\Fernandaspeaks Comme todzor! T’a bièn fan oueu? Ou pa tan?

\Casimirspeaks Ouè, dièn que eunna dzenta \textit{bistecca} \bistecca\ lèi resterie.

\Gerominespeaks Sondza maque! Véyèn se eundovin-o: oueu n’a puré é straqueun! Na oubliavo, n’at eunc\'o la \textit{minestrina}.

\Fernandaspeaks Te sa que t’a belle eundovin-où! Oueu n'a lo menù spésial: polenta é fricand\'o ou tartiffle i beuro avouì lo lapeun!

\Casimirspeaks Mondje! Ad\'on beutta belle eun platte de polenta!

\StageDir{Fernanda teurie foua de pégne pastiille \pillola .}

\Fernandaspeaks Voualà! Polenta é fricandò! Seutte son le noue pillule iofilizaye: te le tappe ba avouì eunna gotta d'éve é t'i plen pe to lo dzor.

\Casimirspeaks Seutte pégne bague?

\Fernandaspeaks  Ouè.

\Gerominespeaks\direct{A Casimir} Fé vèi veure!

\StageDir{Casimir alondze lo bri pe pasé la pillule a Fernanda, mi malerezamente lèi tchi pe tèra. Fernanda lo reprodze eun lèi baillèn, avouì la potse, eun crep desì lo djise.}

\Fernandaspeaks Mi Casimir! T’a fé tchire la Polenta é fricandò! 

\Casimirspeaks Ouè é té t'a to rempli-me de puré!

\Fernandaspeaks \direct{Eun perdèn la pachense} Ouè, ouè. Ad\'on v'ouèide desidoù senque vouillade medzì? N'i pa de ten a pèdre!

\Casimirspeaks Ouè, ouè. N'i désidoù: vou desù la puré é straqueun. Te sa n’i pa l’ive de viya pe dijirì to si medjì.

\Fernandaspeaks Beutta eun sé lo plat ad\'on.

\StageDir{Fernanda beutte ba eunna bella potchaye, eun icretèn tchica desì la coutse.}

\Fernandaspeaks  Eunc\'o an mia?

\Casimirspeaks  Na, na, n'a prao!

\StageDir{Fernanda lèi beutte ba eugn'atra potchaye.}

\Fernandaspeaks Eunc\'o eunna potchà! N'i pa voya que aprì te va i mitcho é te di que Nanda te baille pa prao medjì!

\Casimirspeaks Na, na, seu n'a prao!

\Fernandaspeaks\direct{Eun s'aprotsèn a Geromine} Geromine te vou eunc\'o té?

\Gerominespeaks Mi ouè. Perqué se meudzo pezàn aprì resto rechaye totta l'éproù.

\StageDir{Comme douàn, Fernanda tappe ba eunna potchaye eun icretèn la poua Geromine.}

\Fernandaspeaks  Eunc\'o eunna potchà?

\Gerominespeaks  Na, na, n'a prao!

\StageDir{Sensa lèi baillì lo ten de fenì, Fernanda remplèi pe lo sec\'on cou lo platte de Geromine.}

\Fernandaspeaks\direct{Eun se poulitèn le mandze é le lenette} Mersì madama! Mersì!

\StageDir{Fernanda s'apersèi d'avèi to pouertchà Geromine. Pe rezoudre, pren lo panamàn, lo lètse é lo frotte countre lo vezadzo é le lenette \lenette\ de Geromine.}

\Fernandaspeaks Mondjemé, squezade n'i to pouertcha-vo!

\Gerominespeaks\direct{Eun tsertsèn de pa se fé totchì} Ouè, mersì madama, mersì!

\Fernandaspeaks Voualà é se v'ouèide eunc\'o fan criade-mé. A vo lo baillo!

\Gerominespeaks Na, na, pa fata.

\Fernandaspeaks Bon, ara vou di dentiste.

\Casimirspeaks N'i la fèi que l'è pouza que te remande.

\Fernandaspeaks  Eun per de-z-àn!

\StageDir{Fernanda chor.}

\Gerominespeaks L’et pi caze satte dzor que meudzo so é tcheu le dzor l’a caze eun gou nouo.

\Casimirspeaks Todzor a te lamenté! L'è beun bon so!

\PersEmpourtantaspeaks \direct{Ver l’eunfermì} Gaston!

\Eunfeurmispeaks Ouè mesieu!

\PersEmpourtantaspeaks N’i peucca a l’itsin-a\ldots

\Eunfeurmispeaks Ouè, ad\'on teriade-v\'o si que vo fiyo eun dzen servicho.

\StageDir{L'Aristocrate se teurie si é Gaston comenche a lèi froté l'isteun-a.}

\Eunfeurmispeaks Seuilla?

\PersEmpourtantaspeaks Na, pi aoutre.

\Eunfeurmispeaks Seuilla?

\PersEmpourtantaspeaks Na, tchica pi ba.

\Eunfeurmispeaks Ad\'on seuilla?

\PersEmpourtantaspeaks Ouè brao fran li!

\Eunfeurmispeaks Oh mesieu si bièn contèn de vo fé contèn!

\PersEmpourtantaspeaks Ouè si contèn can te me fé contèn!

\StageDir{Entre eun tsambrì avouì lo papillon \uomoelegante\ é va ver l'aristocrate que i mimo ten se beutte comoddo si la coutse. Gaston lèi refèiche le lenchoueu.}

\Tsambrispeaks Bondzor a tcheut! Bienvenù a l’opetaille moderno. Senque vouillade medjì oueu?

\PersEmpourtantaspeaks Senque v'ouèide?

\Tsambrispeaks \direct{Pren lo menù é li} Ad\'on lo chef vo propoze, comme \textit{entré}:

\begin{itemize}
\item[--]  \textit{moules à la francillon};
\item[--]  \textit{soufflé au fromage de montagne};
\item[--]  \textit{mousse de saumon avec crevettes};
\item[--]  \textit{salade et vinegrette}.
\end{itemize}

pe premì n’en:

\begin{itemize}
\item[--]  \textit{bouillabaisse marseillaise};
\item[--]  \textit{soupe aux legumes}.
\end{itemize}

pe sec\'on:

\begin{itemize}
\item[--] \textit{mérou au bresse bleu};
\item[--]  \textit{encornets a la biscaïenne};
\item[--]  \textit{ratatouille};
\item[--]  \textit{tartiflette}.
\end{itemize}

Pe lo bèye vo propozo Cabernet di 1932 ou Merlot 1927.

\StageDir{Le dou vioù s’avèitson amoddo é arreuvon pa a comprendre \confuso \boh.}

\PersEmpourtantaspeaks Na, na\ldots lèi sen pa. Mé vouillavo eunna \textit{bourguignonne}!

\Tsambrispeaks Eunna \textit{bourguignonne}, n'a gneun problème. Vo apresto totte!

\StageDir{Lo tsambrì chor.}

\PersEmpourtantaspeaks\direct{A Gaston} Que servicho!

\Gerominespeaks \direct{Eun desuèn Casimir} Que bon-a la puré!

\Casimirspeaks\direct{Eun tsertsèn de semblì a Geromine} Que bon-a la puré! Que nen sayoù mé?!

\Gerominespeaks Mi fi-te feun!

\StageDir{L’eunfeurmì dimèn apreste l'aristocrate eun lèi betèn  la servietta i cou.}

\Gerominespeaks Can mimo dedeun sit ipetaille n’en pi dza vi-nen de bague drole!

\Casimirspeaks Ah ouè, bondàn! Va savèi senque l’arèn eunc\'o da vère noutro dzouvin-o!

\Gerominespeaks Que l’opetaille l’ise eun tsandzemèn me vatte, mi moderno tanque a si poueun na!

\StageDir{Entre lo tsambrì avouì eun cabaré \cabaret .}

\Tsambrispeaks Mesieu la voutra \textit{bourguignonne}! Vouillade veure?

\StageDir{L'aristocrate di de ouè avouì la tita. Lo tsambrì levve lo topèn é s'aprotse a la coutse, mi malerezamente s'euntsambotte é vouidze to l'ouillo bolequèn adosse a l'aristocrate. Le dou vioù, dimèn que l'aristocrate braille di mou, s'avèitson é se fan an dzenta riaye.}

\ridocliou

\newpage

\scene[-- Se torne i mictho]

\StageDir{\textsl{Vouése foua campe \vouese}: ``Eun per de dzor aprì''}

\ridoiver

\StageDir{Eun \textit{scène} n’at a la coutse l'aristocrate to fèichà di pià a la tita. Tellamente que l'è fèichà, son eunfermì lèi baille bèye avouì eunna paille. De l'atro coutì, le dou vioù son eun pià eun tren de s'arbeillì é de presté le valize, vu que, finalemàn, chorton de l'opetaille. I mimo ten entre lo medeseun.}

\MedMitchospeaks Bondzor a tcheutte!

\Gerominespeaks Oh Bondzor!

\Casimirspeaks Bondzor medeseun!

\MedMitchospeaks\direct{A Geromine} Ad\'on oueu l’è lo dzor! Fiade le voutre valize que vo mando i mitcho perqué si cou l’è totte eun plase! Saliade-mé la feuille!

\Gerominespeaks\direct{Pa tro convencua} Ouè, ouè!

\StageDir{Medeseun Mitcho baille lo foillette de l'opetaille a Casimir é Geromine.}

\MedMitchospeaks\direct{A Casimir, itèn bièn louèn pe pa sentre lo seun flo} Itade-mé bièn eunc\'o vo.

\Casimirspeaks Ouè mersì, salì!

\MedMitchospeaks\direct{Eun chortèn} Salì  a tcheutte!

\StageDir{Lo medeseun Mitcho chor.}

\Casimirspeaks A forse pouì tornì eun cantin-a a bèye lo vèyo \bicchiererosso\ é  féye la belote!

\Gerominespeaks Ouè, fé maque lo feun, pouèi demàn ti tourna seuilla!

\Casimirspeaks\direct{Eun desuèn la vouése de Geromine} T'i torna seuilla! \direct{Avouì la sin-a vouése} Te tsandze pa té!

\StageDir{Casimir pren la valiza \valigia .}

\Gerominespeaks T'a prèi totte?

\Casimirspeaks Ouè, mé si prest\ldots

\Gerominespeaks La tita?

\Casimirspeaks \ldots te tsandze pa le batiye?

\Gerominespeaks \direct{Eun prégnèn sa valiza} Na! A propoù! Alèn saliì noutro vezeun?

\Casimirspeaks Ouè! \'Eita-l\'o ba lé!

\StageDir{Casimir é Geromine s'aproston a l'aristocrate.}

\Gerominespeaks\direct{A l'aristocrate} Te queutto la bouite di biscouì, te le meudze\ldots ouè, can t'areuve pi a le medjì!

\StageDir{Geromine baille le biscouì a l'eunfermì.}

\Eunfeurmispeaks Mersì madama!

\Casimirspeaks\direct{Eun se moquèn de l'aristocrate} Ad\'on! T'areuve pa a fé plin plin? \direct{Eun sopatèn la tita} T'areuve pa? Alé! Fé plin plin!

\Gerominespeaks\direct{Eun baillèn manforta a Casimir} Ouè betèn si amoddo le queverte é\ldots ah\ldots atèn, mogà dèi se soufflé lo na!

\StageDir{Geromine teurie foua eun motchaou é lo pourte i na de l'aristocrate.}

\Gerominespeaks Souffla lo na, ouè souffla, souffla!

\StageDir{L'aristocrate tsertse de diye de na é de se difendre.}

\PersEmpourtantaspeaks Mmm, mmm, mmm! 

\Gerominespeaks Mi brao!

\Casimirspeaks L'a gnenca eumpli-lo! Aprì la \textit{cannuccia}! Te bèi deun la \textit{cannuccia}\ldots é\ldots é t'a voya de me gratì l'itseun-a eunna miya? Na?\direct{A Geromine} Gratta té!

\Gerominespeaks\direct{Eun gratèn l'itseun-a de Casimir} Ah ah ah, ouè, ouè grato mé, grato mé!

\Casimirspeaks Bon, vo salièn! Conserva-té di gamolle!

\Gerominespeaks Ouè si cou l'è praou fèichà \bendato! Salì!

\PersEmpourtantaspeaks\direct{Malechà \malecha} Mmm, mmm, mmm! 

\Eunfeurmispeaks Salì!

\StageDir{Casimir é Geromine chorton.}

\ridocliou

\DeriLeRido

\RoleNoms{Collaborateur}{Flavio Albaney, Paola Lucianaz, Fabrizio Pession}
\RoleNoms{Costumiste}{Roberta Charrere, Ornella Grivon}
\RoleNoms{Souffleur}{Valeria Brunod}
\RoleNoms{Tramamoublo}{Jean-Pierre Albaney,  Jerome Saccani}

\end{drama}