\title{MATTE\ldots SEN TCHEUTTE MATTE}
\author{Pièse icrita pe Le Digourdì}
\date{Téatro Giacosa de Veulla, 26 avrì 2013}

\maketitle

\fotocopertina{Foto/2013/gruppo.jpg}{Ester Bollon, Giada Grivon, Marco Ducly, Simone Roveyaz, Ilaria Linty, Laurent Chuc, André Comé, Pierre Savioz, Jordy Bollon}{Sophie Comé, Jasmine Comé, Jo\"{e}l Albaney, Jo\"{e}lle Bollon, Francesca Lucianaz, Paolo Cima Sander, Stéphanie Albaney}{2013}

\LinkPiese{Matte\ldots sen tcheutte matte}{https://www.youtube.com/watch?v=AKkdx5w1Mdk}{.75}

\souvenir{Eun souvenir particulié de \og Matte\ldots sen tcheutte matte\fg l'è la fasilit\'o avouì laquelle n'en cherdì la protagoniste de la counta: \og Qui de no s'arbeuille fran mal é l'et a la retsertse de l'amour de sa via?\fg; tcheutte son viondo-se ver mé. Pouò pa diye de na! \\Dèyo diye que aprì seutta pièse si belle meilloraye deun lo chouà di coteill\'on (si pa mé que lo diyo! L'è Laurent Chuc que l'a deu-me-lo!)\ldots é pe sen que regarde l'amour de ma viya pouo diye que ara si mamma d'eun mèinoù mervèilleu, que (boneur!) l'a ren a veure avouì Twitter!
\\N'i eunc\'o eun souvenir de seutta pièse, que me fé todzor mou\-eu\-re di riye: le patèle é le creppe que Ester baillè ba i pouo Marco can eunterprétoon la parodì de SOS tata. De salle baquettaye desì l'itseun-a!}{Jo\"elle Bollon}
\queriaouzitou{
\begin{itemize}
\item[$\bullet$] Pierre Savioz l'a i eun problème tecnique avouì son microfonne. Can l'a fallì chotre pe tsandji-lo, l'a désid\'o de resté foua \textit{scène} pe eun bon momàn, eun fièn eunna mia tracachì Jordy Bollon (que l'ie maque lo sec\'on cou que resitè si lo palque). Mi can l'è entroù l'a bièn esplecoù i pebleuque pequé l'ayè fallì scapé foua!\\

\item[$\bullet$] Pe lo premì cou, la prézentach\'on di-z-atteur l'è itaye féta de fas\'on diffienta reuspé a la tradichonella magniye de querì eun aprì l'atro tcheu le-z-atteur a la feun di spettaclle. Tcheu le-z-atteur son it\'o prézent\'o a l'entérieur de la danse \textit{chorégraphique} finale.
\end{itemize}
}

\Scenographie
\begin{itemize}
\item[$\bullet$] 1 pégno chofà, pe doe personne, é 1 télécomanda;
\item[$\bullet$] 1 sidel de Popcorn;
\item[$\bullet$] 1 pégna tabla, ata tchica mouèn d'eunna caèya;
\item[$\bullet$] 1 fondal gri ou \textit{beige}, lardzo eun car di palque;
\item[$\bullet$] 1 pourtamantì;
\item[$\bullet$] 1 manequeun fenna avouì eun coteill\'on élégàn;
\item[$\bullet$] 1 meriaou;
\item[$\bullet$] 1 poltronna de pi maròn;
\item[$\bullet$] 5 caèye coloraye;
\item[$\bullet$] 1 tabouré de bouque;
\item[$\bullet$] 1 tabla pe maenda é 1 pe traillì avouì l'ordinateur;
\item[$\bullet$] 3 caèye de bouque seumple, pe lo mitcho;
\item[$\bullet$] 1 bal\'on de SPUGNA pe le mèinoù;
\item[$\bullet$] 1 grousa busta de bouque, ata 2 mètre é lardze 3. Atatchà i panel de bouque n'at eun sec\'on panel triangulère que, eun cou iver, moutre eunna fenitra.
\end{itemize}

\setlength{\lengthchar}{3.5cm}

\Character[AUGUSTE]{AUGUSTE}{Viouj}{Vétchot avouì todzor la repounsa presta, \name{Pierre Savioz}}

\Character[JOSEPH]{JOSEPH}{Vioup}{Vétchot que comenche a pédre eun per de creppe, \name{Jordy Bollon}}

\Character[GUILLAUME]{GUILLAUME}{Guillaume}{Imitach\'on de \href{https://it.wikipedia.org/wiki/Enzo_Miccio}{Enzo Miccio} de la trasmech\'on \href{https://it.wikipedia.org/wiki/Ma_come_ti_vesti\%3F!}{\textit{Ma come ti vesti?!}}, \name{Laurent Chuc}}

\Character[CHÉRIE]{CHÉRIE}{Cherie}{Imitach\'on de \href{https://it.wikipedia.org/wiki/Carla_Gozzi}{Carla Gozzi} de la trasmech\'on \textit{Ma come ti vesti?!}, \nameF{Francesca Lucianaz}}

\Character[ALICE]{ALICE}{Alice}{Feuille dzouveunna, sa pa s'arbeillì é l'a pa eunc\'o acapoù l'ommo de sa viya, \nameF{Joëlle Bollon}}

\Character[MARIE\\ DE FELEUNNA]{MARIE DE FELEUNNA}{Marie}{Imitach\'on de \href{https://it.wikipedia.org/wiki/Maria_De_Filippi}{Maria de Filippi}, \nameF{Sophie Comé}}

\Character[TANTEUN]{TANTEUN}{Tanteun}{Eun \textit{tronista} de la trasmech\'on \href{https://it.wikipedia.org/wiki/Uomini_e_donne}{\textit{Uomini e Donne}}, \textit{tombeur de femmes}  que vou acapé la fenna de sa viya, \name{Paolo Cima Sander}}

\Character[LOUISE]{LOUISE}{Louise}{\textit{Trans} inamoroù de Tanteun, \name{André Comé}}

\Character[SPÉRANCE]{SPÉRANCE}{Sperance}{Dzouveunna grasieuza mi bièn déterminaye a gagné lo queur de Tanteun, \nameF{Stéphanie Albaney}}

\Character[TWITTER]{TWITTER}{Twitter}{Gars\'on eumpestoù é vichà, botcha de Alice é Tanteun, \name{Marco Ducly}}

\Character[TANTA FINE]{TANTA FINE}{Tanta}{Imitach\'on d'eunna BABY-SITTER/TATA de la trasmech\'on \href{https://it.wikipedia.org/wiki/SOS_Tata}{\textit{SOS Tata}} que eunsègne comèn terì si le botcha, \nameF{Ester Bollon}}

\Character[IPAD]{IPAD}{Ipad}{Feuille adolésante de Alice é Tanteun, \nameF{Stéphanie Albaney}}

\Character[GEPPINO]{GEPPINO}{Geppino}{Posteill\'on de la trasmech\'on \href{https://it.wikipedia.org/wiki/C\%27\%C3\%A8_posta_per_te_(programma_televisivo)}{\textit{C'è posta per te}}, \name{Simone Roveyaz}}

\Character[TURI TOUNEUN]{TURI TOUNEUN}{Turi}{Imitach\'on d'eun conséillì réjonal de la XIIIa léjislateua, \name{Joël Albaney}}

\Character[]{CHÉRIE E GUILLAUME}{ChEGui}{\quad}

\DramPer

\act[Acte I]

\ridoiver

\scene[-- Samsung\ldots lo quezeun de Balan]


\StageDir{\textsl{Lemie} \lemieSi\ a gotse di palque, iaou acapèn eun vétchot achouatoù si eun chofà. L'è eun tren d'atendre son meillaou amì.}

\begin{drama}

\Vioupspeaks\direct{I pebleuque} Di-me té si ommo! L'è todzor eun retar, maleur! Senque l'ayè tan a fé, me dimando mé. L'a pa de-z-oréo si ommo.

\StageDir{Se sen bouéchì a la pourta.}
			
\Vioupspeaks La fèi l'è lli!			
			
\Vioujspeaks \direct{Dèi dérì le rid\'o} Permì?

\Vioupspeaks Ouè veun maque!

\Vioujspeaks Salì!

\StageDir{L'amì d'Auguste entre.}

\Vioupspeaks Oh lo noutro, comèn l'è?

\Vioujspeaks N'a pamì de demì sèiz\'on!

\Vioupspeaks Sourtoù sit an!

\StageDir{Auguste s'avèitse a l'entor é comenche baillì de pégno creppe i solàn é i meur.}

\Vioupspeaks Senque t'i eun tren de fé?

\Vioujspeaks L’è-tì bon lo simàn?

\Vioupspeaks Lo simàn? Pe fose que l'è bon. T'a eunc\'o finque bailla-me eunna man a eumpasté la betumeuza! Te rapelle pa? L'ion le-z-àn 70! Mimo ten que l'an fé le-z-icoulle a Tsarvensoù.

\Vioujspeaks Fran sen que me tracache! Fé-lo controlì! 

\Vioupspeaks Mi soplé, achouatta-té, macaco!

\StageDir{Auguste s'achouatte protso Joseph.}

\Vioupspeaks Ad\'on, que te counte? T'atégnao pi vito. T'i todzor lo mimo!

\Vioujspeaks N'i rontì la moutra é n'i pamì eugn oréo.

\Vioupspeaks Pequé? Se te ron la moutra te pou pamì sèi quent'aoua l'é?

\Vioujspeaks Na, avèitso le-z-itèile mi dèyo atendre la nite pe sèi l'aoua.

\Vioupspeaks Acoutta eunna bagga, pequé te fé pa comme tcheu le-z-atre Tsarvensolèn que l'an la noua magniye pe sèi l'aoua?

\Vioujspeaks Queunta?

\Vioupspeaks Lo mateun, can van travaillì, se euncrouèijon salla que galoppe pe lo plan d'Ampaillan, ad\'on son tranquilo de teumbré eugn oréo; can, i contréo, la euncrouèijon i Fossà, son belle que tracachà!

\Vioujspeaks Ah!\direct{Eun tsandzèn discoù é eun moutrèn eun poueun douàn lli} Belle dzen lo nouo cadre que t'a pendì i meur?

\Vioupspeaks Queun cadre?

\Vioujspeaks Si lé to neur, dzen!

\Vioupspeaks Si lé to neur? Car\'o?

\Vioujspeaks Ouè, to neur, dzen!

\Vioupspeaks Ouè, dzen si cadre! N'i atseto-lo sit an a la Fèira, lo nouo sculteur, si itrandjì que l'a portoù pe lo premì cou!

\Vioujspeaks Cougniso, cougniso\ldots

\Vioupspeaks L'at eunc\'o finque lo non marc\'o dézò\ldots Samsung! Te lo cougnì? L'è fameu!

\Vioujspeaks Djèique que lo cougniso!

\Vioupspeaks Macaco! Te vèi pa que l'è eunna télévij\'on i LED!

\Vioujspeaks\direct{\'Eton-où} Télévij\'on i senque?

\Vioupspeaks I LED! Ara te l'aleummo\ldots

\StageDir{Joseph tsertse la télécomanda.}

\Vioupspeaks Te baillo eunc\'o le gran de meurga di-z-Amériquèn\ldots

\StageDir{Joseph teurie foua eun grou sidel de Popcorn é lo baille a Auguste. Aprì aleumme la télévij\'on. Can s'aleumme, la lemie é le couleur son tellamente forte que Auguste dèi clloure le joueu!}

\Vioupspeaks Avèitsa que couleur!

\Vioujspeaks Mondjemé! Atèn que beutto le lenette di solèi!

\StageDir{Auguste se beutte eun grou per de lenette di solèi.}

\Vioujspeaks Oh ouè! Ara ouè, fran dzen! Mi\ldots n'at eunna émich\'on que comenche? Senque l'é?

\Vioupspeaks \`Eita\ldots son euntellijante seutte télévij\'on. Baste gnaqué eun bot\'on é te di totte.

\StageDir{Joseph gnaque eun bot\'on é li lo titre de l'émich\'on.}

\Vioupspeaks ``Comèn te t'arbeuille?!".

\Vioujspeaks\direct{Douteu, eun s'avèitsèn le pateun} De pantal\'on, ganeuss\'on, la tsemize é lo tsapì!

\Vioupspeaks Mi senque t'i eun tren de me diye té?

\Vioujspeaks M'arbeuillo pouèi mé!

\Vioupspeaks Mi na! ``Comèn te t’arbeuille?!" l’è lo non de l'émich\'on! Te sa beun senque l'é?

\Vioujspeaks Na! Mé avèitso pa seutte bague.

\Vioupspeaks \ldots é ad\'on dimanda touteun! Donque\ldots te euspleuccon comèn t'arbeillì a la modda.

\Vioujspeaks N'i la fèi que t'avèitse pa tan chouèn seutta émich\'on. T'a todzor le mimo dra de can t'ayè 20 an!

\Vioupspeaks\direct{Eun rièn} Té ouè que t'i seumpateucco! Ara quèi que comenche.

\StageDir{Teuppe \lemieBa .}

\scene[-- Mi comèn te t'arbeuille?!]

\StageDir{Lemie \lemieSi\ a drèite.}

\StageDir{Eun scène n’a Chérie, Guillaume é eun manequeun arbeillà avouì eun dzen coteill\'on. Comme fondal n'at eunna tèila desì laquelle l'è proyettoù lo logo de la trasmech\'on.}

\Guillaumespeaks Bonsouar a tcheutte! Oueu a la trasmech\'on ``Mi comèn te t'arbeuille?!" no éidzèn Alice, 28 an, eumplouayé réjonalla.

\Cheriespeaks T'a bièn deu Guillaume, oueu lèi tsandzèn sa magniye de s’arbeillì. Alice entra maque!

\StageDir{Entre Alice avouì eun pourtamantì  plen de dra.}

\Alicespeaks Bonsouar!

\StageDir{Guillaume é Chérie braillon épouvant\'o.}

\Guillaumespeaks Comme te vèi Chérie, Alice fé caquì!

\Cheriespeaks Guillaume t’i finque tro jantilo. 

\Alicespeaks Mi comèn? Oueu n'i finque betoù le botte de la mima couleur de la tsemize. Can mimo l'è beun vrèi que tcheutte me diyon que m'arbeuillo pa tan bièn\ldots é l'è fran pe sen que si viin-a seuilla: pe vo dimandé eunna man.

\Cheriespeaks L'an rèiz\'on le teun-z-amì! Mi ara alèn veure le pateun que l'at!

\Guillaumespeaks Alèn veure son armouare!

\StageDir{Chérie é Guillaume s'aprotson i pourtamantì de Alice. Chérie pren eunna maille é iternèi.}

\Cheriespeaks Mondjeu! Si allerjeucca i tissù di tchinèis!

\StageDir{Chérie braille comme se l'ise vu lo djablo.}

\Guillaumespeaks Mi senque t’a vi? Eun rat?

\Cheriespeaks Pire! Avèitsa!

\StageDir{Chérie teurie foua eun cachiné\footnote{ \textit{\'Echarpe} eun fransé ou \textit{sciarpa} eun italièn.}.}

\Alicespeaks Dzen sitte! L'a fé-me-l\'o magàn!

\Cheriespeaks Magàn?

\Guillaumespeaks Magàn? Ouè, avouì eunna man é avouì l’atra fiave la sepetta!

\StageDir{Chérie tappe ià lo cachiné.}

\Alicespeaks\direct{Dispéraye} Mi na!

\Cheriespeaks Ouè se pouè pa veure.

\StageDir{Chérie teurie foua eun foulard blan.}

\Cheriespeaks\direct{Ver Alice} \ldots é so?

\Alicespeaks So can fi frette.

\Cheriespeaks\direct{Digoutaye} Mondjeu flèrie finque!

\StageDir{Alice pren lo foulard di man de Chérie.}

\Alicespeaks \ldots avèitsa te fio vère: can fé frette mé fio pouai\ldots

\StageDir{Alice moutre a Guillaume é Chérie comèn se toppe di frette: avouì lo foulard se fèiche la tita.}

\Cheriespeaks Mi l'è pa eun \textit{burqa}! \'E t'i pa pi eunna seur! Ià so!

\StageDir{Alice baille lo foulard a Chérie.}

\Guillaumespeaks\direct{Ver Alice} Avèitsa dérì de té! Lo \textit{dahu}!

\StageDir{Alice se vionde pe veure. Sensa pédre de ten, Chérie tappe ià lo foulard.}

\Alicespeaks Mi n'i pa vi-lo!

\Cheriespeaks Mi que damadzo\ldots ara Alice te prézentèn lo bon gou. Te fièn veure comèn te dériye t'arbeillì.

\StageDir{Guillaume é Chérie moutron a Alice lo manequeun.}

\Cheriespeaks Voualà! N'en cherdì pe té eun tissù de Paris: Louis de Cagnon.

\Alicespeaks Senque?

\Cheriespeaks Louis de Cagnon! Totsa maque lo tissù!

\StageDir{Alice totse eun tissù a case.}

\Guillaumespeaks Mi na! Senque te fé? N'en deu-te Louis de Cagnon!

\Alicespeaks Mi senque vou diye?

\Cheriespeaks Cot\'on! Ouff\ldots

\Guillaumespeaks Ara que te cougnì tcheu le secret de la modda, va fé la spèiza di dra, pren la carte de \textit{crédit} é que lo bon gou\ldots

\ChEGuispeaks \ldots sise avouì té!

\scene[-- A tsasse d'éléganse]

\StageDir{Guillaume fé tsire pe tèra la carte. Alice la pren é galoppe foua totta contenta. Si la tèila veun proyéttaye eunna vidéo iaou Alice entre pe eun négose pe atseté de-z-arbeillemèn. Guillaume é Chérie, dimèn que avèitson la vidéo, dzeudzon sen que combin-e Alice\footnote{ Tanque a la feun de la proyéch\'on, le-z-ach\'on é le battiye de Alice icrite deun si tecste son salle que lleu eunterprète deun la \textit{vidéo}.}.}

\StartVideo{https://youtu.be/boRH83CCIPM}{Comèn te t'arbeuille?!}{

\Alicespeaks\direct{Eun entrèn deun lo négose, ver la serventa} Mé si l'amia de Chérie é Guillaume de la trasmech\'on ``Mi comèn te t'arbeuille?!"\ldots

\Guillaumespeaks Ah! Mé la denoncho!

\Cheriespeaks Mi sel\'on té n'en de-z-amie que s'arbeuillon mal pouai?

\Alicespeaks Fio eun tor, avèitso tchica de bague\ldots

\Guillaumespeaks Eun tor? Mi arbeillaye pouai te fé eun tor i bitche!

\Cheriespeaks Ouè pa de pi! Avèitsèn sen que pren! 

\StageDir{Alice cher eun coteillón bièn dzen mi lo pren pa.}

\Cheriespeaks Mi si lé l'ie dzen! \direct{Eun avèitsèn la vidéo} \'E ara\ldots dzen sise peundeun, i mitcho nen n'i eun per igale\ldots

\StageDir{Alice cher eunna jupe orribla.}

\Alicespeaks Sitte ouè! Avouì seutta jupe sario presta pe fé eun apér\'o

\Guillaumespeaks A Comboé te fé l'apér\'o!

\Cheriespeaks Salla jupe te la eumplèye comme nappa!

\StageDir{Alice pren eun coteillón orriblo.}

\Alicespeaks Ouè! So l'è cheur comèn s'arbeuille Chérie, l'è fran sa magniye de s'arbeillì!

\Cheriespeaks\direct{Scandalizaye} Mi sel\'on té m'arbeuillo pouai? Guillaume! Galoppa! Vito!

\Guillaumespeaks Vou mé! Lèi penso mé!

\StageDir{Guillaume chor de \textit{scène} é i mimo ten entre deun la vidéo. Reprodze Alice pe le beur pateun que l'a djeusto cherdì é lèi pren de coteill\'on élégàn.}

\StageDir{L'écran se tchoué.\newline}
}

\Cheriespeaks\direct{I pebleuque} Pe forteun-a l'è arrevoù Guillaume! Vo rendade contcho comèn s'arbeuille salla fenna? Guillaume!

\StageDir{Entre Guillaume.}

\Guillaumespeaks Chérie, n’i pensou-lei mé. Alice l'è fantastique. Alice entra maque!

\StageDir{Entre Alice avouì eun dzen coteill\'on neur.}

\Cheriespeaks Oh que dzenta que t'i! T'i contenta? T'i dza vi-te i meriaou?

\Alicespeaks Na!

\Cheriespeaks Vou lo prendre to de chouite.

\StageDir{Chérie chor é torne dedeun avouì eun grou meriaou.}

\Cheriespeaks\direct{Ver Alice} T'i presta? Eun, dou, trèi\ldots

\StageDir{Chérie vionde lo meriaou ver Alice.}

\Alicespeaks \textit{Wow}! Mi si fran mé?

\Guillaumespeaks Ouè Alice!

\Alicespeaks Na lèi crèyo pa!

\Cheriespeaks Oué é to mersì a no!

\Alicespeaks Mersì Guillaume, mersì Chérie!

\Cheriespeaks\direct{A Guillaume} Acoutta Guilaume, va veure deun lo poubleuque se n'a lo pappa de Alice, mogà l'a quetsouza a no deu.

\Guillaumespeaks\direct{Ver lo pebleuque} V'ouite lo pappa de Alice? Ah na... eugn éffé l'è tro arbeillà amoddo pe itre lo pappa de Alice\ldots probablemàn l'an pa fé-lo entré.

\Cheriespeaks Ouè cheur. \direct{Ver Alice} Mi ad\'on di-no sen que te nen pense té vi que lo pappa l'è pa.

\Alicespeaks Ara crèyo beun d'itre an mia pi dzenta, fran dzenta\ldots mi n'at eunc\'o eun petchoù problème.

\Cheriespeaks Queun?

\Alicespeaks Me manque eugn ommo!

\Cheriespeaks \ldots é ad\'on senque te atèn? Va lo tsertchì! Galoppa!

\StageDir{Tcheu trèi chorton contèn.}

 \StageDir{Teuppe \lemieBa .}

\scene[-- Dou Sen Pé é eun saout\'on]

\StageDir{Lemie a gotse \lemieSi .}
  
\Vioujspeaks Mi baillon-tì todzor seutta counta foula?

\Vioupspeaks Ouè, ll'è eunna counta noua tcheu le dzor. A propoù de counte\ldots é salla di Sen Pé? Avouì totte le dzi que trebeullon deun la viya é dèyon travaillì tanque a 70 an, pe lo premì cou deun l'istouére l'è all\'o lo Sen Pé eun pench\'on!

\Vioujspeaks Pouè pa fé comme tcheu le-z-atre?

\Vioupspeaks Ouè pequé son de chouà que porton eunc\'o de consecanse! Imajinate-l\'o eun quiya a la pousta pe reterì la pench\'on! 

\Vioujspeaks Prèdza-mé pa de pench\'on! Avouì sise cattro tri que n'i areuvvo pa a la feun di mèis! Mi ara tsandzerèn le bague\ldots pequé mé baillo pi la vouése a sise di saout\'on !

\Vioupspeaks A qui?

\Vioujspeaks Saout\'on \ldots Grillo! Pequé llou, avouì le noue-z-éléch\'on nachonalle, tsadzeré totte!

\Vioupspeaks Le noue-z-éléch\'on nachonalle? Mi se n'en djeusto fé-le eun mèis fi!

\Vioujspeaks Comèn que n'en djeusto fé-le? N'isàn fé-le, le bague sarian tsandjaye. Avèitsa si i Gouvernemàn: n'a todzor le mimo é lo Prézidàn de la Repebleucca\ldots tourna llou!

\Vioupspeaks Eugn éffé t'a beun pa tcheu le tor gneunca té.

\Vioujspeaks\direct{Eun avèitsèn la télévij\'on} Can mimo\ldots la réclamme l'è fenia.

\Vioupspeaks Ouè. Senque gnouerè ara? Dimandèn a Samsung.

\StageDir{Joseph gnaque eun bot\'on de la télécomanda.}

\Vioupspeaks \og Ommo é Fenne\fg\ldots te sa senque l'è?

\Vioujspeaks\direct{Eumbarachà} Na avèitso pa seutte bague.

\Vioupspeaks Ad\'on demanda! L'è suffizàn dimandé a Samsung é llou rep\'on .

\StageDir{Joseph gnaque eungn atro bot\'on.}

\Vioupspeaks Ah ouè n'i comprèi\ldots donque pe la diye avouì de paolle seumple, ll'et eugn ommo que dèi chédre de fenne\ldots tchica comme i martchà di vatse. La prézentatrise l'è Marie de Feleunna, la blounda.

\Vioujspeaks Ouè cougniso\ldots l'ommo l'è\ldots atèn que me veun pa lo non\ldots

\Vioupspeaks \ldots mondjeu eunc\'o mé me veun jamì son non\ldots ah oué! Maurice, Maurice!

\Vioujspeaks Ouè Maurice Costan! Couscrì a té?

\Vioupspeaks Na touteun, l'è bièn pi dzouveun-o. Can mimo mersì, vou deu que pourto fran bièn le-z-àn!

\Vioujspeaks Ara quèi que gnoue.

\StageDir{Auguste se beutte le lenette di solèi.}

\StageDir{Teuppe \lemieBa .}

\scene[-- Ommo é Fenne]

\StageDir{Lemie \lemieSi\ si la drèite di palque.}

\StageDir{I mentèn n'at eunna poltronna eun pi, a gotse trèi caèye coloraye é a drèite eun tabouré de bouque. Comme fondal n'at eunna tèila desì laquelle l'è proyettoù lo logo de la trasmech\'on é la fotografiye de Marie de Feleunna.}

\StageDir{Entre Marie de Feleunna dézò le notte di jénérique:}

\sound{https://www.youtube.com/watch?v=3i1OlH1e1go}{\textit{Uomini e Donne - Sigla}}%\label{uominidonne}

\Mariespeaks Bonsouar a tcheut é bienvin-ì a la 43$^a$ é deriye émich\'on de ``Ommo é Fenne"! Oueu, noutro Costanteun, pe tcheut ``Lo Dzen Tanteun", cherdérè sa compagne de viya. Ad\'on fien-l\'o to de chouite entré: Tanteun!

\sound{https://www.youtube.com/watch?v=3i1OlH1e1go}{\textit{Uomini e Donne - Sigla}}[false]%\label{uominidonne}

\StageDir{Di pebleuque pouye si lo palque Tanteun, tot arbeillà élégàn. Dimèn que fé lo défilé, mande de poteun é saliye le fenne di pebleuque.}

\Tanteunspeaks Bonsouar a tcheut, bonsouar Marie, mi sourtoù bonsouar a totte le feuille que m'avèitson, me mioulon é que son belle itsaoudaye pe mé seutta nite! 

\StageDir{Tanteun é Marie s'achouatton: Tanteun si la poltronna (lo trone) é Marie si lo tabouré de bouque.}

\Tanteunspeaks\direct{Ver Marie} Aprì to lo ten pass\'o eunsemblo avouì totte le feuille que n’en cougnì sen arrevoù a la feun de seutta dzenta trasmech\'on . Te dèyo diye eunna bagga Marie: l'è fenì lo meun ten da fleungàn!

\Mariespeaks  Ouè, te faré praou ataqué a betì la tita a poste!

\Tanteunspeaks La tita a poste? Marie t’arito to de chouite. Dièn que tsertseriyo de fé tchica l'ommo tranquilo pe le prochèn dou ou trèi dzor.

\Mariespeaks Dou ou trèi dzor? Tanteun! Seutta l'è eunna trasmech\'on sérieuza! Fa pa fé seutte bague. Pe eunna tin-a désij\'on pouriye nèitre eunna noua fameuille.

\Tanteunspeaks\direct{Eunna mia perplecse} Va bièn, ad\'on\ldots comenchèn?

\Mariespeaks Ouè, perdèn pa d'atro ten. Fièn entré la premiye feuille. Oueu lo nite n'en Louise, 20 an, eunna mia particuilliye\ldots eun tsertsa de l'ommo de sa viya. Ad\'on fien-là entré. Louise!

\sound{https://www.youtube.com/watch?v=3i1OlH1e1go}{\textit{Uomini e Donne - Sigla}}[false]%\label{uominidonne}

\StageDir{Di pebleuque pouye si lo palque Louise. Pèi rouze é avouì de tsambe eunna mia tro peluye, Louise fé lo défilé eun salièn tcheut. Aprì s'aprotse a Tanteun. Louise prèdzerè todzor avouì eunna vouése déguizaye, bièn femminile.}

\Louisespeaks Bonsouar Tanteun, que dzen te reveure, fio pa d'atro que pensì a té.

\Tanteunspeaks\direct{Eunna mia digout\'o} Bonsouar Louise, \direct{ironique} si fran contèn de te veure.

\Louisespeaks M'achatto?

\Tanteunspeaks Ouè ouè!

\StageDir{Louise s'achatte si eunna caèya a drèite de Tanteun.}

\Mariespeaks La secounda feuille l'è Alice: l'at 28 an, eumplouayé réjonalla é l'è bièn lamaye di noutro Tanteun. L'a maque eun petchoù problème: l'é pa tan bon-a de s'arbeillì. Ad\'on vièn sen que l'a fé, oueu lo nite, pe plère a noutro Tanteun. Alice! Entra!

\sound{https://www.youtube.com/watch?v=3i1OlH1e1go}{\textit{Uomini e Donne - Sigla}}[false]%\label{uominidonne}

\StageDir{Di pebleuque pouye si lo palque Alice avouì lo coteill\'on cherdì pe Guillaume é Chérie. S'aprotse a Marie é Tanteun.}

\Alicespeaks \direct{Dousa é gracheuza} Bonsouar a tcheut, Bonsouar Tanteun. 

\Tanteunspeaks Bonsouar Alice.

\Alicespeaks Spéo que t'ise pass\'o eunna bon-a senà?.

\Tanteunspeaks Ouè bièn amoddo\ldots

\Alicespeaks Mé n'i tsertchà de fé lo poussiblo pe te plére tchica de pi, va bièn?

\Tanteunspeaks\direct{Eun avèitsèn to lo cor de Alice} Ouè, me semble que quètsouza l'è beun meilloroù!

\Alicespeaks Ad\'on vou m'achaté.

\StageDir{Alice s'achouatte protso de Louise.}

\Mariespeaks\direct{I pebleuque} Voualà, é ara la dériye feuille l'è noutra Spérance: lleu s'atèn pa de grouse bague de seutta trasmech\'on,  lleu vou maque eugn ommo pe bèye eun cafì de tenzentèn lo desandro ou pe fé de chortiye caque cou. Ad\'on fièn entré Spérance.

\sound{https://www.youtube.com/watch?v=3i1OlH1e1go}{\textit{Uomini e Donne - Sigla}}[false]%\label{uominidonne}

\StageDir{Di pebleuque pouye si lo palque Spérance avouì eun coteill\'on d'ipaouza. F\'e eun petchoù défilé é s'aprotse a Marie é Tanteun.}

\Sperancespeaks Bonsouar a tcheut, bonsouar Tanteun!

\Tanteunspeaks\direct{\'Eton-où} Bonsouar, amoddo? Mi comèn t'i ar\-be\-illaye?

\Sperancespeaks Squiza, mi n'i fran betoù la premiye baga seumpla é comodda que n'i acapoù pe l'armouare.

\Tanteunspeaks La pi seumpla é comodda?!

\StageDir{Spérance s'achouatte protso de Alice.}

\scene[-- Le-z-arme de l'amour]

\Mariespeaks Ara que no sen tcheutte inque, mé dze vo fériyo to de chouite vère eunna \textit{vidéo} de sen que la combin-à seutta senà noutro dzen Tanteun. \textit{Vidéo}!

\StageDir{Si la tèila veun proyettaye la chortiya amoureuza de Tanteun é Spérance: eunna romantinque sin-a i restoràn \textit{Monte Emilius}.}

\StartVideo{https://youtu.be/1pOl7vbd-v0}{Amour i Monte Emilius}

\Louisespeaks Mi Spérance, spleca-mé eunna baga: pequé t'a pa betoù si grou-z-arbeillemèn blan, que te semble eunna pou\-et\-ta féta i crotset, pe alì prendre eun apéritif ou medjì sin-a?

\Sperancespeaks  N'i la fèi, Louise, que té t'a pa vi amoddo la \textit{vidéo}\ldots é can mimo fran té te prèdze? Que t'i todzor arbeillaye comme eunna patoille? Iaou te crèi de itre? A Rio de Janeiro? Avèitsa que lo carnaval l'è fenì é dèi tchica!

\Louisespeaks Mi Marie, t’a sentì comèn m’eunseurte? Acouta seuilla bella! Mé si eunna patoille mi oumouèn n'i le bale pe deu sen que penso\ldots pa comme vo doe! Eunna reste quèya é l'atra se catse dérì eun grou-z-arbeillemèn blan!

\Alicespeaks  Louise! Rapella-té que pe prédjì maque pe baillì flo a la botse, l’è mioù resté quèi\ldots ou quèyà; \direct{ver Spérance} é té planta-là lé de te arbeillì avouì sise patrasse blan. Pequé Tanteun pourteré mé a l’aouté\ldots avouì mon charme!

\StageDir{Spérance, ofenchaye, se levve é s'apreste pe chotre.}

\Sperancespeaks Marie! Mé eun seutta condich\'on queutto la trasmech\'on. Me sento fran prèiza pe lo \textit{C}.

\StageDir{Alice se levve é eunvite Spérance a s'achouaté.}

\Alicespeaks Mi soplé saquetta! Tourna t'achouaté!

\Sperancespeaks Senque? Di-mé pa sen que dèyo fiye!

\StageDir{Le doe s'achouatton, mi conteneuvvon a tsacoté eun lévèn todzor pi la vouése. Dimèn\ldots}

\Louisespeaks\direct{A Tanteun} Tanteun fièn eunna danse?

\StageDir{Tanteun se levve é comenche a danchì avouì Louise. Marie, dimèn que Alice é Spérance mollon pa de tsacoté, avouì eun jeste fé partì:}

\sound{https://www.youtube.com/watch?v=qiiyq2xrSI0}{Unchained Melody - The Righteous Brothers}

\StageDir{Alice é Spérance s'apersèison de la danche é s'ariton de s'euncheurté. Louise é Tanteun son todzor pi llatoù: s'avèitson, s'acarèson, s'eumbrachon\ldots Tanteun, dousemàn, bèiche le man é sare lo qui de Louise. A si poueun Alice é Spérance se levvon é divijon Tanteun é Louise.}

\Alicespeaks\direct{A Tanteun é Louise} Oh mi senque fiade?

\Tanteunspeaks Eunna pégna danse, ren d'atro.

\Sperancespeaks Me semble fran pa lo ca de fi seutte danse!

\Tanteunspeaks Mondjeu que counte foule!

\StageDir{Tcheutte s'achouatton i leur poste.}

\Alicespeaks Marie! Mé si pa viin-a seuilla pe vère dou mèinoù que rijon to di lon. Dimanderiyo a Mediaset de pourtì foua seutte doe caramban-e!

\Mariespeaks Praou! Praou! Praou! Ara l'è lo mo\-màn iaou noutro Tanteun dèi chédre sa compagne de viya. Tanteun te dèi to de chouite me baillì lo non de la premie feuille de oueu lo nite.

\Tanteunspeaks\direct{Eundesì} L'è defesilo\ldots senque diye. Donque\ldots comencheriyo avouì\ldots Louise.

\StageDir{Louise souspire, spéranseuze.}

\Tanteunspeaks Louise n'i porto-te seuilla perqué tsertsavo de bague noue, diféente, mi si apersi-me que a la feun di contcho l'è fran la normalitoù sen que tsertso. Ad\'on pouì maque te diye que t'i pa la fenna pe mé.

\Louisespeaks \direct{Eun plaouèn} \'E beun Tanteun, te sa pa sen que te te per! T'i eunna grousa déluj\'on .

\Tanteunspeaks Touteun si beun que t'i pa eun gramo gars\'on ,\direct{eumbarachà} eunna grama feuille! Mi t'a pa i de chanse, squeza-mé.

\Mariespeaks Louise, t’i restaye-lèi praou mal\ldots pren eun motchaou.

\StageDir{Marie baille son motchaou a Louise.}

\Louisespeaks Mersì Marie. Tanteun l’ayè bailla-me l’eumpréch\'on de itre bièn euntérésoù a mé. Ara me nen vou!

\StageDir{Louise chor.}

\Mariespeaks Me diplé fran tan pe Louise. Mi no fa alé eun devàn. Tanteun, avouì qui te vou prédjì? Bailla-n\'o eun non.

\Tanteunspeaks Eugn atro non? L'atro non l'è si de Spérance.

\StageDir{Spérance sourì.}

\Tanteunspeaks Spérance\ldots ad\'on la viya l'è pa maque prendre eun seumplo apéritif eunsemblo ou eunna crè sin-a i restoràn. Donque, te diyo que t'i pa té la fenna pe mé.

\Sperancespeaks Mi Tanteun! Sel\'on té dze si arbeilla-me pouèi pe senque? T'i fran eun mar\'on! 

\StageDir{Spérance chor.}

\Tanteunspeaks\direct{I pebleuque} Eunc\'o di mar\'on n'i tchap\'o .

\Mariespeaks Ad\'on dze pouì pensé que la feuille que t'a cherdì l'è\ldots

\Tanteunspeaks\direct{Eun sourièn} Alice!

\StageDir{Alice se levve émochonaye.}

\Tanteunspeaks Mi avèitsa que dzenta, veun sé!\direct{Eun moutrèn son dzin-ao} Achatta-té sé!

\StageDir{Alice s'achouatte si Tanteun.}

\Tanteunspeaks\direct{A Marie} Te vèi que dzenta? \'E l'è pa maque euntellijanta, sa sourtoù restì quèya can se dèi! Caletaye i dzor de oueu bièn eumpourtanta!\direct{Ver Alice} Senque te nen di? Alèn ià?

\StageDir{Alice sourì amoureuza. Le dou chorton avouì lo refrèn de la trasmech\'on .}

\StageDir{Teuppe \lemieBa .}

\scene[-- Tsarvensoù-Becca-de-Noua]

\StageDir{Lemie \lemieSi\ a gotse.}

\Vioujspeaks Baillon de-z-émich\'on bièn euntellijante!

\Vioupspeaks Maleur, maleur! Deh, senque l'è salla réclamme?\direct{Eun avèitsèn la télévij\'on} \`Eita ! To lo bouque que beurle\ldots ll'è eunc\'o eunna verdzasse \ldots é ara peuque eun bonb\'on. Mi senque fé? Levve la patta é\ldots pette?! 

\StageDir{Le dou tchoppon di riye.}

\Vioujspeaks Eun verdzasse que pette!

\Vioupspeaks Mi se pou-tì de bague di janre?! Mé n'i jamì vi de verdzasse pété. Maleur la télévij\'on. N'i todzor deu que medjì de bonb\'on fé pa tan di bièn. Todzor mioù eun dzen toque de saouseuse.

\Vioujspeaks Ouè! Mi tsandzèn discoù:  n'i pamì sentì prédjì de Corméyaou\ldots salla counta\ldots

\Vioupspeaks Queunta? Lo caille (FRANA?) de La Saxe?

\Vioujspeaks Na\ldots que vouillon tsandjì lo non de la quemeua eun  Courmayeur-Mont-Blanc.

\Vioupspeaks Ouè! Te sa pa la dériye: eunc\'o noutro Seunteucco l'a désidoù de tsandjì lo non de la Quemeua. L'a dimand\'o lo condjà i Prézidàn! Te sa comèn vou la querì?

\Vioujspeaks Na.

\Vioupspeaks Tsarvensoù-Becca-de-Noua! 

\StageDir{Le dou tchoppon di riye.}

\Vioupspeaks A fose l'è fenia la réclamme!

\Vioujspeaks A fose !

\Vioupspeaks\direct{Eun prégnèn la télécomanda} Avèitsèn sen que gnoue\ldots  Samsung!

\StageDir{Joseph gnaque eun bot\'on . Le dou saron le joueu pe lie.}

\Vioupspeaks Ara comenche ``Èidza-mé tanta". \direct{Ver Auguste} Te sa senque l'è?

\Vioujspeaks Na, avèitso pa seutte bague!

\Vioupspeaks Ad\'on demanda!\direct{Eun gnaquèn eugn atro bot\'on} Samsung! Donque\ldots eunna fameuille dispéaye\ldots dimande l'èidzo de tanta Fine. Té t'a comprèi quèichouza?

\Vioujspeaks Na\ldots

\Vioupspeaks Mé gnenca; ad\'on quèi é avèitsa.

\StageDir{Teuppe \lemieBa .}

\scene[-- Èidza-mé Tanta]

\StageDir{Lemie \lemieSi\ a drèite, iaou acapèn doe table: eunna pe maenda é l'atra pe travaillì avouì l'ordinateur.  Comme fondal n'at eunna tèila desì laquelle l'è proyettoù lo logo de la trasmech\'on ``Èidza-mé Tanta".}

\StageDir{A gotse de la \textit{scène} n'at Alice. Achouataye, se fé le-z-onlle é avèitse son mèinoù, Twitter, que djouye dézò la tabla. A drèite n'at Tanteun, atatchà a l'ordinateur, bièn eungadjà avouì lo traille. Twitter, botcha fran bièn eumpestoù, molle pa de fé de braillo é de eunfastedjì pappa Tanteun.}

\Tanteunspeaks Vio de bague que n'i eunc\'o a fiye oueu, maladetto d'eun travaille\ldots 

\StageDir{Twitter, avouì eun pistolé a éve, icreutte d'éve i vezadzo di pappa.}

\Tanteunspeaks \ldots é planta-là eunc\'o té souplì, botcha eumpestoù! Avouì totte le bague que\ldots

\StageDir{Twitter icreutte eunc\'o d'éve.}

\Tanteunspeaks\direct{Malechà} Molla!

\Alicespeaks Papi! Papi?

\Tanteunspeaks\direct{Inervoù} Mé si pa papi a té! Véo de cou n'i deu-te que mé si l’ommo! Mar\'on mé que, si dzor a salla trasmech\'on de Marie de Feleunna, n’i cherdi-te. N’iso cherdì Spérance!

\Alicespeaks\direct{Eun lèi baillèn pa fèi} \ldots é beun, papi, te sa senque l'è capit\'o ? 

\Tanteunspeaks Na!

\Alicespeaks Sit an i Grande Fratello nen prègnon maque onze a la plase de doze comme l’an passoù!

\Tanteunspeaks\direct{Ironique} Mi pensa que bague bièn eumpourtante: onze pitoù que doze!

\Twitterspeaks Pappa! Pappa!

\Tanteunspeaks Senque?

\StageDir{Twitter teurie eun bal\'on eun tita a Tanteun.}

\Twitterspeaks Booom!

\Tanteunspeaks\direct{Malechà} Maladetto d'eun botcha eumpestoù! Mi va djouì i mentèn de la rotta réjonalla!

\Alicespeaks\direct{A Twitter, sourienta} Ah lo pop\'on de mé! Papiii!

\StageDir{Silanse.}

\Alicespeaks Papiii!

\Tanteunspeaks Mi senque n'at?

\Alicespeaks Te sa senque l'a deu-me Tine?

\Tanteunspeaks Tin-e? Na\ldots

\Alicespeaks L'a deu-me que son eun tren de tsertchì de bouye pe fé le seuntchiye de Cavalli. 

\Tanteunspeaks\direct{Ironique} Todzor de bague eumpourtante t'a pe la tita.

\StageDir{Twitter teurie eugn atro bal\'on si la tita de Tanteun.}

\Tanteunspeaks Mi te la plante li! Te tchouèyo!

\Alicespeaks Mi na lo meun petchoù\ldots

\Tanteunspeaks\direct{Inervoù, ver Alice} Pren si botcha soplé que me fé gnenca travaillì!

\Alicespeaks Mi l'è tan sayo.

\Tanteunspeaks Tan sayo? Si cou n'a praou. Ara lèi penso mé!

\StageDir{Tanteun gnouye a icrie.}

\Alicespeaks Ouè pensa-lèi té\ldots Twitter veun avouì mamma.

\Tanteunspeaks Queun non l'a-tì salla trasmech\'on iaou veun salla viille, salla tanta, pe baillì eunna man pe seutte \direct{moutre Twitter} bague eumpestaye que te roviyon la viya?

\Alicespeaks Si pa.

\Tanteunspeaks ``\`Eidza-mé Tanta"! Voualà comèn se queurie. Ara lo acappo si Google\ldots

\StageDir{Tanteun icrì eunna \textit{e-mail}. Dimèn Twitter conteneuvve a lo eunfastedjì. Alice, i contréo, djouye avouì lli.}

\Tanteunspeaks\direct{A Twittter, greuffo} Se t'acappo! \direct{Eun icrièn} ``Souplì, chère tanta, si eun pouo ommo de 34 an, resto a Tsarvensoù. N'en eun botcha mar\'on é nen pouèn pamì. Souplì, veun seuilla a Feleunna 36". Voualà! 

\scene[-- Tanta Fine]

\StageDir{Se sen lo son d'eun:}

\effet{https://on.soundcloud.com/ovu3w}{Tchéqueun}

\Tanteunspeaks \ldots é ara qui n'at?

\Alicespeaks Papiii! Va té veure\ldots

\Tanteunspeaks Tourna avouì si Papi?! Papi l'è eun tren de travaillì. Va té veure!

\StageDir{Eun dzemotèn, Alice chor.}

\Alicespeaks\direct{Jantila} Bondzor, entrade maque.

\StageDir{Partèi lo jénérique de la trasmech\'on ``\`Eidza-mé Tanta".}

\sound{https://www.youtube.com/watch?v=BEXJa_A7TZw}{\textit{SOS tata - Sigla}}\label{tata}

\StageDir{Entron Alice é Tanta Fine. Fine l'é arbeillaye avouì eunna londze jupe bleua  é eunna djaquetta de la mima couleur. Avouì eun per de lenette si la poueunte di na, l'è bièn sérieuze é \textit{autoritaire}. Deun eunna man l'at eun dossier, deun l'atra eunna baquetta.}

\Tanteunspeaks\direct{Eun avèitsèn Fine} Mi seutta baga de iaou chor?

\Tantaspeaks\direct{Seuria} Mé si Tanta Fine.

\Tanteunspeaks\direct{Sourprì} Oh Tanta Fine! N'i djeusto mando-vo eunna \textit{mail}.

\Tantaspeaks Donque v'ouite vo Tanteun?

\Tanteunspeaks Ouè.

\Tantaspeaks Amoddo. N'i désid\'o de vegnì péqué si la pi saventa.

\Tanteunspeaks Ad\'on sen a poste!

\StageDir{Fine se vionde é vèi Twitter.}

\Tantaspeaks\direct{Eun s'aprotsèn a Twitter} Que dzen si pop\'on.

\StageDir{Fine sare for lo bouigno de Twitter.}

\Tantaspeaks Tchao! \direct{Eun sarèn eunc\'o pi for} Tchao! Comèn te te queurie?

\StageDir{Twitter, épouvant\'o , se catse dézò la tabla. Alice tsertse de téri-lo foua.}

\Alicespeaks\direct{Ver Twitter} Alé, di-lèi queun non t'a?

\StageDir{Twitter repoùn pa.}

\Tantaspeaks L'è tchica chor?

\Tanteunspeaks Na l'è mar\'on .

\Alicespeaks Mi na, l'è que mon pop\'on l'è tchica timido. \direct{Eun alondzèn la man ver Twitter} Veun seuilla\ldots

\StageDir{Twitter scappe.}

\Tanteunspeaks\direct{Eun braillèn} Mi di-lèi tchao!

\StageDir{Tanta Fine baille eunna baquétaye si lo qui de Twitter.}

\Twitterspeaks Ahia!

\Alicespeaks\direct{A Fine} Llou se queurie Twitter.

\Tantaspeaks\direct{Chocaye} Senque? 

\Alicespeaks Twitter!

\Tantaspeaks Mi v'ouèide de paent\'o ba pe de lé?

\Tanteunspeaks Na, na, pa de paent\'o pe la Basa Valaye.

\Alicespeaks Na, na, mé é mon ommo n’en pensoù de bailli-lei eun non actuel, moderno!

\Tantaspeaks Can mimo\ldots si pa se cougnisade le riille de seutta\ldots

\StageDir{Twitter tappe lo bal\'on countre pappa. Fine, pe pénitense, lèi baille eunna baquétaye si le dèi.}

\Twitterspeaks\direct{Avouì le larme i joueu} Ahia!

\Tanteunspeaks\direct{Satisfé} Ah t'a tchapou-la!

\Tantaspeaks \ldots si pa se cougnisade le riille de seutta trasmech\'on\ldots

\Alicespeaks Na.

\Tantaspeaks Vo fiade maque voutra viya de tcheu le dzor, mé me beutto pe eunna coueugne é can n'a caque bague que va pa \direct{avouì la BACCHETTA lévaye} mé eunterveugno!

\Tanteunspeaks\direct{Ver Fine} Va bièn, ad\'on mé conteneuvvo a travaillì?

\Tantaspeaks Ouè.

\StageDir{Tanta Fine se beutte pe eunna coueugne.}

\scene[-- Le trèi riille]

\Alicespeaks Amoddo\ldots ad\'on té Twitter fé lo sayo. Veun seuilla\ldots

\StageDir{Twitter s'aprotse a mamma é Alice lèi baille lo \textit{ciuccio}.}

\Alicespeaks\ldots fé lo sayo é fé pa malechì pappa. Mé vou presté maenda.

\StageDir{Alice chor. Tanta Fine, sévère, boueuche la baquetta desì la tabla.}

\Tantaspeaks Riilla numér\'o eun! Mi touteun Tanteun!\direct{Eun moutrèn Twitter} L'a caze choui-z-àn, ià si \textit{ciuccio}!

\Tanteunspeaks Proua-lèi té de lo gavì!

\StageDir{Tanta Fine s'aprotse a Twitter, pren lo \textit{ciuccio} é tsertse de teri-lèi-l\'o ià avouì bièn de forse. Twitter molle pa de saré le di. A si poueun, Fine lèi baille eunna baquétaye si lo qui.}

\Twitterspeaks Ahia!

\StageDir{Fine tappe ià lo \textit{ciuccio}. Twitter comenche a plaoué.}

\Tanteunspeaks\direct{Satisfé} T'a tourna tchapou-la avouì la tanta! Te vèi que a forse te comenche a comprendre.

\StageDir{Eun sentèn le braillo de Twitter, entre Alice avouì eunna cachoula é de-z-éze.}

\Alicespeaks Mondje senque l'è capitoù? \direct{Ver Twitter} Senque l'a fé-te?\direct{Malechaye, ver Tanteun} Senque t'a fé-lèi?

\Tanteunspeaks Ara l'è coulpa de mé! Todzor a mé! Dimanda a la Tanta!

\StageDir{Alice preste tabla pe maenda.}

\Alicespeaks Eunc\'o avouì le mèinoù te te la pren! Mi l'è pa poussiblo. \direct{Ver Twitter} Senque l'a fé-te pappa? A propoù Twitter\ldots oueu mamma l'a prestoù catro sate eun pila! Veun t'achatì que l'è preste.

\Twitterspeaks Na, pa voya!

\Alicespeaks\direct{Dousa} Sayo Twitter, fé pa malechì!

\Twitterspeaks N'i pa voya!

\Alicespeaks Twitter!

\Twitterspeaks Pappa! Pappa!

\Tanteunspeaks Senque?

\StageDir{Twitter teurie lo bal\'on eun tita a Tanteun.}

\Tanteunspeaks\direct{Malechà} Mondjeu, lo tchouèyo si botcha!

\Alicespeaks\direct{Eun braillèn} Todzor la mima counta eun si mitcho! Twitter l'è preste! \direct{Eunc\'o pi for} Twitter! Veun seu t'achouaté!

\Twitterspeaks Na n'i pa voya!

\StageDir{Alice areuvve pa a fée achouaté Twitter si la caèya; donque lo pren é lo plache pe tèra protso la tabla.}

\Alicespeaks Achatta-té seu. Todzor la mima counta! Te vou medjì avouì le man? Meudza avouì le man!

\StageDir{Alice pren lo platte é lo beutte pe téra devàn a Twitter. Tanta Fine eunterveun. Avouì la baquetta boueuche si la tabla.}

\Tantaspeaks \direct{Ver Alice} Mi l'è pa pi eun tseun! Riilla noumér\'o dou: se meudze tcheutte a tabla! \direct{Ver Twitter} Pren lo platte.

\StageDir{Twitter ignore Tanta Fine. Fine levve la baquetta. \'Epouvant\'o , Twitter galoppe ià é s'achouatte a tabla. Fine lèi pren lo platte é lo pouze si la tabla.}

\Tantaspeaks Tanteun.

\Tanteunspeaks\direct{Eun avèitsèn l'écran de l'ordinateur} Ouè\ldots

\Tantaspeaks\direct{Eun moutrèn la tabla} Feulla.

\Tanteunspeaks\direct{Avouì reuspé} Squeuza, to de chouite.

\StageDir{Tanteun s'achouatte a tabla.}

\Alicespeaks\direct{A Tanteun} Vèyo que beutte eun riilla eunc\'o té!

\Tanteunspeaks\direct{Eun avèitsèn lo platte} Mi senque t'a combinoù seuilla? Todzor le mime bague. Senque l'è so?

\StageDir{Tanteun pren avouì le man eun \textit{Sofficino} é lo boueuche si la tabla. Di son l'è deur comme eun bèrio : l'è pa it\'o praou didzal\'o . Tanteun, démoraliz\'o, tsertse de medji-lo.}

\Twitterspeaks Mamma! Cacca!

\Alicespeaks\direct{Dispéraye} Acoutta\ldots t'a lo Panpers, fé lé é ita tranquilo. 

\Twitterspeaks Na! Cacca, n'i fata!

\Tanteunspeaks Mi planta-là lé avouì seutta cacca! Reumplèi bièn lo pannoleun é can t'a reumpli-lo mamma te lo tsandze!

\StageDir{Tanta Fine eunterveun eun bouéchèn sa baquetta si la tabla. Tanteun tchappe pouiye.}

\Tantaspeaks Riilla noum\'er\'o trèi: \textit{pipì} é \textit{popò} dedeun lo pisepot.

\Alicespeaks Mi lo aprèn pa avouì la trasmech\'on di Teletubbies?

\Tantaspeaks\direct{Chocaye} Mi soplé! Vou prendre lo pisepot.

\StageDir{Fine chor é, doe seconde aprì, entre avouì lo pisepot. Lo pouze i mentèn di palque.}

\Tantaspeaks Voualà lo pisepot.\direct{Ver Twitter} Veun seuilla.

\Tanteunspeaks\direct{Eun rièn} Pensa-té se areuvve a caquì lé dedeun.

\StageDir{Twitter se boudze pa de la tabla. Fine, todzor avouì sa déterminach\'on, lo pren pe eun bouigno é lo teurie tanque i pispot.}

\Twitter\direct{Dimèn que lèi teurie lo bouigno} Ahia, ahia, ahia!

\StageDir{I mentèn di palque, Twitter comenche a djouì avouì lo pisepot. Fine lèi baille eunna baquétaye si le dèi.}

\Tantaspeaks Mi pa pouèi! Mi sitte l'è fran mar\'on pe dab\'on !

\StageDir{Tanta Fine plache amoddo lo pispot dez\'o lo qui de Twitter.}

\Tantaspeaks Voualà! Poucha! Ara poucha!

\StageDir{Twitter gnouye fé de grimasse.}

\Tantaspeaks\direct{Eun l'euncoadzèn} De pi, op!

\Tanteunspeaks Poucha!

\StageDir{Twitter conteneuvve avouì de grouse grimasse. Alice é Tanteun son surprì.}

\Alicespeaks\direct{Ver Tanteun} Avèitsa! Lo noutro pop\'on !

\Tanteunspeaks L'è eun tren de caquì! Mi pensa té!

\StageDir{Pappa é mamma se levvon é s'aprotson a Twitter. Son émochon-où. Cocolon, avouì eunna matse de complemèn, lo botcha que, to rodzo lo vezadzo, conteneuvve a pouchì.}

\Alicespeaks Lèi crèyo pa!

\Tanteunspeaks Ouè mi que grouse bague que teurie foua!

\Alicespeaks\direct{Ver Fine} Mersì Tanta!

\Tantaspeaks\direct{Eun avèitsèn lo pispot} L'è eun per de dzor que caque pa!

\Alicespeaks Pouo lo noutro Twitter\ldots

\Tanteunspeaks\direct{Ironique ver Alice}Avouì sen que meudze l'è normal itre costipoù!

\Tantaspeaks Can mimo\ldots mé pe oueu n'i fenì, torno demàn. Va bièn? Salì!

\sound{https://www.youtube.com/watch?v=BEXJa_A7TZw}{\textit{SOS tata - Sigla}}[false]%\label{tata}

\StageDir{Tanta Fine chor. Alice é Tanteun, contèn, la saliyon.}

\StageDir{Teuppe \lemieBa.}

\scene[-- La pendeula de la pousta]

\StageDir{Lemie \lemieSi\ a gotse, iaou acapèn maque Auguste achouatoù si lo chofà.}

\StageDir{Entre Joseph.}

\Vioupspeaks Squeza-m\'e, mi seutta \textit{prostata} me fé to di lon galoppé i ben! Mi tracacha-t\'e pa que n'i to sentì lé achouat\'o . \direct{Polemique} Di-me té se n'a fata de eunna tanta que vegnise te die comèn fa élevé ton mèinoù. Sen-tì pa matte?

\Vioujspeaks I dzor de oueu ouè! T'i té que te comprèn ren! T'i  vioù! Vioù!

\Vioupspeaks\direct{Caze malechà} Vioù? Mé? Te rapello que mé si bon a eumpl\'eì lo téléfoneun, alemé é tchouéye l'ordinateur é n'i eunc\'o finque aprèi a eumpl\'eì l'\textit{iPad}! Té t'i bon?

\Vioujspeaks Ouè ara dze aprègno pi eunc\'o mé a eumpl\'eì l'\textit{iPad}, vi que lo baillon a tcheutte. \textit{iPad} pe tcheutte!

\Vioupspeaks Ouè, n'i sentì seutta counta\ldots mi fa douàn  gagnì le-z-éléch\'on !

\StageDir{Le dou riyon.}

\effet{https://soundcloud.com/user-234168361/10oclock-30774}{Pendeula}

\Vioupspeaks La pendeula soun-e\ldots \direct{Ver Auguste} Te sa senque vou diye? Que l'è 9 aoue.

\Vioujspeaks \ldots é ad\'on?

\Vioupspeaks Comenche Walker, fa tsandjì canal! 

\Vioujspeaks N'a pa Walker lo devendro!

\Vioupspeaks Djeusto.

\Vioujspeaks Macaco!

\Vioupspeaks Ll'è ``N'a de pousta pe té"!

\StageDir{Auguste se levve.}

\Vioujspeaks Ad\'on doe meneutte é areuvvo.

\Vioupspeaks Mi iaou te va?

\Vioujspeaks Vou prendre la pousta.

\Vioupspeaks Mi na! Eun télévij\'on ll'è ``N'a de pousta pe té"! Veun seuilla, tranquilo, achouatta-té.

\StageDir{Auguste, perplecse, s'achouatte.}

\Vioujspeaks Eugn'atra émich\'on ?

\Vioupspeaks Ouè, te sa senque l'è?

\Vioujspeaks Na, cougniso pa\ldots

\Vioupspeaks \ldots é ad\'on demanda! Pe seutta n'a pa fata de demandé a Samsung pequé la cougniso eunc\'o mé. Ad\'on, pe ``N'a de pousta pe té" n'a todzor Marie de Feleunna\ldots

\Vioujspeaks Tourna?

\Vioupspeaks Ah ouè, mé la lamo tan! L'è tan dzen la veure salla fenna lé, avouì salla dzenta vouése! Can mimo, comèn diye\ldots pe la fé seumpla: te mandon de pousta pe te prédjì.

\Vioujspeaks Praou?

\Vioupspeaks Praou. T'a comprèi?

\Vioujspeaks Ouè, seumplo.

\Vioupspeaks Ad\'on quèi é avèitsa.

\StageDir{Auguste se beutte le lenette di solèi.}

\StageDir{Teuppe \lemieBa .}

\scene[-- N'a de pousta pe té]

\StageDir{Partèi lo \textit{jénérique} de la trasmech\'on ``N'a de pousta pe té".}

\sound{https://www.youtube.com/watch?v=D9oppHdF--4}{Love Theme - Barry White}\label{pousta}
 
\StageDir{Lemie \lemieSi\ a drèite. Comme fondal n'at eun écran desì lequel l'è proyettoù lo logo de la trasmech\'on. A drèite n'a catro caèye, a gotse eunna caèya é eun tabouré de bouque si loquel acapèn Marie de Feleunna. Can lo refrèn fenèi, Marie comenche a prédjì.}

\Mariespeaks Bonsouar a tcheutte é bienvin-ì a eunna noua émich\'on de ``N’a de pousta pe tè". Oueu lo nite n'en la conta de eunna fameuille que l’a de grou problème pe acapì eun traille i leur mèinoù. Perdériyo pa d'atro ten é dze fériyo to de chouite entrì la fameuille: Tanteun, Alice,  é le dou mèinoù: Twitter é Ipad.

\StageDir{Entre la fameuille é s’achatte a gotse iaou n'a le catro caèye. Tanteun l'a betoù si de pèi blan é Twitter ara l'è eugn ommo dzouveun-o.}

\Mariespeaks Ad\'on Tanteun, va-tì?

\Tanteunspeaks\direct{Sensa forse} Bonsouar Marie. Ouè, ouè va bièn\ldots

\Mariespeaks Se me troumpo pa no sen vi-no caque-z-àn fé pe eunna min-a émich\'on, ti pa?

\Tanteunspeaks\direct{Comme douàn} Ouè,  malerezemàn. Dèi si dzor l’è comenchà mon eunfer!

\Mariespeaks \'Ezajèra pa! T’a beun eunna dzenta fameuille, na?

\Tanteunspeaks Ouè, dzenta\ldots mi oueu sen seuilla pe pourtì a cougnisanse de tcheut noutro problème de travaille.

\Mariespeaks Contegnade maque\ldots

\Tanteunspeaks Ouè\ldots \direct{eun moutrèn Twitter} noutro pégno mar\'on\ldots

\StageDir{Alice baille eunna caoud\'o a Tanteun.}

\Alicespeaks Prèdza amoddo! Sen pa i mitcho!

\Tanteunspeaks Squezade. Noutro pégno Twitter l’a ren voya de travaillì é pa tan voya de itedjì; ad\'on reste todzor i mitcho a bamban-ì, mi no nen pouèn pamì.

\Alicespeaks L’è fran pe so que sen seu avouì té Marie, t’i noutra dériye spéranse. No n'en eun problème é no cougnisèn eunna personna que sel\'on no pouriye no baillì eunna grousa man, comme  l’a dza bailla-la a tan d’atre.

\Twitterspeaks\direct{Contèn} Sen eun tren de prédjì de Turi Touneun!

\Mariespeaks Va bièn. Ad\'on mé dze dimanderio a Geppino de pourté la busta a noutro Tony.

\StageDir{Si la tèila l'è proyettaye eunna vidéo iaou Geppino, lo posteill\'on de Marie, va eun beseclletta, dézò la plodze, a Palase Réjonal pe pourté a Turi Touneun l'eunvitach\'on a \og N'a de pousta pe té\fg. Tony acsepte de partisipé a la trasmech\'on .}

\StartVideo{https://youtu.be/PEOtbdHbDv0}{N'a de pousta pe té}

\scene[-- La busta de Tony]

\Mariespeaks Fériyo to de chouite entré la busta. \direct{Eun braillèn} Geppino! La busta!

\Geppinospeaks Areuvvo, eun momàn\ldots

\StageDir{Geppino pourte eunna grousa busta de bouque, ata dou mètre é lardze trèi, é la plache i mentèn de la \textit{scène}.}

\Mariespeaks Geppino, Tony l’a assétoù de vin-ì inque avouì no?

\Geppinospeaks Ouè Marie, Tony l’a assétoù l'eunvitach\'on é l’è inque avouì no. Vou lo queryì.

\sound{https://www.youtube.com/watch?v=D9oppHdF--4}{Love Theme - Barry White}[false]%\label{pousta}

\StageDir{Entre Turi Touneun. S'achouatte protso a Marie.}

\Turispeaks Salù Marie.

\Mariespeaks Bonsouar Tony, va-tì?

\Turispeaks\direct{Eunna min-a tracachà} Ouè, ouè, va bièn\ldots \textit{plus au moins}. 

\Mariespeaks Te vèyo tchica tracachà\ldots

\Turispeaks Si tchica tracachà ouè! Pequé, pequé\ldots n’i pouiye que me fisan de dimande si lo Consèille réjonal.

\Mariespeaks Mi na! Lèi si mé inque, ita tranquilo. Donque\ldots te cougnì le riille di djouà?

\Turispeaks Na n'i jamì pouì vère.

\Mariespeaks Ad\'on, inque \direct{eun moutrèn la busta de bouque} no-z-ivrérèn  la busta é de l'atro coutì n'arè caqueun que vou vo prédjì. Vo itade quèi é acoutade. A la feun pouade lèi repoundre. To comprèi?

\Turispeaks Ouè, va bièn.

\Mariespeaks\direct{Eun braillèn} Geppino! Veun ivrì la busta!

\StageDir{Geppino entre avouì eun gars\'on pi ate de llou. Atatchà a la grousa busta de bouque n'at eun sec\'on pannel triangulère de bouque que toppe eunna fenitra. Lo gars\'on levve lo pannel é lèi plache déz\'o eun baquette de fas\'on que la busta reustèye iverta. Pe mézo de la fenitra, la fameuille é Tony se vèyon pe lo premì cou.}

\StageDir{Geppino é lo gars\'on chorton.}

\Turispeaks\direct{Surprì, ver Marie} V'ouèide spendì eunna matse de sou pe seutta tecnolojì!

\Mariespeaks Shhht, silanse!

\StageDir{Silanse. Tony é la fameuille, de l'atro coutì de la busta, s'avèiston.}

\Turispeaks\direct{Eun prédzèn plan} Marie\ldots senque vouillon sise?

\Mariespeaks Shhh, ita quèi!

\Twitterspeaks Salì Mesieu Turi Touneun, mé si Twitter 437.

\Turispeaks\direct{Eun rièn} Mi senque l'è? Eunna marca di botte?

\Mariespeaks Shhht!

\Twitterspeaks\direct{Conteneuvve} Mé si séilla péqué vo cheur me cougnisade pa, péqué mé n'i djeusto 19 an é tanque a pocca ten fi pouavo pa votì. \direct{Eun sourièn} Mi ara pouì\ldots é comèn se pouì! Donque si séilla pe vo eusplequé noutro problème: mé n'i pa eun travaille, pappa to lo dzor l'è stouffie, me tratte male é m'euncheurte é mamma l’è stouffie de me difendre. Mé si pamì comèn fie, n'i djeusto vo comme dériye spéanse\ldots pouade vo m'acapé quetsouza?

\Tanteunspeaks\direct{Inervoù} Ouè,  20 an desù la tita é l’a panco portoù eun sou i mitcho!

\Alicespeaks\direct{Douse} Mi na, l'è que noutro petchoù Twitter tsertse eun travaille pa tro pezàn é ad\'on n’en pensoù a vo. Mogà dedeun caque voutro oufficho v'ouèide eunna plase, finque eun \textit{part-time}\ldots

\Turispeaks Madama\ldots \direct{ver Marie} poui-dz\'o prédjì Marie?

\Mariespeaks Ouè, ara ouè.

\Turispeaks\direct{Conteneuvve} Madama sen mal betoù eunc\'o no, n'en pamì de grouse bague, sen a la lemite\ldots sen, sen fran i \textit{sgoccioli}! N'en pamì ren. \direct{Ver Marie} Si pa que fiye Marie, la deriye baga que me veun a pensé l'è\ldots si pa\ldots poui-dz\'o fé eunna téléfon-où pe vère se acappo eunc\'o eunna pégna plase?

\Mariespeaks Se pouriye pa mi fiade maque.

\Turispeaks Ad\'on prouo.

\StageDir{Touri pren lo téléfonne é queurie caquen.}

\scene[-- Pamì de plase]

\Turispeaks\direct{Eun prédzèn i téléfonne} Ouè si mé, si Tony Touneun. Te me passe lo Gran Chef? 

\StageDir{Silanse.}

\Turispeaks\direct{Avouì reuspé} Tchao si Tony. Si pe eunna trasmech\'on , se t'avèitse lo Sinque te m'acappe. Acouta, mé si seuilla é n'at eunna fameuille que l'a fata d'eun poste de travaille\ldots si qué que tcheut lo dimandon dérimente, mi me fa acapé québague, pouì pa fé de fegueue pouai douàn tcheut\ldots mé pensao\ldots se fise CVA? Sariye beun pa mal? \direct{Eun avèitsèn la fameuille} Vo sarie d'acor vo pe CVA? Vo plériye?

\StageDir{La fameuille di de ouè at\'o la tita.}

\Tanteunspeaks Ouè, ouè va bièn!

\Turispeaks\direct{Todzor eun prédzèn i téléfonne} La fameuille di de ouè; avèitsa se t'a quetsouza\ldots ah na. To plen. To fenì eunc\'o lé. Ah pompì! Pe la fameuille va cheur bièn\ldots ah na. Fa atendre ara lo concoù\ldots \textit{Forestale}? Atèn que lèi dimando. \direct{Ver Twitter} T'i bon a eumpléì lo \textit{motosega}?

\Twitterspeaks\direct{Eun bleuffèn} Ouè!

\Turispeaks\direct{Comme douàn} Ouè me di que l'è bon\ldots ah, atèn que lèi dimando eunc\'o sen. \direct{A Twitter} La résetta?

\Twitterspeaks\direct{Todzor eun bleuffèn} Eunc\'o sen si bon.

\Turispeaks\direct{I téléfonne} Ouè, ouè l'è bon\ldots \direct{a Twitter} l'a deu de l'empléì i mitcho, n'a pamì de poste. \direct{I téléfonne} Sen mal betoù, ad\'on te queutto, can fenèi te crio, no sentèn, tchao.

\StageDir{Beutte ba lo téléfonne.}

\Turispeaks Marie si pa, mé n'i prooù de tot, mi n'en pa.

\Ipadspeaks Souplì Tony!\direct{Eun moutrèn Twitter} Lo pachento pamì si strumèn pe lo mitcho! Acapa quetsouza!

\Alicespeaks\direct{Malechaye} Tanteun te fé restì quèya ta megnotta?!

\Turispeaks Megnotta? Marie, sise prèdzon tchica pezàn, sen eun télévij\'on ! Megnotta de sé, megnotta de lé\ldots

\StageDir{Marie reprèn eun man la situach\'on .}

\Mariespeaks Can mimo mé si pa sen que vo diye. N'i jamì i eugn'émich\'on pouèi queurta. Mesieu Tony, vouillade le-z-ambrachì?

\Turispeaks Ouè, ouè bièn volontchì.

\Mariespeaks\direct{Eun braillèn} Geppino! Veun vèi gavì seutta busta i mentèn di pià!

\sound{https://www.youtube.com/watch?v=D9oppHdF--4}{Love Theme - Barry White}[false]%\label{pousta}

\StageDir{Geppino é lo gars\'on dzouveunno pourton ià la busta. Tony eumbrache eun pe cou totta la fameuille. Turi l'è bièn sourièn é dzouze si momàn de télévij\'on pe se fé fé de fotografiye avouì totta la fameuille (de-z-électeur); i contréo, Tanteun é Twitter son malechà avouì llou.}

\scene[-- Matte\ldots sen tcheutte matte]

\Turispeaks Amoddo, mé n'i salia-le. Marie, \textit{ecco}, vi que si seu eun trasmich\'on é vèyo douàn mé la télévij\'on , me vériye voya de fé eun discoù, de diye québague pequé me semble \textit{anche logico}\ldots poui-dz\'o?

\StageDir{Marie l'è an mia tracachaye pe la requète de Turi.}

\Mariespeaks\direct{Douteuza} Si pa, crèyo pa Mesieu Tony.

\Turispeaks\direct{Eun lèi pourtèn ià lo microfonne di man} Mi ouè, alé! Fio vito eun discoù!

\Mariespeaks\direct{Malechaye} Ad\'on fiade!

\StageDir{Turi se plache i mentèn di palque é ataque eun comise politique.}

\Turispeaks Mé vouillavo djeusto diye que ara n'en tsertchà d'èi\-djì tcheutte, \textit{ecco} me semble que n'en tsertchà d'èidjì pi de fameuille; oueu sen pa arrevoù a ten, mi me semble que\ldots \direct{eun tsandzèn discoù} avèitsèn que dzenta fameuille que n'en seu devàn. Seutta ouè que l'è eunna fameuille; ara si botcha l'a pa acapoù de travaille \textit{ma} pourè nen acapé eun demàn! \textit{Ecco}, restèn avouì le pià pe tèra\ldots

\StageDir{Dimèn la fameuille se nen va. Eunc\'o Marie queutte la trasmech\'on .}

\Turispeaks \textit{ma} restèn avouì le pià pe tèra!

\StageDir{Dimèn que Tony conteneuvve avouì son discoù vouido, di palque veugnon portaye ià totte le caèye. Dousemàn partèi eunna tsans\'on; Tony s'avèitse a l'entor é réalize d'itre solette.}

\Turispeaks\direct{CONFUSO} Mi seu\ldots mi seu se nen van tcheut\-te!

\StageDir{Teuppe \lemieBa .}

\Turispeaks Mi seu son tcheu matte!

\StageDir{Lemie \lemieSi .}

\StageDir{Totta la compagnì entre eun danchèn :}

\sound{https://youtu.be/HDjdC5LmlOk}{Matte\ldots sen tcheutte matte}

\ridocliou

\DeriLeRido

\RoleNoms{Chorégraphie}{Jo\"elle Bollon}

\RoleNoms{\`Eidzo réjì}{Flavio Albaney}

\RoleNoms{Vidéo}{André Comé}

\RoleNoms{Souffleur é mezeucca}{Jasmine Comé, Giada Grivon, Ilaria Linty}

\end{drama}