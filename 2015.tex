\title{DISCO FLAMA}
\author{Pièse icrita pe\\ Le Digourdì de Tsarvensoù\\ é\\ Le Beurt é Boun de Pollein}
\date{Téatro Splendor de Veulla, 11 avrì 2015}

\maketitle

\fotocopertina{Foto/2015/gruppo_red.jpg}{Ilaria Linty, Paolo Cima Sander, Christian Brunod, Francesca Lucianaz, Alain Yeuillaz, Tania Giannino, Elisa Viérin, Denise Borbey, Jordy Bollon, Pierre Savioz, Ester Bollon, Jo\"{e}l Albaney, Aline Dalbard, Sophie Comé, Giada Grivon, Jasmine Comé}{Marco Ducly, Valérie Marguerettaz, André Comé, Laurent Chuc, Michel Squinabol, Jo\"{e}lle Bollon, Simone Roveyaz, Marlène Jorrioz, Ilenia Squinabol, Federico Squinabol}{2015}

\LinkPiese{Disco Flama}{https://www.youtube.com/watch?v=MZ9IIh0Y5c8&list=PLBofM-NS_eLJUln45l7VH457fGak_Bk5O&index=12}{.5}

\souvenir{L'idoù de résitì eunsemblo a la compagnì Le Beurt é Boun de Pollein l'ie chortiya foua de la tita de Jo\"{e}l Albaney é Michel Squinabol. Pe totte é doe le compagnì l'è itaye la premiye espérianse de collaborach\'on.

Me rappello eunc\'o que l'è itoù lo premì cou iao tcheu le-z-atteur l'an partesipoù a l'icriteua di teste. Sen divija-no eun trèi groupe (Tsarvensolèn é Pollentch\'on micllà eunsemblo) pe icriye le trèi-z-acte de Disco Flama. Tsaque groupe l'ayè fé eunna  retserte approfondiya pe caractérizé amoddo l'époque deun laquelle lo local (Disco Flama) l'ie reprézentoù: costume, mezeuque, \textit{scénographie} é eunc\'o finque le dzi que l'an viquì réellamente lo local.

Collaboré avouì Pollein l'è itaye eunna dzenta espérianse. Mi can mimo, eunc\'o avouì eugn'atra compagnì, sen  arrev\'o a fé de pastisse (é sitte me lo rappello amoddo pequé la fegueua n'i fé-la mé). A la feun di spettaclle, n'en fé eunna danse de groupe, le Tsarvensolèn avouì eunna maille rodze é le Pollentch\'on avouì eunna bleue. Mé si lo seul que l'è entroù eun \textit{scène} avouì eunna maille diffienta, comme lo pedzeun neur i mentèn di pedzeun dzano. 

Ah! La coulpa l'ie pa de mé, mi de si macao d'eun Jordy Bollon que, dérì le rid\'o, i mentèn de la confuj\'on, l'ayè tchapou-me la mailletta. 
}{Marco Ducly}
\newline
\newline
\og Eun jénéral n'i eun dzen souvenir de Disco Flama, sourtoù pequé tcheut eunsemblo n'en fé eugn énorme traille de retsertse i\-storique. La counta l'è divijaye eun trèi époque diffiente; donque, pe tsaque \textit{scène} sen aloù tsertchì le coteill\'on, le tsans\'on, le danse ou le \textit{cocktail} tipique d'eunna périoda  istorique. Mi sourtoù l'è itoù euntéressàn é amezèn découvrì le querelle euntre Tsarvensolèn é Pollentch\'on, salle di noutro paèn é pappagràn!
\\ Aprì deun la counta n'en eunc\'o bet\'o de personadzo que l'ion pa fran llatoù i Disco Flama, mi que can mimo carattérizaon amoddo eun local valdotèn. Dorina a l'entraye l'ie la Dorina di Trompeurs de Cogne, avouì sa battiya ``V'ouèide-tì prénotoù?"; ét\'o Dino ou mioù\ldots Dino Banana!
\\ For probablo l'è itoù lo premì cou que doe compagnì l'an betoù eunsemblo le forse pe réalizé eunna pièse pe eunna souaré di Printemps Thé\^atral é personellamente seutta collaborach\'on, cheur pa seumpla a niv\'o d'organizach\'on, l'et eun souvenir que vardo avouì plèizì: \og\textit{Alla faccia del campanilismo!}\fg comme dijave lo titre d'eun journal caque dzor aprì lo spettaclle.\fg{}
\newline
\newline
\hspace*{\fill} \textit{Valérie Marguerettaz}

\queriaouzitou{
\begin{itemize}
\item[$\bullet$] \og Disco flama\fg l'è itaye la premiye collaborach\'on avouì eugn atra compagnì téatralla pe réalizé eunna pièse. For probablo l'è it\'o eunc\'o finque lo premì cou deun l'istouére di Printemps Thé\^atral.

\item[$\bullet$] Pe Marlène Jorrioz, future Prézidanta di Digourdì, \og Disco flama \fg l'è son débù deun la compagnì.

\item[$\bullet$] Eunna particularitoù de \og Disco flama\fg l'è la forta contrapozich\'on euntre le dou patoué de Tsarvensoù é Pollein. Doe Quemeue eunna atatchaye a l'atra, mi avouì eunna foneteuca diffienta di patoué: mogà/magà; iaou/iou; bon/boun.  
\end{itemize}
}



\Scenographie
\begin{itemize}
\item[$\bullet$] De fondal gri é maròn pe toppé totta la lardjaou di palque;
\item[$\bullet$] 1 crédense avouì de vèyo é de botèille (bira, veun, dijéstif, ju de frouite\ldots);
\item[$\bullet$] 1 lavandeun;
\item[$\bullet$] 1 réjistrateur de quése;
\item[$\bullet$] 1 ban de la cantin-a;
\item[$\bullet$] 1 caèya ata;
\item[$\bullet$] 1 remiza di dra avouì de pourtamantì;
\item[$\bullet$] 3 fotografie de Little Tony, Mike Bongiorno é de Gène (patr\'on di local);
\item[$\bullet$] 2 pégne panquette de bouque;
\item[$\bullet$] 1 palque de 3 pe 3 mètre, ate eugn'eunpanna, avouì eun tapis rodzo desì;
\item[$\bullet$] 2 pégne cllende de bouque;
\item[$\bullet$] 1 \textit{guitare}, 2 microfonne avouì lo pià, 1 \textit{batterie}, 1 \textit{clavier};
\item[$\bullet$] 1 tabla é 3 caèye;
\item[$\bullet$] 1 \textit{jukebox};
\item[$\bullet$] 3 poltronne base;
\item[$\bullet$] 1 sidel de la llase avouì eunna botèille de Champagne;
\item[$\bullet$] 1 cubo pe fé danchì eunna professioniste de la danse;
\item[$\bullet$] 1 tabla blantse avouì desì 1 \textit{consolle} pe lo DJ. 

\paragraph*{Costume} Di momàn que la pièse se deroule si trèi-z-acte deun lo ten (an '60, '80 é '90), le-z-arbeillemèn, le pereuque é lo maquiadzo seràn difièn é coéràn avouì l'époque istorique reprézentaye.

\end{itemize}

\setlength{\lengthchar}{3.125cm}

\Character[LAURENT]{LAURENT}{Laurent}{Prézentateur tsarvensolèn, \name{Laurent Chuc}}

\Character[TANIA]{TANIA}{Tania}{Prézentatrise pollentchon-a, \nameF{Tania Giannino}}

\Character[GÈNE]{GÈNE}{Gene}{Patr\'on pe le-z-àn '60 di local Disco Flama de Pollein, \name{Jordy Bollon}}

\Character[DORINA]{DORINA}{Dorina}{Serventa di local Disco Flama (40 an), \nameF{Valérie Marguerettaz}}

\Character[WALTER]{WALTER}{Walter}{Eun pollentch\'on cost\'o di-z-àn '60, gran buveur, todzor achouat\'o i banc\'on é amì de Selmo, \name{Alain Yeuillaz}}

\Character[GEMMA]{GEMMA}{Gemma}{Dzouveun-a pollentchon-a, choze de Walter, \nameF{Ilenia Squinabol}}

\Character[ILEANA]{ILEANA}{Ileana}{Dzouveun-a pollentchon-a amia de Gemma é Ketty, \nameF{Aline Dalbard}}

\Character[JOUEUR]{JOUEUR}{Joueur}{\textit{Band} que soun-e i Disco Flama pe le-z-àn '60: eun \textit{batteur} eunterprétoù pe \textsc{André Comé}, eunna \textit{claviériste} eunterprétaye pe \textsc{Elisa Viérin},  eun chanteur é eun \textit{guitariste} eunterprétoù pe \textsc{Jo\"{e}l Albaney} é \textsc{Marco Ducly}. Tcheu le joueur son arbeillà igale: coteill\'on ou tsemiyze grize avouì bièn de payette. Le-z-ommo l'an tcheut eunna craotta rodze.}

\Character[SELMO]{SELMO}{Selmo}{Pollentch\'on amì de Walter, pappa de Giulio é djouyao di tsan, \name{Michel Squinabol}}

\Character[KETTY]{KETTY}{Ketty}{Dzouveun-a tsarvensolentse di-z-an '60, amia de Gemma é dézidéraye pe tcheu le-z-ommo, \nameF{Francesca Lucianaz}}

\Character[PINO]{PINO}{Pino}{Ommo tsarvensolèn di-z-an '60 que lame eustegué, pappa de Vilma, \name{Paolo Cima Sander}}

\Character[GEPPINO]{GEPPINO}{Geppino}{Ommo tsarvensolèn que lame tapé de benzin-a si lo fouà di pompaye de son amì Pino, \name{Simone Roveyaz}}

\Character[BERTA]{BERTA}{Berta}{Feuille de Gène, inamoraye de Giulio, \nameF{Denise Borbey}}

\Character[VILMA]{VILMA}{Vilma}{Feuille de Pino, 20 an, dzenta, sourienta é inamoraye de Giulio, \nameF{Jo\"{e}lle Bollon}}

\Character[VANDA]{VANDA}{Vanda}{Pollentchon-a, 20 an, amia de Vilma é Ornella, \nameF{Ester Bollon}}

\Character[ORNELLA]{ORNELLA}{Ornella}{Pollentchon-a, 18 an, amia de Vilma é Vanda, \nameF{Marlène Jorrioz}}

\Character[DINO]{DINO}{Dino}{Piornats\'on é malereu di veladzo, a tsasse de dzente feuille, \name{Jo\"{e}l Albaney}}

\Character[GIULIO]{GIULIO}{Giulio}{Gars\'on de Selmo, 20 an, beur, bièn timido é inamoroù de Vilma, \name{Christian Brunod}}

\Character[DJOUYAOU I]{DJOUYAOU I}{Djouiyaoueun}{Djouyaou di tsan de Pollein, \name{\\Federico Squinabol}}

\Character[DJOUYAOU II]{DJOUYAOU II}{Djouiyaoudou}{Djouyaou di tsan de Pollein, \name{\\Alain Yeuillaz}}

\Character[MARTEUN]{MARTEUN}{Marteun}{Tsarvensolèn de Roulaz que djouye a tsan avouì sise de Pollein, \name{Marco Ducly}}

\Character[HÉLÈNE]{HÉLÈNE}{Helene}{Serventa trèinasoque di local Disco Flama pe le-z-àn '90, \nameF{Elisa Viérin}}

\Character[CUBISTA]{CUBISTA}{Cubista}{CUBISTA di Disco Flama, \nameF{Sophie Comé}}

\Character[DJ BERTINO]{DJ}{Dj}{Lo \textit{Disk-Jokey} (DJ) di Disco Flama pe le-z-àn '90, \name{Simone Roveyaz}}

\Character[PIERRE]{PIERRE}{Pierre}{Dzouveun-o di-z-an '90 avouì tan de voya de fé fita, \name{Pierre Savioz}}

\Character[LOUIS]{LOUIS}{Louis}{Amì de Pierre, 20 an, entuziaste de fé fita, \name{Jordy Bollon}}

\Character[FEDERICO]{FEDERICO}{Federico}{Amì de Giulio, bièn mastoque é grouchì, \name{Federico Squinabol}}

\Character[MARCO]{MARCO}{Marco}{Amì de Giulio, onéto é sayo, \name{Marco Ducly}}

\Character[MICHEL]{MICHEL}{Michel}{Lo meillaou amì de Giulio, eunna mia stchapeun, \name{Michel Squinabol}}

\Character[VALLET]{VALLET}{Vallet}{Vallet de veulla fenna, avouì la pach\'on pe lo dizarbeillemèn, \nameF{Aline Dalbard}}

\Character[SAMI]{SAMI}{Sami}{Brazilienna \textit{caliente} avouì (litérellamente) le bale, \name{André Comé}}

\Character[PAUL]{PAUL}{Paul}{Gars\'on pégno, bièn arrogàn que sa fé viondé amoddo la lenva can l'è prosto de son amì Auguste, \name{Jo\"{e}l Albaney}}

\Character[AUGUSTE]{AUGUSTE}{Auguste}{Eun cost\'o de gars\'on, ate é ipesse, todzor ià avouì son petchoù amì Paul, \name{Alain Yeuillaz}}

\Character[DJOUYAOU I]{DJOUYAOU I}{Djouiyaoueunf}{Djouyaou di fiolet de Tsarvensoù, \name{Laurent Chuc}}

\Character[DJOUYAOU II]{DJOUYAOU II}{Djouiyaoudouf}{Djouyaou di fiolet de Tsarvensoù, \name{Paolo Cima Sander}}

%\Character[]{}{sof}{\nameF{Sophie Comé}}

\DramPer

\act[\avanSpect\ Avanspettaclle   \avanSpect]

\StageDir{\hspace*{2.5em}Lemie \lemieSi\ si lo \textit{proscenium}.}
 
\StageDir{\hspace*{2.5em}Laurent entre eun fièn le paolle crouijaye.}

\begin{drama}

\Laurentspeaks Avouì seutta nite dèyo pe fose fenì le paolle crouijaye\ldots satte ba\ldots \og Abile, actif, euntellijàn\fg\ldots Digourdì! Tchica comme no Tsarvensolèn.

\StageDir{Laurent icrì la repounsa.}

\Laurentspeaks Dji plan: \og Téatro populéo valdotèn\fg\ldots fasilo! \textit{Printemps}\ldots ah na, comenche avouì la c\ldots

\StageDir{Silanse.}

\Laurentspeaks Ah ouè! \og Consèille réjonal\fg: lèi ite pa to mi lo icrio pégno. Ara, 22 ba: \og Fenèi can te moueue\ldots lo \textit{mutuo}; 17 aoutre: \og Eun tchi no saouton, le vezeun de no le peuccon\fg.

\StageDir{Silanse.}

\Laurentspeaks Le renoille! Eugn éffé maque a Pollein pouon medjì le renoille; 24 ba: \og Can te lo baillon l'è pe l'éternitoù\fg\ldots lo tomb\'o di semeteurio! Donque, 16 aoutre: \og Se di de eunna personna tchica blaye\fg. Comenche avouì la p. Tchica blaye\ldots

\StageDir{Silanse.}

\Laurentspeaks Le Pollentch\'on! Avouì noutra dzenta é grousa Becca-de-Noua devàn, l'è normal que prègnon pa lo solèi! Ara finalemàn sen arevoù a la dériye; 26 ba: \og Quemeua protso de Tsarvensoù\fg\ldots Gressan! Na! Comenche avouì la p.

\StageDir{Silanse.}

\Laurentspeaks Ah ouè! Plan-Feleunna!

\StageDir{Entre Tania.}

\Taniaspeaks\direct{Malechaye} Praou! Praou! Si cou praou! T'a stoufia-me!

\Laurentspeaks\direct{Arrogàn} Mi a té\ldots qui l'a cria-te?

\Taniaspeaks A mé qui l'a cria-me? Mi te te ren countcho de sen que t'i eun tren de diye? Seutta l'è pa la souaré di Digourdì!

\Laurentspeaks\direct{Eun desuèn Tania} L'è pa la souaré di Digourdì! Comèn na?

\Taniaspeaks Oueu lèi sen eunc\'o no, le Beurt é Boun de Pollein. Pe lo premì cou dedeun lo \textit{Printemps} n'a doe compagnì que resiton eunsemblo.

\Laurentspeaks Qui se nen pippe de vo de Pollein!\direct{Eun moutrèn lo pebleuque} N'a pa gneun de Pollein! Son tcheu de Tsarvensoù, avèitsa: Remo, Arthur, Renzo\ldots son tcheu seuilla. Se n'a caqueun de Pollein, lévisa la man!

\StageDir{Tania é Laurent avèitson lo pebleuque.}

\Laurentspeaks Avèitsa, n'a gneun! Ah na\ldots lé ba i fon nen vèyo eun tchica blayo, avouì le renoille que lèi sauton pe l'estomaque. Si pou itre eun di voutre!

\Laurentspeaks Planta-là li, planta-là li, si cou t'a stoufia-me\direct{eun moutrèn le paolle crouijaye} é planta-là de fi seutte counte foule. Can mimo, mé me rapéllo que eun cou, ba i Disco Flama pe le-z-àn '60, mollòn pa pi mi de fi seutte counte\ldots

\Laurentspeaks \ldots pe le-z-àn '60, i Disco Flama, ba a Pollein?

\StageDir{Teuppe \lemieBa .}

\act[Le-z-àn '60]

\ridoiver

\scene[-- Eun \textit{twist} pe comenchì]

\StageDir{Lemie \lemieSi\ si to lo palque.}

\StageDir{La \textit{scène} prinsipale l'è lo local istorique Disco Flama de Pollein, lleu de \textit{rencontre} di Pollentch\'on, iaou chouèn passòn eunc\'o de Tsarvensolèn.\\ I mentèn n'at eunna caèya ata, lo banc\'on, eun lavandeun é eunna crédense avouì plen de botèille é de vèyo. A gotse, n'a la remiza di dra avouì de pourtamantì. A drèite, desì eun petchoù palque cllend\'o, n'a le-z-enstrumèn di joueur; protso di palque lèi son de banquette é lo caro pe danchì.}
\StageDir{Acapèn Gène, propétéo di local
, é Dorina, la serventa. Dérì lo banc\'on, Gène baille bèye i dzi; Dorina beutte eun caro le palt\'o.
Achouat\'o, douàn lo banc\'on, n'at eun cliàn abituel é eun joueur. A drèite, achouataye si eunna banquetta, n'a doe feuille.}

\StageDir{Gène moutre l'aoua i joueur pe lèi fé comprendre que l'è l'aoua de comenchì. Lo joueur bèi ba la dériye gotta é va s'aprestì iaou n'a tcheu le-z-enstrumèn. Pe s'itsaoudé, baille eun per de creppe a case si la batterie.}

\StageDir{Entre la \textit{claviériste}, salie le doe feuille achouataye é arite lo collègue de lleu que l'iye eun tren de fé maque de tapadzo. Proue lo \textit{clavier}, riille l'atchaou di microfonne é proue sa vouése (eunna mia tro ata).}

\StageDir{Entre lo chanteur: tita ata é fier de lli. Eunna di doe feuille, can lo vèi, lo arite pe lèi dimandé eugn \textit{autographe}.}

\Walterspeaks\direct{Ironique, a Gène} T'i aloù tsertchi-le i martchà sise?

\Genespeaks Dèyo finque le paì!

\Dorinaspeaks \ldots é areuvvon tcheut eun aprì l'atro.

\StageDir{Lo chanteur proue lo microfonne avouì eun per de \og sa, sa\fg.}

\StageDir{Entre lo dérì joueur avouì eunna \textit{guitare}.}

\Genespeaks\direct{I dérì joueur, ironique} Can t'a voya va maque\ldots

\StageDir{Lo \textit{guitariste} se plache protso di-z-atre joueur é teurie foua la \textit{guitare} de la catse. Can son tcheu preste, lo chanteur se vionde é baille l'ataque pe lo premì boc\'on de la souaré.}

\StageDir{Partèi la tsans\'on:}
\sound{https://www.youtube.com/watch?v=csISQylhkgc}{Let's Twist Again - Chubby Cheker}

\StageDir{Le doe feuille saouton si pe l'er é atacon a danchì. I banc\'on, tcheutte bouéch\'on lo ten.}

\StageDir{Can fenèi la mezeucca, le doe feuille s'achouaton. Pe to l'acte, can le-z-atteur prèdzon, n'aré todzor eunna mezeucca de fon.}

\scene[-- \textit{Trallallà} i banc\'on]

\Genespeaks\direct{A Dorina} Que dzen coteill\'on que t'a! Totta dzenta teriaye\ldots  te me fé viì le cliàn tsa sensa que béisàn!

\Dorinaspeaks Gène di pa pèi, te sa que mé me jèino! N'i mai betoù la premì baga que n'i acapoù pe l’armouére! Te lo sa que a la feun di countcho mé si eunna fenna bièn seumpla.

\Genespeaks Seumpla comme le pateun que t'a adosse.

\StageDir{Gène poulite lo banc\'on. Acappe pa lo Scottex. De déz\'o lo banc\'on (majiquemàn\footnote{ Noutro pouo Pierre Savioz l'è it\'o to lo premì acte déz\'o lo banc\'on maque pe tapé si lo Scottex pe Gène!}) saoute foua lo Scottex.}

\Genespeaks Si eun tren de squersì. Mé n'i de chanse: té t'a lo \textit{savoir faire}, eunna finesse que pocca de local eun Val d'Outa se pouon permettre.

\Walterspeaks	\direct{Dimèn que bèi, eunna mia tsa} Ouè \textit{savoir faire}, sario mé sen que fée avouì lleu\ldots \direct{a Dorina, triste} mi t'a jamì cacou-me!

\Dorinaspeaks Walter! Resta quèi que t'i finque fianchà! \`Eita\ldots ara areuvve Gemma é te fé pi pasé le voye.

\StageDir{Gemma se levve de la banquetta é, demì malechaye, tsemin-e ver Walter. Dimèn entre Selmo. Sensa diye ren saliye Dorina, paye l'entrée é se plache i banc\'on. Gène lèi soum\'on eunna bira.}

\Gemmaspeaks\direct{A Walter} Walter! Alèn i mitcho? Mamma l'a deu d'itre i mitcho pe onz'aoue, pa pi tar.

\Walterspeaks	L'è eunc\'o vito, dèyo maque  bèye eunna gotta avouì Selmo que l'è djeusto arevoù\ldots é aprì pouèn alé.

\Gemmaspeaks\direct{Per la pachense} Mi comèn bèye eunc\'o eun crep avouì Selmo! T'a gnenca fé-me danchì eun cou seutta nite. Si cou si stouffia. Mamma l'a todzor deu-me: \og Té, dzenta é fin-a, ià avouì eun \textit{trallallà} di janre!\fg.

\Walterspeaks	\textit{Trallallà} mé?

\Gemmaspeaks Fi-té feun! Te me vèi pamì! 

\StageDir{Gemma chor malechaye. Mande eunc\'o a caqué Selmo.}

\Selmospeaks\direct{A Walter} T'a tchapoù sèque!

\Walterspeaks	Ouè, tanque a demàn, aprì lèi pase. N'i beun tchapoù pi sèque é l'è todzor tornaye	.

\Selmospeaks Ah ouè?

\Walterspeaks\direct{Moutchicco} Selmo, bèi eun Martini avouì mé.

\Selmospeaks Ad\'on fenèiso la bira.

\Walterspeaks\direct{A Gène} Gène, beutta dou Martini.

\Genespeaks Acouta\ldots t'a tourna voya de t'eumpiornì eunc\'o oueu comme la senà passaye?

\Dorinaspeaks Beuto mé le dou Martini.

\Genespeaks\direct{A Dorina} Ouè mersì.

\Walterspeaks	M'eumpiornì avouì de Martini? Me fa nen bèye doe botèille, mé si rebeusto! Si pa pi eun dandolèn!

\Genespeaks\direct{A Selmo} Ouè rebeusto! N'i beun vi comèn l'è fenia la senà passaye: l'a to rebetoù douàn lo tabaqueun!

\StageDir{Selmo stchoppe di riye.}

\Genespeaks Lo dzor aprì, tcheu lèi colataon desì!

\Walterspeaks L'iye maque eunna pégna eundijesti\'on\ldots sayò pa pi\'on.

\StageDir{Wally soum\'on le dou Martini.}

\Dorinaspeaks Voualà le Martini son preste! Son 250 lire! 

\Walterspeaks	Sen inque totte le senà é t'a  jamì paya-no a bèye.

\Dorinaspeaks\direct{Eun perdèn la pachense} Walter! Se te vou bèye sensa paì te fa prédjì avouì Gène. Si pa mé la patroun-a di local!

\StageDir{Gène eunterveun pe difendre Dorina.}

\Genespeaks Prèdza mai avouì mé. \direct{A Selmo} Son 20 an que atseutto la fontin-a avouì son pappa é l'a jamì regalou-me ren!

\StageDir{Dimèn areuvve Ketty. Protso de la remiza di dra, tchatchare avouì Dorina.}

\Genespeaks\direct{A Walter} \ldots donque, paya maque, to lo tor!

\Walterspeaks	Ad\'on teun\ldots \direct{eun terièn foua eun per de pise} é varda la resta, acrapì!

\Selmospeaks\direct{Eun levèn lo vèyo} Ad\'on santé! L'an bet\'o eugn'oliva dedeun, mi si pa pequé.

\Walterspeaks Mogà fa medji-la\ldots can mimo santé!

\StageDir{Le dou bèyon eun crep de Martini.}

\scene[-- Cacanéné a qui?]

\StageDir{Ketty, que l'a tsaplet\'o tanque ara avouì Dorina, s'aprotse a Walter.}

\Kettyspeaks\direct{Avouì suspé} Walter, acouta, iaou l'è Gemma?

\StageDir{Selmo can vèi Ketty veun matte. Comenche a s'ajité.}
 
\Walterspeaks	Saré alaye i mitcho.

\Kettyspeaks Comme da couteuma t’aré fé-la  malechì! L'è pa pousiblo Walter, tcheu le cou! Dèij\'on no veure seuilla\ldots

\StageDir{Walter gnoue a itre pi\'on é comenche a se medjì le paolle.}

\Walterspeaks \ldots l'iye tchica greundze, si pa.

\Kettyspeaks Ouè quetèn pédre, ti todzor lo mimo.

\StageDir{Ketty se plache i banc\'on pe comandé a bèye. Selmo l'è rodzo comme eun pèivr\'on.}

\Selmospeaks 	\direct{A Ketty, bièn jèin\'o, comme eun mèinoù} Salì Ketty, comèn te va? 

\Kettyspeaks\direct{Digoutaye} Salì! Amoddo, té?

\Selmospeaks Bièn ara que t'i arrevaye.

\Kettyspeaks \direct{A Gène} Te me fé trèi \textit{Long Island} pe plèizì?

\StageDir{Gène apreste le \textit{cocktail}.}

\Selmospeaks Le payo mé, le payo mé! Payo mé le \textit{Long Island}! Véo te dèyo Gène?

\Genespeaks Son 5000 lire.

\StageDir{Eun pensèn a véo de-z-aoue de traille son 5000 lire, Selmo paye lo contcho.}

\Walterspeaks\direct{A Selmo} Dèi can le fenne bèyon?

\Selmospeaks Té te sa pa de lleu! Lleu veun de l'Amérique! L'è acotemaye a bèye seutte poutreungue; é bèi eunc\'o d'atro.

\StageDir{Sensa remersì Selmo, Ketty pren le \textit{cocktail} é s'achouatte si la banquetta avouì le-z-amie de lleu.} 

\Walterspeaks	N'i comprèi, mi té planta-là lé de pédre le bave dérì salla dandolentse! Saquetta!

\Selmospeaks\direct{Dispéoù é inamoroù} Saquetta\ldots mé can la vèyo m'ajito, tremblo, chouéyo comme eun gadeun!

\Walterspeaks	Ouè se sentèi praou eunc\'o lo flou!

\Selmospeaks\direct{Eun se gneflèn dézò le bri} Te di que se sen?

\Walterspeaks Ouè, t'i lé que te tremble to, l'è normal\ldots te me semble eun cacanéné!

\Selmospeaks\direct{Ofenchà} Cacanéné, a mé? 	

\Walterspeaks Ouè!

\Selmospeaks Ad\'on te me cougnì pa eunc\'o amoddo. Ara te fiyo vére mé qui l'è lo cacanéné.

\StageDir{Selmo va ver le joueur.}

\Selmospeaks\direct{I chanteur} Sotcho! Beutta salla que lamon le feuille!

\StageDir{Le joueur s'apreston. Selmo pren lo microfonne é va protso de Ketty.}

\Selmospeaks Ketty\ldots seutta l'è pe té.

\StageDir{Lo chanteur baille l'ataque i joueur é partèi la tsans\'on:}
\sound{https://www.youtube.com/watch?v=WqinpBMhgys}{Piccola Katy - Pooh (Karaoke)}
\StageDir{Selmo pren coadzo é tsante:
\begin{center}
Oh oh petchouda Ketty (x2)\\
Oh oh\\\vspace{0.15cm}

Petchouda Ketty\\
que dzen t'i arevaye\\
Té t'i eunna feuille\\
tan dzenta é bièn saye\\\vspace{0.15cm}

Ara mé vouillo danchì\\
avouì té\\
pai te fiyo vére\\
qui l'è lo cacanéné\\\vspace{0.15cm}

Lo dérì cou\\
que n'i prouou-lé avouì té\\
t'a plantou-me an locca\\
é t'a rontu-me an di\\\vspace{0.15cm}

Oh oh petchouda Ketty (x2)\\
Oh oh\\\vspace{0.15cm}

Petchouda Ketty\\
la viya l'è dzenta\\
é mé oueu a noua\\
n'i medjà la polenta\\\vspace{0.15cm}

avouì lo lasì\\
de ma chère Tormenta\\
é eun bon bocoùn\\
de fontin-a piquenta\\\vspace{0.15cm}

so pe te diye\\
an baga pe té\\
té t'i la feuille\\
salla djeusta pe mé\\\vspace{0.15cm}

Oh oh petchouda Ketty (x2)\\
Oh oh
\end{center}
 }
 
 \StageDir{Selmo conteneuvve mi lo chanteur lèi gave lo microfonne. Ketty gante la tita.}

\Selmospeaks\direct{A Ketty} Nieunca pai?

\Kettyspeaks Na.

\Selmospeaks Te lame pa comèn tsanto?

\Kettyspeaks Mi sel\'on té a mé euntéresse se t'a medjà la polenta ou la fontin-a piquenta?

\Selmospeaks La polenta l'è de mamma\ldots

\Kettyspeaks Amoddo, ad\'on medza-là avouì mamma!

\StageDir{Selmo tourne i banc\'on ba de moral. L'è dispéoù. Le joueur comprègnon la situach\'on é désidon de soun-ì eun \textit{lento}.}

\StageDir{Partèi la tsans\'on:}
\sound{https://www.youtube.com/watch?v=ISuVYX6RYXQ}{Non ho l'età - Gigliola Cinquetti}

\StageDir{Selmo bèi eun vèyo pe se consolé. Dorina é Gène tsertson de lo convencre de alì tourna da Ketty pe la fé danchì. Selmo pren coadzo é partèi. Pren pe eun bri Ketty é la trèine i mentèn di palque.}

\Kettyspeaks\direct{A Selmo, malechaye} N'i deu de na!

\Selmospeaks \ldots é mé n'i deu de ouè!

\StageDir{Ketty s'achouatte.}

\Selmospeaks\direct{Déterminoù} Mé vouillo danchì avouì té! Veun seuilla!

\StageDir{Selmo trèine pe lo sec\'on cou Ketty i mentèn di palque.}

\Kettyspeaks N'i deu de na!

\StageDir{Selmo se tappe si Ketty é l'eumbrache. Eun la sarèn for comenche a danchì lentamente. Ketty, a si poueun, avouì le joueu reverià, danche avouì Selmo.}

\scene[-- \`Eita comèn se danche!]

\StageDir{Lo volume de la mezeucca bèiche. Selmo é Ketty danchon a drèite di palque.}

\StageDir{Dimèn areuvon dou gars\'on de Tsarvensoù. Entron sensa paì.}

\Dorinaspeaks\direct{I dou Tsarvensolèn} Bonsouar! Pensade de payì?

\Pinospeaks	Mi fa eunc\'o paì a st'aoua?

\Genespeaks\direct{Totchà deun l'orgueuill, ver Pino}	Acrapì que t'i pa d'atro! Seutta l'è pa an crotta! Si si solàn l'an tsemià de dzi comme \direct{eun moutrèn le fotografiye} Petchoù Tony, Mike Bondzor\ldots payade maque l'\textit{entrée}.

\Pinospeaks\direct{Eun bèichèn la crita} Va bièn ad\'on payèn.

\StageDir{Le dou Tsarvensolèn se gavon lo palt\'o, lo baillon a Dorina é payon l'\textit{entrée}.}

\Pinospeaks\direct{A Dorina} Varda maque la resta.

\Dorinaspeaks Voualà le dou beillet, entrade maque ara.

\StageDir{S'avèitson a l'entor é van i banc\'on.}

\Pinospeaks\direct{A Geppino} Senque béyèn?

\Geppinospeaks\direct{A Gène} Dou cornì!

\StageDir{Gène apreste le vèyo.}

\Pinospeaks \direct{Blagueur, ver Walter} Senque te bèi té? D'ive? 

\Walterspeaks	Na, vo sade pa que l'éve fé viì reillèn?

\Pinospeaks	Crèyo beun! Sen no Tsarvensolèn le patr\'on de l'ive de la Becca!

\Geppinospeaks\direct{Gradasse} Djèique, l'è de no!

\Walterspeaks	Fiade pa tan le ganas\'on. L'è pa pi icrì Tsarvensoù i sondz\'on de la Becca. L'éve l'è de tcheutte\ldots é can mimo no vo pourtèn ià le rèine!

\Pinospeaks Le rèine? Pe doe rèine que v'ouèide\ldots

\Walterspeaks \ldots é mogà vo portèn eunc\'o ià le dzente feuille. \direct{Eun moutrèn Selmo} \`Eita lé mon compagn\'on comèn danche\ldots

\Pinospeaks Ouè èita comèn l'è mal betoù!

\Walterspeaks T'a vi comèn la sare? Salla l'è eunna Dandolentse!

\Pinospeaks\direct{Tracachà, ver Geppino} L'è vrèi\ldots

\Walterspeaks Selmo! Veun vèi baillì eun cou de man!
 
\Selmospeaks	Te vèi pa que n'i da fiye?

\Walterspeaks\direct{Eun provoquèn le dou Tsarvensolèn} Comèn véyade l'è bièn eungadjà Selmo.

\StageDir{Pino pouze lo vèyo é s'aprotse déterminoù ver Selmo, mi Walter l'aplante.}

\Pinospeaks\direct{A Walter} Ehi! Totsa-mé pa!

\StageDir{Geppino pouze lo vèyo é se plache i mentèn de Pino é de Walter.}

\Pinospeaks Brao, té pensa a sitte que mé vou avouì l'atro que l'è tchica pi pégno.

\StageDir{Pino rejouèn Selmo, que l'è eun tren todzor de danchì eun \textit{lento} avouì Ketty.}

\Pinospeaks\direct{Eun bouéchèn si l'ipala de Selmo} Comoddo fé lo gagà avouì eun \textit{lento}!

\Geppinospeaks Djèique, comoddo avouì eun \textit{lento}!

\StageDir{Selmo se ditaste de Ketty é se vionde bièn malechà ver Pino.}

\Selmospeaks \ldots é té qui t'i?

\Pinospeaks Mé si eun Tsarvensolèn!

\StageDir{Ketty, eunfastedjaye, s'achouatte si la banquetta.}

\Selmospeaks Mondjeu eunc\'o le Tsarvensolèn pe sèilla\ldots é senque t'ou?

\Pinospeaks Ara te fiyo vére!

\StageDir{Pino va ver la \textit{Band} é  dimande i chanteur de fé eunna tsans\'on pe danchì.}

\Pinospeaks\direct{I chanteur, eun blaguèn eun angllé} \textit{Dik Dik}, fé la \textit{stend alai of de biji of de vuord}!

\StageDir{Lo chanteur fé seumblàn de comprende, s'apreste é baille l'ataque i-z-atre joueur.}

\StageDir{Partèi la tsans\'on:}
\sound{https://www.youtube.com/watch?v=1sqE6P3XyiQ}{You Should Be Dancing - Bee Gees}

\StageDir{Pino é Selmo, a ten de mezeucca, se bataillon eun danchèn.}

\StageDir{Aprì eun per de mouvemàn de danse euntredeutte, le dou s'euncruijon é s'euntsambotton. La mezeucca s'arite.}

\scene[-- Patèle é véyo icllapoù]

\Pinospeaks Maladetto!\direct{A Selmo} T'a fé-lo esprése, sayò eun tren de gagnì!

\Selmospeaks Mi senque te di? Té t'i vin-i-me countre!

\StageDir{Selmo é Pino comenchon a se betì le man adosse.}

\Kettyspeaks\direct{Malechaye} V'ouite fran dou mar\'on!

\StageDir{Ketty chor eun mandèn a caqué Pino é Selmo.}

\Pinospeaks Avèista, t'a fé-la scapé!

\Selmospeaks Té t'a fé-la scapé!

\StageDir{Le dou contenevvon a se pouchì.}

\Pinospeaks	Totsa-mé pa pégno Pollentch\'on!

\Selmospeaks Mé te rounto le corne!

\Pinospeaks Mé te tchouèyo!

\StageDir{Pino é Selmo se baillon ba. Eunterveugnon eunc\'o Geppino é Walter. Lo local ara l'è eun \textit{ring pe boxeur}. Le vèyo s'icllapon, le caèye tsizon pe tèra é Gène avouì totta sa forse tsertse de pouchì foua di local Pino, Geppino é Selmo.}

\StageDir{Aprì eun per de seconde de poueun pe le coute é de patèle i dzoute, Gène tsache foua a piataye Pino, Geppino é Selmo. Walter tourne s'achouatté i banc\'on.}

\Dorinaspeaks\direct{Dispéraye} Que dizastre. L'an to cllapoù le vèyo!

\StageDir{Dorina pren eugn'icaoua é gnouye a recoillì le toque de vèyo.}

\Genespeaks\direct{A l'amia de Ketty} L'an to icllapoù, me fa clloure!

\StageDir{L'amia de Ketty chor ipouvantaye.}

\Genespeaks\direct{I joueur} Me fa clloure, l'è minite!

\StageDir{Le joueur s'avèitson é tcheut eunsemblo fan lo jeste di sou pe fé comprendre a Gène que vouillon itre payà douàn de alì ià.} 

\Genespeaks Ah la paye!

\Walterspeaks Sise l'an eunc\'o lo coadzo de dimandé de sou!

\StageDir{Gène, to pris\'o, pren de sou é paye le joueur. La \textit{band} beutte eun secotse le sou é chor avouì le-z-eunstrumèn.}

\Genespeaks\direct{A Walter} Té te pense-tì de dizarpì?

\Walterspeaks Pensò de bèye eunc\'o eun tor\ldots

\StageDir{Gène lèi tappe eunna bira eun lèi fièn comprendre de chotre di local. Walter pren la botèille é chor.}

\Genespeaks L'an to pouertchà! Ara iaou n'i bet\'o la servietta?

\StageDir{La servietta saoute foua dèi dézò lo banc\'on. Gène la pren é comenche a poulité.}

\Dorinaspeaks Sise dou Dandolèn son todzor le mimo! Djeusto de pastisse combin-\'o.

\Genespeaks L'è pa poussiblo. Ara cllouzèn maque lo local.

\ridocliou

\act[\avanSpect\ Avanspettaclle   \avanSpect]

\StageDir{Lemie \lemieSi\ si lo \textit{proscenium}.}

\StageDir{Entre Tania avouì lo journal eun man.}

\Taniaspeaks Lièn queunte son le noualle: maque de-z-éléch\'on, dèi la premì padze tanque a padze 19 n'a maque de-z-éléch\'on de totte le quemeue. Acappo pa lo manifeste de la fita di renoille\ldots de Pollein n'a ren. La Pro Loco l'aré oublià de beti-lo! Inque n'a totte le quemeue mi de Pollein ren\ldots i contréo n'a totta eunna padze si Tsarvensoù \ldots drolo\ldots queun prodouì tipique n'at a Tsarvensoù?

\StageDir{Silanse, Tania tsemiye eun pensèn.}

\Taniaspeaks Lièn mioù l'articllo\ldots \textit{Festa del Peperoncino}! Seutta l'è dzenta! L'è fran vrèi que a Tsarvensoù son maque bon a medjì de pépérontcheun é bèye d'éve di distributeur.

\Laurentspeaks\direct{Derì le rid\'o} Praou, praou!

\StageDir{Laurent entre.}

\Laurentspeaks\direct{A Tania, malechà} Planta-là lé de prédjì mal de Tsarvensoù! Mé meudzo pa de pépérontcheun é bèyo gnenca l'éve di distributeur!

\Taniaspeaks Vo de Tsarvensoù v'ouèide fata de noutra Grand-Place pe organizì de grouse fite! A Tsarvensoù v'ouèide gnenca de caro!

\Laurentspeaks L'è mioù que no Tsarvensolèn organizisan de fite a la Grand-Place pitoù que vére le dzi catchà dérì le bouèiss\'on que binocolon le feuille que se dezarbeuillon\ldots ou pitoù que vére sise de Pollein que dzouyon a Tsan eun \textit{serie} C!

\Taniaspeaks Grama lenva! \`Eitsa que no de Pollein n'en pi finque gagnà lo \textit{Championnat} de \textit{serie} A!

\Laurentspeaks Ouè si qué! Mi l'iye pe le-z-àn '80, 35 an fé, mersì i dzouyaou de Roulaz é de Feleunna! \'E te rapello que sise de Roulaz é Feleunna son de Tsarvensolèn!

\Taniaspeaks Ara que lèi penso\ldots pe le-z-àn '80\ldots

\act[Le-z-àn '80]

\ridoiver

\scene[-- Dou de peuque]

\StageDir{Lemie \lemieSi.}

\StageDir{Sen todzor i Disco Flama, mi si cou sen pe le-z-àn '80. Acapèn Gène, dérì lo banc\'on, é Dorina, protso de la remiza di dra, avouì 20 an de pi. Tcheu dou l'an le pèi blan. Gène, sourtoù, l'è bièn reillèn deun la jestch\'on di local. Pe bailli-lei eunna man, acapèn protso de llou Berta, sa feuille.}

\StageDir{Lo local l'è tsandjà reuspé a 20 an fé. A gotse n'at eunna tabla avouì trèi caèye rodze, a drèite, a la plase di banquette de bouque n'a de pégne poltronne é a la plase di joueur l'è plachà eun \textit{jukebox}.}

\StageDir{A drèite acapèn trèi feuille dzouveun-e de Tsarvensoù: Ornella, Vilma é Vanda. Son eun tren de danchì é tsaplétì. Vanda l'a djeusto fenì de pipé eunna sigaretta.} 

\Vandaspeaks	Ornella! Que dzenta la noua tsemize que t’a atsetoù!

\Ornellaspeaks Te remersio. N’i dimandoù a pappa le sou péqué mamma l’ariye jamì bailla-me-leu! 32 meulle lire!

\Vandaspeaks	Mi son fran tan!
 
\Ornellaspeaks	 Lo si, mi pappa avouì mé l'è todzor disponiblo pe me baillì de sou\ldots te sa, mé si lo prendre.

\Vilmaspeaks A mé l’ariye deu-me que n’a la crise é de sou n'a pa.

\Vandaspeaks	A mé igale.

\Ornellaspeaks\direct{Eun moutrèn l'\textit{entré}} Avèitsade qui areuvve!

\StageDir{Entre Giulio. S'aprotse a la remiza di dra, paye lo beillet, baille la djacca a Dorina é s'aprotse i banc\'on. Dimèn le trèi feuille conteneuvvon a tsaplétì.}

\Vandaspeaks	Mondje, que tipo, que gramo Pollentch\'on\ldots

\Ornellaspeaks	 \ldots é aprì blagueur, avèitsa que stat!

\StageDir{Vilma ite quèya. Chouèn avèitse Giulio avouì de joueu amoureu.}

\Vandaspeaks	Sit l’è tcheu le desando seuilla é va todzor eun blan.

\Ornellaspeaks Fé finque pèin-a.

\Vilmaspeaks Na\ldots sel\'on mé fé maque seumblàn, l’a cheur de grouse caletaye catchaye.

\Ornellaspeaks Mi senque te di? Te vèi pa que pase totte le nite a bèye solette?

\Vandaspeaks\direct{A Vilma} Aprì avèitsa-l\'o: sit l’at eunc\'o de grou problème de santé.

\StageDir{Dimèn entre Dino. Paye lo beillet é se plache i banc\'on.}

\Ornellaspeaks Ouè mondjeu! Mé que si totte, l’an finque deu-me que l’è it\'o opéroù da mèinoù.

\Vilmaspeaks\direct{Eun perdèn la pachense} Ara plantade-là lé!

\Vandaspeaks	Péqué? Te penserè pa de t’approtchì a salla baga lé?

\Ornellaspeaks	 Ouè, avouì totte le counte é deusquech\'on que l’an i pappa de té é lo seun.

\Vandaspeaks	Son eunc\'o ara eun tren de deusqueté pe l’éve: Pollein é Tsarvensoù alerèn jamì d’accor.

\Ornellaspeaks	 Sen tro diffièn.

\Vilmaspeaks Ara plantade-là lé pe dab\'on.  Se fran vouillade la savèi totta\ldots vo la diyo totta: mé è Giulio son dji dzor que no no véyèn.

\Ornellaspeaks	 Té é Giulio, dji dzor que vo vo véyade?

\Vandaspeaks Mi t'i matta?

\Ornellaspeaks Se lo veun a savèi teun pappa, té te chor pamì pe catro mèis.

\Ornellaspeaks Ouè cheur.

\StageDir{Dimèn Vilma avèitse Giulio. Le dou se sourion, s'avèitson a cats\'on, djouyon avouì leur joueu. Malerezamente, i mentèn de si djouà de regar é sourì, n'a Dino, que se beutte eun tita que Vilma l'è eun tren de lèi proué avouì llou. Di banc\'on l'avèitse, man deun le secotse, preste pe partì a l'ataque.}

\Dinospeaks\direct{A Gène, avouì le joueu ver Vilma} \textit{Bella}, \textit{bella} salla, ah ah!

\Genespeaks Dino\ldots

\Dinospeaks Oh!

\Genespeaks\direct{Eun moutrèn Vilma} Te la vèi salla ba lé i fon?

\Dinospeaks \ldots é beun?

\Genespeaks Sel\'on mé\ldots

\Dinospeaks Lèi reste?

\StageDir{Gène se vionde é pren dou vèyo é eunna botèille.}

\Genespeaks \`Eita, pourta-lèi doe \textit{mentine}\ldots

\Dinospeaks Ah ouè, doe \textit{mente}! Doe! Beutta doe!

\StageDir{Gène reumplèi le vèyo.}

\Dinospeaks\direct{Eun avèitsèn Vilma} \textit{Bella}!\direct{Eun mioulèn Gène} Beutta doe igale! Praou pai! Véo l'è?

\Genespeaks Son 500\ldots

\StageDir{Dino teurie foua le sou é le pouze si lo banc\'on.}

\Dinospeaks Paì é moueure, todzor a ten.

\StageDir{Dino pren le \textit{mentine} é s'aprotse a Vilma.}

\Dinospeaks\direct{A Vilma} Oh \textit{bella}! Danche? Te bèi? 

\Vilmaspeaks\direct{Digoutaye} Te prèdze a mé? Mi pe plèizì!

\Dinospeaks Te bèi?

\StageDir{Vilma se varde louèn de Dino é rejouèn le-z-amie de lleu que dimèn se son tramaye.}

\Vilmaspeaks Bèi solette, lèi manqueriye pi finque.

\Dinospeaks Ad\'on bèyo mé\ldots

\StageDir{Dino se plache pe eunna coueugne é bèi le doe \textit{mentine}.}

\scene[-- L'è de mé]

\StageDir{Vilma déside que l'è lo momàn de s'aprotchì a Giulio. Partèi é va ver lo banc\'on, mi, douàn de lleu, areuvve Berta.}

\Bertaspeaks\direct{Chalereuza, ver Giulio} Giulio! Comèn va?

\Giuliospeaks \direct{A Berta, eun avèitsèn Vilma} Tot amoddo.

\StageDir{Giulio tsertseré todzor de clloure lo discoù avouì Berta pe s'acapé avouì Vilma, que, bièn eunfastedjaye, tsertse de lo aterié avouì de pégno pouteun, de sourì é de joueu douse. }

\Bertaspeaks Giulio senque t'i eun tren de fiye? Te lame ma noua maille?

\Giuliospeaks \direct{A Berta, eun avèitsèn Vilma} Ouè, ouè dzenta!

\Bertaspeaks Dzenta vrèi? L'a atsetoù-me-la pappa. Can mimo Giulio\ldots senque te fé inque todzor to solette?

\Giuliospeaks 	Solette\ldots te sa\ldots mé si tchica defesilo.

\Bertaspeaks Giulio, soplé! T'a stoufia-me! Té te veun tcheu le dzor inque maque pe veure\direct{moutre Vilma} salla saquetta li!\direct{Malechaye} Giulio! L'è finque de Tsarvensoù!

\Giuliospeaks Fi pa ren. A mé pli Vilma!

\Bertaspeaks\direct{Dousa} Mi lèi si mé inque pe té\ldots

\Giuliospeaks Té? Si pa\ldots

\Giuliospeaks Acouta Giulio\ldots veun danchì avouì mé!

\StageDir{Partèi la tsans\'on:}
\sound{https://www.youtube.com/watch?v=g2eH5FSMsoc}{Tuca Tuca - Raffaella Carrà}

\StageDir{Ivron le danse Dino é Vanda. Avouì la squeuza di \textit{Tuca Tuca} Dino alondze tchica tro le man si Vanda que, sensa lèi pensé dou cou, rep\'on avouì eun per de patèle é piataye i qui que spédèison Dino pe eun anglle.}

\StageDir{A souivre, Berta pren Giulio pe eun bri é lo pourte danchì i mentèn di palque.  Vilma, rodze comme eun pèivr\'on pe la jalouzì, galoppe ver la cobla é fé de totte pe pourté ià Giulio: pouche ià Berta, danche avouì Giulio, diplache Berta, teurie pe eun bri Giulio can Berta danche avouì llou. A eun sertèn poueun, le doe feuille teurion Giulio eunna pe eun coutì é eunna pe l'atro.}

\StageDir{Aprì eun per de \og l'è de mé\fg é \og na l'è de mé\fg, Berta tsi pe tèra. Eun plaouèn galoppe ver Gène, que sensa itre tan euntéréchà a sen que l'è capit\'o, console la feuille avouì eun vèyo d'éve. Dimèn, Giulio é Vilma danchon amoureu dou pa di \textit{Tuca Tuca}.}

\StageDir{Fenèi la mezeucca.}

\Giuliospeaks\direct{A Vilma} Alèn bèye caque tsouza?

\Vilmaspeaks\direct{Eun sourièn} Ouè.

\StageDir{Le dou s'aprotson i banc\'on.}

\Giuliospeaks\direct{A Gène} Doe gazeuze.

\Genespeaks\direct{Caze digout\'o} Senque?

\Giuliospeaks Doe gazeuze, mersì!

\StageDir{Giulio se refeuze de soumondre de gazeuza. Lèi pense ad\'on Berta.}

\Giuliospeaks\direct{A Vilma} T'i fran dzenta oueu, finque le pèi\ldots

\Vilmaspeaks Mersì, eunc\'o té\ldots

\Bertaspeaks Giulio! N'i la gazeuza pe té\ldots

\Giuliospeaks Oh mersì.

\StageDir{Berta, bièn dousa é jantila, soum\'on lo vèyo a Giulio.}

\Bertaspeaks\direct{A Vilma} \ldots é seutta l'è pe té\ldots

\StageDir{Berta voueudze si lo coteill\'on de Vilma totta la gazeuza.}

\Vilmaspeaks\direct{Ibaiya} \`Eita seutta beurta saquetta! Giulio! Avèitsa sen que l'a fé-me!

\StageDir{Gène l'è ét\'o ibaì.}

\Genespeaks\direct{A Berta} Mi té t'i foula!

\Vilmaspeaks\direct{A Giulio, eun braillèn pi for}  Avèitsa sen que l'a fé-me!

\Giuliospeaks Tappa-lèi seutta!

\StageDir{Giulio baille son vèyo a Vilma é, sensa lèi pensé dou cou, Vilma lo tappe si lo coteill\'on de Berta.}

\Vilmaspeaks Té t'i fran foula!

\Bertaspeaks Pappa! Avèitsa sen que l'a fé!

\Genespeaks\direct{Bièn malechà avouì Berta} Silanse! Foua, va ià de seu!

\Bertaspeaks Mé salla la pachento pa!

\StageDir{Berta tappe eun panamàn adosse a Vilma. Gène, chocoù, apreste dou nouo vèyo de gazeuza é i mimo ten tsertse de tsachì foua Berta.}

\Genespeaks\direct{A Berta} Foua! Ita quèya!

\StageDir{Gène soum\'on le gazeuze a Giulio é Vilma.}

\Genespeaks\direct{Diplèizì} Squezade-mé\ldots seutte son pe vo é payo to mé.

\Bertaspeaks Mi pappa!

\StageDir{Gène se vionde ver Berta é per la pachense.}

\Genespeaks\direct{Eun braillèn} T'i eunc\'o seu?! N'i deu foua de seuilla! Va si avouì mamma! Foua!

\StageDir{Berta chor eun plaouèn.}

\Dinospeaks Matta! Salla l'è matta!

\scene[-- Ni bou, ni vi]

\StageDir{Entron catro djouyaou di Tsan de Pollein. Le premì trèi s'achouatton a tabla. Eun de leur l'è Selmo.}

\Djouiyaoueunspeaks Salì Gène!

\Genespeaks Bonsouar!

\Dorinaspeaks\direct{I dérì djouyaou} Lo dérì de vo paye l'\textit{entrée}?

\Djouiyaoudouspeaks Todzor lo mimo paye, pa lo dérì.

\StageDir{Lo djouyaou II paye é aprì s'achatte a tabla avouì le-z-atre.}

\Genespeaks\direct{I djouyaou} V'ouèide spounto-la?

\Djouiyaoueunspeaks N'en gagnà! Gène\ldots beutta cattro rodzo!

\Djouiyaoueunspeaks\direct{A sé compagn\'on} V'ouèi vi que pertse que n’i acapoù!

\Djouiyaoudouspeaks L'iye beun pi eunc\'o l'aoua! Se oueu gagnoon pa a Sen-Cretoublo pouoon belle quetì de djouì i Tsan.

\StageDir{Gène é Dorina voueudzon lo rodzo i cattro djouyaou.}

\Selmospeaks	Te pou lo diye. L'iye doe demendze que perdon-o doe tsachà contre eunna. L'iye belle l'aoua de gagni-nen eunna!

\Djouiyaoueunspeaks\direct{Eun moutrèn Marteun} \ldots é fa beun diye que se fise pa it\'o pe lli n'arian perdì eunc\'o le doe demendze devàn!

\Djouiyaoudouspeaks Ara blaga-l\'o pa trop. L’arè djouyà bièn eunna partiya\ldots magà doe! Pa de pi!

\Selmospeaks Ah ouè!\direct{Eun squersèn} Si gramo Tsarvensolèn\ldots pouè fé la séch\'on aoutre eun tchi lli, pitoù que tin-ì tsa seuilla a no!

\Marteunspeaks  Rapella-té que si mé que n’i lévoù la valeur de noutra équipe, rapella-te-l\'o!

\Selmospeaks Mi di pa de counte foule! T'a eunc\'o la mailletta diffienta di-z-atre!

\Marteunspeaks Pe fose! V'ouèide bailla-me vo seutta maille viille!

\Djouiyaoudouspeaks\direct{Todzor eun squersèn} Pe mé te pou tornì a djouì a Tsarvensoù\ldots a Rebatta, i Fiolet ou i pétanque!

\Selmospeaks Ah, ah, ah!

\Djouiyaoueunspeaks Vo de Roulaz n'en jamì si ioù vo plachì! Vegnade a l'icoula de Pollein é votade a Tsarvensoù!

\Djouiyaoudouspeaks Ah, ah! Ni bou, ni vi a Roulaz!

\Selmospeaks Ah, ah!\direct{Eun pouzèn la man si l'ipala de Marteun} Bon, praou ara\ldots pouo Marteun! Queten-lo itì, piatro ataque lo plaouo!

\Djouiyaoueunspeaks\direct{Eun lévèn lo vèyo de rodzo} Dimèn béyèn pe oueu que n'en gagnà! Pe Pollein!

\StageDir{Le cattro djouyaou lèvon lo vèyo.}

\Marteunspeaks Santé pe seutta!\direct{Ironique} Demendze que veun vo fiyo pédre, pai véyèn sen que diade!

\Selmospeaks N'a pa de reusque, magà perdèn touteun.

\StageDir{Le cattro bèyon lo crep.}

\scene[-- Bon tsachaou]

\StageDir{Comenche eun \textit{lento} avouì:}
\sound{https://www.youtube.com/watch?v=9OfoTaLXrUo}{Say You, Say Me - Lionel Richie}
\StageDir{Giulio é Vilma danchon. Aprì eun per de sec\'onde la mezeucca bèiche.}

\Djouiyaoudouspeaks\direct{A Selmo} Avèitsa ton gars\'on Giulio comèn danche!

\Marteunspeaks Sa lèi fiye!

\Selmospeaks	Giulio l'è de caletoù! Si pa se vo rapelade 20 an fi comèn n’i son-où si Tsarvensolèn!

\StageDir{Totta l'équipe di Tsan ri.}

\Djouiyaoueunspeaks Ouè me rapèlo, que riye si dzor! Dimanda eunc\'o a Gène\ldots

\Genespeaks\direct{Eunfastedjà} Que riye? V'ouèide icllapou-me eunna blita de vèyo!

\Djouiyaoudouspeaks Avouì to sen que n'en bi n’en dza paya-te trèi cou le vèyo.

\Dinospeaks\direct{A l'équipe} Payade eun vèyo?

\Djouiyaoueunspeaks Mi ouè! Gène beutta eunc\'o eun vèyo a Dino!

\Dinospeaks\direct{A Gène} Eun vèyo plen! Bèye!

\Selmospeaks \direct{Eun avèitsèn son gars\'on} Can mimo\ldots mé si pa senque vo diye. No n'en la caletoù pe le femalle. N'en de djette é bièn aloù que si arrevoù a lo passé i gars\'on de mé.

\Djouiyaoudouspeaks Aprì salla l'è eunc\'o eunna bella livra!

\Selmospeaks Giulio l’è eun bon tsachaou comme lo pappa.

\Marteunspeaks\direct{A Selmo} Bon tsachaou! Comèn fiyo a fé danchì salla dzenta feuille ba lé i fon. 

\Djouiyaoudouspeaks Selmo va té a lèi fiye vère comèn se danche va.

\Selmospeaks Queunta?

\Marteunspeaks\direct{Eun avèitsèn Ornella} Salla avouì la maille a fleur\ldots

\Djouiyaoueunspeaks Sel\'on mé sitte l'a renque fata d'eun meacllo!

\Selmospeaks Te fiyo vére mé Marteun!

\StageDir{Selmo se levve é pren pe eun bri Marteun.}

\Marteunspeaks Mi na! Squersao!

\Selmospeaks Na, na veun maque! T'a deu-lo, ara te lo fé. Vouì vére comèn te tappe la tsamba.

\StageDir{Selmo é Marteun passon douàn lo banc\'on.}

\Selmospeaks\direct{A Dino} Salì sotcho! Tot amoddo?

\Dinospeaks Paye eun vèyo?

\Selmospeaks Na aprì!

\StageDir{Selmo é Marteun s'aprotson i poltronne iaou son achouataye Ornella é Vanda.}

\Selmospeaks\direct{A Ornella} Squezade demouazella! Mé n'i eugn amì que l'è tchica euntredeutte é sa pa comèn dimandì pe danchì é ad\'on lèi fiyo vère mé\ldots

\StageDir{Ornella sourì. Selmo pren pe man Ornella é Marteun; le pourte i mentèn di palque.}

\Selmospeaks Premì baga: pa pité le pià! Secounda: sara-là amoddo mi\ldots beutta pa ba le man!

\StageDir{Marteun eumbrache Ornella pe danchì.}

\Selmospeaks\direct{I bouigno de Marteun} \ldots é trèijima\ldots souffla-lèi pe le bouigno\ldots leur lamon!

\Marteunspeaks Te di?

\StageDir{Selmo gante la tita é se plache i banc\'on.}

\StageDir{Lo volume de la mezeucca se levve. Aprì eun per de pa, Marteun souffle dou trèi cou pe le bouigno de Ornella.}

\Ornellaspeaks\direct{Eunfastedjaye} Mi va drimì i solèi!

\StageDir{Ornella tsache ià Marteun é tourne s'achouaté protso de Vanda. Marteun, démoralizoù, s'achouatte a tabla avouì Selmo. Lo volume de la mezeucca bèiche.}

\Selmospeaks Senque l'è capitoù?

\Marteunspeaks \og Souffla-lèi pe le bouigno\fg t'a deu-me! Brao! Dzen consèille!

\Djouiyaoudouspeaks T'aré cretou-lèi le bave!

\Selmospeaks Mi l'è pa poussiblo Marteun!

\scene[-- Amour amer]

\StageDir{Giulio s'aprotse a la tabla.}

\Giuliospeaks\direct{A Selmo} Pappa\ldots vi que t’i seuilla veun sé eun momàn\ldots vouì te prézenté ma choze.

\Selmospeaks\direct{\'Emochon-où} Mi cheur! To de chouite! Me perdo pa si momàn!

\StageDir{Selmo se levve é, eunsemblo a Giulio, va i mentèn di palque, iaou n'a Vilma.}

\Giuliospeaks Lleu l'è Vilma!

\Selmospeaks\direct{Eun sarèn la man de Vilma} Oh Vilma, a fose me la fé cougnitre! L'a todzor prédja-me-nèn! Ouè mi\ldots t'a eunc\'o acapou-la dzenta! Comèn t'a fi?

\Giuliospeaks\direct{Timido} Si vin-ì inque i Flama é n'i danchà eun momàn!

\Selmospeaks Brao! \direct{Ver Vilma} Squeza-n\'o eun momàn!

\StageDir{Selmo trame Giulio eunna mia pi louèn de Vilma. Dimèn entre Pino, pappa de Vilma. Se plache i banc\'on pe bèye eun creppe.} 

\Selmospeaks\direct{Eun prédzèn plan i bouigno de Giulio} Acouta botcha de mé\ldots rapella-té eunna baga: proua-nèn pi eun per douàn de betì si fameuille! Piatro aprì l'è tro tar! T'a comprèi?

\Giuliospeaks Ouè, ouè\ldots

\Selmospeaks\direct{Fier} Mi comèn t'a fé a acapé eunna dzenta pèi? Mé si belle contèn!

\StageDir{Pino l'è pris\'o, vou alì ià. Bèi la dériye gotta é s'aprotse a Vilma.}

\Pinospeaks Vilma l’è dza tchica tar, ara alèn i mitcho?

\Vilmaspeaks Na pappa, eunc\'o eun momàn!

\Pinospeaks Eunc\'o eun momàn? L'è dza minite, aprì te sen té mamma?

\Vilmaspeaks Na pappa, maque eun petchoù momàn!

\Pinospeaks Soplé, ara alèn!

\StageDir{Pino pren dézò bri Vilma pe chotre eunsemblo. Le dou van ver la chortia, mi malerezamente euncrouijon Selmo é Giulio. Le doe coble se blocon, paralizaye, eunna douàn l'atra. Le joueu de Pino é Selmo s'eunflamon. Vilma é Giulio son terrorizoù.}

\Selmospeaks Comèn? Té senque te fi sèilla?

\Pinospeaks\direct{Inervoù} Mé senque fiyo? Si vin-ì prende la feuille de mé!

\Selmospeaks\direct{Eunfastedjà} Mi pouì pa lèi crèye! Pino!

\Pinospeaks Selmo!

\Dinospeaks \ldots é Dino!

\Selmospeaks\direct{A Giulio} Mi eunc\'o té te faè fran acapé la feuille de lli? Avouì totte salle que n'at!

\Pinospeaks\direct{A Vilma} Senque l'a deu?

\Vilmaspeaks Ouè pappa, mé lamo Giulio!

\Selmospeaks\direct{Malechà} Senque t'a deu? Té te prèdze a salla baga\direct{eun moutrèn Giulio} lé pai! Avèitsa comèn l'è combin-où! Vilma se l'è maque pe lo non, n'en eunc\'o no-z-atre i Bournì eun Giulio que l'è ren mal!

\Vilmaspeaks Mi na pappa, mé vouì \direct{inamouraye, eun avèitsèn Giulio} llou!

\Selmospeaks\direct{Digout\'o} Mi avèitsa eun que stat l'et! Comèn l'è arbeillà! Salla baga lé!

\Selmospeaks\direct{A Giulio} Eunc\'o té Giulio! Avouì totte salle que n'at aoutre a Grand-Pollein: Rosinetta, Touanetta é totte le-z-atre! Fran seutta te diè acapì?

\Giuliospeaks\direct{Timido} Mi, mi\ldots \direct{eun sequeillèn} m-m-m-é lamo Vilma!

\StageDir{Giulio galoppe ver Vilma é dousemàn lèi pren la man pe la térì ver llou.}

\Giuliospeaks \ldots belle s-s-s-e l'è de Tsarvensoù!

\Pinospeaks\direct{Eun flame, ver Vilma} Mi sen-l\'o! Sequeuille finque!  Semble finque maladdo!

\Dinospeaks Amour! L'è amour!

\StageDir{Avouì le doe man Dino fé lo signe di queur é se tappe i mentèn de la discuch\'on. Selmo é Pino lo tsachon ià.}

\scene[-- Patèle é vèyo icllapoù II]

\StageDir{Selmo s'aprotse ver Pino, lo dèi drette ver lo vezadzo.}

\Selmospeaks \ldots é té prèdza pa mal di botcha de mé!

\Pinospeaks Beutta ba si dèi, piatro te planto eun\ldots

\Selmospeaks \ldots piatro l'è tourna l'aoua!

\Pinospeaks\direct{Avouì la vèin-a i cou} Te te rapelle 20 an fi lo paillano que n'i bailla-te? 

\Selmospeaks\direct{Avouì le nerfe tendì} Te planto seun dèi!

\Pinospeaks\direct{Comme douàn} Me tremblon tourna le man!

\StageDir{Selmo tchappe Pino pe la craotta é le dou comenchon a se baillì ba. Dino se tappe i mentèn pe le divijì, mi nen tchappe pi di-z-atre. Vilma é Giulio scappon ià. Areuvve Gène que, avouì la forse que pou èi eun vioù, tsache foua di local Selmo, Pino é Dino.}

\Dorinaspeaks Que dizastre! Tot icllapoù le vèyo.

\Genespeaks \ldots é l'è eunc\'o finque do-z-aoue! Me fa clloure.

\Genespeaks\direct{I djouyaou di tsan} Gars\'on\ldots dèyo clloure.

\Marteunspeaks Ad\'on beutta eunc\'o eun tor.

\StageDir{Gène pren eun boteill\'on de rodzo é lo régale i trèi djouyaou pe le mandé foua di local. Le trèi lo prègnon contèn é chorton. Dimèn Dorina l'è alaye prendre eugn'icaoua pe ramachì le vèyo cllapoù.}

\Dorinaspeaks\direct{A Ornella é Vanda} Mademouazelle, squezade\ldots cllouzèn lo local. L'è tar. Me diplì.

\StageDir{Ornella é Vanda chorton.}

\Genespeaks Tcheu le-z-àn\ldots

\Dorinaspeaks \ldots pason le-z-àn é la counta l'è todzor la mima.

\ridocliou

\act[\avanSpect\ Avanspettaclle \avanSpect]

\StageDir{Lemie \lemieSi\ si lo \textit{proscenium}.}

\StageDir{Entron Laurent é Tania.}

\Laurentspeaks\direct{A Tania} Pensao\ldots l'a pa de sanse i dzor de oueu no fé la guèra \textit{entre} Valdotèn; l'a pa de sanse se fé la guera \textit{entre} vezeun.

\Taniaspeaks T'a rèiz\'on. Ara comme ara fa pensì i futur. Ara se prèdze de betì eunsemblo le quemeue, de collaborach\'on pe lo bièn di parrotse é sel\'on mé déyèn pensé a quetsouza de nouo. Pe fé eugn izeumplo: senque te nen di se no de Pollein prégnèn catro dzor eun pi voutro medeseun?

\Laurentspeaks Ouè, dzenta idé! Pe eun discoù de collaborach\'on pourian fe pouèi\ldots mi eun tsandzo vo de Pollein no betade pe icrì que can noutro prée l'è maladdo ou l'è pa eun servicho veun si si de Pollein. Pouèi messa comenche eugn oréo, fenèi pi vito é n'en eunc\'o lo ten de bèi eun blan!

\Taniaspeaks Maque a so pensade!

\Laurentspeaks D'acor?

\Taniaspeaks Ouè! Mé n'i eugn'atra idì: senque te di se fièn la patse pe le montagne? Tsamolì, Martsaouchì\ldots comprègno pa pequé le Tsarvensolèn vardon salle de Pollein é le Pollentch\'on vardon salle de Tsarvensoù.

\Laurentspeaks Braa, si d'acor\ldots é pourian eunc\'o troué eugn acor pe le doe formach\'on di pal\'on!

\Taniaspeaks Ah na! So va pa bièn pe no. A Pollein n'en pa la formach\'on di pal\'on; no tegnèn maque i vrèi spor valdotèn: lo Tsan.

\Laurentspeaks Lo Tsan! Te me vou fé crèye que vo de Pollein v'ouèi eunvent\'o lo Tsan? Mi pe case t'a vi de queunta couleur son le maille de sise de Pollein? Te la sa pa la counta?

\Taniaspeaks Queunta couleur di maille é queunta counta?

\Laurentspeaks Ad\'on\ldots l'iye lo 6 juillet 1964, can Jozé Louis Squinabol, nèisì a Barcellona, l'è vin-ì eugn Italì pe tsertchì forteun-a. L'è tramou-se eun Val d'Ousta, l'è arrevoù a Pollein é l'a prèi lo mitcho i Dredjì. Avouì llou, l'a portoù eun sé de l'Espagne si djouà: lo Tsan! Eugn Espagne se queriè pa lo Tsan\ldots l'iye la \textit{Pertica} ou la \textit{Perticas}\ldots é eun seun oneur le Pollentch\'on, vi que lamaon tan si djouà, l'an desid\'o de lèi fé chédre la couleur di maille. Pouè chédre le couleur di péì iaou l'è nèisì, Barcellona, ou salle di son veladzo\ldots Dredjì!

\Taniaspeaks Mi planta-là li!

\Laurentspeaks Aprì doe ou trèi nite iaou l'a pa tan drimì, l'a désid\'o pe Barcellona! Pouèi v'ouèide la maille di Barcellona!

\Taniaspeaks\direct{Stouffie} T'a fenì de di de counte foule? Soplé, pensèn a de bague sérieuze. Mé n'i reflèichì desì eugn'atra baga: senque te pense se vo louyèn la Grand-Place pe trèi ou catro dzor la senà? No divijérian le spèize de manutench\'on!

\Laurentspeaks\direct{Pa convencù} Mé diriyo\ldots fièn eugn atro discoù pe la Grand-Place. Pitoù que le spèize de manutench\'on, mé vo baillério a vo de Pollein l'oneur de fé si voutro territouéo\ldots l'oneur de fé la fita di pépérontcheun!

\Taniaspeaks Na! Na, na! La fita di pépérontcheun vardade-vo-là! No l'è di-z-àn '90 que n'en la Grand-Place é de fite pouèi le vouillèn fran pa.

\Laurentspeaks Te me vou fé crèye que a Pollein pe le-z-àn '90 v'ouèide jamì fé de fite drole pouèi?

\StageDir{Teuppe \lemieBa.}

\act[Le-z-àn '90]

\ridoiver

\scene[-- Sen preste?]

\StageDir{Lemie \lemieSi.}

\StageDir{La streutteua de la \textit{scénographie} l'è la mima, mi avouì eun per de tsandzemèn pe la rendre pi moderna: si lo banc\'on l'è plachà lo sidel de la llase pe lo \textit{Champagne} é a drèite, iaou n'ayè lo \textit{jukebox}, n'at eunna danseuse é la \textit{consolle} di \textit{DJ}. Dérì lo banc\'on, pendia i meur, n'a eunna fotografie de Gène.

Ara la propriétéa di local l'è Berta (feuille de Gène). A la remiza di dra n'a todzor Dorina, avouì dji-z-àn de pèi blan eun pi é eun per de \textit{decibel} eun mouèn. Achouataye li eun livro, mi eun réalit\'o, avouì de grou biicllo di solèi, l'è eun tren de drimì. Dérì lo banc\'on acapèn Hélène, serventa di local. L’è minite é le premì cliàn dérian arrevé de sé a pouza.

Berta entre inervaye.}

\Bertaspeaks Dorina!

\StageDir{Dorina pren pouiye é se rèche.}

\Bertaspeaks Ad\'on? Rècha-teu vèi! Soplé, beutta eun caro mon palt\'o.

\StageDir{Dorina se gave lo palt\'o é lo baille a Dorina, que, mèitchà eundrimiya, lo pen i pourtamantì.}

\Bertaspeaks Acoutta bièn\ldots pen-l\'o louèn de si di cliàn! Pa comme lo dérì cou que aprì s'eumplèi de flou di feun! N'i djeusto portou-lo lavì!

\StageDir{Berta s'aproste i banc\'on.}

\Dorinaspeaks\direct{Eun prédzèn todzèn} Oueu l'è greundze!

\Bertaspeaks\direct{A Hélène, todzor inervaye} Hélène! Hélène senque t'i eun tren de fiye?

\StageDir{Hélène cllou lo journal.}

\Helenespeaks\direct{Naif} Saioù eun tren de lie lo journal.

\Bertaspeaks\direct{Malechaye} Hélène! Sel\'on té mé te payo pe ité inque a lie lo journal?

\Helenespeaks Mi l'è to vouido lo local! N'ayè maque Dorina que drimave.

\Bertaspeaks Acoutta-mé amoddo Hélène! Mé si alaye ià pe fé de commich\'on é n'ayoù deu-te de ité inque pe aprestì totte\ldots

\Helenespeaks Ouè n'i tot aprestoù.

\Bertaspeaks Ouè véyèn\ldots donque, t'a reumplì lo frigo?

\Helenespeaks Ouè.

\Bertaspeaks T'a tsandjà lo fuste de la bira é di gas?

\Helenespeaks Na. 

\Bertaspeaks Comèn na?

\Helenespeaks Na, a la crotta n'ayè pamì de fuste é ad\'on n'i prèi de botèille de bira.

\Bertaspeaks N'i dza de-te-l'o l'atra senà que te fayè atseté le fuste! Te lo sa que le botèille couton bièn pi tchiye!

\Helenespeaks\direct{Eun tsertsèn de squize} Beh\ldots son pi bon-e.

\Bertaspeaks Ouè, te me le paye té. Can mimo\ldots lo Champagne t'a betou-lo deun la llase? 

\Helenespeaks\direct{Eun moutrèn lo sidel si lo banc\'on} Ouè, doe botèille i frique.

\Bertaspeaks Djeusto doe?

\Helenespeaks Ouè baston!

\Bertaspeaks Se baston pa te va pi a la crotta a le prendre i galoppe!

\Helenespeaks Va bièn\ldots ah! N'i eunc\'o aprestoù le frique pe la Caipirinha, la menta pe lo Mohito é\ldots na\ldots n'i oublià lo \textit{lime}!

\Bertaspeaks\direct{Dispéraye} Mi comèn t'a oublià lo \textit{lime}! Te lo sa que oueu l'è eunna souaré eumpourtanta. Oueu l'è la dériye nite da gars\'on pe Giulio é seutta nite si local saré plen de plen: Giulio é tcheutte le-z-amì de lli! Ah\ldots é fé attench\'on que sise lé l'an le man crevaye! Ara feulla to de chouite prendre si \textit{lime}!

\StageDir{Hélène chor de coursa avouì le pèi drette.}

\Dorinaspeaks\direct{A basa vouése} Oueu l'è greundze!

\Dorinaspeaks\direct{A Berta} Moito, Capirina, Pépito\ldots senque son-tì totte seutte potchacaye?

\Bertaspeaks Dorina soplé ataca pa eunc\'o té avouì seutte rentse!

\Dorinaspeaks Pe fose que le dzouveuno i dzor de oueu reston malle pe de dzor.\direct{Mélanconique} I ten de mé, n'ayè pa totte seutte potchaquiye. No béyaon todzor lo Vermouth! L'iye boun lo Vermouth!

\StageDir{Hélène tourne dedeun avouì lo \textit{lime}.}

\Dorinaspeaks\direct{Stouffie} Ouè Dorina, ouè.

\StageDir{Entron lo DJ é la \textit{cubista}, que s'aprotse a Hélène pe ataqué a fé la counta.}

\Djspeaks \direct{A Dorina} Ouélla! Dzenta tanta! Sopatta-té di pidze é de la poussa que oueu ou te mando i paadì o a l'enfeur!

\Dorinaspeaks Malpoulite! Dibroilla-té maque que t'i dza eun retar é Berta l’è dza ajitaye é greundze.

\Bertaspeaks Seutta t'a bièn deu-la Dorina!  \direct{I Dj} Boudza-té é va aprestì le baradziye de té\ldots avouì sen que te payo!

\Djspeaks Ouè, queunse sou? Gneunca sisan ba pe le-z-ivrèye i Milù!

\Bertaspeaks Rabadàn!

\Bertaspeaks\direct{Ver Hélène é la \textit{cubista}} \ldots é vo doe? Senque v'ouite eun tren de fiye?

\Helenespeaks Prédjì!

\Bertaspeaks\direct{A la \textit{cubista}} Va t'aprestì!

\StageDir{Eun silanse la \textit{cubista} se plache si lo cube.}

\Bertaspeaks Hélène t'i todzor eun tren de pédre de ten! Planta-là li! Poulita si banc\'on soplé!

\Helenespeaks Ouè va bièn!

\Bertaspeaks \ldots é t'a vi? Seutta nite n'en finque prèi eunna \textit{cubista}!

\Helenespeaks Pequé? N'ayè pa fata de eunna \textit{cubista} seutta nite.

\Bertaspeaks Mi Hélène! N'i deu-te que oueu l'è la dériye nite da gars\'on pe Giulio!

\Helenespeaks Djeusto! Mi touteun\ldots paì eunna \textit{cubista} pe Giulio!

\Bertaspeaks Si mé que la payo, pa té! \'E aprì pa maque eunna \textit{cubista}\ldots te vèi pi queun squerse n'en organizou-lèi!

\Helenespeaks Ah ouè?

\StageDir{Lo \textit{DJ} fé partì la mezeucca que saré todzor basa de volume dimèn que la counta avanche.}

\scene[-- Dzen si local]

\StageDir{Entron dou dzouveun-o, eun poumpa pe fé fita é s'amuzé. Galoppon drette i banc\'on.}

\Dorinaspeaks Iaou alade vo dou?

\Pierrespeaks Iao alèn? Alèn dedeun, te vèi pa?

\Dorinaspeaks V'ouèide-tì prenotoù?

\Pierrespeaks Prenotoù? Mi se l'è vouido, n'a gneun! Fa-tì eunc\'o prenotì ara?

\Dorinaspeaks Ouè, lo local l'è to prenotoù!

\Dorinaspeaks Can mimo, v'ouèide de chanse! Desì lo carnet n'i eunc\'o do plase libre\ldots vo fiyo entrì. Quetade-mé maque le palt\`o inque a mé.

\Pierrespeaks Oh que jantila, mersì!

\StageDir{Louis é Pierre queutton le palt\`o a Dorina é tournon i banc\'on.}

\Dorinaspeaks \ldots é paì?

\Pierrespeaks Fa eunc\'o paì?\direct{A Louis} Paya té que t'i dzouveun-o.

\Louisspeaks Féo mé, tracacha-té pa.

\StageDir{Louis paye é aprì rejouèn i banc\'on  Pierre, que l'a dza tap\'o lo joueu si Berta é Hélène.}

\Pierrespeaks\direct{Tchica tchapeun} Bonsouar madame!

\Bertaspeaks\direct{Frèide} Bonsouar.

\Pierrespeaks\direct{A Louis} Senque béyèn?

\Louisspeaks Mojito!

\Pierrespeaks Top! \direct{A Hélène} Dou Mojiti!

\Louisspeaks Tsardjà!

\Pierrespeaks Avouì lo \textit{lime}!

\StageDir{Hélène gnouye a aprestì le dou \textit{cocktail}.}

\Louisspeaks\direct{A Pierre, entouziaste, eun avèitsèn lo local} Que poste!

\Pierrespeaks\direct{Eun sourièn} Ouè, que dzen!

\StageDir{Louis vionde la tita ver la \textit{cubista} é reste éton-où.}

\Louisspeaks\direct{Eun moutrèn la \textit{cubista}} \`Eitsa que qui! \`Eitsa-l\'o!

\StageDir{Eunc\'o Pierre l'è eumpréchon-où, le joueu foua de la tita.}

\Pierrespeaks Mi senque l'a salla? Lo fouà i qui! \`Eita comèn danche! Belle que sen a Pollein, l'è dzen si local!

\Louisspeaks Sen jamì entroù, mi la prochène senà sen seu.

\Pierrespeaks Belle demàn se son iver!

\Louisspeaks\direct{A Berta} V'ouite iver demàn?

\Bertaspeaks Ouè sen todzor iver.

\Pierrespeaks Ad\'on prenotoù, pe dou!

\StageDir{Hélène soum\'on le Mojiti.}

\Helenespeaks \ldots é seu son le dou Mojiti!

\StageDir{Pierre é Louis prègnon le Mojiti é fan santé.}

\Pierrespeaks A no! Alèn aoutre lé que n'a de plase pe s'achaté.

\StageDir{Pierre é Louis s'achatton si de pégno chofà a drèite di palque, dèi iaou pouon bièn veure totta la danse de la \textit{cubista}.}

\Bertaspeaks\direct{A Hélène} T'a sentì senque l'an deu le dou gars\'on?

\Helenespeaks Na\ldots

\Bertaspeaks\direct{Fière} \og Que dzen local\fg!

\Helenespeaks\direct{\'Emochon-aye} Pequé n'a no dedeun!

\Bertaspeaks\direct{Grama} Pequé lèi si mé dedeun!

\StageDir{Berta, orgueilleuza, se viounde ver la fotografie di pappa Gène.}

\Bertaspeaks Mon cher pappa, se te pousise veure eunc\'o té\ldots Hélène, té t'a beun cougni-lo pappa, na?

\Helenespeaks Na mé si tro dzouvin-a.

\Bertaspeaks Hélène senque vou diye? Eunc\'o mé si dzouvin-a!

\Helenespeaks Pa comme mé!

\Bertaspeaks\direct{Sensa baillì fèi a Hélène} Pousise veure eunc\'o lli comèn son tsandjaye le bague! Mersì a mé! \`Eitsa que dzen local, pa comme l'iye eun cou. Sarie bièn countèn de mé. Can l'iye eunc\'o lli seu dedeun mé pouò pa fé ren, maque to lo dzor travaillì, travaillì é travaillì! Poulitì, lavì le vèyo, panì le vèyo, poulitì é panì le vèyo! Fiò pa d'atro que travaillì!

\Helenespeaks\direct{Ironique} Squeuza\ldots té te lavave le vèyo, lavave le vèyo, lavave le vèyo?

\Bertaspeaks Ouè, mé lavavo le vèyo, lavavo le vèyo, lavavo le vèyo. Fran mé\ldots é ara te pou veure comèn l'è vin-ì si local! Grase a mé!

\Bertaspeaks Ah! Aprì se sise inque avouì, no é véise Pino é Selmo a tabla eunsemblo, crériye pa a seun joueu! Mi si cou son belle oblidjà a alì d'acor: la senà que veun leur dou mèinoù se marion; l'an fenì de deusquetì. Avouì to sen que l'an fé-lèi pasì!

\Helenespeaks Senque l'an fé-lèi pasì?

\Bertaspeaks Pe de-z-àn é de-z-àn tcheu le cou que s'acapavon seuilla se verlaon!

\Dorinaspeaks Eunna blita de vèyo icllapoù!

\Bertaspeaks Totte icllapavon\ldots é mé poulitavo.

\Helenespeaks Té te poulitave?

\Bertaspeaks Ouè. Can mimo\ldots l'è dza belle tchica tar\ldots Giulio é le-z-amì de lli dérian pa tardì a arevì. Beutta eugn odre lo banc\'on!

\Helenespeaks Ouè, ouè tracàcha-te pa.

\scene[-- Ouassa pe lo moteur]

\StageDir{Entron Giulio é trèi seun amì: Federico, Marco é Michel. S'aprotson i banc\'on eun tsemièn tchica tose: son dza tcheu catro belle tsa.}

\Dorinaspeaks\direct{I catro gars\'on} V'ouèide-tì prénotoù?

\Federicospeaks\direct{Eun desuèn Dorina} Mi comèn v'ouèide-tì prénotoù? 

\Marcospeaks Mi se sen todzor séilla! 

\Michelspeaks Aprì l'è pi maque minite! N'a pa eunc\'o gneun!

\Bertaspeaks Maque minite? L'è dza eun bon momàn que vo-z-atégnavo.

\Federicospeaks\direct{Eun se medzèn le paolle} Squeza-n\'o\ldots no sen arit\'o a fé benzin-a; pe sen que sen arevoù eun retar.

\Marcospeaks N'ay\'on finque fené le boun!

\Michelspeaks Pouèi sen vin-ì a pià di Pon. \direct{Eun s'aprotsèn a Dorina} Sen so que flou de la chao que n'i\ldots

\StageDir{Michel levve lo bri é lo pourte i na de Dorina.}

\Dorinaspeaks Ah na, na! Va maque llouèn! A mé semble que pitoù que fé lo plen a la machina, v'ouite reumpli-vo euntre vo!

\Federicospeaks Ouè, mi pa pi de \textit{super}! \direct{Eun tapèn son palt\'o a Dorina} Tchappa so caramban-a!

\Dorinaspeaks A sentì l’èina que t’a, me semble pitoù que martsade a ouassa.

\StageDir{Michel dimèn paye l'\textit{entrée} pe tcheutte.}

\Giuliospeaks Ouè Dory l’a belle rèizoun!\direct{Eun s'aprotsèn a Berta} Berta, eumbracha-mé que oueu l’è ma dériye nite da garsoun!

\Bertaspeaks\direct{Digoutaye, eun se ditatsèn de Giulio} Ouè, ouè, acoutta, beutta ba le man pe plèizì, soplé!

\StageDir{Le-z-amì de Giulio se plachon i banc\'on.}

\Giuliospeaks\direct{Comme douàn} Desando queun me mario!

\Bertaspeaks\direct{Comme douàn} Ouè lo si Giulio que te te marie, mi te rapello que mé ara si eunna feuille eungadjaye\ldots é aprì te rapello que can te pouè, dji-z-àn fi, t'a pa pi beto-lèi tan a me mandì a si péì!

\Giuliospeaks Say\'on d'atre ten\ldots

\Bertaspeaks\direct{Malechaye} Ouè d'atre ten.\direct{A Dorina} Dorina, le palt\'o de leur pe tèra, pa pi protso di meun!

\Marcospeaks Hélène! Beutta-n\'o eun tor!

\Helenespeaks Senque vo baillo?

\Federicospeaks\direct{Gadeun, eun alondzèn le man} Sario prao mé sen que no baillì, mi maque a mé! 

\StageDir{Hélène fé eun pa eun dérì é Michel bloque Federico.}

\Bertaspeaks\direct{A Federico, seuria} Soplé, beutta ba le man.

\Helenespeaks Donque, senque vo beutto a bèye?

\Marcospeaks\direct{Eun terièn foua le sou} Beutta-n\'o catro sidel de Mimosa!

\Helenespeaks Mimosa? Senque l'è? L'è pa pi la fita di femalle!

\Michelspeaks\textit{A Hélène} Te sa pa senque l'è la Mimosa? L'è lo \textit{cocktail} di momàn! Mi iaou te viquèi?

\Bertaspeaks Mondjeu Hélène! Senque te payo a fiye?

\Federicospeaks Mi queunta Mimosa! Beutta ba catro sezeleun de bira! 

\Helenespeaks Oh sen ouè!

\scene[-- Si le man!]

\StageDir{Lo DJ levve lo volume de la mezeucca.}

\Djspeaks Salì Pollein!

\StageDir{Tcheutte braillon é salion lo DJ.}

\Djspeaks Qui levve pa le man, areuvve pa a demàn!

\StageDir{Partèi la fita: Hélène pourte lo bèye, Berta counte le sou, le-z-amì fan santé pe la dériye nite de Giulio, la \textit{cubista} ataque a danchì. Pierre é Louis lèi beutton eun per de beillette deun le pantal\'on.}

\StageDir{La mezeucca bèiche.}

\Giuliospeaks\direct{A sé-z-amì} Mi salla feuille li que danche? V'ouèide pens\'o a to so pe mé?

\Marcospeaks Ouè, salla li l'è pe té é rapella-té que sen que capite seutta nite i Flama reste i Flama!

\Giuliospeaks Ad\'on alèn aoutre!

\Marcospeaks Partèi a l'ataque!

\StageDir{Giulio s'aprotse eun danchèn ver la \textit{cubista}. Pierre é Louis son todzor a l'entor de lleu.}

\Pierrespeaks\direct{A Giulio} Sotcho! Beutta-té eun quia!

\StageDir{Giulio se vionde ver sé-z-amì; comprèn pa.}

\Marcospeaks\direct{A Giulio} Bailla pa fèi a sise Tsarvensolèn! Dancha maque eunc\'o té!

\StageDir{Marco s'aprotse eunc\'o lli a la \textit{cubista} pe baillì de coadzo a Giulio.}

\Djspeaks Ad\'on Pollein v'ouite tsa?

\StageDir{Tcheutte braillon. La mezeucca se levve: Giulio é Marco baillon campa.}

\StageDir{Lo volume de la mezeucca bèiche.}

\Michelspeaks \direct{A Berta} Ad\'on Beurta\ldots

\Bertaspeaks Berta!

\Michelspeaks \ldots v'ouèide aprest\'o lo squerse?

\Bertaspeaks Oue tranquilo, lèi penso mé.

\Djspeaks Pollein v'ouite eunc\'o tsa?

\StageDir{Tcheutte braillon, levvon le man é danchon. Dimèn Berta s'aprotse i DJ, lèi di quetsouza i bouigno é aprì chor.}

\StageDir{La fita conteneuvve: Giulio veun prèi di-z-amì de llou é tapp\'o si pe l'er eun per de cou. La mezeucca l'è forta.}

\StageDir{To d'eun creppe entre eun vallet de veulla fenna avouì eun manganel eun man.}

\Valletspeaks Ad\'on! Senque l’è si vacarno? Tchouéyade la mezeucca!

\StageDir{La mezeucca se tchoué. La fita s'arite. Tcheutte son ibaì.}

\Valletspeaks\direct{Ver Hélène} Qui l’è la patroun-a di local? Queriade-là!

\Helenespeaks Vou la querì to de chouite!

\Valletspeaks Vito! \direct{I-z-atre prézàn} Vo tcheut, si le man é foua le documàn.

\StageDir{Tcheutte levvon le man.}

\Michelspeaks Ouè mi comèn fièn a terì foua le documàn se n'en le man lévaye?

\Valletspeaks Douàn foua le documàn!

\Pierrespeaks \direct{A Louis} N'ay\'o deu-te de pa queuttì la machina desì lo martsepià, que seuilla a Pollein comprègnon pa seutte bague.

\Valletspeaks Vo dou silanse! V'ouite pa pi dzen di-z-atre!

\Louisspeaks\direct{A Pierre, eun baillèn lo  documàn i vallet} T'ayè beun rèiz\'on.

\Valletspeaks Qui l’é Giulio de Beccafrantse? 

\StageDir{Tcheutte moutron avouì lo dèi Giulio, que dimèn l'iye aprotsa-se a la chortia.}

\Michelspeaks Giulio te tsertson!

\StageDir{Michel pren Giulio déz\'o lo bri é lo plache i mentèn di palque.}

\Valletspeaks Ah v'ouite vo?

\Giuliospeaks\direct{Ajit\'o} Si mé pequé? 

\Valletspeaks Vegnade seuilla. Vo sade pequé mé si inque?

\Giuliospeaks Na\ldots senque n'i fi?

\Valletspeaks Na? Vo sade pa?

\Giuliospeaks Na.

\Valletspeaks Ad\'on vegnade inque, que vo fiyo vin-ì eun devàn to de chouite lo pequé.

\StageDir{Giulio terroriz\'o s'aprotse.}

\Valletspeaks\direct{I-z-amì} Vo! Portade seuilla eunna caèya é èidzade-mé a lo llatì.

\StageDir{Le-z-amì de Giulio prègnon eunna caèya, achouatton Giulio é lo llaton.}

\Giuliospeaks\direct{Todzor pi ajit\'o} Eun lèi pensèn mioù, pou se fiye que can sayò pitchoù magà n'i roboù eun per de caaméle\ldots pa d’atro!

\Valletspeaks Silanse! 

\Giuliospeaks Can n'i robo-le sayò solette!

\StageDir{Giulio ara l'è to llat\'o a la caèya. Le-z-amì se plachon dérì de lli, protso di banc\'on. Tcheutte riyon eun silanse dimèn que lo squerse conteneuvve.}

\Valletspeaks\direct{Arroganta} Ita quèi! Eun manganel desì l'itseun-a t'a jamì tchapou-lo?

\Giuliospeaks Diqué?

\Valletspeaks Eun manganel!

\Giuliospeaks Mondjemé, na! L'icaoua de tenzentèn ouè avouì mamma\ldots mi lo manganel na!

\Valletspeaks Ad\'on n'a todzor eun premì cou!

\StageDir{Lo vallet baille eun crep de manganel si l'itseun-a de Giulio. Le-z-amì riyon for.}

\Valletspeaks \ldots é ara comenche la fita!

\Giuliospeaks La fita?

\StageDir{Partèi eunna mezeucca \textit{dance}. Lo vallet comenche a se dizarbeillì; vionde a l'entor de Giulio, lo acaresse.}

\Giuliospeaks\direct{Bièn pi relassoù} Ouélla!

\StageDir{Le-z-amì de Giulio, Pierre é Louis son électrizoù; l'an le man si pe l'er é se dzouzon lo spétaclle.}

\scene[-- Totchì, mi pa avèitchì!]

\StageDir{A eun sertèn poueun la mezeucca se tchoué é lo vallet, avouì sa craotta, toppe le joueu de Giulio, que dimèn fricotte si la caèya.}

\Djspeaks Oueu l'è eun dzor spésial pe noutro amì Giulio é fran pe sen l'è preste eun cad\'o djeusto pe lli! Si le man gars\'on!

\StageDir{Repartèi la mezeucca.}

\Djspeaks Renque pe té Giulio, renque pe seutta nite, eun cad\'o djeusto arev\'o di Brézil\ldots bouichèn for le man pe Sami!

\StageDir{Entre eun \textit{trans} brazilièn: perruque roze, foulard rodzo, tsemize dzana, \textit{jeans} é de grou pià que chorton foua di-z-\textit{infradito}. La mezeucca tsandze pe créé eunna ambianse pi brazilienne. Lo vallet se plache i banc\'on pe quetì la plase a Sami.}

\StageDir{Sami danche a l'entor de Giulo, se frotte si de lli, lo caèche. Tcheu le-z-atre danchon eun fièn lo tréneun. Eunc\'o Hélène se fé prende pe la fita, mi Berta la redriche to de chouite.} 

\StageDir{La mezeucca bèiche. Sami s'achouatte desì Giulio é lèi prèdze avouì eunna vouése ata é da femalla.}

\Samispeaks Ad\'on Giulo, vegnèn i donque! 

\Giuliospeaks Mondjemé! Que man souple que t'a! 

\Samispeaks T'i to tsa!

\Giuliospeaks Se te me tazeuille pouai sento to fromiatì ba pe l'itsin-a! Gava-mé la benda, gava-mé la benda!

\Samispeaks Na! Totchì mi pa avèitchì!

\Giuliospeaks D'acor, mi si llatoù!

\Samispeaks Ad\'on te diillatto.

\StageDir{Sami diillatte Giulio. Giulio se levve é avouì le man tsertse Sami.}

\Giuliospeaks Iaou t'i Sami?

\Samispeaks Seuilla!

\StageDir{Sami pren Giulio é lo pourte i mentèn di palque pe eunna danse sensuelle, iaou Giulio pou totchì eunna mia de pi. Dimèn le-z-amì conteneuvvon a riye é a fé fita.}

\Giuliospeaks\direct{Eun totsèn le-z-ipale, douteu} Mi\ldots Te me semble pi granta que totar\ldots

\Samispeaks Que dzen que t'i!

\Giuliospeaks Ouè t'a rèiz\'on. Acoutta! Gava-mé ià vito seutta benda pai pouì vére iaou betì le man!

\Samispeaks Bièn cheur mon cher!

\StageDir{Sami gave ià la craotta di joueu de Giulio.}

\Samispeaks\direct{Avouì sa vouése d'ommo} Cou-cou!

\Giuliospeaks Mondjemé la barbuta!

\StageDir{Giulio baille la balta. Tcheutte son pléyà eun catro di riye.}

\Samispeaks Lo \textit{shock} l’a tchoué-lo! 

\StageDir{Le-z-amì se plachon a l'entor de Giulio é tsertson de lo réchì. Sami l'è desì de lli é lèi fé d'er avouì sa perruque. Aprì eun per de sec\'onde, Giulio se rèche.}

\Giuliospeaks\direct{Dizorientoù} Djablo sen que n'i sondjà!

\StageDir{Giulio levve la tita é vèi Sami desì de lli.}

\Giuliospeaks L'iye eugn ommo!

\StageDir{Giulio capote eun sec\'on cou. Tcheutte riyon; i contréo, Berta é Hélène son eunna mia tracachaye pe Giulio. Hélène pourte de coursa eun vèyo d'éve.}

\Helenespeaks\direct{A Michel} D'éve, d'éve!

\StageDir{Michel pren lo vèyo d'éve é eunsemblo i-z-amì lo voueudze si lo vezadzo de Giulio, que se rèche é se teurie si eun catro catro ouette.}

\Marcospeaks L'è reprèi-se!

\Michelspeaks\direct{A Giulio, eun fièn semblàn d'itre tracachà} Pensao que t'ie de l'atro djette!

\Giuliospeaks Mi de squerse pai? Eunc\'o tchica é me vegnave eun creppe!\direct{Ver Sami} \'E eunc\'o té avouì salla vouése n'ayò pa comprèi! Se me véyè Vilma me mariave pamì\ldots é senque l'an totchà seutte man!  Ara si coudzì de alì i coumeun\ldots avouì to sen que n'i totchà!

\Marcospeaks Se t'ioù le copèn!

\Michelspeaks Ara ézajèra pa, t'a dza totchà d'atro!

\StageDir{Giulio chor.}

\Samispeaks Mé vou me tsandjì.

\Michelspeaks Ouè é aprì béyèn eun creppe avouì Giulio!

\StageDir{Sami chor.}

\Michelspeaks\direct{A Hélène} Beutta tourna eun tor.

\Bertaspeaks Douàn paya-l\'o!

\StageDir{Marco beutte le sou si lo banc\'on. Dimèn Louis é Pierre se nen van eun salièn tcheutte. Federico nen profite pe s'achouaté si le poltronne protso de la \textit{cubista}.}

\Louisspeaks\direct{A Marco} No scapèn, mi nen tro riette! Tro seumpateucco.

\Michelspeaks Tournade!

\Pierrespeaks\direct{A Michel} Dzen squerse!

\Michelspeaks Salì sotcho! Fé lo sayo!

\Louisspeaks No véyèn!

\Pierrespeaks\direct{A Hélène} Lo Mojito l'iye pa tan bon!

\StageDir{Berta avèitse mal Hélène.}

\Bertaspeaks Voualà, gnenca bon-a a fé sen.

\scene[-- Dou doblo Zibibbo]

\StageDir{Louis é Pierre son caze foua can  euncrouijon d'atre dou gars\'on: eun bièn ate é ipesse, l'atro base é méyo comme eun pioù. Lo petchoù, Paul, baille eunna eun cou d'ipala a Louis. Louis se reveuriye, mi lo pi grou, Auguste, lo pouche contre Pierre.}

\Pierrespeaks Oh tchappa tchuque! Ad\'on!

\StageDir{Le catro (eun réalit\'o le trèi, Auguste contre Pierre é Louis) comenchon a se pouchì. Auguste l'è braamente pi for. Pierre é Louis se tramon ver la chortiya, a tita ata, eun djabléyèn countre le dou mal polì.}

\Pierrespeaks Avèitsade vèi iaou betade le pià! Alèn ià Louis\ldots mi pensa té sise!

\StageDir{Paul é Auguste se plachon i banc\'on.}

\Louisspeaks Bièn al\'o que t'a tegna-me.

\Pierrespeaks\direct{A Dorina} Le palt\'o, soplé.

\StageDir{Dorina baille le palt\'o a Louis é Pierre é le dou se nen van.}

\Helenespeaks\direct{Ver Paul é Auguste} Vouillade bèye?

\Augustespeaks Dou \textit{Zibibbo}.

\Paulspeaks\direct{Arrogàn, eun bouéchèn la man si lo banc\'on} Fé-l\'o doblo!

\Helenespeaks Ouè.

\Bertaspeaks\direct{Tracachaye, a basa vouése} Hélène! Veun inque\ldots soplé fé attench\'on a sise dou. Le cougnisèn, te lo sa que tcheu le cou veugnon inque maque pe tsertchì rize.

\Helenespeaks Ouè va bièn.
 
\StageDir{Hélène soum\'on le dou Zibibbo. Dimèn entron Vilma, Vanda é Ornella.}

\Dorinaspeaks Iaou alade vo trèi! V'ouèide-tì prénotoù?

\Vilmaspeaks Mi na que n'en pa prénotoù! Comèn te pou-no quetì foua oueu? L’è ma fita!

\Dorinaspeaks Mondjemé! Comencho belle a pèdre la tita! T'i maria-te oueu?

\Vandaspeaks Mi na! Te vèi pa comèn l'è arbeillaye? Oueu l'è sa dériye nite de feuille!

\Dorinaspeaks Oh que dzente seutte fite!
Mé me rapello can n'i fitoù ma dériye nite de démouazella! Sen aloù si i sondz\'on di veladzo, pe eun pailleu\ldots

\StageDir{Vanda se stouffie de seutta counta. Pren Vilma é Ornella é le pouche dedeun lo local. Dorina conteneuvve a prédjì, mi Vanda lèi saoute si eun l'eumbrachèn. Lèi baille dou pouteun é la salie. Aprì rejouèn Vilma é Ornella, que son i mentèn di palque preste pe danchì.}

\Djspeaks Senque vèyon mé joueu! L'è arrevaye l'ipaouza\ldots é seutta tsans\'on l'è totta pe vo! 

\sound{https://www.youtube.com/watch?v=8DNQRtmIMxk}{Whigfield - Saturday Night}

\StageDir{Vilma, Ornella é Vanda atacon eunna danse de groupe. I banc\'on Paul é Auguste bouéchon lo ten avouì lo pià. Michel, Marco é Federico tsertson d'alé dérì le pa de danse que fan le trèi feuille.}

\StageDir{La mezeucca bèiche. Vilma é sé-z-amie conteneuvvon a danchì.}

\Augustespeaks\direct{A Paul} Senque fan totte seutte polaille solette? L’arèn cheur fata de compagnì.

\Paulspeaks Ouè, alèn vère.

\StageDir{Arrogàn, man eun secotse, Paul é Auguste s'aprotson i feuille. Paul pounte Vilma.}

\Paulspeaks\direct{A Vilma} Deh dzenta polaille!

\Vilmaspeaks\direct{Eun sourièn, ipouvantaye} Senque t'ou?

\Paulspeaks T'ou fé catro danse?

\StageDir{Vilma pouche ià Paul, mi llou l'è todzor pi agasàn. Vilma danche pamì. Dimèn, Ornella é Vilma son itaye pouchaye ià pe Auguste.}

\Vilmaspeaks\direct{Seuria} Na mersì.

\StageDir{Vilma vou scapé ià, mi, can se vionde, acape Auguste douàn lleu.}

\Paulspeaks Veun protso de mé que danchèn, alé!

\Vilmaspeaks Na!\direct{Terrorizaye, eun braillèn} Giulio!

\Paulspeaks Veun danchì, bailla campa!

\StageDir{Paul comenche a tazeillì Vilma. Lleu  tsertse de se difendre, mi l'è gnacaye i mentèn de Paul é Auguste.}

\StageDir{Areuvve de coursa Giulio que, a tita ata é sensa pouiye, baille de pégno creppe si l'ipala de Auguste. Auguste, comme se l'ise sentì eunna moutse si l'ipala, se vionde. Vilma dzouze lo momàn pe scapé.}

\Paulspeaks\direct{Ver Vilma} Mi avèista salla critin-a! Critin-a!

\scene[-- Patèle é vèyo icllapoù III]

\StageDir{Giulio l'è drette douàn Auguste; lo avèiste avouì la poueunte di na si pe l'er.}

\Giuliospeaks\direct{A Auguste, avouì coadzo} Vilma l'è de mé! L'è eunna fenna eungadjaye, avouì mé!

\StageDir{Auguste tappe pe tèra Giulio.}

\StageDir{Partèi la tsans\'on:}
\sound{https://www.youtube.com/watch?v=KeKQl6BD7yw}{Dune buggy - Oliver Onions}

\StageDir{Le-z-amì de Giulio pouzon lo vèyo é eunterveugnon pe lèi baillì man forta. Partèi eunna bagarre a la Bud Spencer \& Terence Hill: eun aprì l'atro, Giulio, Michel, Federico é Marco, se batton avouì Auguste, que, sensa de grou-z-ifor, le tappe tcheutte ba pe tèra; Hélène é Berta beutton ià tcheu le vèyo; lo DJ, lo valet é la cubista se catson déz\'o la \textit{consolle}; le feuille baillon ba a Paul, que, eun boquèn deur, scappe foua di local. Dimèn areuvon eunc\'o dou djouyaou di fiolet de Tsarvensoù. Vilma le souplie de èidjì Giulio é sé-z-amì. Eun véyèn Auguste, le dou son pa tan cheur de s'eumpouertchì le man, mi, eun prégnèn coadzo, lèi sauton contre. Chouì contre eun, Giulio, le-z-amì é le dou Tsarvensolèn areuvon a tapé foua di local Auguste.}

\StageDir{La mezeucca se tchoué. Dorina pren eugn'icaoua é comenche a poulité lo solàn.}

\Bertaspeaks\direct{Dispéraye} Que dizastre, que dizastre! N'ay\'o bièn deu-lo que fayè fé attench\'on a sise dou! \direct{Malechaye, i DJ} Mi t'i inutilo té! Gneunca bon de alì le-z-èidjì!\direct{Triste} Hélène, l'an tourna tot ibofou-no!

\Helenespeaks Mé poulito!

\Bertaspeaks Ouè! Té poulita, mé payo!

\Dorinaspeaks Véo de vèyo icllapoù!

\Bertaspeaks Todzor mé que le payo! Mi iaou son fenè le-z-atre?\direct{Eun avèitchèn foua} Gars\'on vegnade! Vegnade dedeun! Si cou vo lo payo mé eun tor!

\StageDir{Entron Giulio, Vilma é tcheu leur amì.}

\Bertaspeaks Vi que v'ouèide tsachà ià sise atacarize\ldots Hélène, beutta eun tor pe tcheutte! Payo mé!

\Helenespeaks Va bièn!

\Bertaspeaks\direct{Eun prédzèn i pebleuque} Qui l'arì jamì deu-lo que eun dzor Pollentch\'on é Tsarvensolèn l’arian betoù de coutì totte  leur discuch\'on é plantou-la li de tsacotì pe de counte foule\ldots va savèi tanque can!

\Djspeaks Pollein! Disco Flama! Eunc\'o oueu n'en fé seunq'aoue di mateun! Ara tcheutte a danchì la dériye tsans\'on:

\sound{https://www.youtube.com/watch?v=ViP87WipSm0}{Be My Lover - La Bouche}

\StageDir{Tcheu le personadzo entron pe danchì. Le-z-atteur tsarvensolèn avouì eunna maille rodza avouì l'icrita \og Tsarvensoù pa de mioù\fg é sise de Pollein avouì eunna maille bleua avouì l'icrita \og Pollentchoùn le pi boun\fg. Ver la feun de la danse le Tsarvensolèn é le Pollentch\'on se divijon eun dou groupe, eun a drèite é eun a gotse di palque. Sembleria que sian tourna preste a se battre é a tsacoté\ldots}

\StageDir{La mezeucca se tchoué. Berta se plache i mentèn di doe fach\'on.}

\Bertaspeaks Oh na! Lèi sen tourna! Pollentch\'on é Tsarvensolèn alérèn jamì d'acor!

\ridocliou

\DeriLeRido

\RoleNoms{Chorégraphie}{Jo\"elle Bollon}

\RoleNoms{Collaborateur}{Flavio Albaney, Enrica Pieiller}

\RoleNoms{Souffleur é mezeucca}{Jasmine Comé, Giada Grivon, Ilaria Linty}

\RoleNoms{Maquillaje}{Muriel Pomat}



\end{drama}